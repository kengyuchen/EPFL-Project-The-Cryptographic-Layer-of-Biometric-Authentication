%%%%%%%%%%%%%%%%%%%%

% Project Name: Semester Project Fall 2024 for EPFL
% File: preliminaries.tex
% Author: Keng-Yu Chen

%%%%%%%%%%%%%%%%%%%%

In this project, we assume

\begin{itemize}
	
	\item $\lambda$ is the security parameter.

	\item $[m]$ denotes the set of integers $\{1, 2, \cdots, m\}$.

	\item $\Z_q$ is the finite field modulo a prime number $q$.

	\item A function $f(n)$ is called \emph{negligible} iff for any integer $c$, $f(n) < \frac{1}{n^c}$ for all sufficiently large $n$. We write it as $f(n) = \negl$, and we may also use $\negl$ to represent an arbitrary negligible function.
	
	\item $\poly$ is the class of polynomial funcions. We may also use $\poly$ to represent an arbitrary polynomial function.
	
	\item We write sampling a value $r$ from a distribution $\mathcal{D}$ as $r \getsdollar \mathcal{D}$. If $S$ is a finite set, then $r \getsdollar S$ means sampling $r$ uniformly from $S$.

	\item The distribution $\mathcal{D}^t$ denotes $t$ identical and independent distributions of $\mathcal{D}$.

	\item A PPT algorithm denotes a probabilistic polynomial time algorithm. Unless otherwise specified, all algorithms run in PPT.

\end{itemize}

%We introduce three types of inner product functional encryption schemes: function hiding functional encryption, two-input functional encryption, and two-client functional encryption. We will instantiate our biometric authentication scheme using these primitives.

We introduce two primitives to instantiate a biometric authentication scheme: function-hiding inner product functional encryption and relatioanl hash.

\begin{definition}[Function-Hiding Inner Product Functional Encryption (adapted from \cite{cryptoeprint:2016/440}) ]
\label{def:fh-IPFE}
	A \emph{function-hiding inner product functional encryption} (fh-IPFE) scheme $\sf FE$ for a field $\mathbb{F}$ and input length $k$ is composed of PPT algorithms $\textsf{FE.Setup}$, $\textsf{FE.KeyGen}$, $\textsf{FE.Enc}$, and $\textsf{FE.Dec}$:

	\begin{itemize}
	
		\item $\textsf{FE.Setup}(1^\lambda) \to \textsf{msk}, \textsf{pp}$: It outputs the public parameter $\textsf{pp}$ and the master secret key $\textsf{msk}$.
	
		\item $\textsf{FE.KeyGen}(\textsf{msk}, \textsf{pp}, \mathbf{x}) \to f_\mathbf{x}$: It generates the functional decryption key $f_\mathbf{x}$ for an input vector $\mathbf{x} \in \mathbb{F}^k$. 
	
		\item $\textsf{FE.Enc}(\textsf{msk}, \textsf{pp}, \mathbf{y}) \to \mathbf{c_y}$: It encrypts the input vector $\mathbf{y} \in \mathbb{F}^k$ to the ciphertext $\mathbf{c_y}$. 
	
		\item $\textsf{FE.Dec}(\textsf{pp}, f_\mathbf{x}, \mathbf{c_y}) \to z$: It outputs a value $z \in \mathbb{F}$ or an error symbol $\bot$.
	
	\end{itemize}
	
	\noindent \textbf{Correctness}: An fh-IPFE scheme \textsf{FE} is \emph{correct} if $\forall (\textsf{msk}, \textsf{pp}) \gets \textsf{FE.Setup}(1^\lambda)$ and $ \mathbf{x}, \mathbf{y} \in \mathbb{F}^k$, we have
	\[
		\textsf{FE.Dec}( \textsf{pp}, \textsf{FE.KeyGen}(\textsf{msk}, \textsf{pp}, \mathbf{x}), \textsf{FE.Enc}(\textsf{msk}, \textsf{pp}, \mathbf{y}) ) = \mathbf{x} \mathbf{y}^T \in \mathbb{F}.
	\]

\end{definition}

Instantiation using an fh-IPFE scheme is given in Section \ref{sec:fh-IPFE-instantiation}.

%\begin{definition}[Two-Input Inner Product Functional Encryption (adapted from \cite{cryptoeprint:2022/441})]
%\label{def:2i-IPFE}
	%A \emph{two-input inner product functional encryption} (2i-IPFE) scheme $\sf FE$ for a field $\mathbb{F}$ and input length $k$ is composed of PPT algorithms $\textsf{FE.Setup}$, $\textsf{FE.KeyGen}$, $\textsf{FE.Enc}$, and $\textsf{FE.Dec}$:

	%\begin{itemize}
	
		%\item $\textsf{FE.Setup}(1^\lambda) \to \textsf{sk}, \textsf{ek}_1, \textsf{ek}_2$: It outputs a secret key $\textsf{sk}$ and two encryption keys $\textsf{ek}_1, \textsf{ek}_2$.
	
		%\item $\textsf{FE.KeyGen}(\textsf{sk}, \mathbf{A}) \to \textsf{dk}_\mathbf{A}$: It generates the functional decryption key $\textsf{dk}_\mathbf{A}$ for a diagonal matrix $\mathbf{A} \in \mathbb{F}^{k \times k}$,  
	
		%\item $\textsf{FE.Enc}(\textsf{ek}_i, \mathbf{x}) \to \mathbf{c_x}$: Given an encryption key, either $\textsf{ek}_1$ or $\textsf{ek}_2$, it encrypts the input vector $\mathbf{x} \in \mathbb{F}^k$ to the ciphertext $\mathbf{c_x}$. 
	
		%\item $\textsf{FE.Dec}(\textsf{dk}_\mathbf{A}, \mathbf{c_x}, \mathbf{c_y}) \to z$: It outputs a value $z \in \mathbb{F}$.
	
	%\end{itemize}
	
	%\noindent \textbf{Correctness}: A 2i-IPFE scheme \textsf{FE} is \emph{correct} if $\forall (\textsf{sk}, \textsf{ek}_1, \textsf{ek}_2) \gets \textsf{FE.Setup}(1^\lambda), \mathbf{A} \in \mathbb{F}^{k \times k}$, and $\mathbf{x}, \mathbf{y} \in \mathbb{F}^k$, we have

	%\[
		%\textsf{FE.Dec}(\textsf{FE.KeyGen}(\textsf{sk},  \mathbf{A}), \textsf{FE.Enc}(\textsf{ek}_1, \mathbf{x}), \textsf{FE.Enc}(\textsf{ek}_2, \mathbf{y}) ) = \mathbf{x} \mathbf{A} \mathbf{y}^T \in \mathbb{F}.
	%\]

%\end{definition}

%Instantiation using a 2i-IPFE is given in Section \ref{sec:2i-IPFE-instantiation}.

%\begin{definition}[Two-Client Inner Product Functional Encryption (adapted from \cite{cryptoeprint:2022/441})]
%\label{def:2c-IPFE}
	%A \emph{two-client inner product functional encryption} (2c-IPFE) scheme $\sf FE$ for a field $\mathbb{F}$ and input length $k$ is composed of PPT algorithms $\textsf{FE.Setup}$, $\textsf{FE.KeyGen}$, $\textsf{FE.Enc}$, and $\textsf{FE.Dec}$:

	%\begin{itemize}
	
		%\item $\textsf{FE.Setup}(1^\lambda) \to \textsf{sk}, \textsf{ek}_1, \textsf{ek}_2$: It outputs a secret key $\textsf{sk}$ and two encryption keys $\textsf{ek}_1, \textsf{ek}_2$.
	
		%\item $\textsf{FE.KeyGen}(\textsf{sk}, \mathbf{A}) \to \textsf{dk}_\mathbf{A}$: It generates the functional decryption key $\textsf{dk}_\mathbf{A}$ for a diagonal matrix $\mathbf{A} \in \mathbb{F}^{k \times k}$,  
	
		%\item $\textsf{FE.Enc}(\ell, \textsf{ek}_i, \mathbf{x}) \to \mathbf{c_x}$: Given a label $\ell$ and an encryption key, either $\textsf{ek}_1$ or $\textsf{ek}_2$, it encrypts the input vector $\mathbf{x} \in \mathbb{F}^k$ to the ciphertext $\mathbf{c_x}$. 
	
		%\item $\textsf{FE.Dec}(\textsf{dk}_\mathbf{A}, \mathbf{c_x}, \mathbf{c_y}) \to z$: It outputs a value $z \in \mathbb{F}$.
	
	%\end{itemize}
	
	%\noindent \textbf{Correctness}: A 2c-IPFE scheme \textsf{FE} is \emph{correct} if $\forall (\textsf{sk}, \textsf{ek}_1, \textsf{ek}_2) \gets \textsf{FE.Setup}(1^\lambda), \mathbf{A} \in \mathbb{F}^{k \times k}$, label $\ell$, and $ \mathbf{x}, \mathbf{y} \in \mathbb{F}^k$, we have
	%\[
		%\textsf{FE.Dec}(\textsf{FE.KeyGen}(\textsf{sk},  \mathbf{A}), \textsf{FE.Enc}(\ell, \textsf{ek}_1, \mathbf{x}), \textsf{FE.Enc}(\ell, \textsf{ek}_2, \mathbf{y}) ) = \mathbf{x} \mathbf{A} \mathbf{y}^T \in \mathbb{F}.
	%\]

%\end{definition}

%Instantiation using a 2c-IPFE is given in Section \ref{sec:2c-IPFE-instantiation}.


\begin{definition}[Relational Hash (adapted from \cite{cryptoeprint:2014/394})]
\label{def:rh}
	Let $R_\lambda$ be a relation over sets $X_\lambda, Y_\lambda$, and $ Z_\lambda$. A \emph{relational hash} scheme \textsf{RH} for $R_\lambda$ consists of PPT algorithms \textsf{RH.KeyGen}, $\textsf{RH.HASH}_1$, $\textsf{RH.HASH}_2$, and \textsf{RH.Verify}:
	
	\begin{itemize}
	
		\item $\textsf{RH.KeyGen}(1^\lambda) \to \textsf{pk}$: It outputs a public hash key \textsf{pk}.  
			
		\item $\textsf{RH.Hash}_1(\textsf{pk}, \mathbf{x}) \to \mathbf{h_x}$: Given a hash key \textsf{pk} and $\mathbf{x} \in X_\lambda$, it outputs a hash $\mathbf{h_x}$.

		\item $\textsf{RH.Hash}_2(\textsf{pk}, \mathbf{y}) \to \mathbf{h_y}$: Given a hash key \textsf{pk} and $\mathbf{y} \in Y_\lambda$, it outputs a hash $\mathbf{h_y}$.

		\item $\textsf{RH.Verify}(\textsf{pk}, \mathbf{h_x}, \mathbf{h_y}, \mathbf{z}) \to r \in \{0, 1\}$: Given a hash key \textsf{pk}, two hashes $\mathbf{h_x}$ and $\mathbf{h_y}$, and $\mathbf{z} \in Z_\lambda$, it verifies whether the relation among $\mathbf{x}, \mathbf{y}$ and $\mathbf{z}$ holds.

	\end{itemize}

	\noindent \textbf{Correctness}: A relational hash scheme \textsf{RH} is \emph{correct} if $\forall \mathbf{x}, \mathbf{y}, \mathbf{z} \in X_\lambda \times Y_\lambda \times Z_\lambda$,
	\[
		\Pr \left [
			\begin{aligned} 
				 &\; \textsf{pk} \gets \textsf{RH.KeyGen}(1^\lambda) \\
				 &\; \mathbf{h_x} \gets \textsf{RH.Hash}_1(\textsf{pk}, \mathbf{x}) \\
				 &\; \mathbf{h_y} \gets \textsf{RH.Hash}_2(\textsf{pk}, \mathbf{y})
			\end{aligned} :
			\textsf{RH.Verify}(\textsf{pk}, \mathbf{h_x}, \mathbf{h_y}, \mathbf{z}) = R(\mathbf{x}, \mathbf{y}, \mathbf{z})
			\right ] = 1 - \negl.
	\]
\end{definition}

\noindent Note that $Z_\lambda$ is an auxiliary input. When the relation $R$ is over two sets $X \times Y$, we ignore $Z$ and write $\textsf{RH.Verify}(\textsf{pk}, \mathbf{h_x}, \mathbf{h_y})$.

Instantiation using a relational hash is given in Section \ref{sec:rh-instantiation}.

