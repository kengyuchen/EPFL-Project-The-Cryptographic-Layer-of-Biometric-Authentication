%%%%%%%%%%%%%%%%%%%%

% Project Name: Semester Project Fall 2024 for EPFL
% File: security_analysis_rh.tex
% Author: Keng-Yu Chen

%%%%%%%%%%%%%%%%%%%%

\begin{definition}[Relational Hash (adapted from \cite{cryptoeprint:2014/394})]
\label{def:rh}
	Let $R$ be a relation over sets $X, Y$, and $Z$. A \emph{relational hash} scheme \textsf{RH} for $R$ consists of PPT algorithms \textsf{RH.KeyGen}, $\textsf{RH.HASH}_1$, $\textsf{RH.HASH}_2$, and \textsf{RH.Verify}:
	
	\begin{itemize}
	
		\item $\textsf{RH.KeyGen}(1^\lambda) \to \textsf{pk}$: It outputs a public hash key \textsf{pk}.  
			
		\item $\textsf{RH.Hash}_1(\textsf{pk}, \mathbf{x}) \to \mathbf{h_x}$: Given a hash key \textsf{pk} and $\mathbf{x} \in X$, it outputs a hash $\mathbf{h_x}$.

		\item $\textsf{RH.Hash}_2(\textsf{pk}, \mathbf{y}) \to \mathbf{h_y}$: Given a hash key \textsf{pk} and $\mathbf{y} \in Y$, it outputs a hash $\mathbf{h_y}$.

		\item $\textsf{RH.Verify}(\textsf{pk}, \mathbf{h_x}, \mathbf{h_y}, \mathbf{z}) \to r \in \{0, 1\}$: Given a hash key \textsf{pk}, two hashes $\mathbf{h_x}$ and $\mathbf{h_y}$, and $\mathbf{z} \in Z$, it verifies whether the relation among $\mathbf{x}, \mathbf{y}$ and $\mathbf{z}$ holds.

	\end{itemize}

	\paragraph{\textbf{Correctness}} A relational hash scheme \textsf{RH} is \emph{correct} if $\forall \mathbf{x}, \mathbf{y}, \mathbf{z} \in X \times Y \times Z$,
	\[
		\Pr \left [
			\begin{aligned} 
				 &\; \textsf{pk} \gets \textsf{RH.KeyGen}(1^\lambda) \\
				 &\; \mathbf{h_x} \gets \textsf{RH.Hash}_1(\textsf{pk}, \mathbf{x}) \\
				 &\; \mathbf{h_y} \gets \textsf{RH.Hash}_2(\textsf{pk}, \mathbf{y})
			\end{aligned} :
			\textsf{RH.Verify}(\textsf{pk}, \mathbf{h_x}, \mathbf{h_y}, \mathbf{z}) = R(\mathbf{x}, \mathbf{y}, \mathbf{z})
			\right ] = 1 - \negl.
	\]
\end{definition}

\noindent Note that $Z_\lambda$ is an auxiliary input. When the relation $R$ is over two sets $X \times Y$, we ignore $Z$ and write $\textsf{RH.Verify}(\textsf{pk}, \mathbf{h_x}, \mathbf{h_y})$.

%-------------------

\subsection{Instantiation with a Relational Hash Scheme}
\label{sec:rh-instantiation}

Let $\textsf{RH} = (\textsf{RH.KeyGen}, \textsf{RH.Hash}_1, \textsf{RH.Hash}_2, \textsf{RH.Verify})$ be a relational hash scheme for the relation $R^\tau$ of Hamming distance proximity parametrized by a constant $\tau$.
\[
	R^\tau = \{ (\mathbf{x}, \mathbf{y}) \mid \textsf{HD}(\mathbf{x}, \mathbf{y}) \leq \tau \wedge \mathbf{x}, \mathbf{y} \in \{0,1\}^k \}
\]
Note that here we ignore the third parameter $Z$.
Let $\textsf{getEnroll}^{\mathcal{O}_{\mathcal{B}}}()$ and $\textsf{getProbe}^{\mathcal{O}_{\mathcal{B}}}()$ both output vectors in $ \{0, 1\}^k$ for all biometric distributions $\mathcal{B} \in \mathbb{B}$, and let 
\[
	\textsf{BioCompare}(\mathbf{b}, \mathbf{b}^\prime) \to
	\begin{cases}
		1 & \text{if } (\mathbf{b}, \mathbf{b}^\prime) \in R^\tau \\
		0 & \text{if } (\mathbf{b}, \mathbf{b}^\prime) \notin R^\tau
	\end{cases} \quad \text{and} \quad
	\textsf{Verify}(s) \to s.
\]
Following \cite{cryptoeprint:2014/394}, we can instantiate a biometric authentication scheme using $\textsf{RH}$.  Let the biometric distribution $\mathcal{B} \subseteq \{0,1\}^k$.

\begin{itemize}

	\item $\textsf{Setup}(1^\lambda)$: It calls $\textsf{RH.KeyGen}(1^\lambda) \to \textsf{pk}$ and outputs $\textsf{esk} \gets \textsf{pk}$, $\textsf{psk} \gets \textsf{pk}$, and $\textsf{csk} \gets \textsf{pk}$.

	\item $\textsf{Enroll}(\textsf{esk}, \mathbf{b})$: Let $\mathbf{x} \gets \mathbf{b}$. It calls $\textsf{RH.Hash}_1(\textsf{pk}, \mathbf{x}) \to \mathbf{h_x}$ and outputs $\mathbf{c_x} \gets \mathbf{h_x}$.

	\item $\textsf{Probe}(\textsf{psk}, \mathbf{b}^\prime)$: Let $\mathbf{y} \gets \mathbf{b}$. It calls $\textsf{RH.Hash}_2(\textsf{pk}, \mathbf{y}) \to \mathbf{h_y}$ and outputs $\mathbf{c_y} \gets \mathbf{h_y}$.

	\item $\textsf{Compare}(\textsf{csk}, \mathbf{c_x}, \mathbf{c_y)}$: It calls $\textsf{RH.Verify}(\textsf{pk}, \mathbf{h_x}, \mathbf{h_y}) \to s$ and outputs the value $s$.

\end{itemize}

By the correctness of the relational hash scheme $\textsf{RH}$, we have (except for a negligible probability),
\[
	r = 1 \Leftrightarrow (\mathbf{x}, \mathbf{y}) = (\mathbf{b}, \mathbf{b}^\prime) \in R^\tau \Leftrightarrow \textsf{HD}(\mathbf{b}, \mathbf{b}^\prime) \leq \tau
\]

The idea behing this construction is that users hold a personal device which runs \textsf{Setup}, \textsf{Enroll}, and \textsf{Probe} using templates coming from a biometric sensor. The message $\mathbf{c_x}$ along with the comparison key $\textsf{csk} = \textsf{pk}$, which is assumed to be public, are sent to a server in order to enroll the user, and $\mathbf{c_y}$ is sent on authentication.


Now, let $\Pi$ be an authentication scheme instantiated by a relational hash scheme \textsf{RH}. We discuss the UF and IND security of $\Pi$ in the following subsections.

%-------------------

\subsection{UF Security of $\Pi$}
\label{sec:security_analysis:rh:uf}

We first recall the unforgeability \cite{cryptoeprint:2014/394} of a relational hash scheme.

\begin{definition}[Unforgeability]

A relational hash scheme \textsf{RH} is called \emph{unforgeable} for the distribution $\mathcal{X}$ if for any adversary $\mathcal{A}$, the following probability is negligible.
\[
	\Pr \left [
		\begin{aligned} 
			 &\; \mathbf{x} \getsdollar \mathcal{X} \\
			 &\; \textsf{pk} \gets \textsf{RH.KeyGen}(1^\lambda) \\
			 &\; \mathbf{h_x} \gets \textsf{RH.Hash}_1(\textsf{pk}, \mathbf{x}) \\
			 &\; \mathbf{\tilde{z}} \gets \mathcal{A}(\textsf{pk}, \mathbf{h_x})
		\end{aligned} :
		\textsf{RH.Verify}(\textsf{pk}, \mathbf{h_x}, \mathbf{\tilde{z}}) = 1
		\right ] = \negl.
\]

\noindent In this work, since we assume the existence of a family $\mathbb{B}$ of biometric distributions and interfaces to interact with it, we extend the notion to \emph{unforgeable for any adversary who has access to $\mathbb{B}$}.
\end{definition}

\begin{theorem}
\label{thm:rh:uf-uf-cx}

Let $\textsf{option} = \{\textsf{esk}, \textsf{psk}, \textsf{csk}, \mathbf{c_x} \}$. If \textsf{RH} is unforgeable for the distribution
\[
	\mathcal{X} = \{ \mathcal{B} \getsdollar \mathbb{B}: \mathbf{b} \gets {\sf getEnroll}^{\mathcal{O}_\mathcal{B}}() \},
\]
for any adversary who has access to $\mathbb{B}$, then $\Pi$ is $\textsf{option}$-UF secure. 

\end{theorem}

In \cite{cryptoeprint:2014/394}, the authors construct an $\textsf{RH}$ that is unforgeable for the uniform distribution over $\{0, 1\}^k$, under the hardness of some computational problems.

\begin{proof}

Recall that the distribution of $\mathbf{c_x}$ in the \textsf{UF} game of the instantiation of Section \ref{sec:rh-instantiation} is
\[
	\left \{
		\begin{aligned} 
			 & \mathcal{B} \getsdollar \mathbb{B} \\
			 & \textsf{pk} \gets \textsf{RH.KeyGen}(1^\lambda) \\
			 & \mathbf{x} = \mathbf{b} \gets \textsf{getEnroll}^{\mathcal{O}_\mathcal{B}}() 
		\end{aligned} :
		\mathbf{c_x} \gets \textsf{RH.Hash}_1(\textsf{pk}, \mathbf{x})
	\right \}
\]
Also recall that $\textsf{Verify}(\textsf{Compare}(\textsf{csk}, \mathbf{c_x}, \mathbf{\tilde{z}} )) = \textsf{RH.Verify}(\textsf{pk}, \mathbf{c_x}, \mathbf{\tilde{z}} )$.
The \textsf{option}-UF security is thus guaranteed by the unforgeability of \textsf{RH}.

\end{proof}

\paragraph{Remark}
As we mentioned in Section \ref{sec:uf_game}, an adversary with \textsf{psk} can enjoy a winning rate of the false positive rate \textsf{FP} of $\mathbb{B}$. Theorem \ref{thm:rh:uf-uf-cx} thus implies that if $\textsf{FP}$ is not negligible, there does not exist an \textsf{RH} that is unforgeable for the distribution $\{ \mathcal{B} \getsdollar \mathbb{B}: \mathbf{b} \gets {\sf getEnroll}^{\mathcal{O}_\mathcal{B}}() \}$ for any adversary who has access to $\mathbb{B}$.


Note that since $\textsf{esk}, \textsf{psk}$, and $ \textsf{csk}$ are all public in this instantiation, it is meaningless to discuss $\mathcal{O}_\textsf{Enroll}, \mathcal{O}_\textsf{Probe}$, or $\mathcal{O}_\textsf{log}$. In addition, for $\textsf{option}$ that includes $\mathcal{O}_\mathcal{B}$ or $\mathcal{O}_\textsf{Probe}^\prime$, as discussed in Section \ref{sec:uf_game}, we cannot achieve \textsf{option}-UF security since $\textsf{psk}$ is public in this instantiation.

For \textsf{option} that includes $\mathcal{O}_\textsf{Enroll}^\prime$, we notice that for the \textsf{RH} construction in \cite{cryptoeprint:2014/394}, there exists an invalid $\textsf{pk}^\prime$ such that $\textsf{RH.Hash}_1(\textsf{pk}^\prime, \mathbf{x})$ directly leaks $\mathbf{x}$. By returning $\textsf{RH.Hash}_2( \textsf{pk}, \mathbf{x} )$, one can break the $\textsf{UF}_{\textsf{option}}$ game with probability $1$.


%-------------------

\subsection{IND Security of $\Pi$}
\label{sec:security_analysis:rh:IND}

For the IND security, since $\textsf{esk}, \textsf{psk}$ and $\textsf{csk}$ are assumed to be public in this instantiation and should given in $\textsf{option}$, by applying Theorem \ref{thm:ind-tp-fp}, $\Pi$ is not $\textsf{option}$-IND secure for any \textsf{option} that includes $\mathbf{c_x}$ or $\mathcal{O}_{\mathbf{c_y}}$.

\begin{theorem}

Let $\textsf{option} = \{\textsf{esk}, \textsf{psk}, \textsf{csk}, \mathbf{c_x}\}$ or $\{\textsf{esk}, \textsf{psk}, \textsf{csk}, \mathcal{O}_{\mathbf{c_y}} \}$. For any distribution family $\mathbb{B}$ that $\textsf{TP} - \textsf{FP} > \frac{1}{\poly}$, and for any relational hash scheme \textsf{RH}, $\Pi$ is not $\textsf{option}$-IND secure.

\end{theorem}

\subsubsection{IND Security for a Particular Biometric Layer}

Recall that in Section \ref{sec:ind_game}, we introduce as an example a particular biometric layer:
\[
	\textsf{getEnroll}^{\mathcal{O}_{\mathcal{B}}}() \to \mathbf{b}^* + \mathcal{E}_{\textsf{Enroll}}  \quad \text{and} \quad \textsf{getProbe}^{\mathcal{O}_{\mathcal{B}}}() \to \mathbf{b}^* + \mathcal{E}_{\textsf{Probe}}
\]
where $\mathbf{b}^* \in \{0, 1\}^k$ is a fixed vector only dependent on $\mathcal{B}$, and $\mathcal{E}_{\textsf{Enroll}}, \mathcal{E}_{\textsf{Probe}} \subseteq \{0, 1\}^k$ are some \emph{error distributions} independent of $\mathcal{B}$.
With the same relational hash \textsf{RH} in Section \ref{sec:rh-instantiation}, we can instantiate another authentication scheme using \textsf{RH}.

\begin{itemize}

	\item $\textsf{Setup}(1^\lambda)$: It runs $\textsf{RH.KeyGen}(1^\lambda) \to \textsf{pk}$ and samples $\mathbf{r} \getsdollar \{0, 1\}^k$. Then it outputs $\textsf{esk} \gets (\textsf{pk}, \mathbf{r})$, $\textsf{psk} \gets (\textsf{pk}, \mathbf{r})$, and $\textsf{csk} \gets \textsf{pk}$.

	\item $\textsf{Enroll}(\textsf{esk}, \mathbf{b})$: Let $\mathbf{x} \gets \mathbf{b}$. It calls $\textsf{RH.Hash}_1(\textsf{pk}, \mathbf{x} + \mathbf{r}) \to \mathbf{h_x}$ and outputs $\mathbf{c_x} \gets \mathbf{h_x}$.

	\item $\textsf{Probe}(\textsf{psk}, \mathbf{b}^\prime)$: Let $\mathbf{y} \gets \mathbf{b}$. It calls $\textsf{RH.Hash}_2(\textsf{pk}, \mathbf{y} + \mathbf{r}) \to \mathbf{h_y}$ and outputs $\mathbf{c_y} \gets \mathbf{h_y}$.

	\item $\textsf{Compare}(\textsf{csk}, \mathbf{c_x}, \mathbf{c_y)}$: It calls $\textsf{RH.Verify}(\textsf{pk}, \mathbf{h_x}, \mathbf{h_y}) \to s$ and outputs the value $s$.

\end{itemize}
Correctness holds because
\[
	\textsf{Compare}(\textsf{csk}, \mathbf{c_x}, \mathbf{c_y)} = 1 \Leftrightarrow \textsf{HD}(\mathbf{x} + \mathbf{r}, \mathbf{y} + \mathbf{r}) \leq \tau \Leftrightarrow \textsf{HD}(\mathbf{x}, \mathbf{y}) \leq \tau = \textsf{BioCompare}(\mathbf{b}, \mathbf{b}^\prime).
\]

With the same argument in Theorem \ref{thm:rh:ind:particular-biometri-layer}, one can prove that this new scheme is now $\{\textsf{csk}, \mathbf{c_x}, \mathcal{O}_{\mathbf{c_y}}\}$-IND secure, albeit at the cost of requiring $\textsf{esk}$ and $\textsf{psk}$ to remain secret.
