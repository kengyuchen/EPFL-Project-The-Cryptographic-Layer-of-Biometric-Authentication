%%%%%%%%%%%%%%%%%%%%

% Project Name: Semester Project Fall 2024 for EPFL
% File: security_game.tex
% Author: Keng-Yu Chen

%%%%%%%%%%%%%%%%%%%%

In this section, we discuss two security notions of a biometric authentication scheme: \emph{unforgeability} and \emph{indistinguishability}.

\subsection{Unforgeability}
\label{sec:uf_game}

To describe the unforgeability of an authentication scheme, we model the ability of an adversary who tries to impersonate a user. The adversary $\mathcal{A}$ is given auxiliary information \textsf{option} that depends on our threat model and tries to find a valid probe message $\mathbf{\tilde{z}}$. The whole game $\textsf{UF}_{\Pi, \mathbb{B}, \textsf{option}}$ is defined in Algorithm \ref{alg:uf_game}.

\begin{figure}[h]
\centering
	\begin{minipage}[t]{0.6\linewidth}
	\centering
	\begin{algorithm}[H]
	\caption{$\textsf{UF}_{\Pi, \mathbb{B}, \textsf{option}}(\mathcal{A})$}
	\label{alg:uf_game}
	\begin{algorithmic}[1]
		\State $\mathcal{B} \getsdollar \mathbb{B}, \mathbb{B} \gets \mathbb{B} \setminus \mathcal{B}$

		\State $\mathsf{esk}, \mathsf{psk}, \mathsf{csk} \gets \mathsf{Setup}(1^\lambda)$
		
		\State $\mathbf{b} \gets \mathsf{getEnroll}^{\mathcal{O}_{\mathcal{B}}}()$

		\State $\mathbf{c_x} \gets \mathsf{Enroll}(\mathsf{esk}, \mathbf{b})$
		
		\State ${\mathbf{\tilde{z}}} \gets \mathcal{A} ( \textsf{option} )$
 
		\If{$\mathbf{\tilde{z}}$ is equal to any output of $\mathcal{O}_{\mathsf{Probe}}$ }
			
			\State \Return $0$
		
		\EndIf

		\State $s \gets \mathsf{Compare}( \mathsf{csk}, \mathbf{c_x}, \mathbf{\tilde{z}} )$

		\State \Return $\mathsf{Verify}(s)$
	\end{algorithmic}
	\end{algorithm}
	\end{minipage}
	
\label{fig:uf_game}
\end{figure}

The auxiliary information \textsf{option} can be nothing or include $\mathsf{esk}, \mathsf{psk}, \mathsf{csk}, \mathbf{c_x}$ or the following oracles:

\begin{itemize}

	\item $\mathcal{O}_{\mathcal{B}}$: It outputs a biometric sample $\mathbf{b} \getsdollar \mathcal{B}$. This oracle and $\mathsf{psk}$ should not be given at the same time; otherwise, there exists a trivial attack with a winning rate $\mathsf{TP}$ by returning $\mathsf{Probe}(\mathsf{psk}, \mathsf{getProbe}^{ \mathcal{O}_\mathcal{B} }() )$.
	
	\item $\mathcal{O}_{\mathsf{Enroll}}(\mathsf{esk}, \cdot)$: On input $\mathbf{b}^\prime$, it outputs the enrollment message $\mathsf{Enroll}(\mathsf{esk}, \mathbf{b}^\prime)$.

	\item $\mathcal{O}_{\mathsf{Probe}}(\mathsf{psk}, \cdot)$: On input $\mathbf{b}^\prime$, it outputs the probe message $\mathsf{Probe}(\mathsf{psk}, \mathbf{b}^\prime)$. If this oracle is given, we require the adversary to return a $\mathbf{\tilde{z}}$ that is not equal to any previous answer of $\mathcal{O}_\mathsf{Probe}$.
	
	\item $\mathcal{O}_\mathsf{log}(\mathsf{csk}, \mathbf{c_x}, \cdot)$: On input $\mathbf{b}^\prime$, it first computes $\mathbf{c_z} \gets \mathsf{Probe}(\mathsf{psk}, \mathbf{b}^\prime)$ and outputs $\mathsf{Verify}(\mathsf{Compare}(\mathsf{csk}, \mathbf{c_x}, \mathbf{c_z} ) )$.
	
	\item $\mathcal{O}_\textsf{Enroll}^\prime (\cdot)$: On input $\textsf{esk}^\prime$, it first samples $\mathbf{b}^\prime \gets \textsf{getEnroll}^{\mathcal{O}_{\mathcal{B}}}()$ and outputs $\textsf{Enroll}(\textsf{esk}^\prime, \mathbf{b}^\prime)$. This oracle is only useful when $\textsf{option}$ does not include $\mathcal{O}_{\mathcal{B}}$.

	\item $\mathcal{O}_\textsf{Probe}^\prime (\cdot)$: On input $\textsf{psk}^\prime$, it first samples $\mathbf{b}^\prime \gets \textsf{getProbe}^{\mathcal{O}_{\mathcal{B}}}()$ and outputs $\textsf{Probe}(\textsf{psk}^\prime, \mathbf{b}^\prime)$. This oracle is only useful when $\textsf{option}$ does not include $\mathcal{O}_{\mathcal{B}}$. This oracle and $\textsf{psk}$ should not be given at the same time; otherwise, there exists a trivial attack with a winning rate $\textsf{TP}$ by returning $\mathcal{O}_{\textsf{Probe}}^\prime (\textsf{psk})$.
	
\end{itemize}


We define the advantage of an adversary $\mathcal{A}$ in the $\textsf{UF}_{\Pi, \mathbb{B}, \textsf{option}}$ game of a scheme $\Pi$ associated with a family $\mathbb{B}$ of distributions as
\[
	\Adv^{\textsf{UF}}_{\Pi, \mathbb{B}, \mathcal{A}, \textsf{option}} := \Pr[\textsf{UF}_{\Pi, \mathbb{B}, \textsf{option}}(\mathcal{A}) \to 1]
\]

An authentication scheme $\Pi$ associated with a family $\mathbb{B}$ of distributions is called \emph{\textsf{option}-unforgeable} (\textsf{option}-UF) if for any PPT adversary $\mathcal{A}$,
\[
	\Adv^{\textsf{UF}}_{\Pi, \mathbb{B}, \mathcal{A}, \textsf{option}} = \negl.
\]

For the rest of this work, if the scheme $\Pi$, the family $\mathbb{B}$ of distributions, and the auxiliary information $\textsf{option}$ are clear from context, we omit the subscript and write the game as $\textsf{UF}(\mathcal{A})$. This abbreviation also holds for all other games.

\paragraph{Choice of \textsf{option}}
\noindent Consider a \textsf{UF} adversary $\mathcal{A}$ in Algorithm \ref{alg:adv:FP}. The $\textsf{option}$ includes either $\textsf{psk}$ or $\mathcal{O}_{\textsf{Probe}}$. The advantage of $\mathcal{A}$ is
\begin{align*}
	& \Pr[\textsf{UF}_{\Pi, \mathbb{B}, \textsf{option}}(\mathcal{A}) \to 1] \\ 
	=\; & \Pr [ \textsf{Verify}(\textsf{Compare}(\textsf{csk}, \mathbf{c_x}, \mathbf{c_y})) = 1 ] \\
	=\; & \Pr \left [
		\begin{aligned}
			& \mathcal{B} \getsdollar \mathbb{B}, \mathbb{B} \gets \mathbb{B} \setminus \mathcal{B} \\
			& \mathbf{b} \gets \textsf{getEnroll}^{\mathcal{O}_{\mathcal{B}} }() \\
			& \mathcal{B}^\prime \getsdollar \mathbb{B} \\
			& \mathbf{b}^\prime \gets \textsf{getProbe}^{\mathcal{O}_{\mathcal{B}^\prime} }()
		\end{aligned}
		: \textsf{Verify}(\textsf{BioCompare}(\mathbf{b}, \mathbf{b}^\prime)) = 1 \right ]. \\
	=\; & \textsf{FP}.
\end{align*}

\noindent If $\textsf{psk}$ and $\mathcal{O}_{\mathcal{B}}$ are given at the same time, the adversary can even win with a probability \textsf{TP} by returning a probe from the distribution $\mathcal{B}$. To prevent such trivial attacks, we add the following requirements:

\begin{itemize}
	\item \textsf{option} only includes $\textsf{psk}$ when \textsf{FP} is negligible.

	\item \label{item:requirement} The adversary is not allowed to return what $\mathcal{O}_\textsf{Probe}$ returns.
	
	\item \textsf{option} cannot include both $\textsf{psk}$ and $\mathcal{O}_{\mathcal{B}}$.
\end{itemize}


\begin{figure}[h]
\centering
	\begin{minipage}[t]{0.6\linewidth}
	\centering
	\begin{algorithm}[H]
	\caption{$\mathcal{A}(\textsf{psk})$ ( or $\mathcal{A}^{\mathcal{O}_\textsf{Probe}}$ ) }
	\label{alg:adv:FP}
	\begin{algorithmic}[1]
		\State $\mathcal{B}^\prime \getsdollar \mathbb{B}$
		
		\State $\mathbf{b}^\prime \gets \textsf{getProbe}^{\mathcal{O}_{\mathcal{B}^\prime }}()$

		\State $\mathbf{c_y} \gets \textsf{Probe}(\textsf{psk}, \mathbf{b}^\prime)$ \Comment{or $\mathbf{c_y} \gets \mathcal{O}_\textsf{Probe}(\mathbf{b}^\prime)$ }

		\State \Return $\mathbf{c_y}$
	\end{algorithmic}
	\end{algorithm}
	\end{minipage}
	
\end{figure}


\paragraph{UF Security with Digital Signature}

We note that we can achieve UF security by a similar approach in \cite{cryptoeprint:2023/481} with a digital signature scheme. Given any authentication scheme $\Pi$ and an sEUF-CMA secure digital signature scheme $\textsf{Sig} = (\textsf{Sig.KeyGen}, \textsf{Sig.Sign}, \textsf{Sig.Verify})$, consider the following scheme $\Pi^\prime$.

\begin{itemize}

	\item $\textsf{Setup}^\prime (1^\lambda)$: Run $(\textsf{esk}, \textsf{psk}, \textsf{csk}) \gets \textsf{Setup}(1^\lambda)$ and $(\textsf{sk}_{\textsf{Sig}}, \textsf{pk}_{\textsf{Sig}}) \gets \textsf{Sig.KeyGen}(1^\lambda)$. Output $\textsf{esk}^\prime \gets \textsf{esk}$, $\textsf{psk}^\prime \gets (\textsf{psk}, \textsf{sk}_{\textsf{Sig}})$, $\textsf{csk}^\prime \gets \textsf{csk}$.

	\item $\textsf{Enroll}^\prime$: The same as $\textsf{Enroll}$.

	\item $\textsf{Probe}^\prime (\textsf{psk}^\prime, \mathbf{b}^\prime)$: Run $\mathbf{c_y} \gets \textsf{Probe}(\textsf{psk}, \mathbf{b}^\prime)$ and $\sigma \gets \textsf{Sig.Sign}(\textsf{sk}_{\textsf{Sig}}, \mathbf{c_y})$. Output $\mathbf{c_y}^\prime \gets (\mathbf{c_y}, \sigma)$.

	\item $\textsf{Compare}^\prime (\textsf{csk}, \mathbf{c_x}, \mathbf{c_y}^\prime)$: If $\textsf{Sig.Verify}(\textsf{pk}_{\textsf{Sig}}, \mathbf{c_y}, \sigma) = 1$, output $\textsf{Compare}(\textsf{csk}, \mathbf{c_x}, \mathbf{c_y})$; otherwise, output $\bot$.

\end{itemize}

An $\textsf{UF}_\textsf{option}$ adversary has to forge a signature $\sigma$ to win the game, so the scheme is $\textsf{option}$-UF secure for any $\textsf{option}$ that does not include $\textsf{psk}$. 

\begin{theorem}
\label{thm:sEUF-CMA-esk-csk}
	Let $\textsf{option} = \{ \textsf{esk}, \textsf{csk}, \mathbf{c_x}, \mathcal{O}_\mathcal{B}, \mathcal{O}_{\textsf{Probe}} \}$. For any authentication scheme $\Pi$, $\Pi^\prime$ is $\textsf{option}$-UF secure. 
\end{theorem}

Note that Theorem \ref{thm:sEUF-CMA-esk-csk} also holds for the following trivial authentication scheme for any biometric layer.
\begin{itemize}
	\item $\textsf{Setup} (1^\lambda)$: $\textsf{esk} = \textsf{psk} = \textsf{csk}$ are all empty strings.
	\item $\textsf{Enroll} (\textsf{esk}, \mathbf{b}) \to \mathbf{b}$.
	\item $\textsf{Probe} (\textsf{psk}, \mathbf{b}^\prime) \to \mathbf{b}^\prime$.
	\item $\textsf{Compare} (\textsf{csk}, \mathbf{c}, \mathbf{c}^\prime) = \textsf{BioCompare}(\mathbf{c}, \mathbf{c}^\prime)$.
\end{itemize}


%-------------------


\subsection{Indistinguishability}
\label{sec:ind_game}

In the game of indistinguishability, we model the ability of an authentication server who tries to identify the user, which describes the privacy leakage of the scheme. The adversary $\mathcal{A}$ is given oracles to two biometric distributions $\mathcal{B}^{(0)}$ and $ \mathcal{B}^{(1)}$ and \textsf{option} that depends on our threat model. It tries to guess from either $\mathcal{B}^{(0)}$ or $ \mathcal{B}^{(1)}$ the enrollment or probe messages are generated. The whole game $\textsf{IND}_{\Pi, \mathbb{B}, \textsf{option}}$ is defined in Algorithm \ref{alg:ind_game}.

\begin{figure}[h]
\centering

	\begin{minipage}[t]{0.55\textwidth}
	\begin{algorithm}[H]
	\caption{$\textsf{IND}_{\Pi, \mathbb{B}, \textsf{option}}(\mathcal{A})$}
	\label{alg:ind_game}
	\begin{algorithmic}[1]
		\State $b \getsdollar \{0, 1\}$

		\State $\mathcal{B}^{(0)} \getsdollar \mathbb{B}, \quad \mathbb{B} \gets \mathbb{B} \setminus \mathcal{B}^{(0)}$

		\State $\mathcal{B}^{(1)} \getsdollar \mathbb{B}, \quad \mathbb{B} \gets \mathbb{B} \setminus \mathcal{B}^{(1)}$

		\State $\textsf{esk}, \textsf{psk}, \textsf{csk} \gets \textsf{Setup}(1^\lambda)$

		\State $\mathbf{b} \gets \textsf{getEnroll}^{\mathcal{O}_{\mathcal{B}^{(b)}}}()$

		\State $\mathbf{c_x} \gets \textsf{Enroll}(\textsf{esk}, \mathbf{b})$

		%\For{$i = 1$ to $t$}

			%\State ${\mathbf{b}^\prime}^{(i)} \gets \textsf{getProbe}^{\mathcal{O}_{\mathcal{B}^{(b)}}}() $
		
			%\State $\mathbf{c_y}^{(i)} \gets \textsf{Probe}( \textsf{psk}, {\mathbf{b}^\prime}^{(i)} )$

		%\EndFor

		%\State In Device-of-User Model:
		
			%\State \hspace{\algorithmicindent} $\tilde{b} \gets \mathcal{A}^{\mathcal{O}_{\mathcal{B}^{(0)}}, \mathcal{O}_{\mathcal{B}^{(1)}} } ( \textsf{csk}, \mathbf{c_x}, \{ \mathbf{c_y}^{(i)} \}_{i=1}^t )$

		%\State In Device-of-Domain Model:
		
			\State $\tilde{b} \gets \mathcal{A}^{\mathcal{O}_{\mathcal{B}^{(0)}}, \mathcal{O}_{\mathcal{B}^{(1)}}} ( \textsf{option} )$

		\State \Return $1_{\tilde{b} = b}$
	\end{algorithmic}
	\end{algorithm}
	\end{minipage}

%\caption{The \textsf{IND} Game}
\label{fig:ind_game}
\end{figure}

The auxiliary information \textsf{option} can be nothing or include $\textsf{esk}, \textsf{psk}, \textsf{csk}, \mathbf{c_x} $ or the following oracles:

\begin{itemize}

	\item $\mathcal{O}_{\mathbf{c_y}}$: It first samples a biometric sample $\mathbf{b}^{\prime} \getsdollar \textsf{getProbe}^{\mathcal{B}^{(b)} }()$ and outputs $\mathbf{c_y} \gets \textsf{Probe}(\textsf{psk}, \mathbf{b}^{\prime})$.

	\item $\mathcal{O}_\textsf{Enroll}(\textsf{esk}, \cdot)$: On input $\mathbf{b}^\prime$, it outputs the enrollment message $\textsf{Enroll}(\textsf{esk}, \mathbf{b}^\prime)$.

	\item $\mathcal{O}_\textsf{Probe}(\textsf{psk}, \cdot)$: On input $\mathbf{b}^\prime$, it outputs the probe message $\textsf{Probe}(\textsf{psk}, \mathbf{b}^\prime)$.
	
	\item $\mathcal{O}_\textsf{CompVrfy}(\textsf{csk}, \cdot, \cdot)$: On input $\mathbf{c}$ and $\mathbf{c}^\prime$, it first computes $s \gets \textsf{Compare}(\textsf{csk}, \mathbf{c}, \mathbf{c}^\prime)$ and outputs $\textsf{Verify}(s)$.
\end{itemize}

\noindent Note that at least one of $\mathbf{c_x}$ and $\mathcal{O}_{\mathbf{c_y}}$ should be given; otherwise, IND security is trivial.

We define the advantage of an adversary $\mathcal{A}$ in the $\textsf{IND}_{\Pi, \mathbb{B}, \textsf{option}}$ game of a scheme $\Pi$ associated with a family of distributions $\mathbb{B}$ as
\[
	\Adv^{\textsf{IND}}_{\Pi, \mathbb{B}, \mathcal{A}, \textsf{option}} := \left |\Pr[\textsf{IND}_{\Pi, \mathbb{B}, \textsf{option}}(\mathcal{A}) \to 1] - \frac{1}{2} \right|.
\]

An authentication scheme $\Pi$ associated with a family $\mathbb{B}$ of distributions is called \emph{\textsf{option}-indistinguishable} (\textsf{option}-IND) if for any PPT adversary $\mathcal{A}$,
\[
	\Adv^{\textsf{IND}}_{\Pi, \mathbb{B}, \mathcal{A}, \textsf{option}} = \negl.
\]


\paragraph{Trivial Attacks}
We note that if $\textsf{TP} - \textsf{FP} > \frac{1}{\poly}$, there are trivial attacks.

\begin{theorem}
\label{thm:ind-tp-fp}
Given any distribution family $\mathbb{B}$ that $\textsf{TP} - \textsf{FP} > \frac{1}{\poly}$, $\Pi$ is not $\textsf{option}$-IND secure for
\begin{itemize}
	\item $\textsf{option} = \{\mathbf{c_x}, \text{ either } \textsf{csk} \text{ or } \mathcal{O}_{\textsf{CompVrfy}}, \text{ either } \textsf{psk} \text{ or } \mathcal{O}_{\textsf{Probe}} \}$

	\item $\textsf{option} = \{\mathcal{O}_{\mathbf{c_y}}, \text{ either } \textsf{csk} \text{ or } \mathcal{O}_{\textsf{CompVrfy}}, \text{ either } \textsf{esk} \text{ or } \mathcal{O}_{\textsf{Enroll}} \}$
\end{itemize}

\end{theorem}

\begin{proof}

Let $\textsf{option}$ be the first case that includes $\mathbf{c_x}$. The second case when $\mathcal{O}_{\mathbf{c_y}}$ is given can be analyzed analogously. Consider the adversary $\mathcal{A}$ in the $\textsf{IND}_{\textsf{option}}$ game in Algorithm \ref{alg:ind-tp-fp}. When the challenge bit $b = 0$, the probability that $\mathcal{A}$ wins is $\textsf{TP}$. When the challenge bit $b = 1$, the probability that $\mathcal{A}$ wins is $1 - \textsf{FP}$. Now, 

\[
	\Adv_{\Pi, \mathbb{B}, \mathcal{A}, \textsf{option}}^{\textsf{IND}} = \left| \Pr[ \textsf{IND}_\Pi(\mathcal{A}) \to 1 ] - \frac{1}{2} \right| = \left| \frac{1}{2} (\textsf{TP} + 1 - \textsf{FP}) - \frac{1}{2} \right| > \frac{1}{\poly}.
\]

\begin{figure}[h]
\centering

	\begin{minipage}[t]{0.85\textwidth}
	\begin{algorithm}[H]
	\caption{$\mathcal{A}^{\mathcal{O}_{\mathcal{B}^{(0)}}, \mathcal{O}_{\mathcal{B}^{(1)}}} (\mathsf{psk}, \mathsf{csk}, \mathbf{c_x})$ or $\mathcal{A}^{\mathcal{O}_{\mathcal{B}^{(0)}}, \mathcal{O}_{\mathcal{B}^{(1)}}, \mathcal{O}_{\mathsf{Probe}}, \mathcal{O}_{\mathsf{CompVrfy}} } (\mathbf{c_x})$}
	\label{alg:ind-tp-fp}
	\begin{algorithmic}[1]

		\State $\mathbf{b}^{(0)} \gets \mathsf{getProbe}^{\mathcal{O}_{\mathcal{B}^{(0)}}}()$
		
		\State $\mathbf{c_y}^{(0)} \gets \mathsf{Probe}(\mathsf{psk}, \mathbf{b}^{(0)})$ \Comment{or $\mathbf{c_y}^{(0)} \gets \mathcal{O}_{\mathsf{Probe}}(\mathbf{b}^{(0)})$ }
		
		\State $r \gets \textsf{Verify}( \textsf{Compare}(\textsf{csk}, \mathbf{c_x}, \mathbf{c_y}^{(0)}) ) $ \Comment{or $r \gets \mathcal{O}_{\textsf{CompVrfy}}( \mathbf{c_x}, \mathbf{c_y}^{(0)}) $}
		
		\State \Return $1 - r$
	\end{algorithmic}
	\end{algorithm}
	\end{minipage}

\end{figure}

\end{proof}


\paragraph{Necessity of IND Security}

Recall the trivial authentication scheme for any biometric layer we introduced in Section \ref{sec:uf_game}. By Theorem \ref{thm:sEUF-CMA-esk-csk}, we can obtain an authentication scheme $\Pi^\prime$ that is $\mathsf{option}$-UF secure for any $\mathsf{option}$ that does not include $\textsf{psk}$.
However, the enrollment and probe messages leak biometric vectors $\mathbf{b}$ and $\mathbf{b}^\prime$ and compromise privacy. Obviously, this scheme is not \textsf{option}-IND secure for an \textsf{option} that includes either $\mathbf{c_x}$ or $\mathcal{O}_{\mathbf{c_y}}$ when $\mathsf{TP} - \mathsf{FP} > \frac{1}{\poly}$.

\paragraph{IND Security for a Particular Biometric Layer}

Let $\mathsf{getEnroll}^{\mathcal{O}_{\mathcal{B}}}(), \textsf{getProbe}^{\mathcal{O}_{\mathcal{B}}}()$ be such that
\[
	\textsf{getEnroll}^{\mathcal{O}_{\mathcal{B}}}() \to \mathbf{b}^*_{\mathcal{B}} \oplus \mathcal{E}_{\textsf{Enroll}}  \quad \text{and} \quad \textsf{getProbe}^{\mathcal{O}_{\mathcal{B}}}() \to \mathbf{b}^*_{\mathcal{B}} \oplus \mathcal{E}_{\textsf{Probe}}
\]
where $\mathbf{b}^*_{\mathcal{B}} \in \{0, 1\}^k$ is a fixed vector only dependent on $\mathcal{B}$, and $\mathcal{E}_{\mathsf{Enroll}}, \mathcal{E}_{\mathsf{Probe}} \subseteq \{0, 1\}^k$ are some \emph{error distributions} independent of $\mathcal{B}$. Let $\mathsf{BioCompare}(\mathbf{b}, \mathbf{b}^\prime) \to 1_{\mathsf{HD}(\mathbf{b}, \mathbf{b}^\prime) \leq \tau}$. Then
\[
	\textsf{TP} = \Pr[ \textsf{HW}(\mathbf{b}^*_{\mathcal{B}} \oplus \mathcal{E}_{\textsf{Enroll}} \oplus \mathbf{b}^*_{\mathcal{B}} \oplus \mathcal{E}_{\textsf{Probe}}) \leq \tau ] = \Pr[ \textsf{HW}(\mathcal{E}_{\textsf{Enroll}} \oplus \mathcal{E}_{\textsf{Probe}}) \leq \tau ]
\]

\noindent We note that previous works such as \cite{10.1145/1030083.1030096,cryptoeprint:2014/394} model biometric template vectors in a similar way.

Now, for this biometric layer, we can construct a simple but IND secure authentication scheme $\Pi$ with the following cryptographic layer.

\begin{itemize}

	\item $\mathsf{Setup} (1^\lambda)$: Sample $\mathbf{r} \getsdollar \{0, 1\}^k$. Output $\mathsf{esk} = \mathsf{psk} \gets \mathbf{r}$, $\mathsf{csk} \gets \epsilon$.

	\item $\mathsf{Enroll}(\mathsf{esk}, \mathbf{b})$: Output $\mathbf{b} \oplus \mathbf{r}$.
	
	\item $\mathsf{Probe}(\mathsf{psk}, \mathbf{b}^\prime)$: Output $\mathbf{b}^\prime \oplus \mathbf{r}$.

	\item $\mathsf{Compare} (\mathsf{csk}, \mathbf{c_x}, \mathbf{c_y})$: If $\mathsf{HD}(\mathbf{c_x}, \mathbf{c_y}) \leq \tau$, return $1$; otherwise, return $0$. 

\end{itemize}
The correctness of $\Pi$ holds by
\[
	\textsf{HD}(\mathbf{c_x}, \mathbf{c_y}) = \textsf{HW}(\mathbf{b} \oplus \mathbf{r} \oplus \mathbf{b}^\prime \oplus \mathbf{r}) = \textsf{HW}(\mathbf{b} \oplus \mathbf{b}^\prime) = \textsf{BioCompare}(\mathbf{b}, \mathbf{b}^\prime).
\]

\begin{theorem}
\label{thm:rh:ind:particular-biometri-layer}
Let $\mathsf{option} = \{\mathsf{csk}, \mathbf{c_x}, \mathcal{O}_{\mathbf{c_y}}\}$. The authentication scheme $\Pi$ is \textsf{option}-IND secure.

\end{theorem}

\begin{proof}

Let $\mathbf{b}^*_0$ and $\mathbf{b}^*_1$ be the fixed vectors of $\mathcal{B}^{(0)}$ and $\mathcal{B}^{(1)}$ in the \textsf{IND} game, respectively. Given any adversary, assume that the number of its queries to $\mathcal{O}_{\mathbf{c_y}}$ is bounded by $t$. For any $\mathbf{v}, \mathbf{v}^{(1)}, \cdots, \mathbf{v}^{(t)}$,
\begin{align*}
	& \Pr[ \mathbf{c_x} = \mathbf{v},\; \mathbf{c_y}^{(1)} = \mathbf{v}^{(1)},\; \cdots,\; \mathbf{c_y}^{(t)} = \mathbf{v}^{(t)} \mid b = 0,\; \mathbf{b}^*_0,\; \mathbf{b}^*_1 ] \\
	&= \Pr[ \mathbf{b}^*_0 \oplus \mathcal{E}_{\textsf{Enroll}} \oplus \mathbf{r} = \mathbf{v},\; \mathbf{b}^*_0 \oplus \mathcal{E}_{\textsf{Probe}} \oplus \mathbf{r} =\mathbf{v}^{(1)},\; \cdots,\; \mathbf{b}^*_0 \oplus \mathcal{E}_{\textsf{Probe}} \oplus \mathbf{r} =\mathbf{v}^{(t)} \mid \mathbf{b}^*_0,\; \mathbf{b}^*_1] \\ 
	&= \Pr[ \mathbf{r} = \mathbf{v} \oplus \mathbf{b}^*_0 \oplus \mathcal{E}_{\textsf{Enroll}} =\mathbf{v}^{(1)} \oplus \mathbf{b}^*_0 \oplus \mathcal{E}_{\textsf{Probe}} = \cdots = \mathbf{v}^{(t)} \oplus \mathbf{b}^*_0 \oplus \mathcal{E}_{\textsf{Probe}} \mid \mathbf{b}^*_0,\; \mathbf{b}^*_1] \\ 
	&= \Pr[ \mathbf{r} = \mathbf{v} \oplus \mathbf{b}^*_1 \oplus \mathcal{E}_{\textsf{Enroll}} = \mathbf{v}^{(1)} \oplus \mathbf{b}^*_1 \oplus \mathcal{E}_{\textsf{Probe}} = \cdots = \mathbf{v}^{(t)} \oplus \mathbf{b}^*_1 \oplus \mathcal{E}_{\textsf{Probe}} \mid \mathbf{b}^*_0,\; \mathbf{b}^*_1] \\ 
	&= \Pr[ \mathbf{b}^*_1 \oplus \mathcal{E}_{\textsf{Enroll}} \oplus \mathbf{r} = \mathbf{v},\; \mathbf{b}^*_1 \oplus \mathcal{E}_{\textsf{Probe}} \oplus \mathbf{r} =\mathbf{v}^{(1)},\; \cdots,\; \mathbf{b}^*_1 \oplus \mathcal{E}_{\textsf{Probe}} \oplus \mathbf{r} =\mathbf{v}^{(t)} \mid \mathbf{b}^*_0,\; \mathbf{b}^*_1] \\ 
	&= \Pr[ \mathbf{c_x} = \mathbf{v},\; \mathbf{c_y}^{(1)} = \mathbf{v}^{(1)},\; \cdots,\; \mathbf{c_y}^{(t)} = \mathbf{v}^{(t)} \mid b = 1,\; \mathbf{b}^*_0,\; \mathbf{b}^*_1  ]
\end{align*}
Hence, the adversary cannot distinguish between $\mathbf{c_x}, \mathbf{c_y}^{(1)}, \cdots, \mathbf{c_y}^{(t)}$ generated from $\mathcal{B}^{(0)}$ and $\mathcal{B}^{(1)}$.

\end{proof}

\paragraph{IND Security with Public-Key Encryption}

We note that we can achieve IND security with a public-key encryption scheme. We first recall the \emph{multi-challenge IND-CPA} security for a public-key encryption scheme.

\begin{definition}[Multi-challenge IND-CPA]

A public-key encryption scheme $\mathsf{PKE} = (\mathsf{PKE.KeyGen}, \mathsf{PKE.Enc}, \mathsf{PKE.Dec})$ is multi-challenge IND-CPA secure if for any adversary $\mathcal{A}$ , the advantage of $\mathcal{A}$ in the game $\mathsf{MC-IND-CPA}_{\mathsf{PKE}}$ is

\[
	\Adv^{\mathsf{MC-IND-CPA}}_{\mathsf{PKE}, \mathcal{A}} := \left| \Pr[ \mathsf{MC-IND-CPA}_{\mathsf{PKE}}(\mathcal{A}) \to 1 ] - \frac{1}{2} \right| = \negl.
\]

\begin{figure}[h]
\centering

	\begin{minipage}[t]{0.6\textwidth}
	\begin{algorithm}[H]
	\caption{$\textsf{MC-IND-CPA}_{\mathsf{PKE}} (\mathcal{A})$ }
	\label{alg:MC-IND-CPA}
	\begin{algorithmic}[1]
		\State $b \getsdollar \{0, 1\}$

		\State $(\mathsf{sk}_{\mathsf{PKE}}, \mathsf{pk}_{\mathsf{PKE}}) \gets \mathsf{PKE.KeyGen}(1^\lambda)$
		
		\State $\tilde{b} \gets \mathcal{A}^{\mathcal{O}_{\textsf{MC-Enc}}}(\mathsf{pk}_{\mathsf{PKE}})$ 
		
		\State \Return $1_{\tilde{b} = b}$
	\end{algorithmic}
	\end{algorithm}
	\end{minipage}

\end{figure}
\end{definition}

\begin{itemize}
	\item $\mathcal{O}_{\textsf{MC-Enc}}(\cdot, \cdot)$: On input $(\mathsf{m}_0, \mathsf{m}_1)$, it outputs $\mathsf{PKE.Enc}(\mathsf{pk}_{\mathsf{PKE}}, \mathsf{m}_b)$
\end{itemize}

Given any authentication scheme $\Pi$ and a multi-challenge IND-CPA secure public-key encryption scheme $\mathsf{PKE} = (\mathsf{PKE.KeyGen}, \allowbreak \mathsf{PKE.Enc}, \mathsf{PKE.Dec})$, consider the following scheme $\Pi^\prime$. 

\begin{itemize}

	\item $\mathsf{Setup}^\prime (1^\lambda)$: Run $(\mathsf{esk}, \mathsf{psk}, \mathsf{csk}) \gets \mathsf{Setup}(1^\lambda)$ and $(\mathsf{sk}_{\mathsf{PKE}}, \mathsf{pk}_{\mathsf{PKE}}) \gets \mathsf{PKE.KeyGen}(1^\lambda)$. Output $\mathsf{esk}^\prime \gets ( \mathsf{esk}, \mathsf{pk}_{\mathsf{PKE}} ) $, $\mathsf{psk}^\prime \gets (\mathsf{psk}, \mathsf{pk}_{\mathsf{PKE}})$, $\mathsf{csk}^\prime \gets (\mathsf{csk}, \mathsf{sk}_{\mathsf{PKE}} )$.

	\item $\mathsf{Enroll}^\prime (\mathsf{esk}^\prime, \mathbf{b})$: Run $\mathbf{c_x} \gets \mathsf{Enroll}(\mathsf{esk}, \mathbf{b})$ and $\mathsf{ct}_{\mathbf{x}} \gets \mathsf{PKE.Enc}(\mathsf{pk}_{\mathsf{PKE}}, \mathbf{c_x})$. Output $\mathbf{c_x}^\prime \gets \mathsf{ct}_{\mathbf{x}}$.

	\item $\mathsf{Probe}^\prime (\textsf{psk}^\prime, \mathbf{b}^\prime)$: Run $\mathbf{c_y} \gets \textsf{Probe}(\textsf{psk}, \mathbf{b}^\prime)$ and $\textsf{ct}_{\mathbf{y}} \gets \textsf{PKE.Enc}(\textsf{pk}_{\textsf{PKE}}, \mathbf{c_y})$. Output $\mathbf{c_y}^\prime \gets \textsf{ct}_{\mathbf{y}}$.

	\item $\textsf{Compare}^\prime (\textsf{csk}^\prime, \mathbf{c_x}^\prime, \mathbf{c_y}^\prime)$: First decrypt $\mathbf{c_x} \gets \textsf{PKE.Dec}(\textsf{sk}_{\textsf{PKE}}, \mathbf{c_x}^\prime)$ and  $\mathbf{c_y} \gets \textsf{PKE.Dec}(\textsf{sk}_{\textsf{PKE}}, \mathbf{c_y}^\prime)$. Output $\textsf{Compare}(\textsf{csk}, \mathbf{c_x}, \mathbf{c_y})$.
\end{itemize}

\begin{theorem}
\label{thm:mc-ind-cpa:ind-esk-psk}
	Let $\textsf{option} = \{ \textsf{esk}, \textsf{psk}, \mathbf{c_x}, \mathcal{O}_{\mathbf{c_y}} \}$. For any authentication scheme $\Pi$, $\Pi^\prime$ is $\textsf{option}$-IND secure. 
\end{theorem}

\begin{proof}

Given an adversary $\mathcal{A}$ in the $\textsf{IND}_{\textsf{option}}$ game, consider the reduction adversary $\mathcal{R}$ in Algrothm \ref{red:mc-ind-cpa:ind-esk-psk-ox-oc} which plays the $\textsf{MC-IND-CPA}_{\textsf{PKE}}$ game by running $\mathcal{A}$. $\mathcal{R}$ simulates $\mathcal{O}_{\mathbf{c_y}}$ by the following steps.

\begin{enumerate}
	\item Sample ${\mathbf{b}^{\prime}}^{(0)} \gets \textsf{getProbe}^{\mathcal{O}_{\mathcal{B}^{(0)}}}()$ and ${\mathbf{b}^{\prime}}^{(1)} \gets \textsf{getProbe}^{\mathcal{O}_{\mathcal{B}^{(1)}}}()$.
	\item Run $\mathbf{c_y}^{(0)} \gets \textsf{Probe}(\textsf{psk}, {\mathbf{b}^{\prime}}^{(0)} )$ and $\mathbf{c_y}^{(1)} \gets \textsf{Probe}(\textsf{psk}, {\mathbf{b}^{\prime}}^{(1)} )$
	\item Output $\mathcal{O}_{\textsf{MC-Enc}}(\mathbf{c_y}^{(0)}, \mathbf{c_y}^{(1)})$.
\end{enumerate}

\begin{figure}[h]
\centering
	
	\begin{minipage}[t]{0.7\linewidth}
	\centering
	\begin{algorithm}[H]
	\caption{$\mathcal{R}^{\mathcal{O}_{\textsf{MC-Enc}}}(\textsf{pk}_{\textsf{PKE}})$}
	\label{red:mc-ind-cpa:ind-esk-psk-ox-oc}
	\begin{algorithmic}[1]
		\State $\mathcal{B}^{(0)} \getsdollar \mathbb{B}, \quad \mathbb{B} \gets \mathbb{B} \setminus \mathcal{B}^{(0)}$
		
		\State $\mathcal{B}^{(1)} \getsdollar \mathbb{B}, \quad \mathbb{B} \gets \mathbb{B} \setminus \mathcal{B}^{(1)}$

		\State $\textsf{esk}, \textsf{psk}, \textsf{csk} \gets \textsf{Setup}(1^\lambda)$

		\State $\mathbf{b}^{(0)} \gets \textsf{getProbe}^{\mathcal{O}_{\mathcal{B}^{(0)}}}(), \mathbf{c_x}^{(0)} \gets \textsf{Enroll}(\textsf{esk}, \mathbf{b}^{(0)})$
		
		\State $\mathbf{b}^{(1)} \gets \textsf{getProbe}^{\mathcal{O}_{\mathcal{B}^{(1)}}}(), \mathbf{c_x}^{(1)} \gets \textsf{Enroll}(\textsf{esk}, \mathbf{b}^{(1)})$
		
		\State $\mathbf{c_x}^\prime \gets \mathcal{O}_{\textsf{MC-Enc}}(\mathbf{c_x}^{(0)}, \mathbf{c_x}^{(1)})$

		\State $\textsf{esk}^\prime \gets (\textsf{esk}, \textsf{pk}_{\textsf{PKE}}), \textsf{psk}^\prime \gets (\textsf{psk}, \textsf{pk}_{\textsf{PKE}})$

		\State $ \tilde{b} \gets {\mathcal{A}}^{\mathcal{O}_{\mathcal{B}^{(0)}}, \mathcal{O}_{\mathcal{B}^{(1)}}, \mathcal{O}_{\mathbf{c_y}} } (\textsf{esk}^\prime, \textsf{psk}^\prime, \mathbf{c_x}^\prime)$ \label{alg:red:ind-cpa:ind-esk-psk-ox-oc}

		\State \Return $\tilde{b}$

	\end{algorithmic}
	\end{algorithm}
	\end{minipage}
	
\end{figure}

\noindent Since $\mathcal{R}$ simulates an $\textsf{IND}_{\textsf{option}}$ game for $\mathcal{A}$, the advantage of $\mathcal{A}$ is
\[
	\Adv^{\textsf{IND}}_{\Pi, \mathbb{B}, \mathcal{A}, \textsf{option}} = \Adv^{\textsf{MC-IND-CPA}}_{\textsf{PKE}, \mathcal{R}} = \negl.
\]

\end{proof}
