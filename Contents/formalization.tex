%%%%%%%%%%%%%%%%%%%%

% Project Name: Semester Project Fall 2024 for EPFL
% File: formalization.tex
% Author: Keng-Yu Chen

%%%%%%%%%%%%%%%%%%%%


\subsection{Biometric Authentication Scheme}

In this section, we formally define a biometric authentication scheme. For this, we first define how we simulate biometric distributions of users.

Assume the existence of a family $\mathbb{B}$ of biometric distributions that are efficiently samplable. We have the following interfaces for all algorithms to interact with $\mathbb{B}$.

\begin{itemize}

	\item $\mathsf{BioSamp}()$: Generate a random distribution $\mathcal{B}$ of $\mathbb{B}$. By this we mean providing either parameters of an efficiently samplable distribution or a PPT algorithm as the sampler. For simplicity, we write $\mathcal{B} \gets \mathsf{BioSamp}()$ as $\mathcal{B} \getsdollar \mathbb{B}$.
	
	\item $\mathsf{BioDelete}(\mathcal{B})$: Delete $\mathcal{B}$ from $\mathbb{B}$. Consequently, no further access to $\mathsf{BioSamp}$ can derive $\mathcal{B}$. For simplicity, we write $\mathsf{BioDelete}(\mathcal{B})$ as $\mathbb{B} \gets \mathbb{B} \setminus \mathcal{B}$.

	\item $\mathsf{TempSamp}(\mathcal{B})$: Let $\mathcal{B}$ be a biometric distribution in $\mathbb{B}$. This algorithm samples a biometric template from $\mathcal{B}$. For simplicity, we write $\mathbf{b} \gets \mathsf{TempSamp}(\mathcal{B})$ as $\mathbf{b} \getsdollar \mathcal{B}$.

\end{itemize}

\begin{definition}[Biometric Authentication Scheme]

A \emph{biometric authentication scheme} $\Pi$ associated with a family $\mathbb{B}$ of biometric distributions is composed of the following algorithms.

\begin{itemize}

	\item $\mathsf{getEnroll}^{\mathcal{O}_{\mathcal{B}}}() \to \mathbf{b}$: Given an oracle $\mathcal{O}_{\mathcal{B}}$, which samples biometric data from a distribution $\mathcal{B} \in \mathbb{B}$, it outputs a biometric template $\mathbf{b}$ for enrollment. In practice, $\mathsf{getEnroll}$ can collect several biometric samples from a user's biometric distribution $\mathcal{B}$ to create a more accurate template.

	\item $\mathsf{getProbe}^{\mathcal{O}_{\mathcal{B}}}() \to \mathbf{b}^\prime$: Given an oracle $\mathcal{O}_{\mathcal{B}}$, which samples biometric data from a distribution $\mathcal{B} \in \mathbb{B}$, it outputs a biometric template $\mathbf{b}^\prime$ for probe. In practice, $\mathsf{getProbe}$ often directly outputs the answer from $\mathcal{O}_{\mathcal{B}}$.

	\item $\mathsf{BioCompare}(\mathbf{b}, \mathbf{b}^\prime) \to s$: Given a biometric template $\mathbf{b}$ from $\mathsf{getEnroll}$ and another template $\mathbf{b}^\prime$ from $\mathsf{getProbe}$, it outputs a score $s$.

	\item $\mathsf{Verify}(s) \to r \in \{0,1\}$: It is a deterministic algorithm that reads the comparison score $s$ and determines whether this is a successful authentication ($r = 1$) or not ($r = 0$).

\end{itemize}
We also call these algorithms the \emph{biometric layer} of $\Pi$. We will add a \emph{cryptographic layer} on top of it in Section \ref{sec:formalization:cryptographic_layer}

\end{definition}


Given an authentication scheme $\Pi$, we can consider its true positive rate and false positive rate.

\begin{definition}[True Positive Rate]
For a biometric distribution $\mathcal{B} \in \mathbb{B}$ and $\mathbf{b} \gets \mathsf{getEnroll}^{\mathcal{O}_\mathcal{B}}()$, define the \emph{true positive rate} \textsf{TP}.

\begin{alignat*}{2}
	\mathsf{TP}(\mathcal{B}, \mathbf{b}) 
	:=&\; \Pr[ \mathbf{b}^\prime \gets \mathsf{getProbe}^{\mathcal{O}_\mathcal{B}}()
	&&: \mathsf{Verify}(\mathsf{BioCompare}(\mathbf{b}, \mathbf{b}^\prime )) = 1 ] \\
	\\
	\mathsf{TP}(\mathcal{B}) 
	:=&\; \Pr \Bigg[ 
		\begin{aligned}	
			& \mathbf{b} \gets \mathsf{getEnroll}^{\mathcal{O}_\mathcal{B}}() \\
			& \mathbf{b}^\prime \gets \mathsf{getProbe}^{\mathcal{O}_\mathcal{B}}()
		\end{aligned}
		&&: \mathsf{Verify}(\mathsf{BioCompare}(\mathbf{b}, \mathbf{b}^\prime)) = 1 \Bigg] \\
	=&\; \mathbb{E}_{ \mathbf{b} \gets \mathsf{getEnroll}^{\mathcal{O}_\mathcal{B}}() }[\mathsf{TP}(\mathcal{B}, \mathbf{b})] \\
	\\
	\mathsf{TP} 
	:=&\; \Pr \vast[
		\begin{aligned}
			& \mathcal{B} \getsdollar \mathbb{B} \\
			& \mathbf{b} \gets \mathsf{getEnroll}^{\mathcal{O}_\mathcal{B} }() \\
			& \mathbf{b}^\prime \gets \mathsf{getProbe}^{\mathcal{O}_\mathcal{B} }()
		\end{aligned}
		&&: \mathsf{Verify}(\mathsf{BioCompare}(\mathbf{b}, \mathbf{b}^\prime )) = 1 \vast] \\
	=&\; \mathbb{E}_{\mathcal{B} \getsdollar \mathbb{B}}[\mathsf{TP}(\mathcal{B})] 
\end{alignat*}

\end{definition}


\begin{definition}[False Positive Rate]
For a biometric distribution $\mathcal{B} \in \mathbb{B}, \mathbb{B} \gets \mathbb{B} \setminus \mathcal{B}$ and $\mathbf{b} \gets \mathsf{getEnroll}^{\mathcal{O}_\mathcal{B}}()$, define the \emph{false positive rate} \textsf{FP}.

\begin{alignat*}{2}
	\mathsf{FP}(\mathbf{b}) 
	:=&\; \Pr \Bigg[ 
		\begin{aligned}
			& \mathcal{B}^\prime \getsdollar \mathbb{B} \\
			& \mathbf{b}^\prime \gets \mathsf{getProbe}^{\mathcal{O}_{\mathcal{B}^\prime}}()
		\end{aligned}
		&&: \mathsf{Verify}(\mathsf{BioCompare}(\mathbf{b}, \mathbf{b}^\prime)) = 1 \Bigg] \\
	\\
	\mathsf{FP}(\mathcal{B}) 
	:=&\; \Pr \vast[ 
		\begin{aligned}
			& \mathcal{B}^\prime \getsdollar \mathbb{B} \\
			& \mathbf{b} \gets \mathsf{getEnroll}^{\mathcal{O}_\mathcal{B} }() \\
			& \mathbf{b}^\prime \gets \mathsf{getProbe}^{\mathcal{O}_{\mathcal{B}^\prime} }()
		\end{aligned}
		&&: \mathsf{Verify}(\mathsf{BioCompare}(\mathbf{b}, \mathbf{b}^\prime)) = 1 \vast] \\
	=&\; \mathbb{E}_{ \mathbf{b} \gets \mathsf{getEnroll}^{\mathcal{O}_\mathcal{B}}() }[\mathsf{FP}(\mathbf{b})] \\
	\\
	\mathsf{FP} 
	:=&\; \Pr \vast[
		\begin{aligned}
			& \mathcal{B} \getsdollar \mathbb{B}, \mathbb{B} \gets \mathbb{B} \setminus \mathcal{B}, \mathcal{B}^\prime \getsdollar \mathbb{B} \\
			& \mathbf{b} \gets \mathsf{getEnroll}^{\mathcal{O}_\mathcal{B} }() \\
			& \mathbf{b}^\prime \gets \mathsf{getProbe}^{\mathcal{O}_{\mathcal{B}^\prime} }()
		\end{aligned}
		&&: \mathsf{Verify}(\mathsf{BioCompare}(\mathbf{b}, \mathbf{b}^\prime)) = 1 \vast] \\
	=&\; \mathbb{E}_{\mathcal{B} \getsdollar \mathbb{B}}[\mathsf{FP}(\mathcal{B})]
\end{alignat*}

\end{definition}

Ideally, we hope $\mathsf{TP}$ to be $1$ and $\mathsf{FP}$ to be negligible. However, due to the inherent nature of biometrics, there might be a non-zero false negative rate $ 1 - \mathsf{TP} > 0$ and a $\mathsf{FP}$ that is not negligible. Our security model and analysis also take these possibilities into consideration.


%-------------------


\subsection{Cryptographic Layer}
\label{sec:formalization:cryptographic_layer}
In this work, we add a cryptographic layer on top of a biometric authentication scheme to protect privacy of users.

\begin{definition}[Cryptographic Layer]

The \emph{cryptographic layer} of a biometric authentication scheme associated with a family $\mathbb{B}$ of biometric distributions is composed of the following algorithms.

\begin{itemize}

	\item $\mathsf{Setup}(1^\lambda) \to \mathsf{esk}, \mathsf{psk}, \mathsf{csk}$: It outputs the enrollment secret key $\mathsf{esk}$, probe secret key $\mathsf{psk}$, and compare secret key $\sf csk$.

	\item $\mathsf{Enroll}(\mathsf{esk}, \mathbf{b}) \to \mathbf{c_x}$: On input a biometric template $\mathbf{b}$, it encodes it into a vector $\mathbf{x}$ and outputs the enrollment message $\mathbf{c_x}$.
	
	\item $\mathsf{Probe}(\mathsf{psk}, \mathbf{b}^\prime) \to \mathbf{c_y}$: On input a biometric template $\mathbf{b}^\prime$, it encodes it into a vector $\mathbf{y}$ and outputs the probe message $\mathbf{c_y}$ .

	\item $\mathsf{Compare}(\mathsf{csk}, \mathbf{c_x}, \mathbf{c_y)} \to s$: It compares the enrollment message $\mathbf{c_x}$ and probe message $\mathbf{c_y}$ and outputs a score $s$.

\end{itemize}

\noindent \textbf{Correctness}: A cryptographic layer is \emph{correct} if for any biometric distributions $\mathcal{B}$ and $\mathcal{B}^\prime$, let $\mathsf{esk}, \mathsf{psk}, \mathsf{csk} \gets \mathsf{Setup}(1^\lambda)$, $\mathbf{b} \gets \mathsf{getEnroll}^{\mathcal{O}_\mathcal{B}}()$, $\mathbf{b}^\prime \gets \mathsf{getProbe}^{\mathcal{O}_{\mathcal{B}^\prime}}()$, $\mathbf{c_x} \gets \mathsf{Enroll}(\mathsf{esk}, \mathbf{b})$, $\mathbf{c_y} \gets \mathsf{Probe}(\mathsf{psk}, \mathbf{b}^\prime)$. Then
	\[
		\Pr \left [
			\mathsf{Compare}(\mathsf{csk}, \mathbf{c_x}, \mathbf{c_y}) = \mathsf{BioCompare}(\mathbf{b}, \mathbf{b}^\prime)
		\right ] = 1.
	\]

\end{definition}

In a real-world biometric system, these algorithms may be run by different parties such as a biometric scanner, a user's secure hardware, a trusted authority that issues keys, and the server.

We provide two instantiations of a biometric authentication scheme with the cryptographic layer in Sections \ref{sec:fh-IPFE-instantiation} and \ref{sec:rh-instantiation}.


%-------------------

\iffalse

We discuss two usage models that employs the authentication scheme $\Pi$.


\subsection{Usage Model – Device-of-User}
\label{sec:dou_model}

In the model described in Figure \ref{fig:model_dou_overview} (an overview), Figure \ref{fig:model_dou_enrollment} (on enrollment), and Figure \ref{fig:model_dou_auth} (on authentication), users authenticate themselves to a server through their own devices and biometric scanners that are shared among different users.
A key distribution service distributes keys for them. In practice, this model applies to the situation when the users access an online service run by the server.

\begin{itemize}

	\item \textsf{User}: The user who enrolls its biometric data and authenticates itself to the server. We assume the user's biometric distribution is $\mathcal{B} \in \mathbb{B}$. 

	\item \textsf{Scanner}: A machine to extract the user's biometric data by querying the oracle $\mathcal{O}_{\mathcal{B}}$.
	
	\item \textsf{Device}: A device belonging to the user. In practice, it can be a desktop or a mobile phone. It processes the  for biometric data through the \textsf{Scanner}.
	
	\item \textsf{KDS}: A key distribution service. It runs $\mathsf{Setup}$ to generate keys and distribute them to \textsf{Device} and \textsf{Server}.
		
	\item \textsf{Server}: The server responsible for authenticating the user. It stores the comparison key $\mathsf{csk}$ and the user's enrollment message $\mathbf{c_x}$. On authentication, it compares the probe message with the registered enrollment message and returns the result.  

\end{itemize}

The Device-of-User model, when instantiated by an fh-IPFE scheme (Section \ref{sec:fh-IPFE-instantiation}), is analogous to the use case presented in \cite{cryptoeprint:2023/481}.
In their model, a user possesses a personal device, such as a smartphone or laptop, and a secure hardware device that runs an initial setup and stores all the keys, which corresponds to our \textsf{KDS}.
On enrollemnt and authentication, the user inputs biometric templates onto the device, which corresponds to our \textsf{Scanner}.
Subsequently, the device transmits the template to the secure hardware for the enrollment or probing processes, which are equivalent to our \textsf{Device}.
In addition, they incorporate a two-factor authentication mechanism.
The secure hardware also executes a digital signature scheme and sign the probe message on authentication.


\input{tikz/dou_model.tex}

%-------------------

\subsection{Usage Model – Device-of-Domain}
\label{sec:dod_model}

In the model described in Figure \ref{fig:model_dod_overview} (an overview), Figure \ref{fig:model_dod_enrollment} (on enrollment), and Figure \ref{fig:model_dod_auth} (on authentication), users first enroll themselves at an enrollment station and then authenticate themselves to a server through devices that belong to a domain.
A key distribution service distributes enrollment keys to the enrollment station, probe keys to the domain, and comparison keys to the server. In practice, a domain can be a department in an organization, and this models applies to the situation when a user wants to access a public service of a department, such as a restricted area or instruments. 

\begin{itemize}

	\item \textsf{User}: The user who enrolls its biometric data at an enrollment station and authenticates itself to the server. We assume the user's biometric distribution is $\mathcal{B} \in \mathbb{B}$.
	
	\item \textsf{Domain}: A domain that owns several devices, all of which share one enrollment key . Only the probe key is stored at each device of a domain. The enrollment key is stored at the enrollment station, and the comparison key is stored at the server. In practice, a domain can be a department, and users enroll and authenticate themselves before accessing a restricted service of this department.

	\item \textsf{Scanner}: A machine to extract the user's biometric data by querying the oracle $\mathcal{O}_{\mathcal{B}}$.
	
	\item \textsf{Station}: An enrollment station responsible for collecting the user's biometric data to enroll them for a domain on the server.

	\item \textsf{Device}: A device belonging to a domain. In practice, it can be a device checking identities for a restricted area or an instrument. It owns a probe key $\sf psk$ and processes the $\sf Probe$ function for enrolled users of this domain.
	
	\item \textsf{KDS}: A key distribution service. It runs $\mathsf{Setup}$ to generate keys and distribute them to \textsf{Station}, \textsf{Domain}, and \textsf{Server}.
		
	\item \textsf{Server}: The server responsible for authenticating the user. It stores the comparison key $\mathsf{csk}$ for each domain and the user's enrollment message $\mathbf{c_x}$. On authentication, it compares the probe message with the registered enrollment message and returns the result.  

\end{itemize}


%-------------------

\input{tikz/dod_model.tex}

\pagebreak

\fi

%-------------------

\iffalse

\subsection{Instantiation with a 2i-IPFE Scheme}
\label{sec:2i-IPFE-instantiation}

Let .

\begin{itemize}

	\item 

	\item : The same as the scheme in \ref{sec:fh-IPFE-instantiation}. 

	\item .

	\item .

	\item .

	\item .

\end{itemize}

By the correctness of the functional encryption scheme $\sf FE$, we have
\[
	s = \mathsf{FE.Dec}(\mathsf{dk}_{\mathbf{I}}, \mathbf{c_x}, \mathbf{c_y}) =  \mathbf{x} \mathbf{I}_{k+2} \mathbf{y}^T = \mathbf{x} \mathbf{y}^T = \| \mathbf{b} - \mathbf{b}^\prime \|^2.
\]
just as the scheme in Section \ref{sec:fh-IPFE-instantiation}


Unlike the previous scheme, instantiated with a 2i-IPFE scheme in this way, the comparison secret key .

In the Device-of-Domain model, the indistinguishability of  enrolled by different samples. Therefore, we must limit the adversary's ability. For example, we can require the adversary to distinguish biometric vectors sampled from distributions in a pre-defined pool, and the adversary can only probe vectors randomly sampled from a distribution in the pool. We can also limit the rate of the probe oracle.

%-------------------

\subsection{Instantiation with a 2c-IPFE Scheme}
\label{sec:2c-IPFE-instantiation}

Note that if labels remain constant, a 2c-IPFE scheme is reduced to a 2i-IPFE scheme. Therefore, we can consider utilizing the label to represent each domain in the Device-of-Domain model. Let .

\begin{itemize}

	\item .

	Note that when the previous 2i-IPFE-based scheme in Section \ref{sec:2i-IPFE-instantiation} is applied to a Device-of-Domain model, we assume that  except different labels.

	\item : The same as the scheme in \ref{sec:2i-IPFE-instantiation}. 

	\item .

	\item .

	\item .

	\item .

	\item 

\end{itemize}

By the correctness of the functional encryption scheme $\sf FE$, if the labels of $\mathbf{c_x}$ and $\mathbf{c_y}$ are the same (they are of the same domain), we have
\[
	s = \mathsf{FE.Dec}(\mathsf{dk}_{\mathbf{I}}, \mathbf{c_x}, \mathbf{c_y}) =  \mathbf{x} \mathbf{I}_{k+2} \mathbf{y}^T = \| \mathbf{b} - \mathbf{b}^\prime \|^2.
\]
just as the scheme in Section \ref{sec:2i-IPFE-instantiation}

When the Device-of-Domain model is instantiated with a 2c-IPFE scheme in this way, the enrollment secret key .

If \textsf{Server} and \textsf{Domain} are both malicious, then the adversary can use . Therefore, we assume at most one party of them can be malicious at the same time. Note that this is the same as the 2i-IPFE-based scheme, where only one of \textsf{Server} and \textsf{Domain} can be malicious.

\fi

%-------------------

