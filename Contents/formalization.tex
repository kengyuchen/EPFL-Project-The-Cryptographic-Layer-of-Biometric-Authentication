%%%%%%%%%%%%%%%%%%%%

% Project Name: Semester Project Fall 2024 for EPFL
% File: formalization.tex
% Author: Keng-Yu Chen

%%%%%%%%%%%%%%%%%%%%


\subsection{Biometric Authentication Scheme}

In this section, we formally define a biometric authentication scheme. For this, we first define how we simulate biometric distributions of users.

Assume the existence of a family $\mathbb{B}$ of biometric distributions that are efficiently samplable. We have the following interfaces for all algorithms to interact with $\mathbb{B}$.

\begin{itemize}

	\item $\textsf{BioSamp}()$: Generate a random distribution $\mathcal{B}$ of $\mathbb{B}$. By this we mean providing either parameters of an efficiently samplable distribution or a PPT algorithm as the sampler. For simplicity, we write $\mathcal{B} \gets \textsf{BioSamp}()$ as $\mathcal{B} \getsdollar \mathbb{B}$.
	
	\item $\textsf{BioDelete}(\mathcal{B})$: Delete $\mathcal{B}$ from $\mathbb{B}$. Consequently, no further access to $\textsf{BioSamp}$ can derive $\mathcal{B}$. For simplicity, we write $\textsf{BioDelete}(\mathcal{B})$ as $\mathbb{B} \gets \mathbb{B} \setminus \mathcal{B}$.

	\item $\textsf{TempSamp}(\mathcal{B})$: Let $\mathcal{B}$ be a biometric distribution in $\mathbb{B}$. This algorithm samples a biometric template from $\mathcal{B}$. For simplicity, we write $\mathbf{b} \gets \textsf{TempSamp}(\mathcal{B})$ as $\mathbf{b} \getsdollar \mathcal{B}$.

\end{itemize}

\begin{definition}[Biometric Authentication Scheme]

A \emph{biomtric authentication shceme} $\Pi$ associated with a family $\mathbb{B}$ of biometric distributions is composed of the following algorithms.

\begin{itemize}

	\item $\textsf{getEnroll}^{\mathcal{O}_{\mathcal{B}}}() \to \mathbf{b}$: Given an oracle $\mathcal{O}_{\mathcal{B}}$, which samples biometric data from a distribution $\mathcal{B} \in \mathbb{B}$, it outputs a biometric template $\mathbf{b}$ for enrollment.

	\item $\textsf{getProbe}^{\mathcal{O}_{\mathcal{B}}}() \to \mathbf{b}^\prime$: Given an oracle $\mathcal{O}_{\mathcal{B}}$, which samples biometric data from a distribution $\mathcal{B} \in \mathbb{B}$, it outputs a biometric template $\mathbf{b}^\prime$ for probe.

	\item $\textsf{BioCompare}(\mathbf{b}, \mathbf{b}^\prime) \to s$: Given two biometric templates $\mathbf{b}$ and $\mathbf{b}^\prime$, it outputs a score $s$. 

	\item $\textsf{Verify}(s) \to r \in \{0,1\}$: It is a deterministic algorithm that reads the comparison score $s$ and determines whether this is a successful authentication ($r = 1$) or not ($r = 0$).

\end{itemize}
We also call these algorithms the \emph{biometric layer} of $\Pi$. We will add a \emph{cryptographic layer} on top of it in Section \ref{sec:formalization:cryptographic_layer}

\end{definition}


Given an authentication scheme $\Pi$, we can consider its true positive rate and false positive rate.

\begin{definition}[True Positive Rate]
For a biometric distribution $\mathcal{B} \in \mathbb{B}$ and $\mathbf{b} \gets \textsf{getEnroll}^{\mathcal{O}_\mathcal{B}}()$, define the \emph{true positive rate} \textsf{TP}.

\begin{alignat*}{2}
	\textsf{TP}(\mathcal{B}, \mathbf{b}) 
	:=&\; \Pr[ \mathbf{b}^\prime \gets \textsf{getProbe}^{\mathcal{O}_\mathcal{B}}()
	&&: \textsf{Verify}(\textsf{BioCompare}(\mathbf{b}, \mathbf{b}^\prime )) = 1 ] \\
	\\
	\textsf{TP}(\mathcal{B}) 
	:=&\; \Pr \Bigg[ 
		\begin{aligned}	
			& \mathbf{b} \gets \textsf{getEnroll}^{\mathcal{O}_\mathcal{B}}() \\
			& \mathbf{b}^\prime \gets \textsf{getProbe}^{\mathcal{O}_\mathcal{B}}()
		\end{aligned}
		&&: \textsf{Verify}(\textsf{BioCompare}(\mathbf{b}, \mathbf{b}^\prime)) = 1 \Bigg] \\
	=&\; \mathbb{E}_{ \mathbf{b} \gets \textsf{getEnroll}^{\mathcal{O}_\mathcal{B}}() }[\textsf{TP}(\mathcal{B}, \mathbf{b})] \\
	\\
	\textsf{TP} 
	:=&\; \Pr \vast[
		\begin{aligned}
			& \mathcal{B} \getsdollar \mathbb{B} \\
			& \mathbf{b} \gets \textsf{getEnroll}^{\mathcal{O}_\mathcal{B} }() \\
			& \mathbf{b}^\prime \gets \textsf{getProbe}^{\mathcal{O}_\mathcal{B} }()
		\end{aligned}
		&&: \textsf{Verify}(\textsf{BioCompare}(\mathbf{b}, \mathbf{b}^\prime )) = 1 \vast] \\
	=&\; \mathbb{E}_{\mathcal{B} \getsdollar \mathbb{B}}[\textsf{TP}(\mathcal{B})] 
\end{alignat*}

\end{definition}


\begin{definition}[False Positive Rate]
For a biometric distribution $\mathcal{B} \in \mathbb{B}, \mathbb{B} \gets \mathbb{B} \setminus \mathcal{B}$ and $\mathbf{b} \gets \textsf{getEnroll}^{\mathcal{O}_\mathcal{B}}()$, define the \emph{false positive rate} \textsf{FP}.

\begin{alignat*}{2}
	\textsf{FP}(\mathbf{b}) 
	:=&\; \Pr \Bigg[ 
		\begin{aligned}
			& \mathcal{B}^\prime \getsdollar \mathbb{B} \\
			& \mathbf{b}^\prime \gets \textsf{getProbe}^{\mathcal{O}_{\mathcal{B}^\prime}}()
		\end{aligned}
		&&: \textsf{Verify}(\textsf{BioCompare}(\mathbf{b}, \mathbf{b}^\prime)) = 1 \Bigg] \\
	\\
	\textsf{FP}(\mathcal{B}) 
	:=&\; \Pr \vast[ 
		\begin{aligned}
			& \mathcal{B}^\prime \getsdollar \mathbb{B} \\
			& \mathbf{b} \gets \textsf{getEnroll}^{\mathcal{O}_\mathcal{B} }() \\
			& \mathbf{b}^\prime \gets \textsf{getProbe}^{\mathcal{O}_{\mathcal{B}^\prime} }()
		\end{aligned}
		&&: \textsf{Verify}(\textsf{BioCompare}(\mathbf{b}, \mathbf{b}^\prime)) = 1 \vast] \\
	=&\; \mathbb{E}_{ \mathbf{b} \gets \textsf{getEnroll}^{\mathcal{O}_\mathcal{B}}() }[\textsf{FP}(\mathbf{b})] \\
	\\
	\textsf{FP} 
	:=&\; \Pr \vast[
		\begin{aligned}
			& \mathcal{B} \getsdollar \mathbb{B}, \mathbb{B} \gets \mathbb{B} \setminus \mathcal{B}, \mathcal{B}^\prime \getsdollar \mathbb{B} \\
			& \mathbf{b} \gets \textsf{getEnroll}^{\mathcal{O}_\mathcal{B} }() \\
			& \mathbf{b} \gets \textsf{getProbe}^{\mathcal{O}_{\mathcal{B}^\prime} }()
		\end{aligned}
		&&: \textsf{Verify}(\textsf{BioCompare}(\mathbf{b}, \mathbf{b}^\prime)) = 1 \vast] \\
	=&\; \mathbb{E}_{\mathcal{B} \getsdollar \mathbb{B}}[\textsf{FP}(\mathcal{B})]
\end{alignat*}

\end{definition}

Ideally, we hope $\textsf{TP}$ to be $1$, and $\textsf{FP}(\mathcal{B})$ to be negligible for any $\mathcal{B} \in \mathbb{B}$. However, due to the inherent nature of biometrics, there might be a nonzero false negative rate $ 1 - \textsf{TP} > 0$ and a non-negligible $\textsf{FP}(\mathcal{B})$. Our security model and analysis also take these possibilities into consideration.


%-------------------


\subsection{Cryptographic Layer}
\label{sec:formalization:cryptographic_layer}
In this work, we add a cryptographic layer on top of $\Pi$ to protect privacy of users. The cryptographic layer includes the following algorithms.

\begin{itemize}

	\item $\textsf{Setup}(1^\lambda) \to \textsf{esk}, \textsf{psk}, \textsf{csk}$: It outputs the enrollment secret key $\textsf{esk}$, probe secret key $\textsf{psk}$, and compare secret key $\sf csk$.

	\item $\textsf{Enroll}(\textsf{esk}, \mathbf{b}) \to \mathbf{c_x}$: On input a biometric template $\mathbf{b}$, it encodes it into a vector $\mathbf{x}$ and outputs the enrollment message $\mathbf{c_x}$.
	
	\item $\textsf{Probe}(\textsf{psk}, \mathbf{b}^\prime) \to \mathbf{c_y}$: On input a biometric template $\mathbf{b}^\prime$, it encodes it into a vector $\mathbf{y}$ and outputs the probe message $\mathbf{c_y}$ .

	\item $\textsf{Compare}(\textsf{csk}, \mathbf{c_x}, \mathbf{c_y)} \to s$: It compares the enrollment message $\mathbf{c_x}$ and probe message $\mathbf{c_y}$ and outputs a score $s$.

\end{itemize}

\noindent \textbf{Correctness}: An authentication scheme $\Pi$ is \emph{correct} if for any biometric distributions $\mathcal{B}$ and $\mathcal{B}^\prime$, let $\textsf{esk}, \textsf{psk}, \textsf{csk} \gets \textsf{Setup}(1^\lambda)$, $\mathbf{b} \gets \textsf{getEnroll}^{\mathcal{O}_\mathcal{B}}()$, $\mathbf{b}^\prime \gets \textsf{getProbe}^{\mathcal{O}_{\mathcal{B}^\prime}}()$, $\mathbf{c_x} \gets \textsf{Enroll}(\textsf{esk}, \mathbf{b})$, $\mathbf{c_y} \gets \textsf{Probe}(\textsf{psk}, \mathbf{b}^\prime)$. Then
	\[
		\Pr \left [
			\textsf{Compare}(\textsf{csk}, \mathbf{c_x}, \mathbf{c_y}) = \textsf{BioCompare}(\mathbf{b}, \mathbf{b}^\prime)
		\right ] = 1 - \negl.
	\]

In a real-world biometric system, these algorithms may be run by different parties such as a biometric scanner, a user's secure hardware, a trusted authority that issues keys, and the server.

Now, we provide two instantiations of a biometric authentication scheme with the cryptographic layer.

%-------------------

\subsubsection{Instantiation with an fh-IPFE Scheme}
\label{sec:fh-IPFE-instantiation}

Let $\textsf{FE} = (\textsf{FE.Setup}, \textsf{FE.KeyGen}, \textsf{FE.Enc}, \textsf{FE.Dec})$ be an fh-IPFE scheme we defined in Definition \ref{def:fh-IPFE}. Following \cite{cryptoeprint:2023/481}, we can instantiate a biometric authentication scheme using $\textsf{FE}$ with the distance metric the Euclidean distance.
Let $\textsf{getEnroll}^{\mathcal{O}_{\mathcal{B}}}()$ and $\textsf{getProbe}^{\mathcal{O}_{\mathcal{B}}}()$ both output vectors in $ \{0, 1, \cdots, m \}^k$ for all biometric distributions $\mathcal{B} \in \mathbb{B}$. 
For a pre-defined real number $\tau \geq 0$, define
\[
	\textsf{BioCompare}(\mathbf{b}, \mathbf{b}^\prime) \to \| \mathbf{b} - \mathbf{b}^\prime\|^2 \quad \text{and} \quad 
	\textsf{Verify}(s) \to 
	\begin{cases} 
		1 & \text{if } \sqrt{s} \leq \tau \\
		0 & \text{if } \sqrt{s} > \tau
	\end{cases}.
\]

Now, let the associated field of $\textsf{FE}$ be $\mathbb{Z}_q$, where $q$ is a prime number larger than the maximum possible Euclidean distance $m^2 \cdot k$. The scheme is instantiated as follows.

\begin{itemize}

	\item $\textsf{Setup}(1^\lambda)$: It calls $\textsf{FE.Setup}(1^\lambda) \to \textsf{msk}, \textsf{pp}$ and outputs $\textsf{esk} \gets (\textsf{msk}, \textsf{pp})$, $\textsf{psk} \gets (\textsf{msk}, \textsf{pp})$ and $\textsf{csk} \gets \textsf{pp}$.

	\item $\textsf{Enroll}(\textsf{esk}, \mathbf{b})$: On input a template vector $\mathbf{b} = (b_1, b_2, \cdots, b_k)$, the algorithm first encodes it as $\mathbf{x} = (x_1, x_2, \cdots, x_{k+2}) = (b_1, b_2, \cdots, b_k, 1, \|\mathbf{b}\|^2)$. Next, it calls $\textsf{FE.KeyGen}(\textsf{msk}, \textsf{pp}, \mathbf{x}) \to f_\mathbf{x}$ and outputs $\mathbf{c_x} \gets f_\mathbf{x}$.

	\item $\textsf{Probe}(\textsf{psk}, \mathbf{b}^\prime)$: On input a template vector $\mathbf{b}^\prime = (b_1^\prime, b_2^\prime, \cdots, b_k^\prime)$, the algorithm first encodes it as $\mathbf{y} = (y_1, y_2, \cdots, y_{k+2}) = (-2b_1^\prime, -2b_2^\prime, \cdots, -2b_k^\prime, \|\mathbf{b}^\prime\|^2, 1)$. Next, it calls $\textsf{FE.Enc}(\textsf{msk}, \textsf{pp}, \mathbf{y}) \to \mathbf{c_y}$ and outputs $\mathbf{c_y}$.

	\item $\textsf{Compare}(\textsf{csk}, \mathbf{c_x}, \mathbf{c_y)}$: It calls $\textsf{FE.Dec}(\textsf{pp}, \mathbf{c_x}, \mathbf{c_y}) \to s$ and outputs the value $s$.

\end{itemize}

By the correctness of the functional encryption scheme $\sf FE$, we have
\[
	s = \textsf{FE.Dec}(\textsf{pp}, \mathbf{c_x}, \mathbf{c_y}) =  \mathbf{x} \mathbf{y}^T = \sum_{i=1}^k -2b_ib_i^\prime + \|\mathbf{b}\|^2 + \|\mathbf{b}^\prime\|^2 = \| \mathbf{b} - \mathbf{b}^\prime \|^2.
\]
which is equal to $\textsf{BioCompare}(\mathbf{b}, \mathbf{b}^\prime)$. Therefore, if two templates $\mathbf{b}$ and $\mathbf{b}^\prime$ are close enough such that $\|\mathbf{b} - \mathbf{b}^\prime\| \leq \tau$, the scheme results in $r = 1$, a successful authentication.

Instantiated with an fh-IPFE scheme in this way, the comparison secret key $\textsf{csk}$ is public, and the enrollment secret key $\textsf{esk}$ and probe secret key $\textsf{psk}$ are the same. Anyone with access to the enrollment message $\mathbf{c_x}$ and either $\textsf{esk}$ or $\textsf{psk}$ can probe any (invalidly encoded) $\mathbf{y}^{\prime} \in \Z_q^{k+2}$ and find $\mathbf{x} {\mathbf{y}^\prime}^T$ to get partial or full information about the biometric template $\mathbf{b}$. Even if the adversary has no $\textsf{esk}$ or $\textsf{psk}$, if it can sample ciphertexts $\mathbf{c_{y}}$ corresponding to some unknown random vectors $\mathbf{y}$, and if the field size $q$ is not large enough, it can also find a forged $\mathbf{c_{y^*}}$ such that $\mathbf{x}\mathbf{y^*}^T \leq \tau$ with a non-negligible probability to impersonate the user by sampling many times offline.

A security analysis of this instantiation in our security model is given in Section \ref{sec:security_analysis:fh-IPFE}.

%-------------------

\subsubsection{Instantiation with a Relational Hash Scheme}
\label{sec:rh-instantiation}

Let $\textsf{RH} = (\textsf{RH.KeyGen}, \textsf{RH.Hash}_1, \textsf{RH.Hash}_2, \textsf{RH.Verify})$ be a relational hash scheme we defined in Definition \ref{def:rh} for the relation $R^\tau$ of Hamming distance proximity parametrized by a constant $\tau$.
\[
	R^\tau = \{ (\mathbf{x}, \mathbf{y}) \mid \textsf{HD}(\mathbf{x}, \mathbf{y}) \leq \tau \wedge \mathbf{x}, \mathbf{y} \in \{0,1\}^k \}
\]
Note that here we ignore the third parameter $Z$.
Let $\textsf{getEnroll}^{\mathcal{O}_{\mathcal{B}}}()$ and $\textsf{getProbe}^{\mathcal{O}_{\mathcal{B}}}()$ both output vectors in $ \{0, 1\}^k$ for all biometric distributions $\mathcal{B} \in \mathbb{B}$, and let 
\[
	\textsf{BioCompare}(\mathbf{b}, \mathbf{b}^\prime) \to
	\begin{cases}
		1 & \text{if } (\mathbf{b}, \mathbf{b}^\prime) \in R^\tau \\
		0 & \text{if } (\mathbf{b}, \mathbf{b}^\prime) \notin R^\tau
	\end{cases} \quad \text{and} \quad
	\textsf{Verify}(s) \to s.
\]
Following \cite{cryptoeprint:2014/394}, we can instantiate a biometric authentication scheme using $\textsf{RH}$.  Let the biometric distribution $\mathcal{B} \subseteq \{0,1\}^k$.

\begin{itemize}

	\item $\textsf{Setup}(1^\lambda)$: It calls $\textsf{RH.KeyGen}(1^\lambda) \to \textsf{pk}$ and outputs $\textsf{esk} \gets \textsf{pk}$, $\textsf{psk} \gets \textsf{pk}$, and $\textsf{csk} \gets \textsf{pk}$.

	\item $\textsf{Enroll}(\textsf{esk}, \mathbf{b})$: Let $\mathbf{x} \gets \mathbf{b}$. It calls $\textsf{RH.Hash}_1(\textsf{pk}, \mathbf{x}) \to \mathbf{h_x}$ and outputs $\mathbf{c_x} \gets \mathbf{h_x}$.

	\item $\textsf{Probe}(\textsf{psk}, \mathbf{b}^\prime)$: Let $\mathbf{y} \gets \mathbf{b}$. It calls $\textsf{RH.Hash}_2(\textsf{pk}, \mathbf{y}) \to \mathbf{h_y}$ and outputs $\mathbf{c_y} \gets \mathbf{h_y}$.

	\item $\textsf{Compare}(\textsf{csk}, \mathbf{c_x}, \mathbf{c_y)}$: It calls $\textsf{RH.Verify}(\textsf{pk}, \mathbf{h_x}, \mathbf{h_y}) \to s$ and outputs the value $s$.

\end{itemize}

By the correctness of the relational hash scheme $\textsf{RH}$, we have (except for a negligible probability),
\[
	r = 1 \Leftrightarrow (\mathbf{x}, \mathbf{y}) = (\mathbf{b}, \mathbf{b}^\prime) \in R^\tau \Leftrightarrow \textsf{HD}(\mathbf{b}, \mathbf{b}^\prime) \leq \tau
\]

A security analysis of this instantiation in our security model is given in Section \ref{sec:security_analysis:rh}.

%-------------------


\iffalse

We discuss two usage models that employs the authentication scheme $\Pi$.


\subsection{Usage Model – Device-of-User}
\label{sec:dou_model}

In the model described in Figure \ref{fig:model_dou_overview} (an overview), Figure \ref{fig:model_dou_enrollment} (on enrollment), and Figure \ref{fig:model_dou_auth} (on authentication), users authenticate themselves to a server through their own devices and biometric scanners that are shared among different users.
A key distribution service distributes keys for them. In practice, this model applies to the situation when the users access an online service run by the server.

\begin{itemize}

	\item \textsf{User}: The user who enrolls its biometric data and authenticates itself to the server. We assume the user's biometric distribution is $\mathcal{B} \in \mathbb{B}$. 

	\item \textsf{Scanner}: A machine to extract the user's biometric data by querying the oracle $\mathcal{O}_{\mathcal{B}}$.
	
	\item \textsf{Device}: A device belonging to the user. In practice, it can be a desktop or a mobile phone. It processes the \textsf{Enroll} and \textsf{Probe} functions for $\textsf{User}$ with keys \textsf{esk} and \textsf{psk}. It queries $\mathcal{O}_{\mathcal{B}}$ for biometric data through the \textsf{Scanner}.
	
	\item \textsf{KDS}: A key distribution service. It runs $\textsf{Setup}$ to generate keys and distribute them to $\textsf{Device}$ and $\textsf{Server}$.
		
	\item \textsf{Server}: The server responsible for authenticating the user. It stores the comparison key \textsf{csk} and the user's enrollment message $\mathbf{c_x}$. On authentication, it compares the probe message with the registered enrollment message and returns the result.  

\end{itemize}

The Device-of-User model, when instantiated by an fh-IPFE scheme (Section \ref{sec:fh-IPFE-instantiation}), is analogous to the use case presented in \cite{cryptoeprint:2023/481}.
In their model, a user possesses a personal device, such as a smartphone or laptop, and a secure hardware device that runs an initial setup and stores all the keys, which corresponds to our \textsf{KDS}.
On enrollemnt and authentication, the user inputs biometric templates onto the device, which corresponds to our \textsf{Scanner}.
Subsequently, the device transmits the template to the secure hardware for the enrollment or probing processes, which are equivalent to our \textsf{Device}.
In addition, they incorporate a two-factor authentication mechanism.
The secure hardware also executes a digital signature scheme and sign the probe message on authentication.


% User-Device with KDS
% Usage Model – Device of User

% Overview
\begin{figure}
\centering

\begin{tikzpicture}[node distance=2cm]

% Gadgets
\node (user1) [inner sep=0pt] at (0,6) 
    {\includegraphics[width=0.1\textwidth]{image/user.jpg}};

\node (user2) [inner sep=0pt] at (0,3) 
    {\includegraphics[width=0.1\textwidth]{image/user.jpg}};

\node (scanner1) [element] at (3,7.5) {{\sf Scanner}};

\node (scanner2) [element] at (3,5.5) {{\sf Scanner}};

\node (scanner3) [element] at (3,3.5) {{\sf Scanner}};

\node (scanner4) [element] at (3,1.5) {{\sf Scanner}};

\node (device1) [element] at (7,6) {\sf Device};
\node (udevice1) [inner sep=0pt] at ($(device1) + (-0.7, +0.95)$)
    {\includegraphics[width=0.03\textwidth]{image/user.jpg}};

\node (device2) [element] at (7,3) {\sf Device};
\node (udevice2) [inner sep=0pt] at ($(device2) + (-0.7, +0.95)$)
    {\includegraphics[width=0.03\textwidth]{image/user.jpg}};

\node (server) [element] at (12,4.5) {\sf Server};

% Texts
% User
\node [align=center] at ($(user1) + (-0.9, 0.9)$) {{\sf User1}};

\node [align=center] at ($(user2) + (-0.9, 0.9)$) {{\sf User2}};

\node [align=center] at ($(udevice1) + (0, 0.5)$) {{\sf User1}};

\node [align=center] at ($(udevice2) + (0, 0.5)$) {{\sf User2}};

% User and Scanner
\draw [darrow, dotted] (user1.east) -- node [align=center, left]{$\mathcal{O}_\mathcal{B}$} (scanner1.west);

\draw [darrow, dotted] (user1.east) -- (scanner2.west);

\draw [darrow, dotted] (user1.east) -- (scanner3.west);

\draw [darrow, dotted] (user1.east) -- (scanner4.west);

\draw [darrow, dotted] (user2.east) -- (scanner1.west);

\draw [darrow, dotted] (user2.east) -- (scanner2.west);

\draw [darrow, dotted] (user2.east) -- (scanner3.west);

\draw [darrow, dotted] (user2.east) -- (scanner4.west);


% Scanner and Device
\draw [darrow, dotted] (scanner1.east) -- node [align=center, above] {$\mathcal{O}_\mathcal{B}$} (device1.west);

\draw [darrow, dotted] (scanner2.east) -- (device1.west);

\draw [darrow, dotted] (scanner3.east) -- (device1.west);

\draw [darrow, dotted] (scanner4.east) -- (device1.west);

\draw [darrow, dotted] (scanner1.east) -- (device2.west);

\draw [darrow, dotted] (scanner2.east) -- (device2.west);

\draw [darrow, dotted] (scanner3.east) -- (device2.west);

\draw [darrow, dotted] (scanner4.east) -- (device2.west);

% Device and Server
\draw [darrow, dotted] (device1.east) -- node [align=center, above, sloped] {${\sf esk}_1, {\sf psk}_1, {\sf csk}_1$} (server.west);

\draw [darrow, dotted] (device2.east) -- node [align=center, above, sloped] {${\sf esk}_2, {\sf psk}_2, {\sf csk}_2$} (server.west);


% Server


\end{tikzpicture}

\caption{An Overview of the Device-of-User Usage Model}
\label{fig:model_dou_overview}
\end{figure}


%-------------------
% Enrollment

\begin{figure}
\centering

\begin{tikzpicture}[node distance=2cm]

% Gadgets
\node (user) [inner sep=0pt] at (0,8) 
    {\includegraphics[width=0.1\textwidth]{image/user.jpg}};

\node (scanner) [element] at (1.5,6.5) {{\sf Scanner}}; 

\node (device) [element] at (7,6.5) {\sf Device};

\node (KDS) [element] at (9.5, 9) {\sf KDS};

\node (server) [element] at (12,6.5) {\sf Server};

% Texts

% KDS
\node [align=center] at ($(KDS.south) + (0, -0.5)$) {${\sf esk}, 
 {\sf psk}, {\sf csk} \gets {\sf Setup}(1^\lambda)$};

% KDS and Device
\draw [arrow] (KDS.west) -| node [align=center, above] {${\sf esk}, {\sf psk}$}  (device.north);

% KDS and Server
\draw [arrow] (KDS.east) -| node [align=center, above] {${\sf csk}$}  (server.north);

% User
\node [align=center] at ($(user) + (-0.9, 0.9)$) {{\sf User}};

% User and Scanner
\draw [arrow] (scanner.west) -| node [align=center, left] {{\sf query} \\ $\mathcal{O}_\mathcal{B}$} (user.south);

\draw [arrow] (user.east) -| node [align=center, right] {{\sf resp.} \\ $\mathbf{b} \getsdollar \mathcal{B}$}  (scanner.north);

% Scanner and Device
\draw [arrow]  ($(device.west) + (-1.5, -2.1)$) -- node[align=center, above] {{\sf query} \\ $\mathcal{O}_\mathcal{B}$} ($(scanner.east) + (0.5, -2.1)$);
\draw [arrow]  ($(scanner.east) + (0.5, -2.3)$) -- node[align=center, below] {{\sf resp.} \\ $\mathbf{b} \getsdollar \mathcal{B}$} ($(device.west) + (-1.5, -2.3)$);

% Device
\node [align=center] at ($(device.south) + (0, -0.5)$) {
    Store ${\sf esk}, {\sf psk}$        
};
\node [align=center] at ($(device.south) + (0, -1.5)$) {
	$\mathbf{x} \gets {\sf encodeEnroll}^{\mathcal{O}_{\mathcal{B}}}()$
};
\node [align=center] at ($(device.south) + (0, -2.5)$) {
    $\mathbf{c_x} \gets {\sf Enroll}({\sf esk, \mathbf{x}})$
};

% Device and Server
\draw [arrow]  ($(device.east) + (1.3, -3.2)$) -- node[align=center, above] {$\mathbf{c_x}$} ($(server.west) + (-0.7, -3.2)$);

% Server
\node [align=center] at ($(server.south) + (0, -0.5)$) {
    Store ${\sf csk}$        
};
\node [align=center] at ($(server.south) + (0, -2.7)$) {
    Store $\mathbf{c_x}$
};


\end{tikzpicture}

\caption{Device-of-User Usage Model on Enrollment}
\label{fig:model_dou_enrollment}
\end{figure}


%-------------------

% Authentication
\begin{figure}[htp]
\centering

\begin{tikzpicture}[node distance=2cm]

% Gadgets
\node (user) [inner sep=0pt] at (0,8) 
    {\includegraphics[width=0.1\textwidth]{image/user.jpg}};

\node (scanner) [element] at (1.5,6.5) {{\sf Scanner}}; 

\node (device) [element] at (7,6.5) {\sf Device};

\node (server) [element] at (12,6.5) {\sf Server};

% Texts

% User
\node [align=center] at ($(user) + (-0.9, 0.9)$) {{\sf User}};

% User and Scanner
\draw [arrow] (scanner.west) -| node [align=center, left] {{\sf query} \\ $\mathcal{O}_\mathcal{B}$} (user.south);

\draw [arrow] (user.east) -| node [align=center, right] {{\sf resp.} \\ $\mathbf{b} \getsdollar \mathcal{B}$}  (scanner.north);

% Device
\node [align=center] at ($(device.south) + (0, -0.9)$) {
    $\mathbf{y} \gets {\sf encodeProbe}^{\mathcal{O}_{\mathcal{B}}}()$
};
\node [align=center] at ($(device.south) + (0, -1.9)$) {
    $\mathbf{c_y} \gets {\sf Probe}({\sf psk, \mathbf{y}})$
};

% % Scanner and Device
% \draw [arrow]  ($(device.west) + (-1.5, -1.5)$) -- node[align=center, above] {{\sf query} \\ $\mathcal{O}_\mathcal{B}$} ($(scanner.east) + (0.5, -1.5)$);
% \draw [arrow]  ($(scanner.east) + (0.5, -1.7)$) -- node[align=center, below] {{\sf resp.} \\ $\mathbf{b} \getsdollar \mathcal{B}$} ($(device.west) + (-1.5, -1.7)$);

% % Device and Server
% \draw [arrow]  ($(device.east) + (1, -3.4)$) -- node[align=center, above] {$\mathbf{c_y}$} ($(server.west) + (-1, -3.4)$);
% \draw [arrow]  ($(server.west) + (-1, -4.4)$) -- node[align=center, above] {$r$} ($(device.east) + (1, -4.4)$);

% % Server
% \node [align=center] at ($(server.south) + (0, -3.4)$) {
%     $s \gets {\sf Compare}({\sf csk}, \mathbf{c_x}, \mathbf{c_y})$ \\
%     $r \gets {\sf Verify}(s)$
% };


% Scanner and Device
\draw [arrow]  ($(device.west) + (-1.5, -1.5)$) -- node[align=center, above] {{\sf query} \\ $\mathcal{O}_\mathcal{B}$} ($(scanner.east) + (0.5, -1.5)$);
\draw [arrow]  ($(scanner.east) + (0.5, -1.7)$) -- node[align=center, below] {{\sf resp.} \\ $\mathbf{b} \getsdollar \mathcal{B}$} ($(device.west) + (-1.5, -1.7)$);

% Device and Server
\draw [arrow]  ($(device.east) + (1, -2.7)$) -- node[align=center, above] {$\mathbf{c_y}$} ($(server.west) + (-1, -2.7)$);
\draw [arrow]  ($(server.west) + (-1, -3.7)$) -- node[align=center, above] {$r$} ($(device.east) + (1, -3.7)$);

% Server
\node [align=center] at ($(server.south) + (0, -2.7)$) {
    $s \gets {\sf Compare}({\sf csk}, \mathbf{c_x}, \mathbf{c_y})$ \\
    $r \gets {\sf Verify}(s)$
};

\end{tikzpicture}

\caption{Device-of-User Usage Model on Authentication}
\label{fig:model_dou_auth}
\end{figure}



%-------------------

\subsection{Usage Model – Device-of-Domain}
\label{sec:dod_model}

In the model described in Figure \ref{fig:model_dod_overview} (an overview), Figure \ref{fig:model_dod_enrollment} (on enrollment), and Figure \ref{fig:model_dod_auth} (on authentication), users first enroll themselves at an enrollment station and then authenticate themselves to a server through devices that belong to a domain.
A key distribution service distributes enrollment keys to the enrollment station, probe keys to the domain, and comparison keys to the server. In practice, a domain can be a department in an organization, and this models applies to the situation when a user wants to access a public service of a department, such as a restricted area or instruments. 

\begin{itemize}

	\item \textsf{User}: The user who enrolls its biometric data at an enrollment station and authenticates itself to the server. We assume the user's biometric distribution is $\mathcal{B} \in \mathbb{B}$.
	
	\item \textsf{Domain}: A domain that owns several devices, all of which share one enrollment key $\textsf{esk}$, one probe key $\textsf{psk}$ and one comparison key $\textsf{csk}$. Only the probe key is stored at each device of a domain. The enrollment key is stored at the enrollment station, and the comparison key is stored at the server. In practice, a domain can be a department, and users enroll and authenticate themselves before accessing a restricted service of this department.

	\item \textsf{Scanner}: A machine to extract the user's biometric data by querying the oracle $\mathcal{O}_{\mathcal{B}}$.
	
	\item \textsf{Station}: An enrollment station responsible for collecting the user's biometric data to enroll them for a domain on the server.

	\item \textsf{Device}: A device belonging to a domain. In practice, it can be a device checking identities for a restricted area or an instrument. It owns a probe key $\sf psk$ and processes the $\sf Probe$ function for enrolled users of this domain.
	
	\item \textsf{KDS}: A key distribution service. It runs $\textsf{Setup}$ to generate keys and distribute them to \textsf{Station}, \textsf{Domain}, and \textsf{Server}.
		
	\item \textsf{Server}: The server responsible for authenticating the user. It stores the comparison key \textsf{csk} for each domain and the user's enrollment message $\mathbf{c_x}$. On authentication, it compares the probe message with the registered enrollment message and returns the result.  

\end{itemize}


%-------------------

% Usage Model – Device of Domain
\begin{figure}
\centering

\begin{tikzpicture}[node distance=2cm]

% Gadgets
\node (user1) [inner sep=0pt] at (0,6) 
    {\includegraphics[width=0.1\textwidth]{image/user.jpg}};

\node (user2) [inner sep=0pt] at (0,3) 
    {\includegraphics[width=0.1\textwidth]{image/user.jpg}};

\node (scanner1) [element] at (3,7.5) {{\sf Scanner}};

\node (scanner2) [element] at (3,5.5) {{\sf Scanner}};

\node (scanner3) [element] at (3,3.5) {{\sf Scanner}};

\node (scanner4) [element] at (3,1.5) {{\sf Scanner}};

\node (station) [rectangle, rounded corners, 
	minimum width=2.5cm, 
	minimum height=1.5cm,
	text centered,
	fill=white!70!black,
	draw=black
] at (7,8) {\sf Station};

\node (device1) [element] at (7, 5.3) {\sf Device};

\node (device2) [element] at (7, 3.7) {\sf Device};

\node (device3) [element] at (7, 1.5) {\sf Device};

\begin{pgfonlayer}{background layer}
\node (domain1) [
    rectangle, rounded corners, 
    minimum width=2.2cm, 
    minimum height=4cm,
    fill=blue!50!white,
    draw=black]
at ($(device1)!0.6!(device2) + (0, 0.5)$) {};
\end{pgfonlayer};

\node (domain1_text) [element,fill=none,draw=none] at ($(domain1.north) + (0, -0.5)$) {\sf Domain1};

\begin{pgfonlayer}{background layer}
\node (domain2) [
    rectangle, rounded corners, 
    minimum width=2.2cm, 
    minimum height=2.1cm,
    fill=orange!50!white,
    draw=black]
at ($(device3) + (0, -0.3)$) {};
\end{pgfonlayer};

\node (domain2_text) [element,fill=none,draw=none] at ($(domain2.south) + (0, 0.4)$) {\sf Domain 2};


\node (server) [element] at (12,4.5) {\sf Server};


% Texts
% User
\node [align=center] at ($(user1) + (-0.9, 0.9)$) {{\sf User1}};

\node [align=center] at ($(user2) + (-0.9, 0.9)$) {{\sf User2}};

% User and Scanner
\draw [darrow, dotted] (user1.east) -- node [align=center, left]{$\mathcal{O}_\mathcal{B}$} (scanner1.west);

\draw [darrow, dotted] (user1.east) -- (scanner2.west);

\draw [darrow, dotted] (user1.east) -- (scanner3.west);

\draw [darrow, dotted] (user1.east) -- (scanner4.west);

\draw [darrow, dotted] (user2.east) -- (scanner1.west);

\draw [darrow, dotted] (user2.east) -- (scanner2.west);

\draw [darrow, dotted] (user2.east) -- (scanner3.west);

\draw [darrow, dotted] (user2.east) -- (scanner4.west);

% Scanner and Device
\draw [darrow, dotted] (scanner1.east) -- (device1.west);

\draw [darrow, dotted] (scanner2.east) -- (device1.west);

\draw [darrow, dotted] (scanner3.east) -- (device1.west);

\draw [darrow, dotted] (scanner4.east) -- (device1.west);

\draw [darrow, dotted] (scanner1.east) -- (device2.west);

\draw [darrow, dotted] (scanner2.east) -- (device2.west);

\draw [darrow, dotted] (scanner3.east) -- (device2.west);

\draw [darrow, dotted] (scanner4.east) -- (device2.west);

\draw [darrow, dotted] (scanner1.east) -- (device3.west);

\draw [darrow, dotted] (scanner2.east) -- (device3.west);

\draw [darrow, dotted] (scanner3.east) -- (device3.west);

\draw [darrow, dotted] (scanner4.east) -- (device3.west);

% Scanner and Station
\draw [darrow, dotted] (scanner1.east) -- node [align=center, above]{$\mathcal{O}_\mathcal{B}$} (station.west);

\draw [darrow, dotted] (scanner2.east) -- (station.west);

\draw [darrow, dotted] (scanner3.east) -- (station.west);

\draw [darrow, dotted] (scanner4.east) -- (station.west);

% Station and Server
\draw [darrow, dotted] (station.east) -- node [align=center, midway, above, sloped] {$\textcolor{blue}{{\sf esk}_1}, \textcolor{orange}{{\sf esk}_2}$} (server.west);


% Device and Server
\draw [darrow, dotted] (device1.east) -- node [blue, align=center, midway, above, sloped] {${\sf psk}_1, {\sf csk}_1$} (server.west);

\draw [darrow, dotted] (device2.east) -- node [blue, align=center, midway, above, sloped] {${\sf psk}_1, {\sf csk}_1$} (server.west);

\draw [darrow, dotted] (device3.east) -- node [orange, align=center, midway, below, sloped] {${\sf psk}_2, {\sf csk}_2$} (server.west);


% Server


\end{tikzpicture}

\caption{An overview of the Device-of-Domain Usage Model}
\label{fig:model_dod_overview}
\end{figure}



%-------------------
% Enrollment

\begin{figure}
\centering

\begin{tikzpicture}[node distance=2cm]

% Gadgets
\node (user) [inner sep=0pt] at (0,8) 
    {\includegraphics[width=0.1\textwidth]{image/user.jpg}};

\node (scanner) [element] at (1.5,6.5) {{\sf Scanner}};

\begin{pgfonlayer}{background layer}
\node (domain) [
    rectangle, rounded corners, 
    minimum width=2.7cm, 
    minimum height=1.8cm,
    fill=blue!50!white,
    draw=black] 
at (4.5, 9.5) {};
\end{pgfonlayer};

\node (domain_text) [element,fill=none,draw=none] at ($(domain.south) + (0, 1.2)$) {\sf Domain};

\node (device1) [
    rectangle, rounded corners, 
    minimum width=0.3cm, 
    minimum height=0.3cm,
    text centered,
    fill=white!70!black,
    draw=black]
at ($(domain.south) + (-0.65, 0.4)$) {\scriptsize{\sf Device}};

\node (device2) [
    rectangle, rounded corners, 
    minimum width=0.3cm, 
    minimum height=0.3cm,
    text centered,
    fill=white!70!black,
    draw=black]
at ($(domain.south) + (0.65, 0.4)$) {\scriptsize{\sf Device}};


\node (station) [element] at (7,6.5) {\sf Station};

\node (KDS) [element] at (9.5, 9.5) {\sf KDS};

\node (server) [element] at (12,6.5) {\sf Server};

% Texts

% KDS
\node [align=center] at ($(KDS.south) + (0, -0.5)$) {${\sf esk}, 
 {\sf psk}, {\sf csk} \gets {\sf Setup}(1^\lambda)$};

% KDS and Station
\draw [arrow] (KDS.west) -| node (esk) [align=center, below=1cm] {}  (station.north);
\node [align=center] at ($(esk) + (-0.3, 0)$) { {\sf esk} };

% KDS and Domain
\draw [arrow] (KDS.west) -- node (psk) [align=center, above=0.1cm] {} (domain.east);
\node [align=center] at ($(psk) + (-0.7, 0)$) { {\sf psk} };

% KDS and Server
\draw [arrow] (KDS.east) -| node [align=center, above] {${\sf csk}$}  (server.north);

% Domain
\node [align=center] at ($(domain.south) + (0, -0.5)$) {Store ${\sf psk}$};

% User
\node [align=center] at ($(user) + (-0.9, 0.9)$) {{\sf User}};

% User and Scanner
\draw [arrow] (scanner.west) -| node [align=center, left] {{\sf query} \\ $\mathcal{O}_\mathcal{B}$} (user.south);

\draw [arrow] (user.east) -| node [align=center, right] {{\sf resp.} \\ $\mathbf{b} \getsdollar \mathcal{B}$}  (scanner.north);

% Scanner and Station
\draw [arrow]  ($(station.west) + (-1.5, -2.1)$) -- node[align=center, above] {{\sf query} \\ $\mathcal{O}_\mathcal{B}$} ($(scanner.east) + (0.5, -2.1)$);
\draw [arrow]  ($(scanner.east) + (0.5, -2.3)$) -- node[align=center, below] {{\sf resp.} \\ $\mathbf{b} \getsdollar \mathcal{B}$} ($(station.west) + (-1.5, -2.3)$);

% Station
\node [align=center] at ($(station.south) + (0, -0.5)$) {
    Store ${\sf esk}$        
};
\node [align=center] at ($(station.south) + (0, -1.5)$) {
	$\mathbf{x} \gets {\sf encodeEnroll}^{\mathcal{O}_{\mathcal{B}}}()$
};
\node [align=center] at ($(station.south) + (0, -2.5)$) {
    $\mathbf{c_x} \gets {\sf Enroll}({\sf esk, \mathbf{x}})$
};

% Station and Server
\draw [arrow]  ($(station.east) + (1.3, -3.2)$) -- node[align=center, above] {$\mathbf{c_x}$} ($(server.west) + (-0.7, -3.2)$);

% Server
\node [align=center] at ($(server.south) + (0, -0.5)$) {
    Store ${\sf csk}$        
};
\node [align=center] at ($(server.south) + (0, -2.7)$) {
    Store $\mathbf{c_x}$
};


\end{tikzpicture}

\caption{Device-of-Domain Usage Model on Enrollment}
\label{fig:model_dod_enrollment}
\end{figure}


%-------------------

% Authentication
\begin{figure}[htp]
\centering

\begin{tikzpicture}[node distance=2cm]

% Gadgets
\node (user) [inner sep=0pt] at (0,8) 
    {\includegraphics[width=0.1\textwidth]{image/user.jpg}};

\node (scanner) [element] at (1.5,6.5) {{\sf Scanner}}; 

\node (device) [element] at (7,6.5) {\sf Device};

\begin{pgfonlayer}{background layer}
\node (domain) [
    rectangle, rounded corners, 
    minimum width=2.2cm, 
    minimum height=2.2cm,
    fill=blue!50!white,
    draw=black] 
at ($(device) + (0, 0.3)$) {};
\end{pgfonlayer};

\node (domain_text) [element,fill=none,draw=none] at ($(domain.south) + (0, 1.8)$) {\sf Domain};


\node (server) [element] at (12,6.5) {\sf Server};

% Texts

% User
\node [align=center] at ($(user) + (-0.9, 0.9)$) {{\sf User}};

% User and Scanner
\draw [arrow] (scanner.west) -| node [align=center, left] {{\sf query} \\ $\mathcal{O}_\mathcal{B}$} (user.south);

\draw [arrow] (user.east) -| node [align=center, right] {{\sf resp.} \\ $\mathbf{b} \getsdollar \mathcal{B}$}  (scanner.north);

% Device
\node [align=center] at ($(device.south) + (0, -0.9)$) {
    $\mathbf{y} \gets {\sf encodeProbe}^{\mathcal{O}_{\mathcal{B}}}()$
};
\node [align=center] at ($(device.south) + (0, -1.9)$) {
    $\mathbf{c_y} \gets {\sf Probe}({\sf psk, \mathbf{y}})$
};

% Scanner and Device
\draw [arrow]  ($(device.west) + (-1.5, -1.5)$) -- node[align=center, above] {{\sf query} \\ $\mathcal{O}_\mathcal{B}$} ($(scanner.east) + (0.5, -1.5)$);
\draw [arrow]  ($(scanner.east) + (0.5, -1.7)$) -- node[align=center, below] {{\sf resp.} \\ $\mathbf{b} \getsdollar \mathcal{B}$} ($(device.west) + (-1.5, -1.7)$);

% Device and Server
\draw [arrow]  ($(device.east) + (1, -2.7)$) -- node[align=center, above] {$\mathbf{c_y}$} ($(server.west) + (-1, -2.7)$);
\draw [arrow]  ($(server.west) + (-1, -3.7)$) -- node[align=center, above] {$r$} ($(device.east) + (1, -3.7)$);

% Server
\node [align=center] at ($(server.south) + (0, -2.7)$) {
    $s \gets {\sf Compare}({\sf csk}, \mathbf{c_x}, \mathbf{c_y})$ \\
    $r \gets {\sf Verify}(s)$
};


\end{tikzpicture}

\caption{Device-of-Domain Usage Model on Authentication}
\label{fig:model_dod_auth}
\end{figure}



\pagebreak

\fi

%-------------------

\iffalse

\subsection{Instantiation with a 2i-IPFE Scheme}
\label{sec:2i-IPFE-instantiation}

Let $\textsf{FE} = (\textsf{FE.Setup}, \textsf{FE.KeyGen}, \textsf{FE.Enc}, \textsf{FE.Dec})$ be a 2i-IPFE scheme we defined in Definition \ref{def:2i-IPFE}. Following the scheme in Section \ref{sec:fh-IPFE-instantiation}, we can instantiate a biometric authentication scheme using $\textsf{FE}$.

\begin{itemize}

	\item $\textsf{Setup}(1^\lambda)$: It calls $\textsf{FE.Setup}(1^\lambda) \to \textsf{sk}, \textsf{ek}_1, \textsf{ek}_2$, $ \textsf{FE.KeyGen}(sk, \mathbf{I}_{k+2}) \to \textsf{dk}_{\mathbf{I}} $, where $\mathbf{I}_{k+2}$ is an identity matrix of size $(k+2) \times (k+2)$. It outputs $\textsf{esk} \gets \textsf{ek}_1$, $\textsf{psk} \gets \textsf{ek}_2$, and $\textsf{csk} \gets \textsf{dk}_{\mathbf{I}}$

	\item $\textsf{encodeEnroll}^{\mathcal{O}_{\mathcal{B}}}(), \textsf{encodeProbe}^{\mathcal{O}_{\mathcal{B}}}()$: The same as the scheme in \ref{sec:fh-IPFE-instantiation}. 

	\item $\textsf{Enroll}(\textsf{esk}, \mathbf{x})$: It calls $\textsf{FE.Enc}(\textsf{ek}_1, \mathbf{x}) \to \mathbf{c_x}$ and outputs $\mathbf{c_x}$.

	\item $\textsf{Probe}(\textsf{psk}, \mathbf{y})$: It calls $\textsf{FE.Enc}(\textsf{ek}_2, \mathbf{y}) \to \mathbf{c_y}$ and outputs $\mathbf{c_y}$.

	\item $\textsf{Compare}(\textsf{csk}, \mathbf{c_x}, \mathbf{c_y)}$: It calls $\textsf{FE.Dec}(\textsf{dk}_{\mathbf{I}}, \mathbf{c_x}, \mathbf{c_y}) \to s$ and outputs the value $s$.

	\item $\textsf{Verify}(s)$: If $\sqrt{s} \leq \tau$, a pre-defined threshold for comparing the closeness of two templates, then it outputs $r = 1$; otherwise, it outputs $r = 0$.

\end{itemize}

By the correctness of the functional encryption scheme $\sf FE$, we have
\[
	s = \textsf{FE.Dec}(\textsf{dk}_{\mathbf{I}}, \mathbf{c_x}, \mathbf{c_y}) =  \mathbf{x} \mathbf{I}_{k+2} \mathbf{y}^T = \mathbf{x} \mathbf{y}^T = \| \mathbf{b} - \mathbf{b}^\prime \|^2.
\]
just as the scheme in Section \ref{sec:fh-IPFE-instantiation}


Unlike the previous scheme, instantiated with a 2i-IPFE scheme in this way, the comparison secret key $\textsf{csk}$ is now secret, and the enrollment secret key $\textsf{esk}$ and probe secret key $\textsf{psk}$ are distinct. Without $\textsf{csk}$, one cannot compare an enrollment message $\mathbf{c_x}$ and a probe message $\mathbf{c_y}$. We can also transmit $\mathbf{c_x}$ in a public channel and store it in a public storage, under necessary security requirements of the 2i-IPFE scheme, such as indistinguishability of $\mathbf{c_x}$.

In the Device-of-Domain model, the indistinguishability of $\mathbf{c_x}$ is against an adversary who has a probe oracle $\textsf{Probe}(\textsf{psk}, \cdot)$. If \textsf{Server} is malicious, then it can use $\textsf{csk}$ to distinguish $\mathbf{c_x}$ enrolled by different samples. Therefore, we must limit the adversary's ability. For example, we can require the adversary to distinguish biometric vectors sampled from distributions in a pre-defined pool, and the adversary can only probe vectors randomly sampled from a distribution in the pool. We can also limit the rate of the probe oracle.

%-------------------

\subsection{Instantiation with a 2c-IPFE Scheme}
\label{sec:2c-IPFE-instantiation}

Note that if labels remain constant, a 2c-IPFE scheme is reduced to a 2i-IPFE scheme. Therefore, we can consider utilizing the label to represent each domain in the Device-of-Domain model. Let $\textsf{FE} = (\textsf{FE.Setup}, \textsf{FE.KeyGen}, \textsf{FE.Enc}, \textsf{FE.Dec})$ be a 2c-IPFE scheme we defined in Definition \ref{def:2c-IPFE}. Following the scheme in Section \ref{sec:2i-IPFE-instantiation}, we can instantiate a biometric authentication scheme using $\textsf{FE}$.

\begin{itemize}

	\item $\textsf{Setup}(1^\lambda)$: It calls $\textsf{FE.Setup}(1^\lambda) \to \textsf{sk}, \textsf{ek}_1, \textsf{ek}_2$, $\textsf{FE.KeyGen}(sk, \mathbf{I}_{k+2}) \to \textsf{dk}_{\mathbf{I}} $, where $\mathbf{I}_{k+2}$ is an identity matrix of size $(k+2) \times (k+2)$. For keys used for \textsf{Domain} $\ell$, it outputs $\textsf{esk} \gets (\ell, \textsf{ek}_1)$, $\textsf{psk} \gets (\ell, \textsf{ek}_2)$, and $\textsf{csk} \gets \textsf{dk}_{\mathbf{I}}$.

	Note that when the previous 2i-IPFE-based scheme in Section \ref{sec:2i-IPFE-instantiation} is applied to a Device-of-Domain model, we assume that $\textsf{Setup}$ is run once for each domain to generate different $\textsf{esk}, \textsf{psk}, \textsf{csk}$. In the scheme in this section, however, $\textsf{Setup}$ is run only once for all the domains, and each domain shares the same $\textsf{csk}$ and the same $\textsf{esk}, \textsf{psk}$ except different labels.

	\item $\textsf{encodeEnroll}^{\mathcal{O}_{\mathcal{B}}}(), \textsf{encodeProbe}^{\mathcal{O}_{\mathcal{B}}}()$: The same as the scheme in \ref{sec:2i-IPFE-instantiation}. 

	\item $\textsf{Enroll}(\textsf{esk}, \mathbf{x})$: It calls $\textsf{FE.Enc}(\ell, \textsf{ek}_1, \mathbf{x}) \to \mathbf{c_x}$ and outputs $\mathbf{c_x}$.

	\item $\textsf{Probe}(\textsf{psk}, \mathbf{y})$: It calls $\textsf{FE.Enc}(\ell, \textsf{ek}_2, \mathbf{y}) \to \mathbf{c_y}$ and outputs $\mathbf{c_y}$.

	\item $\textsf{Compare}(\textsf{csk}, \mathbf{c_x}, \mathbf{c_y)}$: It calls $\textsf{FE.Dec}(\textsf{dk}_{\mathbf{I}}, \mathbf{c_x}, \mathbf{c_y}) \to s$ and outputs the value $s$.

	\item $\textsf{Verify}(s)$: If $\sqrt{s} \leq \tau$, a pre-defined threshold for comparing the closeness of two templates, then it outputs $r = 1$; otherwise, it outputs $r = 0$.

	\item $\textsf{BioCompare}$

\end{itemize}

By the correctness of the functional encryption scheme $\sf FE$, if the labels of $\mathbf{c_x}$ and $\mathbf{c_y}$ are the same (they are of the same domain), we have
\[
	s = \textsf{FE.Dec}(\textsf{dk}_{\mathbf{I}}, \mathbf{c_x}, \mathbf{c_y}) =  \mathbf{x} \mathbf{I}_{k+2} \mathbf{y}^T = \| \mathbf{b} - \mathbf{b}^\prime \|^2.
\]
just as the scheme in Section \ref{sec:2i-IPFE-instantiation}

When the Device-of-Domain model is instantiated with a 2c-IPFE scheme in this way, the enrollment secret key $\textsf{esk}$ and probe secret key $\textsf{psk}$ are now shared among all the devices, regardless of their domains. Therefore, to let a malicious or broken \textsf{Domain} not threaten other honest ones, one needs to make sure given $\textsf{esk}$ or $\textsf{psk}$, $\mathbf{c_x}$ still does not leak information about $\mathbf{x}$. This is different from the scheme in Section \ref{sec:2i-IPFE-instantiation}, where we only need seurity against an adversary who has a probe oracle $\textsf{Probe}(\textsf{psk}, \cdot)$.

If \textsf{Server} and \textsf{Domain} are both malicious, then the adversary can use $\textsf{csk}$ to distinguish $\mathbf{c_x}$ and even recover $\mathbf{x}$. Therefore, we assume at most one party of them can be malicious at the same time. Note that this is the same as the 2i-IPFE-based scheme, where only one of \textsf{Server} and \textsf{Domain} can be malicious.

\fi

%-------------------

