%%%%%%%%%%%%%%%%%%%%

% Semester Project Fall 2024 for EPFL
% Author: Keng-Yu Chen

%%%%%%%%%%%%%%%%%%%%

%%%%%%%%%%%%%%%%%%%%

% Semester Project for EPFL
% Author: Keng-Yu Chen

%%%%%%%%%%%%%%%%%%%%

\documentclass[a4paper, 12pt]{article}
\usepackage[utf8]{inputenc}
\usepackage[margin=3cm]{geometry}
\setlength{\headheight}{14.49998pt}

% Include Packages
\usepackage{mathrsfs}
\usepackage{color}
\usepackage{amsthm,amsmath,amssymb}
\usepackage{bm}
\usepackage{stmaryrd}
\usepackage{hyperref}
\usepackage{graphicx}
\usepackage{multicol}
\usepackage{wrapfig}
\usepackage{subcaption}
\hypersetup{colorlinks=true,linkcolor=blue,citecolor=blue,filecolor=magenta,urlcolor=blue}

\usepackage{fancyhdr}
\usepackage{thmtools}

\usepackage{algorithm}
\usepackage{algpseudocode}
\renewcommand{\algorithmicrequire}{\textbf{Input:}}
\renewcommand{\algorithmicensure}{\textbf{Output:}}



% Set Self-defined Words
\DeclareMathOperator*{\argmax}{arg\!\max}
\DeclareMathOperator*{\argmin}{arg\!\min}

\newcommand{\negl}{\mathsf{negl}}
\newcommand{\poly}{\mathsf{poly}}
\newcommand{\Adv}{\mathbf{Adv}}

\newcommand{\R}{\mathbb{R}}
\newcommand{\N}{\mathbb{N}}
\newcommand{\Z}{\mathbb{Z}}
\newcommand{\Q}{\mathbb{Q}}

\newcommand\getsdollar{\mathrel{{\leftarrow}\vcenter{\hbox{\tiny\rmfamily\upshape\$}}}}


% Convert block[name] into a theorem-like format
\newtheorem{theorem}{Theorem}

\theoremstyle{definition}
\newtheorem{definition}{Definition}

\theoremstyle{definition}
\newtheorem{assumption}{Assumption}

\theoremstyle{plain}
\newtheorem{corollary}{Corollary}



% Add Referemces
\usepackage[
	backend=biber,
	style=alphabetic,
	sorting=ynt
]{biblatex}
\addbibresource{reference.bib}


\title{\textbf{Biometrics Authentication: Formalization and Instantiation}}
\author{Keng-Yu Chen}

\date{\today}



\begin{document}

%% Title
\maketitle

%-------------------

%% Header and Foot
\pagestyle{fancy}
\fancyhf{}
\fancyhead[L]{Semester Project}
\fancyhead[C]{Biometric Authentication}
\fancyhead[R]{Keng-Yu Chen}
\fancyfoot[C]{\thepage}

%-------------------

%This project formalizes the biometric authentication scheme, including its structure, usage, and security analysis with a security game model.

%-------------------


\begin{abstract}
\label{sec:abstract}

In this project, we focus on the cryptographic layer which is added on the top of biometric authentication for privacy reasons. We first formalize a biometric authentication scheme and propose security models for two security properties of interest: \emph{unforgeability} and \emph{indistinguishability}. Unforgeability refers to an adversary's ability to impersonate a user, while indistinguishability evaluates the server's knowledge of users' biometrics, related to privacy preservation. Subsequently, we analyze two existing instantiations of biometric authentication built on two cryptographic primitives: function-hiding inner product functional encryption and relational hash. Our results demonstrate conditions under which these schemes achieve security within out security model, and we propose a simple way to strengthen the system based on functional encryption by adding a digital signature in the cryptographic layer.

\end{abstract}


\section{Introduction}

Biometric authentication offers an error-tolerant approach to user verification. Despite its convenience, unlike traditional authentication methods, servers have to verify users' identities by comparing the similarity of enrolled and probed data instead of their equivalence. An authentication method based on comparing hashes of two templates thus fails. Additionally, unlike a user-defined password, biometrics reveal sensitive personal information and cannot be changed, raising significant privacy concerns. Furthermore, the inherent nature of biometrics data can introduce a non-negligible false positive rate. These issues make designing a biometric authentication scheme and analyzing its security challenging and highlight the importance of a rigorous study in this domain.

[More to say about previous works and a summary of this project ...]


\section{Preliminaries}
\label{sec:preliminaries}

In this project, we assume

\begin{itemize}
	
	\item $\lambda$ is the security parameter.

	\item $[m]$ denotes the set of integers $\{1, 2, \cdots, m\}$.

	\item $\Z_q$ is the finite field modulo a prime number $q$.

	\item A function $f(n)$ is called \emph{negligible} iff for any integer $c$, $f(n) < \frac{1}{n^c}$ for all sufficiently large $n$. We write it as $f(n) = \negl$, and we may also use $\negl$ to represent an arbitrary negligible function.
	
	\item $\poly$ is the class of polynomial funcions. We may also use $\poly$ to represent an arbitrary polynomial function.
	
	\item We write sampling a value $r$ from a distribution $\mathcal{D}$ as $r \getsdollar \mathcal{D}$. If $S$ is a finite set, then $r \getsdollar S$ means sampling $r$ uniformly from $S$.

	\item The distribution $\mathcal{D}^t$ denotes $t$ identical and independent distributions of $\mathcal{D}$.

	\item A PPT algorithm denotes a probabilistic polynomial time algorithm. Unless otherwise specified, all algorithms run in PPT.

\end{itemize}

%We introduce three types of inner product functional encryption schemes: function hiding functional encryption, two-input functional encryption, and two-client functional encryption. We will instantiate our biometric authentication scheme using these primitives.

We introduce two primitives to instantiate a biometric authentication scheme: function-hiding inner product functional encryption and relatioanl hash.

\begin{definition}[Function-Hiding Inner Product Functional Encryption (adapted from \cite{cryptoeprint:2016/440}) ]
\label{def:fh-IPFE}
	A \emph{function-hiding inner product functional encryption} (fh-IPFE) scheme $\sf FE$ for a field $\mathbb{F}$ and input length $k$ is composed of PPT algorithms $\textsf{FE.Setup}$, $\textsf{FE.KeyGen}$, $\textsf{FE.Enc}$, and $\textsf{FE.Dec}$:

	\begin{itemize}
	
		\item $\textsf{FE.Setup}(1^\lambda) \to \textsf{msk}, \textsf{pp}$: It outputs the public parameter $\textsf{pp}$ and the master secret key $\textsf{msk}$.
	
		\item $\textsf{FE.KeyGen}(\textsf{msk}, \textsf{pp}, \mathbf{x}) \to f_\mathbf{x}$: It generates the functional decryption key $f_\mathbf{x}$ for an input vector $\mathbf{x} \in \mathbb{F}^k$. 
	
		\item $\textsf{FE.Enc}(\textsf{msk}, \textsf{pp}, \mathbf{y}) \to \mathbf{c_y}$: It encrypts the input vector $\mathbf{y} \in \mathbb{F}^k$ to the ciphertext $\mathbf{c_y}$. 
	
		\item $\textsf{FE.Dec}(\textsf{pp}, f_\mathbf{x}, \mathbf{c_y}) \to z$: It outputs a value $z \in \mathbb{F}$ or an error symbol $\bot$.
	
	\end{itemize}
	
	\noindent \textbf{Correctness}: An fh-IPFE scheme \textsf{FE} is \emph{correct} if $\forall (\textsf{msk}, \textsf{pp}) \gets \textsf{FE.Setup}(1^\lambda)$ and $ \mathbf{x}, \mathbf{y} \in \mathbb{F}^k$, we have
	\[
		\textsf{FE.Dec}( \textsf{pp}, \textsf{FE.KeyGen}(\textsf{msk}, \textsf{pp}, \mathbf{x}), \textsf{FE.Enc}(\textsf{msk}, \textsf{pp}, \mathbf{y}) ) = \mathbf{x} \mathbf{y}^T \in \mathbb{F}.
	\]

\end{definition}

Instantiation using an fh-IPFE scheme is given in Section \ref{sec:fh-IPFE-instantiation}.

%\begin{definition}[Two-Input Inner Product Functional Encryption (adapted from \cite{cryptoeprint:2022/441})]
%\label{def:2i-IPFE}
	%A \emph{two-input inner product functional encryption} (2i-IPFE) scheme $\sf FE$ for a field $\mathbb{F}$ and input length $k$ is composed of PPT algorithms $\textsf{FE.Setup}$, $\textsf{FE.KeyGen}$, $\textsf{FE.Enc}$, and $\textsf{FE.Dec}$:

	%\begin{itemize}
	
		%\item $\textsf{FE.Setup}(1^\lambda) \to \textsf{sk}, \textsf{ek}_1, \textsf{ek}_2$: It outputs a secret key $\textsf{sk}$ and two encryption keys $\textsf{ek}_1, \textsf{ek}_2$.
	
		%\item $\textsf{FE.KeyGen}(\textsf{sk}, \mathbf{A}) \to \textsf{dk}_\mathbf{A}$: It generates the functional decryption key $\textsf{dk}_\mathbf{A}$ for a diagonal matrix $\mathbf{A} \in \mathbb{F}^{k \times k}$,  
	
		%\item $\textsf{FE.Enc}(\textsf{ek}_i, \mathbf{x}) \to \mathbf{c_x}$: Given an encryption key, either $\textsf{ek}_1$ or $\textsf{ek}_2$, it encrypts the input vector $\mathbf{x} \in \mathbb{F}^k$ to the ciphertext $\mathbf{c_x}$. 
	
		%\item $\textsf{FE.Dec}(\textsf{dk}_\mathbf{A}, \mathbf{c_x}, \mathbf{c_y}) \to z$: It outputs a value $z \in \mathbb{F}$.
	
	%\end{itemize}
	
	%\noindent \textbf{Correctness}: A 2i-IPFE scheme \textsf{FE} is \emph{correct} if $\forall (\textsf{sk}, \textsf{ek}_1, \textsf{ek}_2) \gets \textsf{FE.Setup}(1^\lambda), \mathbf{A} \in \mathbb{F}^{k \times k}$, and $\mathbf{x}, \mathbf{y} \in \mathbb{F}^k$, we have

	%\[
		%\textsf{FE.Dec}(\textsf{FE.KeyGen}(\textsf{sk},  \mathbf{A}), \textsf{FE.Enc}(\textsf{ek}_1, \mathbf{x}), \textsf{FE.Enc}(\textsf{ek}_2, \mathbf{y}) ) = \mathbf{x} \mathbf{A} \mathbf{y}^T \in \mathbb{F}.
	%\]

%\end{definition}

%Instantiation using a 2i-IPFE is given in Section \ref{sec:2i-IPFE-instantiation}.

%\begin{definition}[Two-Client Inner Product Functional Encryption (adapted from \cite{cryptoeprint:2022/441})]
%\label{def:2c-IPFE}
	%A \emph{two-client inner product functional encryption} (2c-IPFE) scheme $\sf FE$ for a field $\mathbb{F}$ and input length $k$ is composed of PPT algorithms $\textsf{FE.Setup}$, $\textsf{FE.KeyGen}$, $\textsf{FE.Enc}$, and $\textsf{FE.Dec}$:

	%\begin{itemize}
	
		%\item $\textsf{FE.Setup}(1^\lambda) \to \textsf{sk}, \textsf{ek}_1, \textsf{ek}_2$: It outputs a secret key $\textsf{sk}$ and two encryption keys $\textsf{ek}_1, \textsf{ek}_2$.
	
		%\item $\textsf{FE.KeyGen}(\textsf{sk}, \mathbf{A}) \to \textsf{dk}_\mathbf{A}$: It generates the functional decryption key $\textsf{dk}_\mathbf{A}$ for a diagonal matrix $\mathbf{A} \in \mathbb{F}^{k \times k}$,  
	
		%\item $\textsf{FE.Enc}(\ell, \textsf{ek}_i, \mathbf{x}) \to \mathbf{c_x}$: Given a label $\ell$ and an encryption key, either $\textsf{ek}_1$ or $\textsf{ek}_2$, it encrypts the input vector $\mathbf{x} \in \mathbb{F}^k$ to the ciphertext $\mathbf{c_x}$. 
	
		%\item $\textsf{FE.Dec}(\textsf{dk}_\mathbf{A}, \mathbf{c_x}, \mathbf{c_y}) \to z$: It outputs a value $z \in \mathbb{F}$.
	
	%\end{itemize}
	
	%\noindent \textbf{Correctness}: A 2c-IPFE scheme \textsf{FE} is \emph{correct} if $\forall (\textsf{sk}, \textsf{ek}_1, \textsf{ek}_2) \gets \textsf{FE.Setup}(1^\lambda), \mathbf{A} \in \mathbb{F}^{k \times k}$, label $\ell$, and $ \mathbf{x}, \mathbf{y} \in \mathbb{F}^k$, we have
	%\[
		%\textsf{FE.Dec}(\textsf{FE.KeyGen}(\textsf{sk},  \mathbf{A}), \textsf{FE.Enc}(\ell, \textsf{ek}_1, \mathbf{x}), \textsf{FE.Enc}(\ell, \textsf{ek}_2, \mathbf{y}) ) = \mathbf{x} \mathbf{A} \mathbf{y}^T \in \mathbb{F}.
	%\]

%\end{definition}

%Instantiation using a 2c-IPFE is given in Section \ref{sec:2c-IPFE-instantiation}.


\begin{definition}[Relational Hash (adapted from \cite{cryptoeprint:2014/394})]
\label{def:rh}
	Let $R_\lambda$ be a relation over sets $X_\lambda, Y_\lambda$, and $ Z_\lambda$. A \emph{relational hash} scheme \textsf{RH} for $R_\lambda$ consists of PPT algorithms \textsf{RH.KeyGen}, $\textsf{RH.HASH}_1$, $\textsf{RH.HASH}_2$, and \textsf{RH.Verify}:
	
	\begin{itemize}
	
		\item $\textsf{RH.KeyGen}(1^\lambda) \to \textsf{pk}$: It outputs a public hash key \textsf{pk}.  
			
		\item $\textsf{RH.Hash}_1(\textsf{pk}, \mathbf{x}) \to \mathbf{h_x}$: Given a hash key \textsf{pk} and $\mathbf{x} \in X_\lambda$, it outputs a hash $\mathbf{h_x}$.

		\item $\textsf{RH.Hash}_2(\textsf{pk}, \mathbf{y}) \to \mathbf{h_y}$: Given a hash key \textsf{pk} and $\mathbf{y} \in Y_\lambda$, it outputs a hash $\mathbf{h_y}$.

		\item $\textsf{RH.Verify}(\textsf{pk}, \mathbf{h_x}, \mathbf{h_y}, \mathbf{z}) \to r \in \{0, 1\}$: Given a hash key \textsf{pk}, two hashes $\mathbf{h_x}$ and $\mathbf{h_y}$, and $\mathbf{z} \in Z_\lambda$, it verifies whether the relation among $\mathbf{x}, \mathbf{y}$ and $\mathbf{z}$ holds.

	\end{itemize}

	\noindent \textbf{Correctness}: A relational hash scheme \textsf{RH} is \emph{correct} if $\forall \mathbf{x}, \mathbf{y}, \mathbf{z} \in X_\lambda \times Y_\lambda \times Z_\lambda$,
	\[
		\Pr \left [
			\begin{aligned} 
				 &\; \textsf{pk} \gets \textsf{RH.KeyGen}(1^\lambda) \\
				 &\; \mathbf{h_x} \gets \textsf{RH.Hash}_1(\textsf{pk}, \mathbf{x}) \\
				 &\; \mathbf{h_y} \gets \textsf{RH.Hash}_2(\textsf{pk}, \mathbf{y})
			\end{aligned} :
			\textsf{RH.Verify}(\textsf{pk}, \mathbf{h_x}, \mathbf{h_y}, \mathbf{z}) = R(\mathbf{x}, \mathbf{y}, \mathbf{z})
			\right ] = 1 - \negl.
	\]
\end{definition}

\noindent Note that $Z_\lambda$ is an auxiliary input. When the relation $R$ is over two sets $X \times Y$, we ignore $Z$ and write $\textsf{RH.Verify}(\textsf{pk}, \mathbf{h_x}, \mathbf{h_y})$.

Instantiation using a relational hash is given in Section \ref{sec:rh-instantiation}.


%-------------------


\section{Formalization}
\label{sec:formalization}

\subsection{Biometric Authentication Scheme}

In this section, we formally define a biometric authentication scheme. For this, we first define how we simulate biometric distributions of users.

Assume the existence of a family $\mathbb{B}$ of biometric distributions that are efficiently samplable. We have the following interfaces for all algorithms to interact with $\mathbb{B}$.

\begin{itemize}

	\item $\textsf{BioSamp}()$: Generate a random distribution $\mathcal{B}$ of $\mathbb{B}$. By this we mean providing either parameters of an efficiently samplable distribution or a PPT algorithm as the sampler. For simplicity, we write $\mathcal{B} \gets \textsf{BioSamp}()$ as $\mathcal{B} \getsdollar \mathbb{B}$.
	
	\item $\textsf{BioDelete}(\mathcal{B})$: Delete $\mathcal{B}$ from $\mathbb{B}$. Consequently, no further access to $\textsf{BioSamp}$ can derive $\mathcal{B}$. For simplicity, we write $\textsf{BioDelete}(\mathcal{B})$ as $\mathbb{B} \gets \mathbb{B} \setminus \mathcal{B}$.

	\item $\textsf{TempSamp}(\mathcal{B})$: Let $\mathcal{B}$ be a biometric distribution in $\mathbb{B}$. This algorithm samples a biometric template from $\mathcal{B}$. For simplicity, we write $\mathbf{b} \gets \textsf{TempSamp}(\mathcal{B})$ as $\mathbf{b} \getsdollar \mathcal{B}$.

\end{itemize}

\begin{definition}[Biometric Authentication Scheme]

A \emph{biomtric authentication shceme} $\Pi$ associated with a family $\mathbb{B}$ of biometric distributions is composed of the following algorithms.

\begin{itemize}

	\item $\textsf{getEnroll}^{\mathcal{O}_{\mathcal{B}}}() \to \mathbf{b}$: Given an oracle $\mathcal{O}_{\mathcal{B}}$, which samples biometric data from a distribution $\mathcal{B} \in \mathbb{B}$, it outputs a biometric template $\mathbf{b}$ for enrollment.

	\item $\textsf{getProbe}^{\mathcal{O}_{\mathcal{B}}}() \to \mathbf{b}^\prime$: Given an oracle $\mathcal{O}_{\mathcal{B}}$, which samples biometric data from a distribution $\mathcal{B} \in \mathbb{B}$, it outputs a biometric template $\mathbf{b}^\prime$ for probe.

	\item $\textsf{BioCompare}(\mathbf{b}, \mathbf{b}^\prime) \to s$: Given two biometric templates $\mathbf{b}$ and $\mathbf{b}^\prime$, it outputs a score $s$. 

	\item $\textsf{Verify}(s) \to r \in \{0,1\}$: It is a deterministic algorithm that reads the comparison score $s$ and determines whether this is a successful authentication ($r = 1$) or not ($r = 0$).

\end{itemize}

\end{definition}

Given an authentication scheme $\Pi$, we can consider its true positive rate and false positive rate.

\begin{definition}[True Positive Rate]
For a biometric distribution $\mathcal{B} \in \mathbb{B}$ and $\mathbf{b} \gets \textsf{getEnroll}^{\mathcal{O}_\mathcal{B}}()$, define the \emph{true positive rate} \textsf{TP}.

\begin{align*}
	\textsf{TP}(\mathcal{B}, \mathbf{b}) 
	:=&\; \Pr[ \mathbf{b}^\prime \gets \textsf{getProbe}^{\mathcal{O}_\mathcal{B}}(): \textsf{Verify}(\textsf{BioCompare}(\mathbf{b}, \mathbf{b}^\prime )) = 1 ] \\
	\textsf{TP}(\mathcal{B}) 
	:=&\; \Pr \left[ 
		\begin{aligned}	
			& \mathbf{b} \gets \textsf{getEnroll}^{\mathcal{O}_\mathcal{B}}() \\
			& \mathbf{b}^\prime \gets \textsf{getProbe}^{\mathcal{O}_\mathcal{B}}()
		\end{aligned}
		: \textsf{Verify}(\textsf{BioCompare}(\mathbf{b}, \mathbf{b}^\prime)) = 1 \right] \\
	=&\; \mathbb{E}_{ \mathbf{b} \gets \textsf{getEnroll}^{\mathcal{O}_\mathcal{B}}() }[\textsf{TP}(\mathcal{B}, \mathbf{b})] \\
	\textsf{TP} 
	:=&\; \Pr \left[
		\begin{aligned}
			& \mathcal{B} \getsdollar \mathbb{B} \\
			& \mathbf{b} \gets \textsf{getEnroll}^{\mathcal{O}_\mathcal{B} }() \\
			& \mathbf{b}^\prime \gets \textsf{getProbe}^{\mathcal{O}_\mathcal{B} }()
		\end{aligned}
		: \textsf{Verify}(\textsf{BioCompare}(\mathbf{b}, \mathbf{b}^\prime )) = 1 \right] \\
	=&\; \mathbb{E}_{\mathcal{B} \getsdollar \mathbb{B}}[\textsf{TP}(\mathcal{B})] 
\end{align*}

\end{definition}


\begin{definition}[False Positive Rate]
For a biometric distribution $\mathcal{B} \in \mathbb{B}, \mathbb{B} \gets \mathbb{B} \setminus \mathcal{B}$ and $\mathbf{b} \gets \textsf{getEnroll}^{\mathcal{O}_\mathcal{B}}()$, define the \emph{false positive rate} \textsf{FP}.

\begin{align*}
	\textsf{FP}(\mathbf{b}) 
	:=&\; \Pr \left[ 
		\begin{aligned}
			& \mathcal{B}^\prime \getsdollar \mathbb{B} \\
			& \mathbf{b}^\prime \gets \textsf{getProbe}^{\mathcal{O}_{\mathcal{B}^\prime}}()
		\end{aligned}
		: \textsf{Verify}(\textsf{BioCompare}(\mathbf{b}, \mathbf{b}^\prime)) = 1 \right] \\
	\textsf{FP}(\mathcal{B}) 
	:=&\; \Pr \left[ 
		\begin{aligned}
			& \mathcal{B}^\prime \getsdollar \mathbb{B} \\
			& \mathbf{b} \gets \textsf{getEnroll}^{\mathcal{O}_\mathcal{B} }() \\
			& \mathbf{b}^\prime \gets \textsf{getProbe}^{\mathcal{O}_{\mathcal{B}^\prime} }()
		\end{aligned}
		: \textsf{Verify}(\textsf{BioCompare}(\mathbf{b}, \mathbf{b}^\prime)) = 1 \right] \\
	=&\; \mathbb{E}_{ \mathbf{b} \gets \textsf{getEnroll}^{\mathcal{O}_\mathcal{B}}() }[\textsf{FP}(\mathbf{b})] \\
	\textsf{FP} 
	:=&\; \Pr \left[
		\begin{aligned}
			& \mathcal{B} \getsdollar \mathbb{B}, \mathbb{B} \gets \mathbb{B} \setminus \mathcal{B}, \mathcal{B}^\prime \getsdollar \mathbb{B} \\
			& \mathbf{b} \gets \textsf{getEnroll}^{\mathcal{O}_\mathcal{B} }() \\
			& \mathbf{b} \gets \textsf{getProbe}^{\mathcal{O}_{\mathcal{B}^\prime} }()
		\end{aligned}
		: \textsf{Verify}(\textsf{BioCompare}(\mathbf{b}, \mathbf{b}^\prime)) = 1 \right] \\
	=&\; \mathbb{E}_{\mathcal{B} \getsdollar \mathbb{B}}[\textsf{FP}(\mathcal{B})]
\end{align*}

\end{definition}

Ideally, we hope $\textsf{TP}$ to be $1$, and $\textsf{FP}(\mathcal{B})$ to be negligible for any $\mathcal{B} \in \mathbb{B}$. However, due to the inherent nature of biometrics, there might be a nonzero false negative rate $ 1 - \textsf{TP} > 0$ and a non-negligible $\textsf{FP}(\mathcal{B})$. Our security model and analysis also take these possibilities into consideration.


%-------------------


\subsection{Cryptographic Layer}

In this work, we add a cryptographic layer on top of $\Pi$ to protect privacy of users. The cryptographic layer includes the following algorithms.

\begin{itemize}

	\item $\textsf{Setup}(1^\lambda) \to \textsf{esk}, \textsf{psk}, \textsf{csk}$: It outputs the enrollment secret key $\textsf{esk}$, probe secret key $\textsf{psk}$, and compare secret key $\sf csk$.

	\item $\textsf{Enroll}(\textsf{esk}, \mathbf{b}) \to \mathbf{c_x}$: On input a biometric template $\mathbf{b}$, it encodes it into a vector $\mathbf{x}$ and outputs the enrollment message $\mathbf{c_x}$.
	
	\item $\textsf{Probe}(\textsf{psk}, \mathbf{b}^\prime) \to \mathbf{c_y}$: On input a biometric template $\mathbf{b}^\prime$, it encodes it into a vector $\mathbf{y}$ and outputs the probe message $\mathbf{c_y}$ .

	\item $\textsf{Compare}(\textsf{csk}, \mathbf{c_x}, \mathbf{c_y)} \to s$: It compares the enrollment message $\mathbf{c_x}$ and probe message $\mathbf{c_y}$ and outputs a score $s$.

\end{itemize}

\noindent \textbf{Correctness}: An authenticaion scheme $\Pi$ is \emph{correct} if for any biometric distributions $\mathcal{B}$ and $\mathcal{B}^\prime$, let $\textsf{esk}, \textsf{psk}, \textsf{csk} \gets \textsf{Setup}(1^\lambda)$, $\mathbf{b} \gets \textsf{getEnroll}^{\mathcal{O}_\mathcal{B}}()$, $\mathbf{b}^\prime \gets \textsf{getProbe}^{\mathcal{O}_{\mathcal{B}^\prime}}()$, $\mathbf{c_x} \gets \textsf{Enroll}(\textsf{esk}, \mathbf{b})$, $\mathbf{c_y} \gets \textsf{Probe}(\textsf{psk}, \mathbf{b}^\prime)$. Then
	\[
		\Pr \left [
			\textsf{Compare}(\textsf{csk}, \mathbf{c_x}, \mathbf{c_y}) = \textsf{BioCompare}(\mathbf{b}, \mathbf{b}^\prime)
		\right ] = 1 - \negl.
	\]

In a real-world biometric system, these algorithms may be run by different parties such as a biometric scanner, a user's secure hardware, a trusted authority that issues keys, and the server.

Now, we provide two instantiations of a biometric authentication scheme with the cryptographic layer.

%-------------------

\subsubsection{Instantiation with an fh-IPFE Scheme}
\label{sec:fh-IPFE-instantiation}

Let $\textsf{FE} = (\textsf{FE.Setup}, \textsf{FE.KeyGen}, \textsf{FE.Enc}, \textsf{FE.Dec})$ be an fh-IPFE scheme we defined in Definition \ref{def:fh-IPFE}. Following \cite{cryptoeprint:2023/481}, we can instantiate a biometric authentication scheme using $\textsf{FE}$ with the distance metric the Euclidean distance.
Let $\textsf{getEnroll}^{\mathcal{O}_{\mathcal{B}}}()$ and $\textsf{getProbe}^{\mathcal{O}_{\mathcal{B}}}()$ both output vectors in $ \{0, 1, \cdots, m \}^k$ for all biometric distributions $\mathcal{B} \in \mathbb{B}$. 
For a pre-defined real number $\tau \geq 0$, define
\[
	\textsf{BioCompare}(\mathbf{b}, \mathbf{b}^\prime) \to \| \mathbf{b} - \mathbf{b}^\prime\|^2 \quad \text{and} \quad 
	\textsf{Verify}(s) \to 
	\begin{cases} 
		1 & \text{if } \sqrt{s} \leq \tau \\
		0 & \text{if } \sqrt{s} > \tau
	\end{cases}.
\]

Now, let the associated field of $\textsf{FE}$ be $\mathbb{Z}_q$, where $q$ is a prime number larger than the maximum possible Euclidean distance $m^2 \cdot k$. The scheme is instantiated as follows.

\begin{itemize}

	\item $\textsf{Setup}(1^\lambda)$: It calls $\textsf{FE.Setup}(1^\lambda) \to \textsf{msk}, \textsf{pp}$ and outputs $\textsf{esk} \gets (\textsf{msk}, \textsf{pp})$, $\textsf{psk} \gets (\textsf{msk}, \textsf{pp})$ and $\textsf{csk} \gets \textsf{pp}$.

	\item $\textsf{Enroll}(\textsf{esk}, \mathbf{b})$: On input a template vector $\mathbf{b} = (b_1, b_2, \cdots, b_k)$, the algorithm first encodes it as $\mathbf{x} = (x_1, x_2, \cdots, x_{k+2}) = (b_1, b_2, \cdots, b_k, 1, \|\mathbf{b}\|^2)$. Next, it calls $\textsf{FE.KeyGen}(\textsf{msk}, \textsf{pp}, \mathbf{x}) \to f_\mathbf{x}$ and outputs $\mathbf{c_x} \gets f_\mathbf{x}$.

	\item $\textsf{Probe}(\textsf{psk}, \mathbf{b}^\prime)$: On input a template vector $\mathbf{b}^\prime = (b_1^\prime, b_2^\prime, \cdots, b_k^\prime)$, the algorithm first encodes it as $\mathbf{y} = (y_1, y_2, \cdots, y_{k+2}) = (-2b_1^\prime, -2b_2^\prime, \cdots, -2b_k^\prime, \|\mathbf{b}^\prime\|^2, 1)$. Next, it calls $\textsf{FE.Enc}(\textsf{msk}, \textsf{pp}, \mathbf{y}) \to \mathbf{c_y}$ and outputs $\mathbf{c_y}$.

	\item $\textsf{Compare}(\textsf{csk}, \mathbf{c_x}, \mathbf{c_y)}$: It calls $\textsf{FE.Dec}(\textsf{pp}, \mathbf{c_x}, \mathbf{c_y}) \to s$ and outputs the value $s$.

\end{itemize}

By the correctness of the functional encryption scheme $\sf FE$, we have
\[
	s = \textsf{FE.Dec}(\textsf{pp}, \mathbf{c_x}, \mathbf{c_y}) =  \mathbf{x} \mathbf{y}^T = \sum_{i=1}^k -2b_ib_i^\prime + \|\mathbf{b}\|^2 + \|\mathbf{b}^\prime\|^2 = \| \mathbf{b} - \mathbf{b}^\prime \|^2.
\]
which is equal to $\textsf{BioCompare}(\mathbf{b}, \mathbf{b}^\prime)$. Therefore, if two templates $\mathbf{b}$ and $\mathbf{b}^\prime$ are close enough such that $\|\mathbf{b} - \mathbf{b}^\prime\| \leq \tau$, the scheme results in $r = 1$, a successful authentication.

Instantiated with an fh-IPFE scheme in this way, the comparison secret key $\textsf{csk}$ is public, and the enrollment secret key $\textsf{esk}$ and probe secret key $\textsf{psk}$ are the same. Anyone with access to the enrollment message $\mathbf{c_x}$ and either $\textsf{esk}$ or $\textsf{psk}$ can probe any (invalidly encoded) $\mathbf{y}^{\prime} \in \Z_q^{k+2}$ and find $\mathbf{x} {\mathbf{y}^\prime}^T$ to get partial or full information about the biometric template $\mathbf{b}$. Even if the adversary has no $\textsf{esk}$ or $\textsf{psk}$, if it can sample ciphertexts $\mathbf{c_{y}}$ corresponding to some unknown random vectors $\mathbf{y}$, and if the field size $q$ is not large enough, it can also find a forged $\mathbf{c_{y^*}}$ such that $\mathbf{x}\mathbf{y^*}^T \leq \tau$ with a non-negligible probability to impersonate the user by sampling many times offline.

A security analysis of this instantiation in our security model is given in Section \ref{sec:security_analysis:fh-IPFE}.

%-------------------

\subsubsection{Instantiation with a Relational Hash Scheme}
\label{sec:rh-instantiation}

Let $\textsf{RH} = (\textsf{RH.KeyGen}, \textsf{RH.Hash}_1, \textsf{RH.Hash}_2, \textsf{RH.Verify})$ be a relational hash scheme we defined in Definition \ref{def:rh} for the relation $R^\tau$ of Hamming distance proximity parametrized by a constant $\tau$.
\[
	R^\tau = \{ (\mathbf{x}, \mathbf{y}) \mid \textsf{HD}(\mathbf{x}, \mathbf{y}) \leq \tau \wedge \mathbf{x}, \mathbf{y} \in \{0,1\}^k \}
\]
Note that here we ignore the third parameter $Z$.
Let $\textsf{getEnroll}^{\mathcal{O}_{\mathcal{B}}}()$ and $\textsf{getProbe}^{\mathcal{O}_{\mathcal{B}}}()$ both output vectors in $ \{0, 1\}^k$ for all biometric distributions $\mathcal{B} \in \mathbb{B}$, and let 
\[
	\textsf{BioCompare}(\mathbf{b}, \mathbf{b}^\prime) \to
	\begin{cases}
		1 & \text{if } (\mathbf{b}, \mathbf{b}^\prime) \in R^\tau \\
		0 & \text{if } (\mathbf{b}, \mathbf{b}^\prime) \notin R^\tau
	\end{cases} \quad \text{and} \quad
	\textsf{Verify}(s) \to s.
\]
Following \cite{cryptoeprint:2014/394}, we can instantiate a biometric authentication scheme using $\textsf{RH}$.  Let the biometric distribution $\mathcal{B} \subseteq \{0,1\}^k$.

\begin{itemize}

	\item $\textsf{Setup}(1^\lambda)$: It calls $\textsf{RH.KeyGen}(1^\lambda) \to \textsf{pk}$ and outputs $\textsf{esk} \gets \textsf{pk}$, $\textsf{psk} \gets \textsf{pk}$, and $\textsf{csk} \gets \textsf{pk}$.

	\item $\textsf{Enroll}(\textsf{esk}, \mathbf{b})$: Let $\mathbf{x} \gets \mathbf{b}$. It calls $\textsf{RH.Hash}_1(\textsf{pk}, \mathbf{x}) \to \mathbf{h_x}$ and outputs $\mathbf{c_x} \gets \mathbf{h_x}$.

	\item $\textsf{Probe}(\textsf{psk}, \mathbf{b}^\prime)$: Let $\mathbf{y} \gets \mathbf{b}$. It calls $\textsf{RH.Hash}_2(\textsf{pk}, \mathbf{y}) \to \mathbf{h_y}$ and outputs $\mathbf{c_y} \gets \mathbf{h_y}$.

	\item $\textsf{Compare}(\textsf{csk}, \mathbf{c_x}, \mathbf{c_y)}$: It calls $\textsf{RH.Verify}(\textsf{pk}, \mathbf{h_x}, \mathbf{h_y}) \to s$ and outputs the value $s$.

\end{itemize}

By the correctness of the relational hash scheme $\textsf{RH}$, we have (except for a negligible probability),
\[
	r = 1 \Leftrightarrow (\mathbf{x}, \mathbf{y}) = (\mathbf{b}, \mathbf{b}^\prime) \in R^\tau \Leftrightarrow \textsf{HD}(\mathbf{b}, \mathbf{b}^\prime) \leq \tau
\]

A security analysis of this instantiation in our security model is given in Section \ref{sec:security_analysis:rh}.

%-------------------


\iffalse

We discuss two usage models that employs the authentication scheme $\Pi$.


\subsection{Usage Model – Device-of-User}
\label{sec:dou_model}

In the model described in Figure \ref{fig:model_dou_overview} (an overview), Figure \ref{fig:model_dou_enrollment} (on enrollment), and Figure \ref{fig:model_dou_auth} (on authentication), users authenticate themselves to a server through their own devices and biometric scanners that are shared among different users.
A key distribution service distributes keys for them. In practice, this model applies to the situation when the users access an online service run by the server.

\begin{itemize}

	\item \textsf{User}: The user who enrolls its biometric data and authenticates itself to the server. We assume the user's biometric distribution is $\mathcal{B} \in \mathbb{B}$. 

	\item \textsf{Scanner}: A machine to extract the user's biometric data by querying the oracle $\mathcal{O}_{\mathcal{B}}$.
	
	\item \textsf{Device}: A device belonging to the user. In practice, it can be a desktop or a mobile phone. It processes the \textsf{Enroll} and \textsf{Probe} functions for $\textsf{User}$ with keys \textsf{esk} and \textsf{psk}. It queries $\mathcal{O}_{\mathcal{B}}$ for biometric data through the \textsf{Scanner}.
	
	\item \textsf{KDS}: A key distribution service. It runs $\textsf{Setup}$ to generate keys and distribute them to $\textsf{Device}$ and $\textsf{Server}$.
		
	\item \textsf{Server}: The server responsible for authenticating the user. It stores the comparison key \textsf{csk} and the user's enrollment message $\mathbf{c_x}$. On authentication, it compares the probe message with the registered enrollment message and returns the result.  

\end{itemize}

The Device-of-User model, when instantiated by an fh-IPFE scheme (Section \ref{sec:fh-IPFE-instantiation}), is analogous to the use case presented in \cite{cryptoeprint:2023/481}.
In their model, a user possesses a personal device, such as a smartphone or laptop, and a secure hardware device that runs an initial setup and stores all the keys, which corresponds to our \textsf{KDS}.
On enrollemnt and authentication, the user inputs biometric templates onto the device, which corresponds to our \textsf{Scanner}.
Subsequently, the device transmits the template to the secure hardware for the enrollment or probing processes, which are equivalent to our \textsf{Device}.
In addition, they incorporate a two-factor authentication mechanism.
The secure hardware also executes a digital signature scheme and sign the probe message on authentication.


% User-Device with KDS
% Usage Model – Device of User

% Overview
\begin{figure}
\centering

\begin{tikzpicture}[node distance=2cm]

% Gadgets
\node (user1) [inner sep=0pt] at (0,6) 
    {\includegraphics[width=0.1\textwidth]{image/user.jpg}};

\node (user2) [inner sep=0pt] at (0,3) 
    {\includegraphics[width=0.1\textwidth]{image/user.jpg}};

\node (scanner1) [element] at (3,7.5) {{\sf Scanner}};

\node (scanner2) [element] at (3,5.5) {{\sf Scanner}};

\node (scanner3) [element] at (3,3.5) {{\sf Scanner}};

\node (scanner4) [element] at (3,1.5) {{\sf Scanner}};

\node (device1) [element] at (7,6) {\sf Device};
\node (udevice1) [inner sep=0pt] at ($(device1) + (-0.7, +0.95)$)
    {\includegraphics[width=0.03\textwidth]{image/user.jpg}};

\node (device2) [element] at (7,3) {\sf Device};
\node (udevice2) [inner sep=0pt] at ($(device2) + (-0.7, +0.95)$)
    {\includegraphics[width=0.03\textwidth]{image/user.jpg}};

\node (server) [element] at (12,4.5) {\sf Server};

% Texts
% User
\node [align=center] at ($(user1) + (-0.9, 0.9)$) {{\sf User1}};

\node [align=center] at ($(user2) + (-0.9, 0.9)$) {{\sf User2}};

\node [align=center] at ($(udevice1) + (0, 0.5)$) {{\sf User1}};

\node [align=center] at ($(udevice2) + (0, 0.5)$) {{\sf User2}};

% User and Scanner
\draw [darrow, dotted] (user1.east) -- node [align=center, left]{$\mathcal{O}_\mathcal{B}$} (scanner1.west);

\draw [darrow, dotted] (user1.east) -- (scanner2.west);

\draw [darrow, dotted] (user1.east) -- (scanner3.west);

\draw [darrow, dotted] (user1.east) -- (scanner4.west);

\draw [darrow, dotted] (user2.east) -- (scanner1.west);

\draw [darrow, dotted] (user2.east) -- (scanner2.west);

\draw [darrow, dotted] (user2.east) -- (scanner3.west);

\draw [darrow, dotted] (user2.east) -- (scanner4.west);


% Scanner and Device
\draw [darrow, dotted] (scanner1.east) -- node [align=center, above] {$\mathcal{O}_\mathcal{B}$} (device1.west);

\draw [darrow, dotted] (scanner2.east) -- (device1.west);

\draw [darrow, dotted] (scanner3.east) -- (device1.west);

\draw [darrow, dotted] (scanner4.east) -- (device1.west);

\draw [darrow, dotted] (scanner1.east) -- (device2.west);

\draw [darrow, dotted] (scanner2.east) -- (device2.west);

\draw [darrow, dotted] (scanner3.east) -- (device2.west);

\draw [darrow, dotted] (scanner4.east) -- (device2.west);

% Device and Server
\draw [darrow, dotted] (device1.east) -- node [align=center, above, sloped] {${\sf esk}_1, {\sf psk}_1, {\sf csk}_1$} (server.west);

\draw [darrow, dotted] (device2.east) -- node [align=center, above, sloped] {${\sf esk}_2, {\sf psk}_2, {\sf csk}_2$} (server.west);


% Server


\end{tikzpicture}

\caption{An Overview of the Device-of-User Usage Model}
\label{fig:model_dou_overview}
\end{figure}


%-------------------
% Enrollment

\begin{figure}
\centering

\begin{tikzpicture}[node distance=2cm]

% Gadgets
\node (user) [inner sep=0pt] at (0,8) 
    {\includegraphics[width=0.1\textwidth]{image/user.jpg}};

\node (scanner) [element] at (1.5,6.5) {{\sf Scanner}}; 

\node (device) [element] at (7,6.5) {\sf Device};

\node (KDS) [element] at (9.5, 9) {\sf KDS};

\node (server) [element] at (12,6.5) {\sf Server};

% Texts

% KDS
\node [align=center] at ($(KDS.south) + (0, -0.5)$) {${\sf esk}, 
 {\sf psk}, {\sf csk} \gets {\sf Setup}(1^\lambda)$};

% KDS and Device
\draw [arrow] (KDS.west) -| node [align=center, above] {${\sf esk}, {\sf psk}$}  (device.north);

% KDS and Server
\draw [arrow] (KDS.east) -| node [align=center, above] {${\sf csk}$}  (server.north);

% User
\node [align=center] at ($(user) + (-0.9, 0.9)$) {{\sf User}};

% User and Scanner
\draw [arrow] (scanner.west) -| node [align=center, left] {{\sf query} \\ $\mathcal{O}_\mathcal{B}$} (user.south);

\draw [arrow] (user.east) -| node [align=center, right] {{\sf resp.} \\ $\mathbf{b} \getsdollar \mathcal{B}$}  (scanner.north);

% Scanner and Device
\draw [arrow]  ($(device.west) + (-1.5, -2.1)$) -- node[align=center, above] {{\sf query} \\ $\mathcal{O}_\mathcal{B}$} ($(scanner.east) + (0.5, -2.1)$);
\draw [arrow]  ($(scanner.east) + (0.5, -2.3)$) -- node[align=center, below] {{\sf resp.} \\ $\mathbf{b} \getsdollar \mathcal{B}$} ($(device.west) + (-1.5, -2.3)$);

% Device
\node [align=center] at ($(device.south) + (0, -0.5)$) {
    Store ${\sf esk}, {\sf psk}$        
};
\node [align=center] at ($(device.south) + (0, -1.5)$) {
	$\mathbf{x} \gets {\sf encodeEnroll}^{\mathcal{O}_{\mathcal{B}}}()$
};
\node [align=center] at ($(device.south) + (0, -2.5)$) {
    $\mathbf{c_x} \gets {\sf Enroll}({\sf esk, \mathbf{x}})$
};

% Device and Server
\draw [arrow]  ($(device.east) + (1.3, -3.2)$) -- node[align=center, above] {$\mathbf{c_x}$} ($(server.west) + (-0.7, -3.2)$);

% Server
\node [align=center] at ($(server.south) + (0, -0.5)$) {
    Store ${\sf csk}$        
};
\node [align=center] at ($(server.south) + (0, -2.7)$) {
    Store $\mathbf{c_x}$
};


\end{tikzpicture}

\caption{Device-of-User Usage Model on Enrollment}
\label{fig:model_dou_enrollment}
\end{figure}


%-------------------

% Authentication
\begin{figure}[htp]
\centering

\begin{tikzpicture}[node distance=2cm]

% Gadgets
\node (user) [inner sep=0pt] at (0,8) 
    {\includegraphics[width=0.1\textwidth]{image/user.jpg}};

\node (scanner) [element] at (1.5,6.5) {{\sf Scanner}}; 

\node (device) [element] at (7,6.5) {\sf Device};

\node (server) [element] at (12,6.5) {\sf Server};

% Texts

% User
\node [align=center] at ($(user) + (-0.9, 0.9)$) {{\sf User}};

% User and Scanner
\draw [arrow] (scanner.west) -| node [align=center, left] {{\sf query} \\ $\mathcal{O}_\mathcal{B}$} (user.south);

\draw [arrow] (user.east) -| node [align=center, right] {{\sf resp.} \\ $\mathbf{b} \getsdollar \mathcal{B}$}  (scanner.north);

% Device
\node [align=center] at ($(device.south) + (0, -0.9)$) {
    $\mathbf{y} \gets {\sf encodeProbe}^{\mathcal{O}_{\mathcal{B}}}()$
};
\node [align=center] at ($(device.south) + (0, -1.9)$) {
    $\mathbf{c_y} \gets {\sf Probe}({\sf psk, \mathbf{y}})$
};

% % Scanner and Device
% \draw [arrow]  ($(device.west) + (-1.5, -1.5)$) -- node[align=center, above] {{\sf query} \\ $\mathcal{O}_\mathcal{B}$} ($(scanner.east) + (0.5, -1.5)$);
% \draw [arrow]  ($(scanner.east) + (0.5, -1.7)$) -- node[align=center, below] {{\sf resp.} \\ $\mathbf{b} \getsdollar \mathcal{B}$} ($(device.west) + (-1.5, -1.7)$);

% % Device and Server
% \draw [arrow]  ($(device.east) + (1, -3.4)$) -- node[align=center, above] {$\mathbf{c_y}$} ($(server.west) + (-1, -3.4)$);
% \draw [arrow]  ($(server.west) + (-1, -4.4)$) -- node[align=center, above] {$r$} ($(device.east) + (1, -4.4)$);

% % Server
% \node [align=center] at ($(server.south) + (0, -3.4)$) {
%     $s \gets {\sf Compare}({\sf csk}, \mathbf{c_x}, \mathbf{c_y})$ \\
%     $r \gets {\sf Verify}(s)$
% };


% Scanner and Device
\draw [arrow]  ($(device.west) + (-1.5, -1.5)$) -- node[align=center, above] {{\sf query} \\ $\mathcal{O}_\mathcal{B}$} ($(scanner.east) + (0.5, -1.5)$);
\draw [arrow]  ($(scanner.east) + (0.5, -1.7)$) -- node[align=center, below] {{\sf resp.} \\ $\mathbf{b} \getsdollar \mathcal{B}$} ($(device.west) + (-1.5, -1.7)$);

% Device and Server
\draw [arrow]  ($(device.east) + (1, -2.7)$) -- node[align=center, above] {$\mathbf{c_y}$} ($(server.west) + (-1, -2.7)$);
\draw [arrow]  ($(server.west) + (-1, -3.7)$) -- node[align=center, above] {$r$} ($(device.east) + (1, -3.7)$);

% Server
\node [align=center] at ($(server.south) + (0, -2.7)$) {
    $s \gets {\sf Compare}({\sf csk}, \mathbf{c_x}, \mathbf{c_y})$ \\
    $r \gets {\sf Verify}(s)$
};

\end{tikzpicture}

\caption{Device-of-User Usage Model on Authentication}
\label{fig:model_dou_auth}
\end{figure}



%-------------------

\subsection{Usage Model – Device-of-Domain}
\label{sec:dod_model}

In the model described in Figure \ref{fig:model_dod_overview} (an overview), Figure \ref{fig:model_dod_enrollment} (on enrollment), and Figure \ref{fig:model_dod_auth} (on authentication), users first enroll themselves at an enrollment station and then authenticate themselves to a server through devices that belong to a domain.
A key distribution service distributes enrollment keys to the enrollment station, probe keys to the domain, and comparison keys to the server. In practice, a domain can be a department in an organization, and this models applies to the situation when a user wants to access a public service of a department, such as a restricted area or instruments. 

\begin{itemize}

	\item \textsf{User}: The user who enrolls its biometric data at an enrollment station and authenticates itself to the server. We assume the user's biometric distribution is $\mathcal{B} \in \mathbb{B}$.
	
	\item \textsf{Domain}: A domain that owns several devices, all of which share one enrollment key $\textsf{esk}$, one probe key $\textsf{psk}$ and one comparison key $\textsf{csk}$. Only the probe key is stored at each device of a domain. The enrollment key is stored at the enrollment station, and the comparison key is stored at the server. In practice, a domain can be a department, and users enroll and authenticate themselves before accessing a restricted service of this department.

	\item \textsf{Scanner}: A machine to extract the user's biometric data by querying the oracle $\mathcal{O}_{\mathcal{B}}$.
	
	\item \textsf{Station}: An enrollment station responsible for collecting the user's biometric data to enroll them for a domain on the server.

	\item \textsf{Device}: A device belonging to a domain. In practice, it can be a device checking identities for a restricted area or an instrument. It owns a probe key $\sf psk$ and processes the $\sf Probe$ function for enrolled users of this domain.
	
	\item \textsf{KDS}: A key distribution service. It runs $\textsf{Setup}$ to generate keys and distribute them to \textsf{Station}, \textsf{Domain}, and \textsf{Server}.
		
	\item \textsf{Server}: The server responsible for authenticating the user. It stores the comparison key \textsf{csk} for each domain and the user's enrollment message $\mathbf{c_x}$. On authentication, it compares the probe message with the registered enrollment message and returns the result.  

\end{itemize}


%-------------------

% Usage Model – Device of Domain
\begin{figure}
\centering

\begin{tikzpicture}[node distance=2cm]

% Gadgets
\node (user1) [inner sep=0pt] at (0,6) 
    {\includegraphics[width=0.1\textwidth]{image/user.jpg}};

\node (user2) [inner sep=0pt] at (0,3) 
    {\includegraphics[width=0.1\textwidth]{image/user.jpg}};

\node (scanner1) [element] at (3,7.5) {{\sf Scanner}};

\node (scanner2) [element] at (3,5.5) {{\sf Scanner}};

\node (scanner3) [element] at (3,3.5) {{\sf Scanner}};

\node (scanner4) [element] at (3,1.5) {{\sf Scanner}};

\node (station) [rectangle, rounded corners, 
	minimum width=2.5cm, 
	minimum height=1.5cm,
	text centered,
	fill=white!70!black,
	draw=black
] at (7,8) {\sf Station};

\node (device1) [element] at (7, 5.3) {\sf Device};

\node (device2) [element] at (7, 3.7) {\sf Device};

\node (device3) [element] at (7, 1.5) {\sf Device};

\begin{pgfonlayer}{background layer}
\node (domain1) [
    rectangle, rounded corners, 
    minimum width=2.2cm, 
    minimum height=4cm,
    fill=blue!50!white,
    draw=black]
at ($(device1)!0.6!(device2) + (0, 0.5)$) {};
\end{pgfonlayer};

\node (domain1_text) [element,fill=none,draw=none] at ($(domain1.north) + (0, -0.5)$) {\sf Domain1};

\begin{pgfonlayer}{background layer}
\node (domain2) [
    rectangle, rounded corners, 
    minimum width=2.2cm, 
    minimum height=2.1cm,
    fill=orange!50!white,
    draw=black]
at ($(device3) + (0, -0.3)$) {};
\end{pgfonlayer};

\node (domain2_text) [element,fill=none,draw=none] at ($(domain2.south) + (0, 0.4)$) {\sf Domain 2};


\node (server) [element] at (12,4.5) {\sf Server};


% Texts
% User
\node [align=center] at ($(user1) + (-0.9, 0.9)$) {{\sf User1}};

\node [align=center] at ($(user2) + (-0.9, 0.9)$) {{\sf User2}};

% User and Scanner
\draw [darrow, dotted] (user1.east) -- node [align=center, left]{$\mathcal{O}_\mathcal{B}$} (scanner1.west);

\draw [darrow, dotted] (user1.east) -- (scanner2.west);

\draw [darrow, dotted] (user1.east) -- (scanner3.west);

\draw [darrow, dotted] (user1.east) -- (scanner4.west);

\draw [darrow, dotted] (user2.east) -- (scanner1.west);

\draw [darrow, dotted] (user2.east) -- (scanner2.west);

\draw [darrow, dotted] (user2.east) -- (scanner3.west);

\draw [darrow, dotted] (user2.east) -- (scanner4.west);

% Scanner and Device
\draw [darrow, dotted] (scanner1.east) -- (device1.west);

\draw [darrow, dotted] (scanner2.east) -- (device1.west);

\draw [darrow, dotted] (scanner3.east) -- (device1.west);

\draw [darrow, dotted] (scanner4.east) -- (device1.west);

\draw [darrow, dotted] (scanner1.east) -- (device2.west);

\draw [darrow, dotted] (scanner2.east) -- (device2.west);

\draw [darrow, dotted] (scanner3.east) -- (device2.west);

\draw [darrow, dotted] (scanner4.east) -- (device2.west);

\draw [darrow, dotted] (scanner1.east) -- (device3.west);

\draw [darrow, dotted] (scanner2.east) -- (device3.west);

\draw [darrow, dotted] (scanner3.east) -- (device3.west);

\draw [darrow, dotted] (scanner4.east) -- (device3.west);

% Scanner and Station
\draw [darrow, dotted] (scanner1.east) -- node [align=center, above]{$\mathcal{O}_\mathcal{B}$} (station.west);

\draw [darrow, dotted] (scanner2.east) -- (station.west);

\draw [darrow, dotted] (scanner3.east) -- (station.west);

\draw [darrow, dotted] (scanner4.east) -- (station.west);

% Station and Server
\draw [darrow, dotted] (station.east) -- node [align=center, midway, above, sloped] {$\textcolor{blue}{{\sf esk}_1}, \textcolor{orange}{{\sf esk}_2}$} (server.west);


% Device and Server
\draw [darrow, dotted] (device1.east) -- node [blue, align=center, midway, above, sloped] {${\sf psk}_1, {\sf csk}_1$} (server.west);

\draw [darrow, dotted] (device2.east) -- node [blue, align=center, midway, above, sloped] {${\sf psk}_1, {\sf csk}_1$} (server.west);

\draw [darrow, dotted] (device3.east) -- node [orange, align=center, midway, below, sloped] {${\sf psk}_2, {\sf csk}_2$} (server.west);


% Server


\end{tikzpicture}

\caption{An overview of the Device-of-Domain Usage Model}
\label{fig:model_dod_overview}
\end{figure}



%-------------------
% Enrollment

\begin{figure}
\centering

\begin{tikzpicture}[node distance=2cm]

% Gadgets
\node (user) [inner sep=0pt] at (0,8) 
    {\includegraphics[width=0.1\textwidth]{image/user.jpg}};

\node (scanner) [element] at (1.5,6.5) {{\sf Scanner}};

\begin{pgfonlayer}{background layer}
\node (domain) [
    rectangle, rounded corners, 
    minimum width=2.7cm, 
    minimum height=1.8cm,
    fill=blue!50!white,
    draw=black] 
at (4.5, 9.5) {};
\end{pgfonlayer};

\node (domain_text) [element,fill=none,draw=none] at ($(domain.south) + (0, 1.2)$) {\sf Domain};

\node (device1) [
    rectangle, rounded corners, 
    minimum width=0.3cm, 
    minimum height=0.3cm,
    text centered,
    fill=white!70!black,
    draw=black]
at ($(domain.south) + (-0.65, 0.4)$) {\scriptsize{\sf Device}};

\node (device2) [
    rectangle, rounded corners, 
    minimum width=0.3cm, 
    minimum height=0.3cm,
    text centered,
    fill=white!70!black,
    draw=black]
at ($(domain.south) + (0.65, 0.4)$) {\scriptsize{\sf Device}};


\node (station) [element] at (7,6.5) {\sf Station};

\node (KDS) [element] at (9.5, 9.5) {\sf KDS};

\node (server) [element] at (12,6.5) {\sf Server};

% Texts

% KDS
\node [align=center] at ($(KDS.south) + (0, -0.5)$) {${\sf esk}, 
 {\sf psk}, {\sf csk} \gets {\sf Setup}(1^\lambda)$};

% KDS and Station
\draw [arrow] (KDS.west) -| node (esk) [align=center, below=1cm] {}  (station.north);
\node [align=center] at ($(esk) + (-0.3, 0)$) { {\sf esk} };

% KDS and Domain
\draw [arrow] (KDS.west) -- node (psk) [align=center, above=0.1cm] {} (domain.east);
\node [align=center] at ($(psk) + (-0.7, 0)$) { {\sf psk} };

% KDS and Server
\draw [arrow] (KDS.east) -| node [align=center, above] {${\sf csk}$}  (server.north);

% Domain
\node [align=center] at ($(domain.south) + (0, -0.5)$) {Store ${\sf psk}$};

% User
\node [align=center] at ($(user) + (-0.9, 0.9)$) {{\sf User}};

% User and Scanner
\draw [arrow] (scanner.west) -| node [align=center, left] {{\sf query} \\ $\mathcal{O}_\mathcal{B}$} (user.south);

\draw [arrow] (user.east) -| node [align=center, right] {{\sf resp.} \\ $\mathbf{b} \getsdollar \mathcal{B}$}  (scanner.north);

% Scanner and Station
\draw [arrow]  ($(station.west) + (-1.5, -2.1)$) -- node[align=center, above] {{\sf query} \\ $\mathcal{O}_\mathcal{B}$} ($(scanner.east) + (0.5, -2.1)$);
\draw [arrow]  ($(scanner.east) + (0.5, -2.3)$) -- node[align=center, below] {{\sf resp.} \\ $\mathbf{b} \getsdollar \mathcal{B}$} ($(station.west) + (-1.5, -2.3)$);

% Station
\node [align=center] at ($(station.south) + (0, -0.5)$) {
    Store ${\sf esk}$        
};
\node [align=center] at ($(station.south) + (0, -1.5)$) {
	$\mathbf{x} \gets {\sf encodeEnroll}^{\mathcal{O}_{\mathcal{B}}}()$
};
\node [align=center] at ($(station.south) + (0, -2.5)$) {
    $\mathbf{c_x} \gets {\sf Enroll}({\sf esk, \mathbf{x}})$
};

% Station and Server
\draw [arrow]  ($(station.east) + (1.3, -3.2)$) -- node[align=center, above] {$\mathbf{c_x}$} ($(server.west) + (-0.7, -3.2)$);

% Server
\node [align=center] at ($(server.south) + (0, -0.5)$) {
    Store ${\sf csk}$        
};
\node [align=center] at ($(server.south) + (0, -2.7)$) {
    Store $\mathbf{c_x}$
};


\end{tikzpicture}

\caption{Device-of-Domain Usage Model on Enrollment}
\label{fig:model_dod_enrollment}
\end{figure}


%-------------------

% Authentication
\begin{figure}[htp]
\centering

\begin{tikzpicture}[node distance=2cm]

% Gadgets
\node (user) [inner sep=0pt] at (0,8) 
    {\includegraphics[width=0.1\textwidth]{image/user.jpg}};

\node (scanner) [element] at (1.5,6.5) {{\sf Scanner}}; 

\node (device) [element] at (7,6.5) {\sf Device};

\begin{pgfonlayer}{background layer}
\node (domain) [
    rectangle, rounded corners, 
    minimum width=2.2cm, 
    minimum height=2.2cm,
    fill=blue!50!white,
    draw=black] 
at ($(device) + (0, 0.3)$) {};
\end{pgfonlayer};

\node (domain_text) [element,fill=none,draw=none] at ($(domain.south) + (0, 1.8)$) {\sf Domain};


\node (server) [element] at (12,6.5) {\sf Server};

% Texts

% User
\node [align=center] at ($(user) + (-0.9, 0.9)$) {{\sf User}};

% User and Scanner
\draw [arrow] (scanner.west) -| node [align=center, left] {{\sf query} \\ $\mathcal{O}_\mathcal{B}$} (user.south);

\draw [arrow] (user.east) -| node [align=center, right] {{\sf resp.} \\ $\mathbf{b} \getsdollar \mathcal{B}$}  (scanner.north);

% Device
\node [align=center] at ($(device.south) + (0, -0.9)$) {
    $\mathbf{y} \gets {\sf encodeProbe}^{\mathcal{O}_{\mathcal{B}}}()$
};
\node [align=center] at ($(device.south) + (0, -1.9)$) {
    $\mathbf{c_y} \gets {\sf Probe}({\sf psk, \mathbf{y}})$
};

% Scanner and Device
\draw [arrow]  ($(device.west) + (-1.5, -1.5)$) -- node[align=center, above] {{\sf query} \\ $\mathcal{O}_\mathcal{B}$} ($(scanner.east) + (0.5, -1.5)$);
\draw [arrow]  ($(scanner.east) + (0.5, -1.7)$) -- node[align=center, below] {{\sf resp.} \\ $\mathbf{b} \getsdollar \mathcal{B}$} ($(device.west) + (-1.5, -1.7)$);

% Device and Server
\draw [arrow]  ($(device.east) + (1, -2.7)$) -- node[align=center, above] {$\mathbf{c_y}$} ($(server.west) + (-1, -2.7)$);
\draw [arrow]  ($(server.west) + (-1, -3.7)$) -- node[align=center, above] {$r$} ($(device.east) + (1, -3.7)$);

% Server
\node [align=center] at ($(server.south) + (0, -2.7)$) {
    $s \gets {\sf Compare}({\sf csk}, \mathbf{c_x}, \mathbf{c_y})$ \\
    $r \gets {\sf Verify}(s)$
};


\end{tikzpicture}

\caption{Device-of-Domain Usage Model on Authentication}
\label{fig:model_dod_auth}
\end{figure}



\pagebreak

\fi

%-------------------

\iffalse

\subsection{Instantiation with a 2i-IPFE Scheme}
\label{sec:2i-IPFE-instantiation}

Let $\textsf{FE} = (\textsf{FE.Setup}, \textsf{FE.KeyGen}, \textsf{FE.Enc}, \textsf{FE.Dec})$ be a 2i-IPFE scheme we defined in Definition \ref{def:2i-IPFE}. Following the scheme in Section \ref{sec:fh-IPFE-instantiation}, we can instantiate a biometric authentication scheme using $\textsf{FE}$.

\begin{itemize}

	\item $\textsf{Setup}(1^\lambda)$: It calls $\textsf{FE.Setup}(1^\lambda) \to \textsf{sk}, \textsf{ek}_1, \textsf{ek}_2$, $ \textsf{FE.KeyGen}(sk, \mathbf{I}_{k+2}) \to \textsf{dk}_{\mathbf{I}} $, where $\mathbf{I}_{k+2}$ is an identity matrix of size $(k+2) \times (k+2)$. It outputs $\textsf{esk} \gets \textsf{ek}_1$, $\textsf{psk} \gets \textsf{ek}_2$, and $\textsf{csk} \gets \textsf{dk}_{\mathbf{I}}$

	\item $\textsf{encodeEnroll}^{\mathcal{O}_{\mathcal{B}}}(), \textsf{encodeProbe}^{\mathcal{O}_{\mathcal{B}}}()$: The same as the scheme in \ref{sec:fh-IPFE-instantiation}. 

	\item $\textsf{Enroll}(\textsf{esk}, \mathbf{x})$: It calls $\textsf{FE.Enc}(\textsf{ek}_1, \mathbf{x}) \to \mathbf{c_x}$ and outputs $\mathbf{c_x}$.

	\item $\textsf{Probe}(\textsf{psk}, \mathbf{y})$: It calls $\textsf{FE.Enc}(\textsf{ek}_2, \mathbf{y}) \to \mathbf{c_y}$ and outputs $\mathbf{c_y}$.

	\item $\textsf{Compare}(\textsf{csk}, \mathbf{c_x}, \mathbf{c_y)}$: It calls $\textsf{FE.Dec}(\textsf{dk}_{\mathbf{I}}, \mathbf{c_x}, \mathbf{c_y}) \to s$ and outputs the value $s$.

	\item $\textsf{Verify}(s)$: If $\sqrt{s} \leq \tau$, a pre-defined threshold for comparing the closeness of two templates, then it outputs $r = 1$; otherwise, it outputs $r = 0$.

\end{itemize}

By the correctness of the functional encryption scheme $\sf FE$, we have
\[
	s = \textsf{FE.Dec}(\textsf{dk}_{\mathbf{I}}, \mathbf{c_x}, \mathbf{c_y}) =  \mathbf{x} \mathbf{I}_{k+2} \mathbf{y}^T = \mathbf{x} \mathbf{y}^T = \| \mathbf{b} - \mathbf{b}^\prime \|^2.
\]
just as the scheme in Section \ref{sec:fh-IPFE-instantiation}


Unlike the previous scheme, instantiated with a 2i-IPFE scheme in this way, the comparison secret key $\textsf{csk}$ is now secret, and the enrollment secret key $\textsf{esk}$ and probe secret key $\textsf{psk}$ are distinct. Without $\textsf{csk}$, one cannot compare an enrollment message $\mathbf{c_x}$ and a probe message $\mathbf{c_y}$. We can also transmit $\mathbf{c_x}$ in a public channel and store it in a public storage, under necessary security requirements of the 2i-IPFE scheme, such as indistinguishability of $\mathbf{c_x}$.

In the Device-of-Domain model, the indistinguishability of $\mathbf{c_x}$ is against an adversary who has a probe oracle $\textsf{Probe}(\textsf{psk}, \cdot)$. If \textsf{Server} is malicious, then it can use $\textsf{csk}$ to distinguish $\mathbf{c_x}$ enrolled by different samples. Therefore, we must limit the adversary's ability. For example, we can require the adversary to distinguish biometric vectors sampled from distributions in a pre-defined pool, and the adversary can only probe vectors randomly sampled from a distribution in the pool. We can also limit the rate of the probe oracle.

%-------------------

\subsection{Instantiation with a 2c-IPFE Scheme}
\label{sec:2c-IPFE-instantiation}

Note that if labels remain constant, a 2c-IPFE scheme is reduced to a 2i-IPFE scheme. Therefore, we can consider utilizing the label to represent each domain in the Device-of-Domain model. Let $\textsf{FE} = (\textsf{FE.Setup}, \textsf{FE.KeyGen}, \textsf{FE.Enc}, \textsf{FE.Dec})$ be a 2c-IPFE scheme we defined in Definition \ref{def:2c-IPFE}. Following the scheme in Section \ref{sec:2i-IPFE-instantiation}, we can instantiate a biometric authentication scheme using $\textsf{FE}$.

\begin{itemize}

	\item $\textsf{Setup}(1^\lambda)$: It calls $\textsf{FE.Setup}(1^\lambda) \to \textsf{sk}, \textsf{ek}_1, \textsf{ek}_2$, $\textsf{FE.KeyGen}(sk, \mathbf{I}_{k+2}) \to \textsf{dk}_{\mathbf{I}} $, where $\mathbf{I}_{k+2}$ is an identity matrix of size $(k+2) \times (k+2)$. For keys used for \textsf{Domain} $\ell$, it outputs $\textsf{esk} \gets (\ell, \textsf{ek}_1)$, $\textsf{psk} \gets (\ell, \textsf{ek}_2)$, and $\textsf{csk} \gets \textsf{dk}_{\mathbf{I}}$.

	Note that when the previous 2i-IPFE-based scheme in Section \ref{sec:2i-IPFE-instantiation} is applied to a Device-of-Domain model, we assume that $\textsf{Setup}$ is run once for each domain to generate different $\textsf{esk}, \textsf{psk}, \textsf{csk}$. In the scheme in this section, however, $\textsf{Setup}$ is run only once for all the domains, and each domain shares the same $\textsf{csk}$ and the same $\textsf{esk}, \textsf{psk}$ except different labels.

	\item $\textsf{encodeEnroll}^{\mathcal{O}_{\mathcal{B}}}(), \textsf{encodeProbe}^{\mathcal{O}_{\mathcal{B}}}()$: The same as the scheme in \ref{sec:2i-IPFE-instantiation}. 

	\item $\textsf{Enroll}(\textsf{esk}, \mathbf{x})$: It calls $\textsf{FE.Enc}(\ell, \textsf{ek}_1, \mathbf{x}) \to \mathbf{c_x}$ and outputs $\mathbf{c_x}$.

	\item $\textsf{Probe}(\textsf{psk}, \mathbf{y})$: It calls $\textsf{FE.Enc}(\ell, \textsf{ek}_2, \mathbf{y}) \to \mathbf{c_y}$ and outputs $\mathbf{c_y}$.

	\item $\textsf{Compare}(\textsf{csk}, \mathbf{c_x}, \mathbf{c_y)}$: It calls $\textsf{FE.Dec}(\textsf{dk}_{\mathbf{I}}, \mathbf{c_x}, \mathbf{c_y}) \to s$ and outputs the value $s$.

	\item $\textsf{Verify}(s)$: If $\sqrt{s} \leq \tau$, a pre-defined threshold for comparing the closeness of two templates, then it outputs $r = 1$; otherwise, it outputs $r = 0$.

	\item $\textsf{BioCompare}$

\end{itemize}

By the correctness of the functional encryption scheme $\sf FE$, if the labels of $\mathbf{c_x}$ and $\mathbf{c_y}$ are the same (they are of the same domain), we have
\[
	s = \textsf{FE.Dec}(\textsf{dk}_{\mathbf{I}}, \mathbf{c_x}, \mathbf{c_y}) =  \mathbf{x} \mathbf{I}_{k+2} \mathbf{y}^T = \| \mathbf{b} - \mathbf{b}^\prime \|^2.
\]
just as the scheme in Section \ref{sec:2i-IPFE-instantiation}

When the Device-of-Domain model is instantiated with a 2c-IPFE scheme in this way, the enrollment secret key $\textsf{esk}$ and probe secret key $\textsf{psk}$ are now shared among all the devices, regardless of their domains. Therefore, to let a malicious or broken \textsf{Domain} not threaten other honest ones, one needs to make sure given $\textsf{esk}$ or $\textsf{psk}$, $\mathbf{c_x}$ still does not leak information about $\mathbf{x}$. This is different from the scheme in Section \ref{sec:2i-IPFE-instantiation}, where we only need seurity against an adversary who has a probe oracle $\textsf{Probe}(\textsf{psk}, \cdot)$.

If \textsf{Server} and \textsf{Domain} are both malicious, then the adversary can use $\textsf{csk}$ to distinguish $\mathbf{c_x}$ and even recover $\mathbf{x}$. Therefore, we assume at most one party of them can be malicious at the same time. Note that this is the same as the 2i-IPFE-based scheme, where only one of \textsf{Server} and \textsf{Domain} can be malicious.

\fi

%-------------------


\section{Security Games}
\label{sec:security_game}

In this section, we discuss two security notions of a biometric authentication scheme: \emph{unforgeability} and \emph{indistinguishability}.

\subsection{Unforgeability}
\label{sec:uf_game}

To describe the unforgeability of an authentication scheme, we model the ability of an adversary who tries to impersonate a user. The adversary $\mathcal{A}$ is given auxiliary information \textsf{option} that depends on our threat model and tries to find a valid probe message $\mathbf{\tilde{z}}$. The whole game $\textsf{UF}_{\Pi, \mathbb{B}, \textsf{option}}$ is defined in Algorithm \ref{alg:uf_game}.

\begin{figure}[h]
\centering
	\begin{minipage}[t]{0.6\linewidth}
	\centering
	\begin{algorithm}[H]
	\caption{$\textsf{UF}_{\Pi, \mathbb{B}, \textsf{option}}(\mathcal{A})$}
	\label{alg:uf_game}
	\begin{algorithmic}[1]
		\State $\mathcal{B} \getsdollar \mathbb{B}, \mathbb{B} \gets \mathbb{B} \setminus \mathcal{B}$

		\State $\textsf{esk}, \textsf{psk}, \textsf{csk} \gets \textsf{Setup}(1^\lambda)$
		
		\State $\mathbf{b} \gets \textsf{getEnroll}^{\mathcal{O}_{\mathcal{B}}}()$

		\State $\mathbf{c_x} \gets \textsf{Enroll}(\textsf{esk}, \mathbf{b})$
		
		\State ${\mathbf{\tilde{z}}} \gets \mathcal{A} ( \textsf{option} )$
 
		\If{$\mathbf{\tilde{z}}$ is equal to any output of $\mathcal{O}_{\textsf{Probe}}$ }
			
			\State \Return $0$
		
		\EndIf

		\State $s \gets \textsf{Compare}( \textsf{csk}, \mathbf{c_x}, \mathbf{\tilde{z}} )$

		\State \Return $\textsf{Verify}(s)$
	\end{algorithmic}
	\end{algorithm}
	\end{minipage}
	
\label{fig:uf_game}
\end{figure}

The auxiliary information \textsf{option} can be nothing or include $\mathbf{c_x}, \textsf{esk}, \textsf{psk}, \textsf{csk}$ or the following oracles:

\begin{itemize}

	\item $\mathcal{O}_{\mathcal{B}}$: It outputs a biometric sample $\mathbf{b} \getsdollar \mathcal{B}$. This oracle and $\textsf{psk}$ should not be given at the same time; otherwise, there exists a trivial attack with a winning rate $\textsf{TP}$ by returning $\textsf{Probe}(\textsf{psk}, \textsf{getProbe}^{ \mathcal{O}_\mathcal{B} }() )$.
	
	\item $\mathcal{O}_\textsf{Enroll}(\textsf{esk}, \cdot)$: On input $\mathbf{b}^\prime$, it outputs the enrollment message $\textsf{Enroll}(\textsf{esk}, \mathbf{b}^\prime)$.

	\item $\mathcal{O}_\textsf{Probe}(\textsf{psk}, \cdot)$: On input $\mathbf{b}^\prime$, it outputs the probe message $\textsf{Probe}(\textsf{psk}, \mathbf{b}^\prime)$. If this oracle is given, we require the adversary to return a $\mathbf{\tilde{z}}$ that is not equal to any previous answer of $\mathcal{O}_\textsf{Probe}$.
	
	\item $\mathcal{O}_\textsf{log}(\textsf{csk}, \mathbf{c_x}, \cdot)$: On input $\mathbf{b}^\prime$, it first computes $\mathbf{c_z} \gets \textsf{Probe}(\textsf{psk}, \mathbf{b}^\prime)$ and outputs $\textsf{Verify}(\textsf{Compare}(\textsf{csk}, \mathbf{c_x}, \mathbf{c_z} ) )$.
	
	\item $\mathcal{O}_\textsf{Enroll}^\prime (\cdot)$: On input $\textsf{esk}^\prime$, it first samples $\mathbf{b}^\prime \gets \textsf{getEnroll}^{\mathcal{O}_{\mathcal{B}}}()$ and outputs $\textsf{Enroll}(\textsf{esk}^\prime, \mathbf{b}^\prime)$. This oracle is only useful when $\textsf{option}$ does not include $\mathcal{O}_{\mathcal{B}}$.

	\item $\mathcal{O}_\textsf{Probe}^\prime (\cdot)$: On input $\textsf{psk}^\prime$, it first samples $\mathbf{b}^\prime \gets \textsf{getProbe}^{\mathcal{O}_{\mathcal{B}}}()$ and outputs $\textsf{Probe}(\textsf{psk}^\prime, \mathbf{b}^\prime)$. This oracle is only useful when $\textsf{option}$ does not include $\mathcal{O}_{\mathcal{B}}$, and this oracle and $\textsf{psk}$ should not be given at the same time; otherwise, there exists a trivial attack with a winning rate $\textsf{TP}$ by returning $\mathcal{O}_{\textsf{Probe}}^\prime (\textsf{psk})$.
	
\end{itemize}

The requirement that the adversary should return a $\mathbf{\tilde{z}}$ that is not equal to any previous answer of $\mathcal{O}_\textsf{Probe}$ is to prevent a trivial attack that leverages \textsf{TP} or \textsf{FP} when it is non-negligible. If \textsf{option} includes $\mathcal{O}_\mathcal{B}$ and either $\textsf{psk}$ or $\mathcal{O}_\textsf{Probe}$, the adversary can enjoy a winning rate \textsf{TP}. Therefore, we rule out the case that $\textsf{option}$ includes both $\textsf{psk}$ and $\mathcal{O}_\mathcal{B}$, and we forbid the adversary to return what $\mathcal{O}_\textsf{Probe}$ returns.
If \textsf{option} has only $\textsf{psk}$ or $\mathcal{O}_\textsf{Probe}$, the $\textsf{UF}$ adversary $\mathcal{A}$ in Algorithm \ref{alg:adv:FP} can still enjoy a winning rate $\textsf{FP}$, if we place no restriction on the adversary's answer. Therefore, we only consider $\textsf{psk}$ in \textsf{option} when \textsf{FP} is non-negligible, and we restrict the adversary's answer when $\mathcal{O}_\textsf{Probe}$ is given.

%If $\textsf{option}$ includes $\mathcal{O}_{\textsf{Enroll}}^\prime$ and either $\textsf{psk}$ or $\mathcal{O}_\textsf{Probe}$, and if we place no restriction on an \textsf{UF} game adversary's answer, the adversary in Algorithm \ref{alg:adv:FP2} can win with a probability
%\[
	%\Pr[ \textsf{Verify}( \textsf{BioCompare}(\mathbf{x}^{(0)}, \mathbf{y}) ) = 1 \mid \textsf{Verify}( \textsf{BioCompare}(\mathbf{x}^{(1)}, \mathbf{y}) ) = 1]
%\]
%where $\mathbf{x}^{(0)}, \mathbf{x}^{(1)}$ are generated from $\textsf{encodeEnroll}^{\mathcal{O}_\mathcal{B}}()$ and $\mathbf{y} \gets \textsf{encodeProbe}^{\mathcal{O}_{\mathcal{B}^\prime}}() $.
%This value is in general not negligible.
%The expected number of repetitions is $\mathbb{E}_{\mathcal{B} \getsdollar \mathbb{B}}\left[\frac{1}{\textsf{FP}(\mathcal{B})} \right]$. If $\textsf{FP}(\mathcal{B})$ is non-negligible, the adversary can return the answer in an expected polynomial time. A similar adversary also exists when $\textsf{option}$ includes $\mathcal{O}_{\textsf{Probe}}^\prime$ and $\mathcal{O}_\textsf{Probe}$.

 
%\begin{figure}[h]
%\centering
	%\begin{minipage}[t]{0.6\linewidth}
	%\centering
	%\begin{algorithm}[H]
	%\caption{$\mathcal{A}^{\mathcal{O}_\textsf{Enroll}^\prime}(\textsf{psk})$ (or  $\mathcal{A}^{\mathcal{O}_\textsf{Enroll}^\prime, \mathcal{O}_\textsf{Probe}}$ ) }
	%\label{alg:adv:FP2}
	%\begin{algorithmic}[1]
		%\State $\textsf{esk}^\prime, \textsf{psk}^\prime, \textsf{csk}^\prime \gets \textsf{Setup}(1^\lambda)$

		%\Repeat
		
			%\State $\mathcal{B}^\prime \getsdollar \mathbb{B}$
		
			%\State $\mathbf{y}^\prime \gets \textsf{encodeProbe}^{\mathcal{O}_{\mathcal{B}^\prime }}()$ 

			%\State $\mathbf{c_y}^\prime \gets \textsf{Probe}(\textsf{psk}^\prime, \mathbf{y}^\prime)$

			%\State $\mathbf{c_x}^\prime \gets \mathcal{O}_\textsf{Enroll}^\prime (\textsf{esk}^\prime)$

		%\Until{ $\textsf{Verify}(\textsf{Compare}(\textsf{csk}^\prime, \mathbf{c_x}^\prime, \mathbf{c_y}^\prime )) = 1$ }

		%\State $\mathbf{c_y} \gets \textsf{Probe}(\textsf{psk}, \mathbf{y}^\prime)$  \Comment{ or $\mathbf{c_y} \gets \mathcal{O}_\textsf{Probe}(\mathbf{y}^\prime)$ } 

		%\State \Return $\mathbf{c_y}$
	%\end{algorithmic}
	%\end{algorithm}
	%\end{minipage}
	
%\end{figure}

\begin{figure}[h]
\centering
	\begin{minipage}[t]{0.6\linewidth}
	\centering
	\begin{algorithm}[H]
	\caption{$\mathcal{A}(\textsf{psk})$ ( or $\mathcal{A}^{\mathcal{O}_\textsf{Probe}}$ ) }
	\label{alg:adv:FP}
	\begin{algorithmic}[1]
		\State $\mathcal{B}^\prime \getsdollar \mathbb{B}$
		
		\State $\mathbf{b}^\prime \gets \textsf{getProbe}^{\mathcal{O}_{\mathcal{B}^\prime }}()$

		\State $\mathbf{c_y} \gets \textsf{Probe}(\textsf{psk}, \mathbf{b}^\prime)$ \Comment{or $\mathbf{c_y} \gets \mathcal{O}_\textsf{Probe}(\mathbf{b}^\prime)$ }

		\State \Return $\mathbf{c_y}$
	\end{algorithmic}
	\end{algorithm}
	\end{minipage}
	
\end{figure}


We define the advantage of an adversary $\mathcal{A}$ in the $\textsf{UF}_{\Pi, \mathbb{B}, \textsf{option}}$ game of a scheme $\Pi$ associated with a family $\mathbb{B}$ of distributions as
\[
	\Adv^{\textsf{UF}}_{\Pi, \mathbb{B}, \mathcal{A}, \textsf{option}} := \Pr[\textsf{UF}_{\Pi, \mathbb{B}, \textsf{option}}(\mathcal{A}) \to 1]
\]

An authentication scheme $\Pi$ associated with a family $\mathbb{B}$ of distributions is called \emph{\textsf{option}-unforgeable} (\textsf{option}-UF) if for any PPT adversary $\mathcal{A}$,
\[
	\Adv^{\textsf{UF}}_{\Pi, \mathbb{B}, \mathcal{A}, \textsf{option}} = \negl.
\]

For the rest of this work, if the scheme $\Pi$, the family $\mathbb{B}$ of distributions, and the auxiliary information $\textsf{option}$ are clear from context, we omit the subscript and write the game as $\textsf{UF}(\mathcal{A})$. This abbreviation also holds for all other games.


%-------------------


\subsection{Indistinguishability}
\label{sec:ind_game}


In the game of indistinguishability, we model the ability of an authentication server who tries to identify the user, which describes the privacy leakage of the scheme. The adversary $\mathcal{A}$ is given oracles to two biometric distributions $\mathcal{B}^{(0)}$ and $ \mathcal{B}^{(1)}$, the comparison key $\textsf{csk}$, an enrollment message $\mathbf{c_x}$, and a list of $t$ probe messages $\{ \mathbf{c_y}^{(i)} \}_{i=1}^t$. It tries to guess from either $\mathcal{B}^{(0)}$ or $ \mathcal{B}^{(1)}$ these messages are generated. The whole game is defined in Algorithm \ref{alg:ind_game}.

\begin{figure}[h]
\centering

	\begin{minipage}[t]{0.55\textwidth}
	\begin{algorithm}[H]
	\caption{$\textsf{IND}_{\Pi, \mathbb{B}}(\mathcal{A})$}
	\label{alg:ind_game}
	\begin{algorithmic}[1]
		\State $b \getsdollar \{0, 1\}$

		\State $\mathcal{B}^{(0)} \getsdollar \mathbb{B}, \quad \mathbb{B} \gets \mathbb{B} \setminus \mathcal{B}^{(0)}$

		\State $\mathcal{B}^{(1)} \getsdollar \mathbb{B}, \quad \mathbb{B} \gets \mathbb{B} \setminus \mathcal{B}^{(1)}$

		\State $\textsf{esk}, \textsf{psk}, \textsf{csk} \gets \textsf{Setup}(1^\lambda)$

		\State $\mathbf{b} \gets \textsf{getEnroll}^{\mathcal{O}_{\mathcal{B}^{(b)}}}()$

		\State $\mathbf{c_x} \gets \textsf{Enroll}(\textsf{esk}, \mathbf{b})$

		\For{$i = 1$ to $t$}

			\State ${\mathbf{b}^\prime}^{(i)} \gets \textsf{getProbe}^{\mathcal{O}_{\mathcal{B}^{(b)}}}() $
		
			\State $\mathbf{c_y}^{(i)} \gets \textsf{Probe}( \textsf{psk}, {\mathbf{b}^\prime}^{(i)} )$

		\EndFor

		%\State In Device-of-User Model:
		
			%\State \hspace{\algorithmicindent} $\tilde{b} \gets \mathcal{A}^{\mathcal{O}_{\mathcal{B}^{(0)}}, \mathcal{O}_{\mathcal{B}^{(1)}} } ( \textsf{csk}, \mathbf{c_x}, \{ \mathbf{c_y}^{(i)} \}_{i=1}^t )$

		%\State In Device-of-Domain Model:
		
			\State $\tilde{b} \gets \mathcal{A}^{\mathcal{O}_{\mathcal{B}^{(0)}}, \mathcal{O}_{\mathcal{B}^{(1)}}} ( \textsf{csk}, \mathbf{c_x}, \{ \mathbf{c_y}^{(i)} \}_{i=1}^t )$

		\State \Return $1_{\tilde{b} = b}$
	\end{algorithmic}
	\end{algorithm}
	\end{minipage}

%\caption{The \textsf{IND} Game}
\label{fig:ind_game}
\end{figure}

%Note that in Device-of-Domain model, a probe oracle is given to the adversary.

%\begin{itemize}

	%\item $\mathcal{O}_{\textsf{Probe}}^{\textsf{samp}}(\cdot)$: On input an index $i$, it first samples $\mathbf{y}^\prime \getsdollar \textsf{encodeProbe}^{\mathcal{O}_{\textsf{samp}}(i)}$, which uses $\mathcal{O}_{\textsf{samp}}(i)$ to answer biometric queries, and outputs $\textsf{Probe}(\textsf{psk}, \mathbf{y}^\prime )$. 

%\end{itemize}

%We provide $\mathcal{O}_{\textsf{Probe}}^{\textsf{samp}}(\cdot)$ instead of $\mathcal{O}_{\textsf{Probe}}(\textsf{psk}, \cdot)$. This is to avoid the trivial attack where the adversary probes samples from the oracles $\mathcal{O}_{\mathcal{B}^{(0)}}$ and $\mathcal{O}_{\mathcal{B}^{(1)}}$ and compare the results with $\mathbf{c_x}$.

We define the advantage of an adversary $\mathcal{A}$ in the \textsf{IND} game of a scheme $\Pi$ associated with a family of distributions $\mathbb{B}$ as
\[
	\Adv^{\textsf{IND}}_{\Pi, \mathbb{B}, \mathcal{A}} := \left |\Pr[\textsf{IND}_{\Pi}(\mathcal{A}) \to 1] - \frac{1}{2} \right|.
\]

An authentication scheme $\Pi$ associated with a family $\mathbb{B}$ of distributions is called \emph{indistinguishable (IND)} if for any PPT adversary $\mathcal{A}$,
\[
	\Adv^{\textsf{IND}}_{\Pi, \mathbb{B}, \mathcal{A}} = \negl.
\]


Let $\textsf{Sig} = (\textsf{Sig.KeyGen}, \allowbreak \textsf{Sig.Sign}, \textsf{Sig.Verify})$ be an sEUF-CMA digital signature scheme, consider the following authentication scheme. Let $\textsf{esk}$ be empty, $\textsf{psk}$ be the signing secret key $\textsf{sk}_\textsf{Sig}$, and $\textsf{csk}$ be the verification public key $\textsf{pk}_\textsf{Sig}$. Let 
\begin{gather*}
	\textsf{Enroll}(\textsf{esk}, \mathbf{b}) \to \mathbf{b}, \quad \textsf{Probe}(\textsf{psk}, \mathbf{b}^\prime ) \to (\mathbf{b}^\prime , \sigma = \textsf{Sig.Sign}(\textsf{sk}_\textsf{Sig}, \mathbf{b}^\prime ) ) \\
	\textsf{Compare}(\textsf{csk}, \mathbf{b}, (\mathbf{b}^\prime, \sigma)) = \begin{cases}
		\textsf{BioCompare}(\mathbf{b}, \mathbf{b}^\prime ) & \text{if } \textsf{Sig.Verify}(\textsf{pk}_\textsf{Sig}, \mathbf{b}^\prime, \sigma) = 1 \\
		\bot & \text{if } \textsf{Sig.Verify}(\textsf{pk}_\textsf{Sig}, \mathbf{b}^\prime, \sigma) = 0
	\end{cases}
\end{gather*}

An $\textsf{UF}_\textsf{option}$ adversary has to forge a signature $\sigma$ to win the game, so the scheme is $\textsf{option}$-UF for any $\textsf{option}$ that does not include $\textsf{psk}$. However, the enrollment and probe messages leak biometric vectors $\mathbf{b}$ and $\mathbf{b}^\prime$. Obviously, this scheme is not IND, and we use this example emphasize the necessity of the game of indistinguishability.

%-------------------

\section{Security Analysis: fh-IPFE-based Instantiation}
\label{sec:security_analysis:fh-IPFE}

Let $\Pi$ be an authentication scheme instantiated by an fh-IPFE scheme \textsf{FE} as in Section \ref{sec:fh-IPFE-instantiation}. We discuss the UF and IND security of $\Pi$ in this section. For this, we first define two security notions of \textsf{FE}.

\subsection{fh-IND Security of \textsf{FE}}

Given an fh-IPFE scheme \textsf{FE}, we define the \textsf{fh-IND} game \cite{cryptoeprint:2016/440} in Algorithm \ref{alg:ind-fh-IPFE}.

\begin{figure}[h]
\centering

	\begin{minipage}[t]{0.4\textwidth}
	\begin{algorithm}[H]
	\caption{$\textsf{fh-IND}_{\textsf{FE}}(\mathcal{A})$}
	\label{alg:ind-fh-IPFE}
	\begin{algorithmic}[1]
		\State $b \getsdollar \{0, 1\}$

		\State $\textsf{msk}, \textsf{pp} \gets \textsf{FE.Setup}(1^\lambda)$

		\State $\tilde{b} \gets \mathcal{A}^{\mathcal{O}_{\textsf{KeyGen}}, \mathcal{O}_{\textsf{Enc}}} ( \textsf{pp} )$

		\State \Return $1_{\tilde{b} = b}$
	\end{algorithmic}
	\end{algorithm}
	\end{minipage}

\label{fig:ind-fh-IPFE}
\end{figure}

\begin{itemize}

	\item $\mathcal{O}_{\textsf{KeyGen}}(\cdot, \cdot)$: On input pair $(\mathbf{x}^{(0)}, \mathbf{x}^{(1)})$, it outputs $\textsf{FE.KeyGen}(\textsf{msk}, \textsf{pp}, \mathbf{x}^{(b)} )$.

	\item $\mathcal{O}_{\textsf{Enc}}(\cdot, \cdot)$: On input pair $(\mathbf{y}^{(0)}, \mathbf{y}^{(1)})$, it outputs $\textsf{FE.Enc}(\textsf{msk}, \textsf{pp}, \mathbf{y}^{(b)} )$.

\end{itemize}

\noindent To avoid trivial attacks, we consider \emph{admissible adversaries}.

\begin{definition}[Admissible Adversary]

	Let $\mathcal{A}$ be an adversary in an \textsf{fh-IND} game, and let $ (\mathbf{x}_1^{(0)}, \mathbf{x}_1^{(1)}), \cdots, (\mathbf{x}_{Q_K}^{(0)}, \mathbf{x}_{Q_K}^{(1)})$ be its queries to $\mathcal{O}_{\textsf{KeyGen}}$ and $(\mathbf{y}_1^{(0)}, \mathbf{y}_1^{(1)}), \cdots, (\mathbf{y}_{Q_E}^{(0)}, \mathbf{y}_{Q_E}^{(1)})$ be its queries to $\mathcal{O}_{\textsf{Enc}}$.
	We say $\mathcal{A}$ is \emph{admissible} if $\forall i \in [Q_K], \forall j \in [Q_E]$,
\[
	{\mathbf{x}^{(0)}_{i}} {\mathbf{y}^{(0)}_{j}}^T = {\mathbf{x}^{(1)}_{i}} {\mathbf{y}^{(1)}_{j}}^T
\]

\end{definition}


\begin{definition}[fh-IND Security]

	An fh-IPFE scheme \textsf{FE} is called fh-IND secure if for any admissible adversary $\mathcal{A}$, the advantage of $\mathcal{A}$ in the $\textsf{fh-IND}$ game in Algorithm \ref{alg:ind-fh-IPFE} is

\[
	\Adv_{\textsf{FE}, \mathcal{A}}^{\textsf{fh-IND}} := \left| \Pr[\textsf{fh-IND}_{\textsf{FE}}(\mathcal{A}) \to 1 ] - \frac{1}{2} \right| = \negl.
\]

\end{definition}

We note that fh-IND security is a standard notion for an fh-IPFE, and constructions in \cite{cryptoeprint:2015/1255, 10.1007/978-3-319-45871-7_24, cryptoeprint:2016/440} are proven fh-IND. However, fh-IND security may not be sufficient for the UF security of the instantiation in Section \ref{sec:fh-IPFE-instantiation}. 

\begin{theorem}

An instantiation $\Pi$ using the construction in \cite{cryptoeprint:2016/440} is not \textsf{option}-UF for any \textsf{option}.

\end{theorem}

\noindent We recall the construction in \cite{cryptoeprint:2016/440} in Appendix \ref{sec:fh-IPFE-construction}.

\begin{proof}

Let $\mathcal{A}$ be a \textsf{UF} game adversary that returns $(K_1, K_2) = (1, (1, \cdots, 1))$. Then, in the decryption,
\[
	D_1 = e(g_1, g_2)^{0} = 1 \quad \text{and} \quad D_2 = e(g_1, g_2)^0 = 1
\]
As $D_1^0 = D_2$, the decryption returns $0$ and let the adversary win the game with probability $1$.

\end{proof}

%-------------------

\subsection{RUF Security of \textsf{FE}}

We also define the $\textsf{RUF}^{\mathcal{O}, \gamma}_\textsf{FE}$ game in Algorithm \ref{alg:ruf-fh-IPFE} for a real number $\gamma$.

\begin{figure}[h]
\centering

	\begin{minipage}[t]{0.55\textwidth}
	\begin{algorithm}[H]
	\caption{$\textsf{RUF}^{\mathcal{O}, \gamma}_{\textsf{FE}}(\mathcal{A})$}
	\label{alg:ruf-fh-IPFE}
	\begin{algorithmic}[1]
		\State $\mathbf{r} \getsdollar \mathbb{F}^{k}$ \label{alg:oracle-ruf-fh-IPFE:r}

		\State $\textsf{msk}, \textsf{pp} \gets \textsf{FE.Setup}(1^\lambda)$

		\State $\mathbf{c} \gets \textsf{FE.KeyGen}(\textsf{msk}, \textsf{pp}, \mathbf{r})$

		\State $\mathbf{\tilde{z}} \gets \mathcal{A}^{\mathcal{O}} ( \textsf{pp}, \mathbf{c} )$

		\If{$\mathbf{\tilde{z}}$ is equal to any output of $\mathcal{O}^\prime_{\textsf{Enc}}$ }
			
			\State \Return $0$
		
		\EndIf

		\State $s \gets \textsf{FE.Dec}(\textsf{pp}, \mathbf{c}, \mathbf{\tilde{z}} )$

		\State \Return $1_{s \leq \gamma}$
	\end{algorithmic}
	\end{algorithm}
	\end{minipage}

\end{figure}

The oracle $\mathcal{O}$ can be nothing or include the following options based on the threat model.

\begin{itemize}

	\item $\mathcal{O}^\prime_{\textsf{KeyGen}}(\cdot)$: On input $\mathbf{x}^\prime$, it outputs $\textsf{FE.KeyGen}(\textsf{msk}, \textsf{pp}, \mathbf{x}^\prime)$.
	
	\item $\mathcal{O}^\prime_{\textsf{Enc}}(\cdot)$: On input $\mathbf{y}^\prime$, it outputs $\textsf{FE.Enc}(\textsf{msk}, \textsf{pp}, \mathbf{y}^\prime)$. The adversary is required to return $\mathbf{\tilde{z}}$ that is not equal to any output of this oracle.
\end{itemize}

\begin{definition}[RUF Security]

	An fh-IPFE scheme \textsf{FE} is called $\mathcal{O}$-RUF secure for a real number $\gamma$ if for any adversary $\mathcal{A}$, the advantage of $\mathcal{A}$ in the $\textsf{RUF}^{\mathcal{O}, \gamma}_\textsf{FE}$ game in Algorithm \ref{alg:ruf-fh-IPFE} is

\[
	\Adv_{\textsf{FE}, \mathcal{A}}^{\textsf{RUF}, \mathcal{O}, \gamma} := \Pr[\textsf{RUF}^{\mathcal{O}, \gamma}_{\textsf{FE}}(\mathcal{A}) \to 1 ] = \negl.
\]

\noindent We say $\textsf{FE}$ is RUF secure if it is $\{ \mathcal{O}^\prime_{\textsf{KeyGen}}, \mathcal{O}^\prime_{\textsf{Enc}} \}$-RUF secure.

\end{definition}


\subsubsection{Achievability of RUF Security}

We note that RUF security is a new security notion of fh-IPFE. In this section, we provide two theorems to obtain an $\mathcal{O}^\prime_{\textsf{KeyGen}}$-RUF and an RUF scheme, respectively.

\begin{assumption}
\label{assump}
Let $\mathbf{x} \in \mathbb{F}^k, \mathbf{c} \gets \textsf{FE.KeyGen}(\textsf{msk}, \textsf{pp}, \mathbf{x})$. Assume that $\textsf{FE.Dec}(\textsf{pp}, \mathbf{c}, \mathbf{z})$ only returns when $\mathbf{z}$ corresponds to a \emph{nonzero} vector $\mathbf{v} \in \mathbb{F}^k$. That is, assume that for any $\mathbf{z}$, there can only be two possibilities.

\begin{itemize}
	\item There exists a vector $\mathbf{v} \in \mathbb{F}^k \setminus \{\mathbf{0}\}$ such that for any $\mathbf{x} \in \mathbb{F}^k, \mathbf{c} \gets \textsf{FE.KeyGen}(\textsf{msk}, \textsf{pp}, \mathbf{x})$, and $\mathbf{c_v} \gets \textsf{FE.KeyGen}(\textsf{msk}, \textsf{pp}, \mathbf{v})$, 
	\[
		\textsf{FE.Dec}(\textsf{pp}, \mathbf{c}, \mathbf{z}) = \textsf{FE.Dec}(\textsf{pp}, \mathbf{c}, \mathbf{c_v}).
	\]
	\item For any $\mathbf{x} \in \mathbb{F}^k$ and $ \mathbf{c} \gets \textsf{FE.KeyGen}(\textsf{msk}, \textsf{pp}, \mathbf{x})$, $\textsf{FE.Dec}(\textsf{pp}, \mathbf{c}, \mathbf{z}) \to \bot$.

\end{itemize}
Note that this implies $\textsf{FE}$ rejects zero vector $\mathbf{0}$ as the input of $\textsf{FE.Enc}$.
\end{assumption}

\begin{theorem}
\label{thm:fh-IPFE:ind-ruf}
Given Assumption \ref{assump}. If \textsf{FE} is fh-IND, then $\textsf{FE}$ is $\mathcal{O}^\prime_{\textsf{KeyGen}}$-RUF for any $\gamma \leq \|\mathbb{F}\|$.

\end{theorem}

\begin{proof}
Given an adversary $\mathcal{A}$ in the $\textsf{RUF}^{\mathcal{O}^\prime_{\textsf{KeyGen}}, \gamma}_{\textsf{FE}}$ game for any $\gamma < \|\mathbb{F}\|$. Let $t$ be an integer, consider the reduction adversary $\mathcal{R}$ in Algorithm \ref{alg:red:ind-ruf}. $\mathcal{R}$ simulates $\mathcal{O}_\textsf{KeyGen}^\prime(\mathbf{x}^\prime)$ by $\mathcal{O}_\textsf{KeyGen}(\mathbf{x}^\prime, \mathbf{x}^\prime)$.
If there exists an $s_i \neq \bot$ in Line \ref{alg:red:ind-ruf:s}, by Assumption \ref{assump}, let $\mathbf{\tilde{z}}$ correspond to a vector $\mathbf{\tilde{v}}$.

\begin{figure}[h]
\centering
	
	\begin{minipage}[t]{0.5\linewidth}
	\centering
	\begin{algorithm}[H]
	\caption{$\mathcal{R}^{\mathcal{O}_{\textsf{KeyGen}}, \mathcal{O}_{\textsf{Enc}}}(\textsf{pp})$}
	\label{alg:red:ind-ruf}
	\begin{algorithmic}[1]
		\State $\mathbf{r}^{(0)}, \mathbf{r}^{(1)} \getsdollar \mathbb{F}^{k}$
		
		\State $\mathbf{c} \gets \mathcal{O}_{\textsf{KeyGen}}(\mathbf{r}^{(0)}, \mathbf{r}^{(1)})$ 

		\State ${\mathbf{\tilde{z}}} \gets {\mathcal{A}}^{\mathcal{O}^\prime_{\textsf{KeyGen}}} (\textsf{pp}, \mathbf{c})$

		\For{$i = 1$ to $t$}
		
			\State $\mathbf{r}_i \getsdollar \mathbb{F}^{k}$

			\State $\mathbf{c}_i \gets \mathcal{O}_{\textsf{KeyGen}}(\mathbf{r}^{(0)}, \mathbf{r}_i)$

			\State $s_i \gets \textsf{FE.Dec}( \textsf{pp}, \mathbf{c}_i, \mathbf{\tilde{z}} )$ \label{alg:red:ind-ruf:s}
	
		\EndFor	
		
		\If{$\bigwedge_{i=1}^t s_i \leq \gamma$} \label{alg:red:ind-ruf:verify}
			\State \Return $\tilde{b} = 0$
		\Else
			\State \Return $\tilde{b} \getsdollar \{0, 1\}$
		\EndIf

	\end{algorithmic}
	\end{algorithm}
	\end{minipage}
	
\end{figure}


If the challenge bit $b = 0$, then by Assumption \ref{assump}, any $s_i \neq \bot$ in Line \ref{alg:red:ind-ruf:s} implies all $s_i \neq \bot$ and $s_i = s_j$ for any $i, j$. Therefore, the probability that all $s_i \leq \gamma$ in Line \ref{alg:red:ind-ruf:verify} is
\begin{align*}
	\Pr\left[ \bigwedge_{i=1}^t s_i \leq \gamma \mid b = 0 \right]
	&= \Pr\left[ s_1 \neq \bot \mid b = 0 \right] \cdot \Pr \left[ s_1 \leq \gamma \mid b = 0 \wedge s_1 \neq \bot \right] \\
	&= \Pr \left[ s_1 \neq \bot \mid b = 0 \right] \cdot \Pr \left[ \mathbf{r}^{(0)} \mathbf{\tilde{v}}^T \leq \gamma \mid b = 0 \wedge s_1 \neq \bot \right] \\
	&= \Pr \left[s_1 \neq \bot \mid b = 0 \right] \cdot \Pr \left[ \textsf{FE.Dec}(\textsf{pp}, \mathbf{c}, \mathbf{\tilde{z}}) \leq \gamma \mid b = 0 \wedge s_1 \neq \bot \right] \\
	&= \Pr \left[s_1 \neq \bot \mid b = 0 \right] \cdot \Pr \left[ \textsf{RUF}^{\mathcal{O}^\prime_{\textsf{KeyGen}}, \gamma}(\mathcal{A}) \to 1 \mid b = 0 \wedge s_1 \neq \bot \right] \\ 
	&= \Pr \left[ \textsf{RUF}^{\mathcal{O}^\prime_{\textsf{KeyGen}}, \gamma}(\mathcal{A}) \to 1 \right] 
\end{align*}

If the challenge bit $b = 1$, for any $i \in [t]$,
\begin{align*}
	\Pr[ s_i \leq \gamma \mid b = 1 ]
	&= \Pr[s_i \neq \bot \mid b = 1] \cdot \Pr[ s_i \leq \gamma \mid b = 1 \wedge s_i \neq \bot] \\
	&= \Pr[s_i \neq \bot \mid b = 1] \cdot \Pr[ \mathbf{r}_i \mathbf{\tilde{v}}^T \leq \gamma \mid b = 1 \wedge s \neq \bot]
\end{align*}
Note that $\mathbf{r}_i$ is independent of $\mathbf{\tilde{z}}$ and thus independent of $\mathbf{\tilde{v}}$. Hence, $\Pr[\mathbf{r}_i \mathbf{\tilde{v}}^T \leq \gamma \mid b = 1 \wedge s_i \neq \bot] = \frac{\gamma}{\| \mathbb{F} \|}$ and
\[
	\Pr \left[ \bigwedge_{i=1}^t s_i \leq \gamma \mid b = 1 \right] = \Pr \left[ \bigwedge_{i=1}^t s_i \neq \bot \mid b = 1 \right] \cdot \left( \frac{\gamma}{\| \mathbb{F} \|} \right)^t \leq \left( \frac{\gamma}{\| \mathbb{F} \|} \right)^t
\]

In conclusion,
\begin{align*}
	\Pr[\textsf{fh-IND}(\mathcal{R}) \to 1] 
	&= \frac{1}{2} + \frac{1}{4} \left( \Pr \left[ \bigwedge_{i=1}^t s_i \leq \gamma \mid b = 0 \right] - \Pr \left[ \bigwedge_{i=1}^t s_i \leq \gamma \mid b = 1 \right] \right) \\
	&\geq \frac{1}{2} + \frac{1}{4} \left( \Pr[ \textsf{RUF}^{\mathcal{O}^\prime_{\textsf{KeyGen}}, \gamma}(\mathcal{A}) \to 1] - \left( \frac{\gamma}{\| \mathbb{F} \|} \right)^t \right) \\
	&\geq \frac{1}{2} + \frac{1}{4} \left( \Pr[ \textsf{RUF}^{\mathcal{O}^\prime_{\textsf{KeyGen}}, \gamma}(\mathcal{A}) \to 1] -  e^{- t \cdot (1 - \frac{\gamma}{\| \mathbb{F} \|}) } \right)
\end{align*}

\noindent Take $t$ be any integer larger than $\frac{\lambda}{ 1 - \frac{\gamma}{\| \mathbb{F} \| }}$. Since $\Adv_{\textsf{FE}, \mathcal{R}}^\textsf{fh-IND} = \left| \Pr[\textsf{fh-IND}(\mathcal{R}) \to 1] - \frac{1}{2} \right|$ and $e^{-t \cdot (1 - \frac{\gamma}{\| \mathbb{F} \|}) } \leq e^{-\lambda}$ are negligible,
\[
	\Pr[ \textsf{RUF}^{\mathcal{O}^\prime_{\textsf{KeyGen}}, \gamma}(\mathcal{A}) \to 1]  \leq e^{- t \cdot (1 - \frac{\gamma}{\| \mathbb{F} \|}) } + 4 \cdot \Adv_{\textsf{FE}, \mathcal{R}}^\textsf{fh-IND}  = \negl.
\]

\end{proof}


Let $\textsf{Sig} = ( \textsf{Sig.KeyGen}, \textsf{Sig.Sign}, \textsf{Sig.Verify} )$ be an sEUF-CMA signature scheme. By adding $\textsf{Sig}$, an fh-IPFE scheme $\textsf{FE}$ can be upgraded to an RUF scheme $\textsf{FE}^\prime$. 

\begin{itemize}

	\item $\textsf{FE}^\prime \textsf{.Setup}(1^\lambda)$: Run $\textsf{FE.Setup}(1^\lambda) \to (\textsf{msk}, \textsf{pp})$ and $\textsf{Sig.KeyGen}(1^\lambda) \to (\textsf{sk}_{\textsf{Sig}}, \textsf{pk}_{\textsf{Sig}} )$. Output $\textsf{msk}^\prime = (\textsf{msk}, \textsf{sk}_\textsf{Sig})$ and $\textsf{pp}^\prime = (\textsf{pp}, \textsf{pk}_{\textsf{Sig}})$.

	\item $\textsf{FE}^\prime \textsf{.KeyGen}(\textsf{msk}^\prime, \mathbf{x})$: Run $\textsf{FE.KeyGen}(\textsf{msk}, \mathbf{x}) \to f_\mathbf{x}$ and output $f_\mathbf{x}$.

	\item $\textsf{FE}^\prime \textsf{.Enc}(\textsf{msk}^\prime, \mathbf{y})$: Run $\textsf{FE.Enc}(\textsf{msk}, \mathbf{y}) \to \mathbf{c_y}$ and sign $\mathbf{c_y}$ by $\textsf{Sig.Sign}(\textsf{sk}_{\textsf{Sig}}, \mathbf{c_y}) \to \sigma$. Output $\mathbf{c_y}^\prime = (\mathbf{c_y}, \sigma)$.

	\item $\textsf{FE}^\prime \textsf{.Dec}(\textsf{pp}^\prime, f_\mathbf{x}, \mathbf{c_y}^\prime )$: Output the decryption $\textsf{FE.Dec}(\textsf{pp}, f_\mathbf{x}, \mathbf{c_y})$ if the verification $\textsf{Sig.Verify}(\textsf{pk}_{\textsf{Sig}}, \mathbf{c_y}, \sigma ) = 1$. Otherwise, output $\bot$.

\end{itemize}

\begin{theorem}

For any fh-IPFE $\textsf{FE}$, $\textsf{FE}^\prime$ is an RUF fh-IPFE for any $\gamma$.

\end{theorem}

\begin{proof}

Given an adversary $\mathcal{A}$ in the $\textsf{RUF}^{\mathcal{O}_{\textsf{KeyGen}}^\prime, \mathcal{O}_{\textsf{Enc}}^\prime, \gamma}_{\textsf{FE}^\prime}$ game, consider the reduction adversary $\mathcal{R}$ in Algorithm \ref{alg:red:sEUF-CMA:RUF} which plays the \textsf{sEUF-CMA} game of $\textsf{Sig}$. $\mathcal{R}$ is given a verification public key $\textsf{pk}_{\textsf{Sig}}$ and a signing oracle $\mathcal{O}_{\textsf{Sig}}$ and returns a forged message-signature pair that is not equal to any previous answer of $\mathcal{O}_{\textsf{Sig}}$. To run $\mathcal{A}$, $\mathcal{R}$ simulates each oracle in the following way.

\begin{itemize}
	\item $\mathcal{O}_{\textsf{KeyGen}}^\prime(\mathbf{x}^\prime)$: Return $\textsf{FE.KeyGen}(\textsf{msk}, \mathbf{x})$.

	\item $\mathcal{O}_{\textsf{Enc}}^\prime(\mathbf{y}^\prime)$: Run $\textsf{FE.Enc}(\textsf{msk}, \mathbf{y}) \to \mathbf{c_y}$ and call the signing oracle $\mathcal{O}_{\textsf{Sign}}(\mathbf{c_y}) \to \sigma$. Output $\mathbf{c_y}^\prime = (\mathbf{c_y}, \sigma)$.
\end{itemize}

\begin{figure}[h]
\centering
	
	\begin{minipage}[t]{0.5\linewidth}
	\centering
	\begin{algorithm}[H]
	\caption{$\mathcal{R}^{\mathcal{O}_{\textsf{Sign}}}( \textsf{pk}_{\textsf{Sig}} )$}
	\label{alg:red:sEUF-CMA:RUF}
	\begin{algorithmic}[1]

		\State $\mathbf{r} \gets \mathbb{F}^k$

		\State $\textsf{msk}, \textsf{pp} \gets \textsf{FE.Setup}(1^\lambda)$

		\State $\mathbf{c} \gets \textsf{FE.KeyGen}(\textsf{msk}, \textsf{pp}, \mathbf{r})$

		\State $\textsf{pp}^\prime \gets (\textsf{pp}, \textsf{pk}_{\textsf{Sig}})$

		\State $\mathbf{\tilde{z}} \gets {\mathcal{A}}^{\mathcal{O}_{\textsf{KeyGen}}^\prime, \mathcal{O}_{\textsf{Enc}}^\prime } (\textsf{pp}^\prime, \mathbf{c})$ \label{alg:red:ind-uf-OB-Enroll:A}

		\State Parse $(\mathbf{c_z}, \sigma^\prime) \gets \mathbf{\tilde{z}}$

		\State \Return $(\mathbf{c_z}, \sigma^\prime)$
	\end{algorithmic}
	\end{algorithm}
	\end{minipage}
	
\end{figure}

$\mathcal{R}$ perfectly simulates a $\textsf{RUF}$ game for $\mathcal{A}$, and if $\mathcal{A}$ wins the $\textsf{RUF}$ game, $(\mathbf{c_z}, \sigma^\prime)$ is not equal to any previous answer of $\mathcal{O}_{\textsf{Enc}}^\prime$, and therefoere not equal to any previous message-signature pair $(\mathbf{c_y}, \sigma)$ given from the signing oracle $\mathcal{O}_{\textsf{Sign}}$. Now, since $\textsf{Sig}$ is sEUF-CMA,
\[
	\Pr[\textsf{RUF}^{ \mathcal{O}^\prime_{\textsf{KeyGen}}, \gamma}(\mathcal{A}) \to 1] \leq \Pr[ \textsf{Sig.Verify}(\textsf{pk}_{\textsf{Sig}}, \mathbf{c_z}, \sigma^\prime) = 1 ] = \negl.
\]


\end{proof}

%-------------------


\subsection{UF Security of $\Pi$}
\label{sec:security_analysis:fh-IPFE:uf}

We first consider \textsf{option}-UF security when $\textsf{option}$ includes $\mathcal{O}_\textsf{Enroll}$. Note that in this instantiation, $\textsf{csk}$ is the public parameter $\textsf{pp}$ of \textsf{FE} and assumed to be given to all adversaries. 

\begin{theorem}
\label{thm:fh-IPFE:ind-uf-OB-Enroll}
	Let $\textsf{option} = \{ \mathbf{c_x}, \textsf{csk}, \mathcal{O}_\mathcal{B}, \mathcal{O}_{\textsf{Enroll}} \}$. For any distribution family $\mathbb{B}$, if \textsf{FE} is fh-IND and $\mathcal{O}^\prime_{\textsf{KeyGen}}$-RUF for a $\gamma \geq \tau^2$, then $\Pi$ is $\textsf{option}$-UF. 
\end{theorem}


\begin{proof}

Given an adversary $\mathcal{A}$ in the $\textsf{UF}_{\textsf{option}}$ game, consider the reduction adversary $\mathcal{R}$ in Algorithm \ref{alg:red:ind-uf-OB-Enroll} which plays the \textsf{fh-IND} game. $\mathcal{R}$ runs $\mathcal{A}$ and simulates $\mathcal{O}_{\textsf{Enroll}}( \textsf{esk}, \mathbf{b}^\prime )$ by first encoding $\mathbf{b}^\prime = (b_1^\prime, \cdots, b_k^\prime)$ into $\mathbf{x}^\prime = (b_1^\prime, \cdots, b_k^\prime, 1, \|\mathbf{b}^\prime\|^2)$ and calling $\mathcal{O}_{\textsf{KeyGen}}(\mathbf{x}^\prime, \mathbf{x}^\prime)$ given in the \textsf{fh-IND} game.  Note that since $\mathcal{R}$ never calls $\mathcal{O}_{\textsf{Enc}}$, it is an admissible adversary.

\begin{figure}[h]
\centering
	
	\begin{minipage}[t]{0.5\linewidth}
	\centering
	\begin{algorithm}[H]
	\caption{$\mathcal{R}^{\mathcal{O}_{\textsf{KeyGen}}, \mathcal{O}_{\textsf{Enc}}}(\textsf{pp})$}
	\label{alg:red:ind-uf-OB-Enroll}
	\begin{algorithmic}[1]
		\State $\mathcal{B} \getsdollar \mathbb{B}, \quad \mathbb{B} \gets \mathbb{B} \setminus \mathcal{B}$ \label{alg:red:ind-uf-OB-Enroll:B}

		\State $\mathbf{b} = (b_1, \cdots, b_k) \gets \textsf{getEnroll}^{\mathcal{O}_{\mathcal{B}}}()$

		\State $\mathbf{x} \gets (b_1, \cdots, b_k, 1, \|\mathbf{b}\|^2)$

		\State $\mathbf{r} \getsdollar \mathbb{F}^{k+2}$
		
		\State $\mathbf{c} \gets \mathcal{O}_{\textsf{KeyGen}}(\mathbf{x}, \mathbf{r})$ \label{alg:red:ind-uf-OB-Enroll:c}

		\State ${\mathbf{\tilde{z}}} \gets {\mathcal{A}}^{\mathcal{O}_\mathcal{B}, \mathcal{O}_\textsf{Enroll} } ( \mathbf{c}, \textsf{pp})$ \label{alg:red:ind-uf-OB-Enroll:A}

		\State $s \gets \textsf{FE.Dec}( \textsf{pp}, \mathbf{c}, \mathbf{\tilde{z}} )$

		\If{$\textsf{Verify}(s) = 1$} \label{alg:red:ind-uf-OB-Enroll:verify}
			\State \Return $\tilde{b} = 0$
		\Else
			\State \Return $\tilde{b} \getsdollar \{0, 1\}$
		\EndIf

	\end{algorithmic}
	\end{algorithm}
	\end{minipage}
	
\end{figure}

	If the challenge bit $b = 0$, then $\mathcal{R}$ perfectly simulates a $\textsf{UF}_{\textsf{option}}$ game for $\mathcal{A}$. Therefore, the probability that $\textsf{Verify}(s) = 1$ in Line \ref{alg:red:ind-uf-OB-Enroll:verify} is $\Pr[\textsf{UF}_{\textsf{option}}(\mathcal{A}) \to 1]$.

	For the case when the challenge bit $b = 1$, consider an adversary $\mathcal{A}^\prime$ in Algorithm \ref{alg:adv:ind-uf-OB-Enroll} in the $\textsf{RUF}^{ \mathcal{O}^\prime_{\textsf{KeyGen}} }$ game. $\mathcal{A}^\prime$ runs Line \ref{alg:red:ind-uf-OB-Enroll:B} and \ref{alg:red:ind-uf-OB-Enroll:A} of $\mathcal{R}$ and simulates $\mathcal{O}_{\textsf{Enroll}}( \textsf{esk}, \mathbf{b}^\prime )$ by first encoding $\mathbf{b}^\prime$ into $\mathbf{x}^\prime$ as before and calling $\mathcal{O}_\textsf{KeyGen}^\prime(\mathbf{x}^\prime)$ given in the $\textsf{RUF}^{ \mathcal{O}^\prime_{\textsf{KeyGen}} }$ game. 


\begin{figure}[h]
\centering

	\begin{minipage}[t]{0.45\textwidth}
	\begin{algorithm}[H]
	\caption{${\mathcal{A}^\prime}^{\mathcal{O}^\prime_{\textsf{KeyGen}} }(\textsf{pp}, \mathbf{c})$}
	\label{alg:adv:ind-uf-OB-Enroll}
	\begin{algorithmic}[1]
		\State $\mathcal{B} \getsdollar \mathbb{B}, \quad \mathbb{B} \gets \mathbb{B} \setminus \mathcal{B}$ 
		
		\State $\mathbf{\tilde{z}} \gets {\mathcal{A}}^{\mathcal{O}_{\mathcal{B}}, \mathcal{O}_\textsf{Enroll} } (\mathbf{c}, \textsf{pp})$

		\State \Return $\mathbf{\tilde{z}}$
	\end{algorithmic}
	\end{algorithm}
	\end{minipage}

\end{figure}

Now, if the challenge bit $b = 1$, then $\mathcal{R}$ perfectly simulates $\mathcal{A}^\prime$ in the $\textsf{RUF}^{ \mathcal{O}^\prime_{\textsf{KeyGen}} }$ game. The probability that $\textsf{Verify}(s) = 1$, which is equivalent to $s \leq \tau^2$, in Line \ref{alg:red:ind-uf-OB-Enroll:verify} is $\Pr[\textsf{RUF}^{\mathcal{O}^\prime_{\textsf{KeyGen}}, \tau^2}_{\textsf{FE}}(\mathcal{A}^\prime) \to 1 ]$

In conclusion, since $\gamma \geq \tau^2$,
\begin{align*}
	\Pr[\textsf{fh-IND}(\mathcal{R}) \to 1] 
	&= \Pr[b = 0] \cdot \left( \Pr[\textsf{Verify}(s) = 1 \mid b = 0] + \frac{1}{2} \cdot \Pr[\textsf{Verify}(s) = 0 \mid b = 0] \right) \\
	&\quad + \Pr[b = 1] \cdot \frac{1}{2} \cdot \Pr[\textsf{Verify}(s) = 0 \mid b = 1] \\
	&= \frac{1}{2} + \frac{1}{4} \left( \Pr[\textsf{Verify}(s) = 1 \mid b = 0] - \Pr[\textsf{Verify}(s) = 1 \mid b = 1] \right) \\
	&= \frac{1}{2} + \frac{1}{4} \left( \Pr[\textsf{UF}_{\textsf{option}}(\mathcal{A}) \to 1] - \Pr[\textsf{RUF}^{\mathcal{O}^\prime_{\textsf{KeyGen}}, \tau^2}_{\textsf{FE}}(\mathcal{A}^\prime) \to 1 ] \right) \\
	&\geq \frac{1}{2} + \frac{1}{4} \left( \Pr[\textsf{UF}_\textsf{option}(\mathcal{A}) \to 1] - \Pr[\textsf{RUF}^{\mathcal{O}^\prime_{\textsf{KeyGen}}, \gamma}_{\textsf{FE}}(\mathcal{A}^\prime) \to 1 ] \right)
\end{align*}

\noindent Since both $\Adv_{\textsf{FE}, \mathcal{R}}^\textsf{fh-IND} = \left| \Pr[\textsf{fh-IND}(\mathcal{R}) \to 1] - \frac{1}{2} \right|$ and $\Adv_{\textsf{FE}, \mathcal{A}^\prime}^{\textsf{RUF},\mathcal{O}^\prime_{\textsf{KeyGen}}, \gamma} = \Pr[\textsf{RUF}^{\mathcal{O}^\prime_{\textsf{KeyGen}}, \gamma}_{\textsf{FE}}(\mathcal{A}^\prime) \to 1 ]$ are negligilbe,
\[
	\Pr[\textsf{UF}_\textsf{option}(\mathcal{A}) \to 1] \leq 4 \cdot \Adv_{\textsf{FE}, \mathcal{R}}^\textsf{fh-IND} + \Adv_{\textsf{FE}, \mathcal{A}^\prime}^{\textsf{RUF},\mathcal{O}^\prime_{\textsf{KeyGen}}, \gamma} = \negl.
\]

\end{proof}


For $\textsf{option}$ that includes $\mathcal{O}_{\textsf{Probe}}$, we first note that for any $d \in \Z_q$ and any nonzero vector $\mathbf{r} \in \Z_q^{k+2}$, there exists a vector $\mathbf{y} \in \Z_q^{k+2}$ such that $\mathbf{r}\mathbf{y}^T = d$.

\begin{theorem}
\label{thm:fh-IPFE:ind-uf-OB-Probe}
	Let $\textsf{option} = \{\mathbf{c_x}, \textsf{csk}, \mathcal{O}_\mathcal{B}, \mathcal{O}_\textsf{Probe}\}$. For any distribution family $\mathbb{B}$, if \textsf{FE} is fh-IND and $\mathcal{O}^\prime_{\textsf{Enc}}$-RUF for a $\gamma \geq \tau^2$, then $\Pi$ is $\textsf{option}$-UF. 
\end{theorem}


\begin{proof}

Given an adversary $\mathcal{A}$ in the $\textsf{UF}_\textsf{option}$ game, consider the reduction adversary $\mathcal{R}$ in Algorithm \ref{alg:red:ind-uf-OB-Probe} which plays the \textsf{fh-IND} game. $\mathcal{R}$ runs $\mathcal{A}$ and simulates $\mathcal{O}_{\textsf{Probe}}$ in the following way.

\begin{itemize}

	\item $\mathcal{O}_{\textsf{Probe}}( \textsf{psk}, \mathbf{b}^\prime )$: On input $\mathbf{b}^\prime = (b_1^\prime, \cdots, b_k^\prime)$, it first encodes it as $\mathbf{y}^\prime = (-2b_1^\prime, \cdots, \allowbreak -2b_k^\prime, \|\mathbf{b}^\prime\|^2, 1)$. Next, it computes $d \gets \mathbf{x}{\mathbf{y}^\prime}^T$ and finds a vector $\mathbf{y}^{\prime\prime}$ such that $\mathbf{r}{\mathbf{y}^{\prime\prime}}^T = d$. Finally, it calls $\mathcal{O}_{\textsf{Enc}}(\mathbf{y}^\prime, {\mathbf{y}^{\prime\prime}})$, which is given by the \textsf{fh-IND} game, and returns the result.

\end{itemize}

\noindent Note that $(\mathbf{x}, \mathbf{r})$ is the only query of $\mathcal{R}$ to $\mathcal{O}_{\textsf{KeyGen}}$, and for any query $( \mathbf{y}^\prime, {\mathbf{y}^{\prime\prime}} )$ to $\mathcal{O}_{\textsf{Enc}}$, it satisfies $\mathbf{x}{\mathbf{y}^\prime}^T = \mathbf{r}{\mathbf{y}^{\prime\prime}}^T$. Hence, $\mathcal{R}$ is an admissible adversary.

\begin{figure}[h]
\centering
	
	\begin{minipage}[t]{0.6\linewidth}
	\centering
	\begin{algorithm}[H]
	\caption{$\mathcal{R}^{\mathcal{O}_{\textsf{KeyGen}}, \mathcal{O}_{\textsf{Enc}}}(\textsf{pp})$}
	\label{alg:red:ind-uf-OB-Probe}
	\begin{algorithmic}[1]
		\State $\mathcal{B} \getsdollar \mathbb{B}, \quad \mathbb{B} \gets \mathbb{B} \setminus \mathcal{B}$ \label{alg:red:ind-uf-OB-Probe:B}

		\State $\mathbf{b} = (b_1, \cdots, b_k) \gets \textsf{getEnroll}^{\mathcal{O}_{\mathcal{B}}}()$

		\State $\mathbf{x} \gets (b_1, \cdots, b_k, 1, \|\mathbf{b}\|^2)$

		\State $\mathbf{r} \getsdollar \mathbb{F}^{k+2}$

		\State $\mathbf{c} \gets \mathcal{O}_{\textsf{KeyGen}}(\mathbf{x}, \mathbf{r})$

		\State ${\mathbf{\tilde{z}}} \gets \mathcal{A}^{\mathcal{O}_{\mathcal{B}}, \mathcal{O}_{\textsf{Probe}} } ( \mathbf{c}, \textsf{pp})$

		\If{$\mathbf{\tilde{z}}$ is equal to any output of $\mathcal{O}_\textsf{Probe}$}

			\State \Return $\bot$

		\EndIf

		\State $s \gets \textsf{FE.Dec}( \textsf{pp}, \mathbf{c}, \mathbf{\tilde{z}} )$

		\If{$\textsf{Verify}(s) = 1$} \label{alg:red:ind-uf-OB-Probe:verify}
			\State \Return $\tilde{b} = 0$
		\Else
			\State \Return $\tilde{b} \getsdollar \{0, 1\}$
		\EndIf

	\end{algorithmic}
	\end{algorithm}
	\end{minipage}
	
\end{figure}

If the challenge bit $b = 0$, then $\mathcal{R}$ perfectly simulates a $\textsf{UF}_\textsf{option}$ game for $\mathcal{A}$. Therefore, the probability that $\textsf{Verify}(s) = 1$ in Line \ref{alg:red:ind-uf-OB-Probe:verify} is $\Pr[\textsf{UF}_\textsf{option}(\mathcal{A}) \to 1]$.

For the case when the challenge bit $b = 1$, consider an adversary $\mathcal{A}^\prime$ in Algorithm \ref{alg:adv:ind-uf-OB-Probe} in the $\textsf{RUF}^{ \mathcal{O}^\prime_{\textsf{Enc}} }$ game. $\mathcal{A}^\prime$ runs $\mathcal{A}$ and simulates $\mathcal{O}_{\textsf{Probe}}$ in the following way.

\begin{itemize}

	\item $\mathcal{O}_{\textsf{Probe}}( \textsf{psk}, \mathbf{b}^\prime )$: It first encodes $\mathbf{b}^\prime$ into $\mathbf{y}^\prime$ as before. Next, it computes $d \gets \mathbf{x}^{(*)}{\mathbf{y}^\prime}^T$ and finds a vector $\mathbf{y}^{\prime\prime}$ such that $\mathbf{r}{\mathbf{y}^{\prime\prime}}^T = d$. Finally, it calls $\mathcal{O}^\prime_{\textsf{Enc}} (\mathbf{y}^{\prime\prime} )$ , which is given by the $\textsf{RUF}^{ \mathcal{O}^\prime_{\textsf{Enc}} }$ game, and returns the result.

\end{itemize}

\begin{figure}[h]
\centering
	
	\begin{minipage}[t]{0.8\linewidth}
	\centering
	\begin{algorithm}[H]
	\caption{$ {\mathcal{A}^\prime}^{ \mathcal{O}^\prime_{\textsf{Enc}} } (\textsf{pp}, \mathbf{c}) $}
	\label{alg:adv:ind-uf-OB-Probe}
	\begin{algorithmic}[1]
		\State $\mathcal{B} \getsdollar \mathbb{B}, \quad \mathbb{B} \gets \mathbb{B} \setminus \mathcal{B}$ \label{alg:adv:ind-uf-OB-Probe:B}
		
		\State $\mathbf{x}^{(*)} \gets \textsf{encodeEnroll}^{ \mathcal{O}_{\mathcal{B}} } ()$

		\State Sample $k+2$ linearly independent vectors $\{ \mathbf{e}^{(i)} \}_{i=1}^{k+2}$.

		\For{$i=1$ to $k+2$}
			\State $\mathbf{c}^{(i)} \gets \mathcal{O}^\prime_{\textsf{Enc}}(\mathbf{e}^{(i)})$.

			\State $d_i \gets \textsf{FE.Dec}(\textsf{pp}, \mathbf{c}, \mathbf{c}^{(i)})$.
		\EndFor

		\State Find the vector $\mathbf{r}$ by solving the linear system $\{ \mathbf{r} {\mathbf{e}^{(i)}}^T = d_i \}_{i=1}^{k+2}$.

		\If{$\mathbf{r} = \mathbf{0}$}

			\State \Return $\bot$

		\EndIf

		\State ${\mathbf{\tilde{z}}} \gets \mathcal{A}^{\mathcal{O}_{\mathcal{B}}, \mathcal{O}_{\textsf{Probe}} } ( \mathbf{c}, \textsf{pp})$
		
		\State \Return ${\mathbf{\tilde{z}}}$
	\end{algorithmic}
	\end{algorithm}
	\end{minipage}
	
\end{figure}

To make $\mathcal{R}$ simulate $\mathcal{A}^\prime$ in the $\textsf{RUF}^{ \mathcal{O}^\prime_{\textsf{Enc}} }$ game, we still need to ensure two conditions.

\begin{itemize}

	\item $\mathbf{r} \neq \mathbf{0}$. Otherwise, $\mathcal{A}^\prime$ cannot simulate $\mathcal{O}_\textsf{Probe}$. 

	\item $\mathbf{\tilde{z}} \neq \mathbf{c}^{(i)}$ for all $i$. The answers of $\mathcal{O}_\textsf{Probe}$ have already been checked in $\mathcal{R}$. 
\end{itemize}

Let $\mathcal{A}^\prime$ play a tweaked $\textsf{RUF}_\textsf{FE}^{\mathcal{O}^\prime_{\textsf{Enc}}, \tau^2}$ game which does not check that $\mathbf{\tilde{z}}$ is not equal to $\mathbf{c}^{(i)}$ for all $i$. That is, the game only checks whether $\mathbf{\tilde{z}}$ is not equal to any output of $\mathcal{O}^\prime_\textsf{Enc}$ called by $\mathcal{O}_\textsf{Probe}$ of $\mathcal{A}$. Let the returned value of this game be $V$. We have Equation \ref{equ:ind-uf-OB-Probe:1} and \ref{equ:ind-uf-OB-Probe:2}. The former one is a relation between $\mathcal{R}$ playing $\textsf{fh-IND}$ game when the challenge bit $b=1$ and $V$, and the latter is a relation between $\mathcal{A}^\prime$ playing a regular $\textsf{RUF}_\textsf{FE}^{\mathcal{O}^\prime_{\textsf{Enc}}, \tau^2}$ game and the tweaked one.

\begin{gather}
	\Pr[\textsf{Verify}(s) = 1 \mid b = 1 \wedge \mathbf{r} \neq \mathbf{0}] = \Pr[V = 1] \label{equ:ind-uf-OB-Probe:1} \\
	\Pr[\textsf{RUF}_\textsf{FE}^{\mathcal{O}^\prime_{\textsf{Enc}}, \tau^2}(\mathcal{A}^\prime) \to 1] = \Pr\left[ V = 1 \mid \bigwedge_{i=1}^{k+2} \mathbf{\tilde{z}} \neq \mathbf{c}^{(i)} \right] \label{equ:ind-uf-OB-Probe:2}
\end{gather}

\noindent For Equation \ref{equ:ind-uf-OB-Probe:1}, consider that

\begin{align*}
	\Pr[\textsf{Verify}(s) = 1 \mid b = 1]
	&= \Pr[\textsf{Verify}(s) = 1 \mid b = 1 \wedge \mathbf{r} \neq \mathbf{0}] \cdot \Pr[\mathbf{r} \neq \mathbf{0}] \\
	&+ \Pr[\textsf{Verify}(s) = 1 \mid b = 1 \wedge \mathbf{r} = \mathbf{0}] \cdot \Pr[\mathbf{r} = \mathbf{0}] \\
	&\leq \Pr[V = 1] + \Pr[\mathbf{r} = 0] \\
	&= \Pr[V = 1] + \frac{1}{q^{k+2}} 
\end{align*}

\noindent For Equation \ref{equ:ind-uf-OB-Probe:2}, consider that

\begin{align*}
	\Pr[\textsf{RUF}_\textsf{FE}^{\mathcal{O}^\prime_{\textsf{Enc}}, \tau^2}(\mathcal{A}^\prime) \to 1] 
	&= \Pr\left[ V = 1 \mid \bigwedge_{i=1}^{k+2} \mathbf{\tilde{z}} \neq \mathbf{c}^{(i)} \right] \\ 
	& \geq \Pr[V = 1] - \Pr \left[ \neg  \left( \bigwedge_{i=1}^{k+2} \mathbf{\tilde{z}} \neq \mathbf{c}^{(i)} \right) \right] \\
	& = \Pr[V = 1] - \Pr \left[ \bigvee_{i=1}^{k+2} \mathbf{\tilde{z}} = \mathbf{c}^{(i)} \right] \\
	& \geq \Pr[V = 1] - \sum_{i=1}^{k+2} \Pr[\mathbf{\tilde{z}} = \mathbf{c}^{(i)}].
\end{align*}

\noindent Note that each $\mathbf{c}^{(i)} = \textsf{FE.Enc}(\textsf{msk}, \textsf{pp}, \mathbf{e}^{(i)})$ for some uniform nonzero vector $\mathbf{e}^{(i)}$. Also note that distinct vectors in $\Z_q^{k+2}$ will have different encryptions due to the correctness of $\textsf{FE}$. Therefore, $\Pr[\mathbf{\tilde{z}} = \mathbf{c}^{(i)}] \leq \frac{1}{q^{k+2} - 1}$ and
\[
	\Pr[\textsf{RUF}_\textsf{FE}^{\mathcal{O}^\prime_{\textsf{Enc}}, \tau^2}(\mathcal{A}^\prime) \to 1] \geq \Pr[V = 1] - \frac{k+2}{q^{k+2}-1}.
\]

\noindent Combining both results from Equation \ref{equ:ind-uf-OB-Probe:1} and \ref{equ:ind-uf-OB-Probe:2}, we derive
\[
	\Pr[\textsf{Verify}(s) = 1 \mid b = 1] \leq \Pr[V = 1] + \frac{1}{q^{k+2}} \leq \Pr[\textsf{RUF}_\textsf{FE}^{\mathcal{O}^\prime_{\textsf{Enc}}, \tau^2}(\mathcal{A}^\prime) \to 1] + \frac{k+2}{q^{k+2}-1} + \frac{1}{q^{k+2}}.
\]
Finally, similar to the proof of Theorem \ref{thm:fh-IPFE:ind-uf-OB-Enroll}, we derive

\begin{align*}
	\Pr[\textsf{fh-IND}(\mathcal{R}) \to 1]
	&= \frac{1}{2} + \frac{1}{4} \left( \Pr[\textsf{Verify}(s) = 1 \mid b = 0] - \Pr[\textsf{Verify}(s) = 1 \mid b = 1] \right) \\
	&\geq \frac{1}{2} + \frac{1}{4} \left( \Pr[\textsf{UF}_\textsf{option}(\mathcal{A}) \to 1] - \Pr[\textsf{RUF}_\textsf{FE}^{\mathcal{O}^\prime_{\textsf{Enc}}, \tau^2}(\mathcal{A}^\prime) \to 1] - \frac{k+2}{q^{k+2}-1} - \frac{1}{q^{k+2}} \right) \\
	&\geq \frac{1}{2} + \frac{1}{4} \left( \Pr[\textsf{UF}_\textsf{option}(\mathcal{A}) \to 1] - \Pr[\textsf{RUF}_\textsf{FE}^{\mathcal{O}^\prime_{\textsf{Enc}}, \gamma}(\mathcal{A}^\prime) \to 1] - \frac{k+2}{q^{k+2}-1} - \frac{1}{q^{k+2}} \right) \\
\end{align*}

\noindent Since both $\Adv_{\textsf{FE}, \mathcal{R}}^\textsf{fh-IND} = \left| \Pr[\textsf{fh-IND}(\mathcal{R}) \to 1] - \frac{1}{2} \right|$ and $\Adv_{\textsf{FE}, \mathcal{A}^\prime}^{\textsf{RUF}, \mathcal{O}^\prime_{\textsf{Enc}}, \gamma} = \Pr[\textsf{RUF}^{\mathcal{O}^\prime_{\textsf{Enc}}, \gamma}_{\textsf{FE}}(\mathcal{A}^\prime) \to 1 ]$ are negligible,
\[
	\Pr[\textsf{UF}_\textsf{option}(\mathcal{A}) \to 1] \leq 4 \cdot \Adv_{\textsf{FE}, \mathcal{R}}^\textsf{fh-IND} + \Adv_{\textsf{FE}, \mathcal{A}^\prime}^{\textsf{RUF}, \mathcal{O}^\prime_{\textsf{Enc}}, \gamma} + \frac{k+2}{q^{k+2}-1} + \frac{1}{q^{k+2}} = \negl.
\]

\end{proof}

Unfortunately, for the instantiation in Section \ref{sec:fh-IPFE-instantiation}, we cannot achieve UF security when the adversary has \textsf{psk}, even if the false positive rate is negligible. The adversary can simply compute $\mathbf{c} \gets \textsf{Probe}(\textsf{psk}, \mathbf{0})$ and return $\mathbf{c}$. The same results also hold for $\textsf{option}$ that includes $\textsf{esk}$ since both $\textsf{psk}$ and $\textsf{esk}$ are equal to $\textsf{msk}$ and allow the adversary to run $\textsf{FE.Enc}(\textsf{msk}, \textsf{pp}, \mathbf{v})$ for any vector $\mathbf{v}$. We state this result formally in the following theorem.

\begin{theorem}

Let $\textsf{option}$ include $\textsf{esk}$ or $\textsf{psk}$. For any distribution family $\mathbb{B}$ and functional encryption $\textsf{FE}$, $\Pi$ is not \textsf{option}-UF.

\end{theorem}


%-------------------


\subsection{IND Security of $\Pi$}
\label{sec:security_analysis:fh-IPFE:IND}

For the IND security, we first consider the following definition and assumption on the biometric distribution family $\mathbb{B}$.

\begin{definition}
For an authentication scheme $\Pi$, a distribution $\mathcal{B} \in \mathbb{B}$, and an integer $t$, define the distribution $\mathcal{D}_\mathcal{B}(t)$ as
\[
	\mathcal{D}_\mathcal{B}(t) = \left( \textsf{BioCompare}(\mathbf{b}, \mathbf{b}^{(1)}), \textsf{BioCompare}(\mathbf{b}, \mathbf{b}^{(2)}), \cdots, \textsf{BioCompare}(\mathbf{b}, \mathbf{b}^{(t)}) \right)
\]
where $\mathbf{b} \gets \textsf{getEnroll}^{\mathcal{O}_\mathcal{B}}()$ and $ \mathbf{b}^{(i)} \gets \textsf{getProbe}^{\mathcal{O}_\mathcal{B}}()$ for all $i \in [t]$.

\end{definition}

\begin{assumption}
\label{assump:equal_distance}
Let $t$ be an integer. Assume that for any two distributions $\mathcal{B}^{(0)}$ and $\mathcal{B}^{(1)}$ in the biometric distribution family $\mathbb{B}$, $\mathcal{D}_{ \mathcal{B}^{(0)} }(t)$ and $\mathcal{D}_{ \mathcal{B}^{(1)} }(t)$ are the same. 

\end{assumption}

Note that indistinguishability between $\mathcal{D}_{ \mathcal{B}^{(0)} }(t)$ and $\mathcal{D}_{ \mathcal{B}^{(1)} }(t)$ is a necessary condition to achieve IND security because
\[
	\left( \textsf{Compare}(\textsf{csk}, \mathbf{c_x}, \mathbf{c_y}^{(1)}), \cdots, \textsf{Compare}(\textsf{csk}, \mathbf{c_x}, \mathbf{c_y}^{(t)}) \right) = \mathcal{D}_{ \mathcal{B}^{(b)} }(t)
\]
where $b$ is the challenge bit.


\begin{theorem}
\label{thm:fh-IPFE:ind-ind}
For any distribution family $\mathbb{B}$ satisfying Assumption \ref{assump:equal_distance} and having a true positive rate $\textsf{TP} > \frac{1}{\poly}$, if \textsf{FE} is fh-IND, then $\Pi$ is IND.

\end{theorem}



\begin{proof}

Given an adversary $\mathcal{A}$ in the $\textsf{IND}$ game, consider the reduction adversary $\mathcal{R}$ in Algorithm \ref{alg:red:ind-ind} which plays the \textsf{fh-IND} game by running $\mathcal{A}$.

\begin{figure}[h]
\centering
	
	\begin{minipage}[t]{0.75\linewidth}
	\centering
	\begin{algorithm}[H]
	\caption{$\mathcal{R}^{\mathcal{O}_{\textsf{KeyGen}}, \mathcal{O}_{\textsf{Enc}}}(\textsf{pp})$}
	\label{alg:red:ind-ind}
	\begin{algorithmic}[1]
		\State $\mathcal{B}^{(0)} \getsdollar \mathbb{B}, \quad \mathbb{B} \gets \mathbb{B} \setminus \mathcal{B}^{(0)}$ \label{alg:red:ind-ind:B0}
		
		\State $\mathcal{B}^{(1)} \getsdollar \mathbb{B}, \quad \mathbb{B} \gets \mathbb{B} \setminus \mathcal{B}^{(1)}$ \label{alg:red:ind-ind:B1}

		\State $\mathbf{b}^{(0)} \gets \textsf{getEnroll}^{\mathcal{O}_{\mathcal{B}^{(0)}}}(), \mathbf{x}^{(0)} \gets (b_1^{(0)}, \cdots, b_k^{(0)}, 1, \|\mathbf{b}^{(0)}\|^2)$
		
		\State $\mathbf{b}^{(1)} \gets \textsf{getEnroll}^{\mathcal{O}_{\mathcal{B}^{(1)}}}(), \mathbf{x}^{(1)} \gets (b_1^{(1)}, \cdots, b_k^{(1)}, 1, \|\mathbf{b}^{(1)}\|^2)$
		
		\State $\mathbf{c_x} \gets \mathcal{O}_{\textsf{KeyGen}}(\mathbf{x}^{(0)}, \mathbf{x}^{(1)})$ \label{alg:red:ind-ind:cx}

		\For{$i = 1$ to $t$}
			
			\State ${\mathbf{b}^\prime}^{(0)} \gets \textsf{getProbe}^{\mathcal{O}_{\mathcal{B}^{(0)}}}()$

			\State $\mathbf{y}^{(0)} \gets (-2{b_1^\prime}^{(0)}, \cdots, -2{b_k^\prime}^{(0)}, \| {\mathbf{b}^\prime}^{(0)} \|^2, 1) $

			\Repeat 
				
				\State ${\mathbf{b}^\prime}^{(1)} \gets \textsf{getProbe}^{\mathcal{O}_{\mathcal{B}^{(1)}}}()$

				\State $\mathbf{y}^{(1)} \gets (-2{b_1^\prime}^{(1)}, \cdots, -2{b_k^\prime}^{(1)}, \| {\mathbf{b}^\prime}^{(1)} \|^2, 1)$

			\Until{ $\mathbf{x}^{(0)} {\mathbf{y}^{(0)}}^T = \mathbf{x}^{(1)} {\mathbf{y}^{(1)}}^T$ } \label{alg:red:ind-ind:while}

				%\State $\mathbf{y}^{(1)} \gets \textsf{encodeProbe}^{\mathcal{O}_{\mathcal{B}^{(1)}}}()$

			%\EndWhile

			\State $\mathbf{c_y}^{(i)} \gets \mathcal{O}_{\textsf{Enc}}(\mathbf{y}^{(0)}, \mathbf{y}^{(1)})$ \label{alg:red:ind-ind:cy}

		\EndFor

		\State $\tilde{b} \gets {\mathcal{A}}^{\mathcal{O}_{\mathcal{B}^{(0)}}, \mathcal{O}_{\mathcal{B}^{(1)}} } (\textsf{pp}, \mathbf{c_x}, \{ \mathbf{c_y}^{(i)} \}_{i=1}^t )$ \label{alg:red:ind-ind:A}

		\State \Return $\tilde{b}$

	\end{algorithmic}
	\end{algorithm}
	\end{minipage}
	
\end{figure}

\noindent Note that $(\mathbf{x}^{(0)}, \mathbf{x}^{(1)})$ is the only query of $\mathcal{R}$ to $\mathcal{O}_{\textsf{KeyGen}}$, and for any query $( \mathbf{y}^{(0)}, {\mathbf{y}^{(1)}} )$ to $\mathcal{O}_{\textsf{Enc}}$, it satisfies $\mathbf{x}^{(0)}{\mathbf{y}^{(0)}}^T = \mathbf{x}^{(1)}{\mathbf{y}^{(1)}}^T$. Hence, $\mathcal{R}$ is an admissible adversary.

The probability that Line \ref{alg:red:ind-ind:while} is satisfied is
\begin{align*}
	\Pr[\mathcal{D}_{\mathcal{B}^{(0)}}(1) = \mathcal{D}_{\mathcal{B}^{(1)}}(1)] 
	&\geq \sum_{i=0}^\tau \Pr[\mathcal{D}_{\mathcal{B}^{(0)}}(1) = i]^2 \qquad \text{(Assumption \ref{assump:equal_distance})} \\
	&\geq \frac{1}{\tau + 1} \cdot \left( \sum_{i=0}^\tau \Pr[\mathcal{D}_{\mathcal{B}^{(0)}}(1) = i] \right)^2 \\
	&= \frac{1}{\tau + 1} \cdot \left( \Pr \left[
		\begin{aligned}
			& \mathbf{b} \gets \textsf{getEnroll}^{\mathcal{O}_{\mathcal{B}^{(0)}} }() \\
			& \mathbf{b}^\prime \gets \textsf{getProbe}^{\mathcal{O}_{\mathcal{B}^{(0)}} }()
		\end{aligned}
		: \| \mathbf{b} - \mathbf{b}^\prime \| \leq \tau \right] \right)^2 \\
	&= \frac{ \textsf{TP}(\mathcal{B}^{(0)})^2 }{\tau + 1} = \frac{ \textsf{TP}^2 }{\tau + 1} \qquad \text{(Assumption \ref{assump:equal_distance})} 
\end{align*}
The expected number of repetitions is bounded above by $\frac{\tau + 1}{ \textsf{TP}^2 }$. Moreover, the probability that it is satisfied within $T$ repetitions is at least
\[
	1 - (1 - \frac{ \textsf{TP}^2 }{\tau + 1})^T \geq 1 - e^{-T \cdot \frac{ \textsf{TP}^2 }{\tau + 1}}
\]
We can reach a $1 - \negl.$ probability that the loop will end within $T$ times by setting a polynomial-size $T$.

Now, we show that $\mathcal{R}$ perfectly simulate an \textsf{IND} game for $\mathcal{A}$. If the challenge bit $b$ of the \textsf{fh-IND} game is $0$, $\mathbf{c_x}$ and $\mathbf{c_y}^{(i)}$ for all $i \in [t]$ are generated from $\mathcal{B}^{(0)}$ and have the same distributions as the inputs for an adversary in \textsf{IND} game.
If the challenge bit $b$ is $1$, we show that distributions of $\mathbf{c_x}, \{ \mathbf{c_y}^{(i)} \}_{i=1}^t$ also follow the same distribution given Assumption \ref{assump:equal_distance}.

Let $b^\prime \in \{0, 1\}$, define distributions 
\begin{align*}
	\mathbf{X}^{(b^\prime)} &= \{ \mathbf{b}^{(b^\prime)} \gets \textsf{getEnroll}^{\mathcal{O}_{\mathcal{B}^{(b^\prime)}}}() : \mathbf{x}^{(b^{\prime})} \gets (b_1^{(b^{\prime})}, \cdots, b_k^{(b^{\prime})}, 1, \|\mathbf{b}^{ (b^{\prime}) }\|^2) \} \\
	\mathbf{Y}^{ (b^\prime) }_i &= \{ \mathbf{b}^{(b^\prime)} \gets \textsf{getProbe}^{\mathcal{O}_{\mathcal{B}^{(b^\prime)}}}() : \mathbf{y}^{(b^{\prime})} \gets ( -2b_1^{(b^{\prime})}, \cdots, -2b_k^{(b^{\prime})}, \|\mathbf{b}^{ (b^{\prime}) }\|^2, 1) \} \\
	\{ \mathbf{Y}^{ (b^\prime) }_i \}_{ i \in [t] } &= (\mathbf{Y}^{ (b^\prime) }_1, \cdots, \mathbf{Y}^{ (b^\prime) }_t) \quad  \text{ ($t$ identical and independent distributions) }
\end{align*}

Let $\mathbf{Y}^\prime_i$ be the distribution of $\mathbf{y}^{(1)}$ derived after the loop in Line \ref{alg:red:ind-ind:while} in the $i$-th iteration.
For any $\{ d_i \}_{i=1}^t, d_i > 0$,
\begin{align*}
\Pr \left[ \bigwedge_{i=1}^t \mathbf{X}^{(0)} {\mathbf{Y}^{(0)}_i}^T = d_i^2 \right] 
&= \Pr \left[ \mathcal{D}_{\mathcal{B}^{(0)}}(t) = (d_1, \cdots, d_t) \right] \\
&= \Pr \left[ \mathcal{D}_{\mathcal{B}^{(1)}}(t) = (d_1, \cdots, d_t) \right] = \Pr \left[ \bigwedge_{i=1}^t \mathbf{X}^{(1)} {\mathbf{Y}^{(1)}_i}^T = d_i^2 \right]
\end{align*}

Hence, for any $\mathbf{x}$ and $\{ \mathbf{y}_i \}_{i=1}^t$,
\begin{align*}
	& \Pr [\mathbf{X}^{(1)} = \mathbf{x}, \mathbf{Y}^\prime_1 = \mathbf{y}_1, \cdots, \mathbf{Y}^\prime_t = \mathbf{y}_t] \\
	&= \sum_{d_1, \cdots, d_t} \left( \Pr \left[\mathbf{X}^{(1)} = \mathbf{x}, \mathbf{Y}^{(1)}_1 = \mathbf{y}_1, \cdots, \mathbf{Y}^{(1)}_t = \mathbf{y}_t \mid \bigwedge_{i=1}^t \mathbf{X}^{(1)} {\mathbf{Y}^{(1)}_i}^T = d_i^2 \right] \right. \\
	&\qquad \qquad \left. \times \Pr \left[\bigwedge_{i=1}^t \mathbf{X}^{(0)} {\mathbf{Y}_i^{(0)}}^T = d_i^2 \right] \right) \\
	&= \sum_{d_1, \cdots, d_t} \left( \Pr \left[\mathbf{X}^{(1)} = \mathbf{x}, \mathbf{Y}^{(1)}_1 = \mathbf{y}_1, \cdots, \mathbf{Y}^{(1)}_t = \mathbf{y}_t \mid \bigwedge_{i=1}^t \mathbf{X}^{(1)} {\mathbf{Y}^{(1)}_i}^T = d_i^2 \right] \right. \\
	&\qquad \qquad \left. \times \Pr \left[\bigwedge_{i=1}^t \mathbf{X}^{(1)} {\mathbf{Y}_i^{(1)}}^T = d_i^2 \right] \right) \\
	&= \Pr[\mathbf{X}^{(1)} = \mathbf{x}, \mathbf{Y}^{(1)}_1 = \mathbf{y}_1, \cdots, \mathbf{Y}^{(1)}_t = \mathbf{y}_t ]
\end{align*}

\noindent which implies $\mathcal{R}$ also perfectly simulate an \textsf{IND} game for $\mathcal{A}$ when the challenge bit $b = 1$.

In conclusion, 
\[
	\Adv_{\textsf{FE}, \mathcal{R}}^\textsf{fh-IND} = \Adv_{\Pi, \mathbb{B}, \mathcal{A}}^\textsf{IND} = \negl.
\]
which holds for all adversaries $\mathcal{A}$ in the \textsf{IND} game. This implies the IND security of $\Pi$.


\end{proof}


%-------------------


\section{Security Analysis: Relational Hash-based Instantiation}
\label{sec:security_analysis:rh}

Let $\Pi$ be an authentication scheme instantiated by a relational hash scheme \textsf{RH} as in Section \ref{sec:rh-instantiation}. We discuss the UF and IND security of $\Pi$ in this section. Note that in this instantiation, $\textsf{esk}, \textsf{psk}, \textsf{csk}$ are all public hash keys $\textsf{pk}$ of \textsf{FE} and assumed to be given to all adversaries.

Given a relational scheme \textsf{RH} for a relation $R \subseteq X \times Y$, we first define the unforgeability \cite{cryptoeprint:2014/394} of \textsf{RH}.

\begin{definition}[Unforgeability]

A relational hash scheme \textsf{RH} is called \emph{unforgeable} for the distribution $\mathcal{X}$ if for any adversary $\mathcal{A}$, the following probability is negligible.
\[
	\Pr \left [
		\begin{aligned} 
			 &\; \mathbf{x} \getsdollar \mathcal{X} \\
			 &\; \textsf{pk} \gets \textsf{RH.KeyGen}(1^\lambda) \\
			 &\; \mathbf{h_x} \gets \textsf{RH.Hash}_1(\textsf{pk}, \mathbf{x}) \\
			 &\; \mathbf{\tilde{z}} \gets \mathcal{A}(\textsf{pk}, \mathbf{h_x})
		\end{aligned} :
		\textsf{RH.Verify}(\textsf{pk}, \mathbf{h_x}, \mathbf{\tilde{z}}) = 1
		\right ] = \negl.
\]

\end{definition}

%Note that in \cite{cryptoeprint:2014/394}, the authors proposed an unforgeable relational scheme for the Hamming distance proximity relation $R^\tau$ for any $\tau < k$.

%-------------------

\subsection{UF Security of $\Pi$}
\label{sec:security_analysis:rh:uf}

We first consider $\textsf{option}$ that includes $\mathbf{c_x}$.


\begin{theorem}
\label{thm:rh:uf-uf-cx}

Let $\textsf{option} = \{\mathbf{c_x}, \textsf{esk}, \textsf{psk}, \textsf{csk}\}$. If \textsf{RH} is unforgeable for the distribution
\[
	\mathcal{X} = \{ \mathcal{B} \getsdollar \mathbb{B}: \mathbf{b} \gets {\sf getEnroll}^{\mathcal{O}_\mathcal{B}}() \mid \mathbb{B} \},
\]
then $\Pi$ is $\textsf{option}$-UF. 

\end{theorem}

In \cite{cryptoeprint:2014/394}, the authors construct an $\textsf{RH}$ that is unforgeable for the uniform distribution over $\{0, 1\}^k$, under the hardness of some computational problems. Note that we need to provide knowledge of $\mathbb{B}$ in the distribution $\mathcal{X}$.

\begin{proof}

Recall that the distribution of $\mathbf{c_x}$ in the \textsf{UF} game of the instantiation of Section \ref{sec:rh-instantiation} is
\[
	\left \{
		\begin{aligned} 
			 & \mathcal{B} \getsdollar \mathbb{B} \\
			 & \textsf{pk} \gets \textsf{RH.KeyGen}(1^\lambda) \\
			 & \mathbf{x} = \mathbf{b} \gets \textsf{getEnroll}^{\mathcal{O}_\mathcal{B}}() 
		\end{aligned} :
		\mathbf{c_x} \gets \textsf{RH.Hash}_1(\textsf{pk}, \mathbf{x})
	\right \}
\]
Also recall that $\textsf{Verify}(\textsf{Compare}(\textsf{csk}, \mathbf{c_x}, \mathbf{\tilde{z}} )) = \textsf{RH.Verify}(\textsf{pk}, \mathbf{c_x}, \mathbf{\tilde{z}} )$.
The \textsf{option}-UF security is thus guaranteed by the unforgeability of \textsf{RH}.

\end{proof}

\paragraph{Remark}
As we mentioned in Section \ref{sec:choice-of-option}, an adversary with \textsf{psk} can enjoy a winning rate of the false positive rate \textsf{FP} of $\mathbb{B}$. Theorem \ref{thm:rh:uf-uf-cx} thus implies that if $\textsf{FP}$ is not negligible, there does not exist an \textsf{RH} that is unforgeable for the distribution $\{ \mathcal{B} \getsdollar \mathbb{B}: \mathbf{b} \gets {\sf getEnroll}^{\mathcal{O}_\mathcal{B}}() \mid \mathbb{B} \}$.


Note that since $\textsf{esk}, \textsf{psk}$, and $ \textsf{csk}$ are all public in this instantiation, it is meaningless to discuss $\mathcal{O}_\textsf{Enroll}, \mathcal{O}_\textsf{Probe}$, or $\mathcal{O}_\textsf{log}$. In addition, for $\textsf{option}$ that includes $\mathcal{O}_\mathcal{B}$ or $\mathcal{O}_\textsf{Probe}^\prime$, as discussed in Section \ref{sec:choice-of-option}, we cannot achieve \textsf{option}-UF security since $\textsf{psk}$ is public in this instantiation.

For \textsf{option} that includes $\mathcal{O}_\textsf{Enroll}^\prime$, we notice that for the \textsf{RH} construction in \cite{cryptoeprint:2014/394}, there exists an invalid $\textsf{pk}^\prime$ such that $\textsf{RH.Hash}_1(\textsf{pk}^\prime, \mathbf{x})$ directly leaks $\mathbf{x}$. By returning $\textsf{RH.Hash}_2( \textsf{pk}, \mathbf{x} )$, one can break the $\textsf{UF}_{\textsf{option}}$ game with probability $1$.


%-------------------

\subsection{IND Security of $\Pi$}
\label{sec:security_analysis:rh:IND}

For the IND security, we have a negative result for $\Pi$.

\begin{theorem}

For any distribution family $\mathbb{B}$ that $\textsf{TP} - \textsf{FP} > \frac{1}{\poly}$, and for any relational hash scheme \textsf{RH}, $\Pi$ is not IND for any $t \geq 0$.

\end{theorem}

\begin{proof}

Consider the adversary $\mathcal{A}$ in Algorithm \ref{alg:rh:ind}. When the challenge bit $b = 0$, the probability that $\mathcal{A}$ wins is $\textsf{TP}$. When the challenge bit $b = 1$, the probability that $\mathcal{A}$ wins is $1 - \textsf{FP}$. Now, 

\[
	\Adv_{\Pi, \mathbb{B}, \mathcal{A}}^{\textsf{IND}} = \left| \Pr[ \textsf{IND}_\Pi(\mathcal{A}) \to 1 ] - \frac{1}{2} \right| = \left| \frac{1}{2} (\textsf{TP} + 1 - \textsf{FP}) - \frac{1}{2} \right| > \frac{1}{\poly}.
\]

\begin{figure}[h]
\centering

	\begin{minipage}[t]{0.65\textwidth}
	\begin{algorithm}[H]
	\caption{$\mathcal{A}^{\mathcal{O}_{\mathcal{B}^{(0)}}, \mathcal{O}_{\mathcal{B}^{(1)}}} ( \textsf{csk} = \textsf{pk}, \mathbf{c_x}, \{ \mathbf{c_y}^{(i)} \}_{i=1}^t )$}
	\label{alg:rh:ind}
	\begin{algorithmic}[1]

		\State $\mathbf{y}^{(0)} = \mathbf{b}^{(0)} \gets \textsf{getProbe}^{\mathcal{O}_{\mathcal{B}^{(0)}}}()$
		
		\State $\mathbf{h_y}^{(0)} \gets \textsf{RH.Hash}_2(\textsf{pk}, \mathbf{y}^{(0)})$
		
		\If{ $\textsf{RH.Verify}(\textsf{pk}, \mathbf{c_x}, \mathbf{h_y}^{(0)}) = 1$}
		
			\State \Return $0$
		
		\Else
			
			\State \Return $1$
		
		\EndIf
	\end{algorithmic}
	\end{algorithm}
	\end{minipage}

\end{figure}

\end{proof}

We note that this insecurity result holds whenever $\textsf{psk}$ is public. When $\textsf{esk}$ is public, one can also use $\mathbf{c_y}^{(i)}$ to verify from which distribution the chalenge ciphertexts are generated. We write this observation formally in the following theorem.

\begin{theorem}

Given any distribution family $\mathbb{B}$ that $\textsf{TP} - \textsf{FP} > \frac{1}{\poly}$. If $\textsf{psk}$ is public, $\Pi$ is not IND for any $t \geq 0$. If $\textsf{esk}$ is public, $\Pi$ is not IND for any $t \geq 1$.

\end{theorem}



%-------------------


\appendix

%-------------------

\section{Construction in \cite{cryptoeprint:2016/440}}
\label{sec:fh-IPFE-construction}

Let $\mathbb{G}_1$ and $\mathbb{G}_2$ be two groups of order a prime number $q$ with generators $g_1$ and $g_2$, respectively. Let $e: \mathbb{G}_1 \times \mathbb{G}_2 \to \mathbb{G}_T$ be a mapping to a target group $\mathbb{G}_T$ also of order $q$. 

\begin{definition}[Bilinear asymmetric group \cite{cryptoeprint:2016/440}]
\label{bilinear-group}

A tuple $(\mathbb{G}_1, \mathbb{G}_2, \mathbb{G}_T, q, e)$ is a \emph{bilinear asymmetric group} if the following hold.

\begin{itemize}

	\item Group operations in $\mathbb{G}_1, \mathbb{G}_2$, and $\mathbb{G}_T$ and mapping $e$ are efficiently computable.

	\item $e$ is bilinear. That is, for $x, y \in \Z_q$, $e(g_1^x, g_2^y) = e(g_1, g_2)^{xy}$.

	\item $e$ is non-degenerate. That is, $e(g_1, g_2) \neq 1$, the identity element of $\mathbb{G}_T$.

\end{itemize}

\end{definition}

For a vector $\mathbf{v} = (v_1, v_2, \cdots, v_n) \in \Z_q^n$ and a group element $g$ in group of order $q$, we write $g^\mathbf{v}$ to denote the vector of group elements $(g^{v_1}, g^{v_2}, \cdots, g^{v_n})$. Moreover, for $k \in \Z_q$ and $\mathbf{v}, \mathbf{w} \in \Z_q^n$, we write $(g^{\mathbf{v}})^k = g^{k \cdot \mathbf{v}}$ and $g^\mathbf{v} \cdot g^\mathbf{w} =  g^{\mathbf{v} + \mathbf{w}}$. Finally, the pairing operation is extended to vectors.
\[
	e(g_1^{\mathbf{v}}, g_2{^\mathbf{w}}) = \prod_{i \in [n]} e(g_1^{v_i}, g_2^{w_i}) = e(g_1, g_2)^{\mathbf{v}\mathbf{w}^T}.
\]

We now recall the fh-IPFE construction $\textsf{FE}$ in \cite{cryptoeprint:2016/440}.

\begin{itemize}

	\item $\textsf{FE.Setup}(1^\lambda)$: Sample an asymmetric bilinear group $(\mathbb{G}_1, \mathbb{G}_2, \mathbb{G}_T, q, e)$ and choose generators $g_1 \in \mathbb{G}_1$ and $g_2 \in \mathbb{G}_2$. Sample $\mathbf{B} \in \mathbb{GL}_n(\Z_q)$ and find $\mathbf{B}^* = \det(\mathbf{B}) \cdot (\mathbf{B}^{-1})^T$. Finally, output the public parameter $\textsf{pp} = (\mathbb{G}_1, \mathbb{G}_2, \mathbb{G}_T, q, e)$ and the master secret key $\textsf{msk} = (\textsf{pp}, g_1, g_2, \mathbf{B}, \mathbf{B}^*)$.
	
	\item $\textsf{FE.KeyGen}(\textsf{msk}, \textsf{pp}, \mathbf{x})$: Sample $\alpha \getsdollar \Z_q$ and output
	\[
		 f_\mathbf{x} = (K_1, K_2) = \left( g_1^{\alpha \cdot \det(\mathbf{B})}, g_1^{\alpha \cdot \mathbf{x} \cdot \mathbf{B}} \right)
	\]
	
	\item $\textsf{FE.Enc}(\textsf{msk}, \textsf{pp}, \mathbf{y})$: Sample $\beta \getsdollar \Z_q$ and output
	\[
		 \mathbf{c_y} = (C_1, C_2) = \left( g_2^{\beta}, g_2^{\beta \cdot \mathbf{y} \cdot \mathbf{B}^*} \right)
	\]
	
	\item $\textsf{FE.Dec}(\textsf{pp}, f_\mathbf{x}, \mathbf{c_y}) \to z$: Parse $f_\mathbf{x} = (K_1, K_2)$ and $\mathbf{c_y} = (C_1, C_2)$ and compute 
	\[
		D_1 = e(K_1, C_1) \quad \text{and} \quad D_2 = e(K_2, C_2)
	\]
	Solve the discrete logarithm to find $z$ such that $D_1^z = D_2$ and output $z$. If it fails to find such $z$, output $\bot$.

\end{itemize}

\paragraph{Correctness}
We have
\[
	D_1 = e(K_1, C_1) = e(g_1, g_2)^{\alpha \cdot \beta \cdot \det(\mathbf{B})}
\] and 
\[
	D_2 = e(K_2, C_2) = e(g_1, g_2)^{\alpha \cdot \beta \cdot \mathbf{x} \cdot \mathbf{B} \cdot (\mathbf{B}^*)^T \cdot \mathbf{y}^T \cdot } = e(g_1, g_2)^{\alpha \cdot \beta \cdot \det(\mathbf{B}) \cdot \mathbf{x}\mathbf{y}^T }.
\]
Therefore, $(D_1)^{\mathbf{x}\mathbf{y}^T} = D_2$.

\paragraph{Remark}
In this construction, $q$ is exponential to $\lambda$ to achieve security, and decryption relies on some priori knowledge of possible ranges of the inner product $\mathbf{x} \mathbf{y}^T$. For example, for the instantiation in Section \ref{sec:fh-IPFE-instantiation}, one can enumerate $z \in \{0, 1, \cdots, \tau \}$ and return $\bot$ when no valid $z \leq \tau$ such that $D_1^z = D_2$ is found.





%-------------------


\newpage

%-------------------

\section{One-Way Game}

\begin{figure}[h]
\centering

	\begin{minipage}[t]{0.55\textwidth}
	\begin{algorithm}[H]
	\caption{$\textsf{OW}_{\Pi, \mathbb{B}}(\mathcal{A})$}
	\label{alg:ow_game}
	\begin{algorithmic}[1]

		\State $\mathcal{B} \getsdollar \mathbb{B}, \quad \mathbb{B} \gets \mathbb{B} \setminus \mathcal{B}$

		\State $\textsf{esk}, \textsf{psk}, \textsf{csk} \gets \textsf{Setup}(1^\lambda)$

		\State $\mathbf{b} \gets \textsf{getEnroll}^{\mathcal{O}_{\mathcal{B}}}()$

		\State $\mathbf{c_x} \gets \textsf{Enroll}(\textsf{esk}, \mathbf{b})$

		\For{$i = 1$ to $t$}

			\State ${\mathbf{b}^\prime}^{(i)} \gets \textsf{getProbe}^{\mathcal{O}_{\mathcal{B}}}() $
		
			\State $\mathbf{c_y}^{(i)} \gets \textsf{Probe}( \textsf{psk}, {\mathbf{b}^\prime}^{(i)} )$

		\EndFor
	
			\State $\mathbf{\tilde{b}} \gets \mathcal{A}( \textsf{csk}, \mathbf{c_x}, \{ \mathbf{c_y}^{(i)} \}_{i=1}^t )$

		\State $s \gets \textsf{BioCompare}( \mathbf{\tilde{b}}, \mathbf{b})$
		\State \Return $ \textsf{Verify}(s) $
	\end{algorithmic}
	\end{algorithm}
	\end{minipage}

\label{fig:ow_game}
\end{figure}


%-------------------
%% Backup File

% \section*{Backup}
\subsection*{Privacy Game}

In the privacy game, we model the ability of the server who tries to learn the biometric template of a user. The adversary $\mathcal{A}$ is given the enrollment message $\mathbf{c_x}$, $t$ probe messages $\{ \mathbf{c_y}_i\}_{i=1}^t$ for some integer $t$, and oracle $\mathcal{O}$ and tries to find the template $\mathbf{b}$. The whole game is defined in Figure \ref{fig:privacy_game}.


\begin{figure}[h]
	\begin{center}
		\begin{tabular}{l c}
			${\sf Priv}_{\Pi}(\mathcal{A})$\\

			\hline

			${\sf csk}, {\sf pp_s} \gets {\sf SSetup(1^\lambda)}$ \\

			${\sf esk}, {\sf psk}, {\sf pp_u} \gets {\sf USetup}(1^\lambda, {\sf pp_s})$ \\

			${\sf pp}:= ({\sf pp_s}, {\sf pp_u})$ \\

			$ \mathbf{b} \getsdollar \mathcal{B}, \mathbf{x} \gets {\sf encodeEnroll}^{\mathcal{O}_{\sf aux}}(\mathbf{b})$ \\

			$\mathbf{c_x} \gets {\sf Enroll}({\sf esk}, {\sf pp}, \mathbf{x})$ \\

			$\{ \mathbf{b}_i \}_{i=1}^t \getsdollar \mathcal{B}^t$ \\ 

			$\{ \mathbf{c_y}_i \}_{i=1}^t \gets \left\{ {\sf Probe}( {\sf psk}, {\sf pp}, {\sf encodeProbe}^{\mathcal{O}_{\sf aux}}(\mathbf{b}_i) ) \right \}_{i=1}^t$ \\

			${\mathbf{\tilde{b}}} \gets \mathcal{A}^{\mathcal{O}} ( \sf{pp}, \mathbf{c_x}, \{ \mathbf{c_y}_i \}_{i=1}^t )$ \\

			\textbf{return} $1_{\mathbf{\tilde{b}} = \mathbf{b}}$
			
		\end{tabular}
	\end{center}
	\caption{The Privacy Game}
	\label{fig:privacy_game}
\end{figure}


The given oracle $\mathcal{O}$ can be any or more of the following three oracles:


\begin{itemize}

	\item $\mathcal{O}_{\sf Enroll}({\sf esk}, {\sf pp}, \cdot)$: On input $\mathbf{x}$, it outputs the enrollment message ${\sf Enroll}({\sf esk}, {\sf pp}, \mathbf{x})$.

	\item $\mathcal{O}_{\sf Probe}({\sf psk}, {\sf pp}, \cdot)$: On input $\mathbf{y}$, it outputs the probe message ${\sf Probe}({\sf psk}, {\sf pp}, \mathbf{y})$.

	\item $\mathcal{O}_{\sf Compare}({\sf csk}, {\sf pp}, \cdot, \cdot)$: On input $\mathbf{c_x}$ and $\mathbf{c_y}$, it outputs the comparison result ${\sf Compare}( {\sf csk}, {\sf pp}, \mathbf{c_x}, \mathbf{c_y} )$.

\end{itemize}

Note that if the enrollment secret key ${\sf esk}$, the probe secret key ${\sf psk}$, or the comparison secret key ${\sf csk}$ is an empty string in the scheme, then the corresponding oracles are naturally and implicitly given since the adversary can compute them herself.

We define the success probability of an adversary $\mathcal{A}$ in the privacy game of the scheme $\Pi$ as
\[
	\Adv^{\sf Priv}_{\Pi, \mathcal{A}^\mathcal{O}} := \Pr[{\sf Priv}_{\Pi}(\mathcal{A}^\mathcal{O}) \to 1].
\]

The authentication scheme $\Pi$ is called \emph{privacy} secure if for any PPT adversary $\mathcal{A}$,
\[
	\Adv^{\sf Priv}_{\Pi, \mathcal{A}^\mathcal{O}} = \negl.
\]

%-------------------

\subsection*{IND Game}
For empty comparison key ${\sf csk}$, to avoid trivial attacks, we need to restrict the adversary's queries.

\begin{definition}[Admissible Adversary]
\label{def:admissible}
	For an adversary $\mathcal{A}^\mathcal{O}$ in an IND game of a scheme $\Pi$ with empty comparison secret key, let $(\mathbf{x}_0^1, \mathbf{x}_1^1), (\mathbf{x}_0^2, \mathbf{x}_1^2), \cdots, (\mathbf{x}_0^{Q_{\sf Enroll}}, \mathbf{x}_1^{Q_{\sf Enroll}})$ be the queries to $\mathcal{O}_{\sf Enroll}^{\sf IND}$, and let $(\mathbf{y}_0^1, \mathbf{y}_1^1), (\mathbf{y}_0^2, \mathbf{y}_1^2),$ $\cdots, (\mathbf{y}_0^{Q_{\sf Probe}}, \mathbf{y}_1^{Q_{\sf Probe}})$ be the queries to $\mathcal{O}_{\sf Probe}^{\sf IND}$. We say the adversary $\mathcal{A}$ is \emph{admissible} if
	\[
		\begin{aligned}
			& {\sf Compare}( {\sf pp}, {\sf Enroll}({\sf esk}, {\sf pp}, \mathbf{x}_0^i), {\sf Probe}({\sf psk}, {\sf pp}, \mathbf{y}_0^j) ) \\
			= \; & {\sf Compare}( {\sf pp}, {\sf Enroll}({\sf esk}, {\sf pp}, \mathbf{x}_1^i), {\sf Probe}({\sf psk}, {\sf pp}, \mathbf{y}_1^j) )
		\end{aligned}
		\quad \forall i \in [Q_{\sf Enroll}], \forall j \in [Q_{\sf Probe}]
	\]
	

\end{definition}

This prevents the adversary's win from simply querying unmatched enrollment and probe message pairs and finding the difference from the public $\sf Compare$ function.

We define the success probability of an adversary $\mathcal{A}$ in the IND game of the scheme $\Pi$ as
\[
	\Adv^{\sf IND}_{\Pi, \mathcal{A}^\mathcal{O}} := \Pr[{\sf IND}_{\Pi}(\mathcal{A}^\mathcal{O}) \to 1].
\]

The authentication scheme $\Pi$ is called \emph{IND} secure if for all PPT adversaries $\mathcal{A}$ when $\Pi$ has a non-trivial comparison secret key, or for all admissible PPT adversaries $\mathcal{A}$ when $\Pi$ does not have a non-trivial comparison secret key,
\[
	\Adv^{\sf IND}_{\Pi, \mathcal{A^\mathcal{O}}} = \negl.
\]



%-------------------
%% Reference List
\nocite{*}
\printbibliography


\end{document}
