%%%%%%%%%%%%%%%%%%%%

% Project Name: Semester Project Fall 2024 for EPFL
% File: main.tex
% Author: Keng-Yu Chen

%%%%%%%%%%%%%%%%%%%%

\input{header.tex}


\begin{document}

%% Title
\title{\textbf{The Cryptographic Layer of Biometric Authentication}}
\author{Keng-Yu Chen \\
  \multicolumn{1}{p{.7\textwidth}}{} \\
  \multicolumn{1}{p{.7\textwidth}}{\centering \textbf{Supervisor}: Serge Vaudenay}\\
  \multicolumn{1}{p{.7\textwidth}}{\centering LASEC, EPFL}\\
  \\
}
\date{\today}

\maketitle

%-------------------

%% Header and Foot
\pagestyle{fancy}
\fancyhf{}
\fancyhead[L]{LASEC}
\fancyhead[R]{Keng-Yu Chen}
\fancyfoot[C]{\thepage}

%-------------------
% Abstract
\input{Contents/abstract}

%-------------------
% Introduction
\section{Introduction}
\label{sec:introdiction}
\input{Contents/introduction.tex}

%-------------------
% Fomalization
\section{Formalization}
\label{sec:formalization}
%%%%%%%%%%%%%%%%%%%%

% Project Name: Semester Project Fall 2024 for EPFL
% File: formalization.tex
% Author: Keng-Yu Chen

%%%%%%%%%%%%%%%%%%%%


\subsection{Biometric Authentication Scheme}

In this section, we formally define a biometric authentication scheme. For this, we first define how we simulate biometric distributions of users.

Assume the existence of a family $\mathbb{B}$ of biometric distributions that are efficiently samplable. We have the following interfaces for all algorithms to interact with $\mathbb{B}$.

\begin{itemize}

	\item $\mathsf{BioSamp}()$: Generate a random distribution $\mathcal{B}$ of $\mathbb{B}$. By this we mean providing either parameters of an efficiently samplable distribution or a PPT algorithm as the sampler. For simplicity, we write $\mathcal{B} \gets \mathsf{BioSamp}()$ as $\mathcal{B} \getsdollar \mathbb{B}$.
	
	\item $\mathsf{BioDelete}(\mathcal{B})$: Delete $\mathcal{B}$ from $\mathbb{B}$. Consequently, no further access to $\mathsf{BioSamp}$ can derive $\mathcal{B}$. For simplicity, we write $\mathsf{BioDelete}(\mathcal{B})$ as $\mathbb{B} \gets \mathbb{B} \setminus \mathcal{B}$.

	\item $\mathsf{TempSamp}(\mathcal{B})$: Let $\mathcal{B}$ be a biometric distribution in $\mathbb{B}$. This algorithm samples a biometric template from $\mathcal{B}$. For simplicity, we write $\mathbf{b} \gets \mathsf{TempSamp}(\mathcal{B})$ as $\mathbf{b} \getsdollar \mathcal{B}$.

\end{itemize}

\begin{definition}[Biometric Authentication Scheme]

A \emph{biometric authentication scheme} $\Pi$ associated with a family $\mathbb{B}$ of biometric distributions is composed of the following algorithms.

\begin{itemize}

	\item $\mathsf{getEnroll}^{\mathcal{O}_{\mathcal{B}}}() \to \mathbf{b}$: Given an oracle $\mathcal{O}_{\mathcal{B}}$, which samples biometric data from a distribution $\mathcal{B} \in \mathbb{B}$, it outputs a biometric template $\mathbf{b}$ for enrollment. In practice, $\mathsf{getEnroll}$ can collect several biometric samples from a user's biometric distribution $\mathcal{B}$ to create a more accurate template.

	\item $\mathsf{getProbe}^{\mathcal{O}_{\mathcal{B}}}() \to \mathbf{b}^\prime$: Given an oracle $\mathcal{O}_{\mathcal{B}}$, which samples biometric data from a distribution $\mathcal{B} \in \mathbb{B}$, it outputs a biometric template $\mathbf{b}^\prime$ for probe. In practice, $\mathsf{getProbe}$ often directly outputs the answer from $\mathcal{O}_{\mathcal{B}}$.

	\item $\mathsf{BioCompare}(\mathbf{b}, \mathbf{b}^\prime) \to s$: Given a biometric template $\mathbf{b}$ from $\mathsf{getEnroll}$ and another template $\mathbf{b}^\prime$ from $\mathsf{getProbe}$, it outputs a score $s$.

	\item $\mathsf{Verify}(s) \to r \in \{0,1\}$: It is a deterministic algorithm that reads the comparison score $s$ and determines whether this is a successful authentication ($r = 1$) or not ($r = 0$).

\end{itemize}
We also call these algorithms the \emph{biometric layer} of $\Pi$. We will add a \emph{cryptographic layer} on top of it in Section \ref{sec:formalization:cryptographic_layer}

\end{definition}


Given an authentication scheme $\Pi$, we can consider its true positive rate and false positive rate.

\begin{definition}[True Positive Rate]
For a biometric distribution $\mathcal{B} \in \mathbb{B}$ and $\mathbf{b} \gets \mathsf{getEnroll}^{\mathcal{O}_\mathcal{B}}()$, define the \emph{true positive rate} \textsf{TP}.

\begin{alignat*}{2}
	\mathsf{TP}(\mathcal{B}, \mathbf{b}) 
	:=&\; \Pr[ \mathbf{b}^\prime \gets \mathsf{getProbe}^{\mathcal{O}_\mathcal{B}}()
	&&: \mathsf{Verify}(\mathsf{BioCompare}(\mathbf{b}, \mathbf{b}^\prime )) = 1 ] \\
	\\
	\mathsf{TP}(\mathcal{B}) 
	:=&\; \Pr \Bigg[ 
		\begin{aligned}	
			& \mathbf{b} \gets \mathsf{getEnroll}^{\mathcal{O}_\mathcal{B}}() \\
			& \mathbf{b}^\prime \gets \mathsf{getProbe}^{\mathcal{O}_\mathcal{B}}()
		\end{aligned}
		&&: \mathsf{Verify}(\mathsf{BioCompare}(\mathbf{b}, \mathbf{b}^\prime)) = 1 \Bigg] \\
	=&\; \mathbb{E}_{ \mathbf{b} \gets \mathsf{getEnroll}^{\mathcal{O}_\mathcal{B}}() }[\mathsf{TP}(\mathcal{B}, \mathbf{b})] \\
	\\
	\mathsf{TP} 
	:=&\; \Pr \vast[
		\begin{aligned}
			& \mathcal{B} \getsdollar \mathbb{B} \\
			& \mathbf{b} \gets \mathsf{getEnroll}^{\mathcal{O}_\mathcal{B} }() \\
			& \mathbf{b}^\prime \gets \mathsf{getProbe}^{\mathcal{O}_\mathcal{B} }()
		\end{aligned}
		&&: \mathsf{Verify}(\mathsf{BioCompare}(\mathbf{b}, \mathbf{b}^\prime )) = 1 \vast] \\
	=&\; \mathbb{E}_{\mathcal{B} \getsdollar \mathbb{B}}[\mathsf{TP}(\mathcal{B})] 
\end{alignat*}

\end{definition}


\begin{definition}[False Positive Rate]
For a biometric distribution $\mathcal{B} \in \mathbb{B}, \mathbb{B} \gets \mathbb{B} \setminus \mathcal{B}$ and $\mathbf{b} \gets \mathsf{getEnroll}^{\mathcal{O}_\mathcal{B}}()$, define the \emph{false positive rate} \textsf{FP}.

\begin{alignat*}{2}
	\mathsf{FP}(\mathbf{b}) 
	:=&\; \Pr \Bigg[ 
		\begin{aligned}
			& \mathcal{B}^\prime \getsdollar \mathbb{B} \\
			& \mathbf{b}^\prime \gets \mathsf{getProbe}^{\mathcal{O}_{\mathcal{B}^\prime}}()
		\end{aligned}
		&&: \mathsf{Verify}(\mathsf{BioCompare}(\mathbf{b}, \mathbf{b}^\prime)) = 1 \Bigg] \\
	\\
	\mathsf{FP}(\mathcal{B}) 
	:=&\; \Pr \vast[ 
		\begin{aligned}
			& \mathcal{B}^\prime \getsdollar \mathbb{B} \\
			& \mathbf{b} \gets \mathsf{getEnroll}^{\mathcal{O}_\mathcal{B} }() \\
			& \mathbf{b}^\prime \gets \mathsf{getProbe}^{\mathcal{O}_{\mathcal{B}^\prime} }()
		\end{aligned}
		&&: \mathsf{Verify}(\mathsf{BioCompare}(\mathbf{b}, \mathbf{b}^\prime)) = 1 \vast] \\
	=&\; \mathbb{E}_{ \mathbf{b} \gets \mathsf{getEnroll}^{\mathcal{O}_\mathcal{B}}() }[\mathsf{FP}(\mathbf{b})] \\
	\\
	\mathsf{FP} 
	:=&\; \Pr \vast[
		\begin{aligned}
			& \mathcal{B} \getsdollar \mathbb{B}, \mathbb{B} \gets \mathbb{B} \setminus \mathcal{B}, \mathcal{B}^\prime \getsdollar \mathbb{B} \\
			& \mathbf{b} \gets \mathsf{getEnroll}^{\mathcal{O}_\mathcal{B} }() \\
			& \mathbf{b}^\prime \gets \mathsf{getProbe}^{\mathcal{O}_{\mathcal{B}^\prime} }()
		\end{aligned}
		&&: \mathsf{Verify}(\mathsf{BioCompare}(\mathbf{b}, \mathbf{b}^\prime)) = 1 \vast] \\
	=&\; \mathbb{E}_{\mathcal{B} \getsdollar \mathbb{B}}[\mathsf{FP}(\mathcal{B})]
\end{alignat*}

\end{definition}

Ideally, we hope $\mathsf{TP}$ to be $1$ and $\mathsf{FP}$ to be negligible. However, due to the inherent nature of biometrics, there might be a non-zero false negative rate $ 1 - \mathsf{TP} > 0$ and a $\mathsf{FP}$ that is not negligible. Our security model and analysis also take these possibilities into consideration.


%-------------------


\subsection{Cryptographic Layer}
\label{sec:formalization:cryptographic_layer}
In this work, we add a cryptographic layer on top of a biometric authentication scheme to protect privacy of users.

\begin{definition}[Cryptographic Layer]

The \emph{cryptographic layer} of a biometric authentication scheme associated with a family $\mathbb{B}$ of biometric distributions is composed of the following algorithms.

\begin{itemize}

	\item $\mathsf{Setup}(1^\lambda) \to \mathsf{esk}, \mathsf{psk}, \mathsf{csk}$: It outputs the enrollment secret key $\mathsf{esk}$, probe secret key $\mathsf{psk}$, and compare secret key $\sf csk$.

	\item $\mathsf{Enroll}(\mathsf{esk}, \mathbf{b}) \to \mathbf{c_x}$: On input a biometric template $\mathbf{b}$, it encodes it into a vector $\mathbf{x}$ and outputs the enrollment message $\mathbf{c_x}$.
	
	\item $\mathsf{Probe}(\mathsf{psk}, \mathbf{b}^\prime) \to \mathbf{c_y}$: On input a biometric template $\mathbf{b}^\prime$, it encodes it into a vector $\mathbf{y}$ and outputs the probe message $\mathbf{c_y}$ .

	\item $\mathsf{Compare}(\mathsf{csk}, \mathbf{c_x}, \mathbf{c_y)} \to s$: It compares the enrollment message $\mathbf{c_x}$ and probe message $\mathbf{c_y}$ and outputs a score $s$.

\end{itemize}

\noindent \textbf{Correctness}: A cryptographic layer is \emph{correct} if for any biometric distributions $\mathcal{B}$ and $\mathcal{B}^\prime$, let $\mathsf{esk}, \mathsf{psk}, \mathsf{csk} \gets \mathsf{Setup}(1^\lambda)$, $\mathbf{b} \gets \mathsf{getEnroll}^{\mathcal{O}_\mathcal{B}}()$, $\mathbf{b}^\prime \gets \mathsf{getProbe}^{\mathcal{O}_{\mathcal{B}^\prime}}()$, $\mathbf{c_x} \gets \mathsf{Enroll}(\mathsf{esk}, \mathbf{b})$, $\mathbf{c_y} \gets \mathsf{Probe}(\mathsf{psk}, \mathbf{b}^\prime)$. Then
	\[
		\Pr \left [
			\mathsf{Compare}(\mathsf{csk}, \mathbf{c_x}, \mathbf{c_y}) = \mathsf{BioCompare}(\mathbf{b}, \mathbf{b}^\prime)
		\right ] = 1.
	\]

\end{definition}

In a real-world biometric system, these algorithms may be run by different parties such as a biometric scanner, a user's secure hardware, a trusted authority that issues keys, and the server.

We provide two instantiations of a biometric authentication scheme with the cryptographic layer in Sections \ref{sec:fh-IPFE-instantiation} and \ref{sec:rh-instantiation}.


%-------------------

\iffalse

We discuss two usage models that employs the authentication scheme $\Pi$.


\subsection{Usage Model – Device-of-User}
\label{sec:dou_model}

In the model described in Figure \ref{fig:model_dou_overview} (an overview), Figure \ref{fig:model_dou_enrollment} (on enrollment), and Figure \ref{fig:model_dou_auth} (on authentication), users authenticate themselves to a server through their own devices and biometric scanners that are shared among different users.
A key distribution service distributes keys for them. In practice, this model applies to the situation when the users access an online service run by the server.

\begin{itemize}

	\item \textsf{User}: The user who enrolls its biometric data and authenticates itself to the server. We assume the user's biometric distribution is $\mathcal{B} \in \mathbb{B}$. 

	\item \textsf{Scanner}: A machine to extract the user's biometric data by querying the oracle $\mathcal{O}_{\mathcal{B}}$.
	
	\item \textsf{Device}: A device belonging to the user. In practice, it can be a desktop or a mobile phone. It processes the  for biometric data through the \textsf{Scanner}.
	
	\item \textsf{KDS}: A key distribution service. It runs $\mathsf{Setup}$ to generate keys and distribute them to \textsf{Device} and \textsf{Server}.
		
	\item \textsf{Server}: The server responsible for authenticating the user. It stores the comparison key $\mathsf{csk}$ and the user's enrollment message $\mathbf{c_x}$. On authentication, it compares the probe message with the registered enrollment message and returns the result.  

\end{itemize}

The Device-of-User model, when instantiated by an fh-IPFE scheme (Section \ref{sec:fh-IPFE-instantiation}), is analogous to the use case presented in \cite{cryptoeprint:2023/481}.
In their model, a user possesses a personal device, such as a smartphone or laptop, and a secure hardware device that runs an initial setup and stores all the keys, which corresponds to our \textsf{KDS}.
On enrollemnt and authentication, the user inputs biometric templates onto the device, which corresponds to our \textsf{Scanner}.
Subsequently, the device transmits the template to the secure hardware for the enrollment or probing processes, which are equivalent to our \textsf{Device}.
In addition, they incorporate a two-factor authentication mechanism.
The secure hardware also executes a digital signature scheme and sign the probe message on authentication.


\input{tikz/dou_model.tex}

%-------------------

\subsection{Usage Model – Device-of-Domain}
\label{sec:dod_model}

In the model described in Figure \ref{fig:model_dod_overview} (an overview), Figure \ref{fig:model_dod_enrollment} (on enrollment), and Figure \ref{fig:model_dod_auth} (on authentication), users first enroll themselves at an enrollment station and then authenticate themselves to a server through devices that belong to a domain.
A key distribution service distributes enrollment keys to the enrollment station, probe keys to the domain, and comparison keys to the server. In practice, a domain can be a department in an organization, and this models applies to the situation when a user wants to access a public service of a department, such as a restricted area or instruments. 

\begin{itemize}

	\item \textsf{User}: The user who enrolls its biometric data at an enrollment station and authenticates itself to the server. We assume the user's biometric distribution is $\mathcal{B} \in \mathbb{B}$.
	
	\item \textsf{Domain}: A domain that owns several devices, all of which share one enrollment key . Only the probe key is stored at each device of a domain. The enrollment key is stored at the enrollment station, and the comparison key is stored at the server. In practice, a domain can be a department, and users enroll and authenticate themselves before accessing a restricted service of this department.

	\item \textsf{Scanner}: A machine to extract the user's biometric data by querying the oracle $\mathcal{O}_{\mathcal{B}}$.
	
	\item \textsf{Station}: An enrollment station responsible for collecting the user's biometric data to enroll them for a domain on the server.

	\item \textsf{Device}: A device belonging to a domain. In practice, it can be a device checking identities for a restricted area or an instrument. It owns a probe key $\sf psk$ and processes the $\sf Probe$ function for enrolled users of this domain.
	
	\item \textsf{KDS}: A key distribution service. It runs $\mathsf{Setup}$ to generate keys and distribute them to \textsf{Station}, \textsf{Domain}, and \textsf{Server}.
		
	\item \textsf{Server}: The server responsible for authenticating the user. It stores the comparison key $\mathsf{csk}$ for each domain and the user's enrollment message $\mathbf{c_x}$. On authentication, it compares the probe message with the registered enrollment message and returns the result.  

\end{itemize}


%-------------------

\input{tikz/dod_model.tex}

\pagebreak

\fi

%-------------------

\iffalse

\subsection{Instantiation with a 2i-IPFE Scheme}
\label{sec:2i-IPFE-instantiation}

Let .

\begin{itemize}

	\item 

	\item : The same as the scheme in \ref{sec:fh-IPFE-instantiation}. 

	\item .

	\item .

	\item .

	\item .

\end{itemize}

By the correctness of the functional encryption scheme $\sf FE$, we have
\[
	s = \mathsf{FE.Dec}(\mathsf{dk}_{\mathbf{I}}, \mathbf{c_x}, \mathbf{c_y}) =  \mathbf{x} \mathbf{I}_{k+2} \mathbf{y}^T = \mathbf{x} \mathbf{y}^T = \| \mathbf{b} - \mathbf{b}^\prime \|^2.
\]
just as the scheme in Section \ref{sec:fh-IPFE-instantiation}


Unlike the previous scheme, instantiated with a 2i-IPFE scheme in this way, the comparison secret key .

In the Device-of-Domain model, the indistinguishability of  enrolled by different samples. Therefore, we must limit the adversary's ability. For example, we can require the adversary to distinguish biometric vectors sampled from distributions in a pre-defined pool, and the adversary can only probe vectors randomly sampled from a distribution in the pool. We can also limit the rate of the probe oracle.

%-------------------

\subsection{Instantiation with a 2c-IPFE Scheme}
\label{sec:2c-IPFE-instantiation}

Note that if labels remain constant, a 2c-IPFE scheme is reduced to a 2i-IPFE scheme. Therefore, we can consider utilizing the label to represent each domain in the Device-of-Domain model. Let .

\begin{itemize}

	\item .

	Note that when the previous 2i-IPFE-based scheme in Section \ref{sec:2i-IPFE-instantiation} is applied to a Device-of-Domain model, we assume that  except different labels.

	\item : The same as the scheme in \ref{sec:2i-IPFE-instantiation}. 

	\item .

	\item .

	\item .

	\item .

	\item 

\end{itemize}

By the correctness of the functional encryption scheme $\sf FE$, if the labels of $\mathbf{c_x}$ and $\mathbf{c_y}$ are the same (they are of the same domain), we have
\[
	s = \mathsf{FE.Dec}(\mathsf{dk}_{\mathbf{I}}, \mathbf{c_x}, \mathbf{c_y}) =  \mathbf{x} \mathbf{I}_{k+2} \mathbf{y}^T = \| \mathbf{b} - \mathbf{b}^\prime \|^2.
\]
just as the scheme in Section \ref{sec:2i-IPFE-instantiation}

When the Device-of-Domain model is instantiated with a 2c-IPFE scheme in this way, the enrollment secret key .

If \textsf{Server} and \textsf{Domain} are both malicious, then the adversary can use . Therefore, we assume at most one party of them can be malicious at the same time. Note that this is the same as the 2i-IPFE-based scheme, where only one of \textsf{Server} and \textsf{Domain} can be malicious.

\fi

%-------------------



%-------------------
% Security Games
\section{Security Games}
\label{sec:security_game}
%%%%%%%%%%%%%%%%%%%%

% Project Name: Semester Project Fall 2024 for EPFL
% File: security_game.tex
% Author: Keng-Yu Chen

%%%%%%%%%%%%%%%%%%%%

In this section, we discuss two security notions of a biometric authentication scheme: \emph{unforgeability} and \emph{indistinguishability}.

\subsection{Unforgeability}
\label{sec:uf_game}

To describe the unforgeability of an authentication scheme, we model the ability of an adversary who tries to impersonate a user. The adversary $\mathcal{A}$ is given auxiliary information \textsf{option} that depends on our threat model and tries to find a valid probe message $\mathbf{\tilde{z}}$. The whole game $\textsf{UF}_{\Pi, \mathbb{B}, \textsf{option}}$ is defined in Algorithm \ref{alg:uf_game}.

\begin{figure}[h]
\centering
	\begin{minipage}[t]{0.6\linewidth}
	\centering
	\begin{algorithm}[H]
	\caption{$\textsf{UF}_{\Pi, \mathbb{B}, \textsf{option}}(\mathcal{A})$}
	\label{alg:uf_game}
	\begin{algorithmic}[1]
		\State $\mathcal{B} \getsdollar \mathbb{B}, \mathbb{B} \gets \mathbb{B} \setminus \mathcal{B}$

		\State $\mathsf{esk}, \mathsf{psk}, \mathsf{csk} \gets \mathsf{Setup}(1^\lambda)$
		
		\State $\mathbf{b} \gets \mathsf{getEnroll}^{\mathcal{O}_{\mathcal{B}}}()$

		\State $\mathbf{c_x} \gets \mathsf{Enroll}(\mathsf{esk}, \mathbf{b})$
		
		\State ${\mathbf{\tilde{z}}} \gets \mathcal{A} ( \textsf{option} )$
 
		\If{$\mathbf{\tilde{z}}$ is equal to any output of $\mathcal{O}_{\mathsf{Probe}}$ }
			
			\State \Return $0$
		
		\EndIf

		\State $s \gets \mathsf{Compare}( \mathsf{csk}, \mathbf{c_x}, \mathbf{\tilde{z}} )$

		\State \Return $\mathsf{Verify}(s)$
	\end{algorithmic}
	\end{algorithm}
	\end{minipage}
	
\label{fig:uf_game}
\end{figure}

The auxiliary information \textsf{option} can be nothing or include $\mathsf{esk}, \mathsf{psk}, \mathsf{csk}, \mathbf{c_x}$ or the following oracles:

\begin{itemize}

	\item $\mathcal{O}_{\mathcal{B}}$: It outputs a biometric sample $\mathbf{b} \getsdollar \mathcal{B}$. This oracle and $\mathsf{psk}$ should not be given at the same time; otherwise, there exists a trivial attack with a winning rate $\mathsf{TP}$ by returning $\mathsf{Probe}(\mathsf{psk}, \mathsf{getProbe}^{ \mathcal{O}_\mathcal{B} }() )$.
	
	\item $\mathcal{O}_{\mathsf{Enroll}}(\mathsf{esk}, \cdot)$: On input $\mathbf{b}^\prime$, it outputs the enrollment message $\mathsf{Enroll}(\mathsf{esk}, \mathbf{b}^\prime)$.

	\item $\mathcal{O}_{\mathsf{Probe}}(\mathsf{psk}, \cdot)$: On input $\mathbf{b}^\prime$, it outputs the probe message $\mathsf{Probe}(\mathsf{psk}, \mathbf{b}^\prime)$. If this oracle is given, we require the adversary to return a $\mathbf{\tilde{z}}$ that is not equal to any previous answer of $\mathcal{O}_\mathsf{Probe}$.
	
	\item $\mathcal{O}_\mathsf{log}(\mathsf{csk}, \mathbf{c_x}, \cdot)$: On input $\mathbf{b}^\prime$, it first computes $\mathbf{c_z} \gets \mathsf{Probe}(\mathsf{psk}, \mathbf{b}^\prime)$ and outputs $\mathsf{Verify}(\mathsf{Compare}(\mathsf{csk}, \mathbf{c_x}, \mathbf{c_z} ) )$.
	
	\item $\mathcal{O}_\textsf{Enroll}^\prime (\cdot)$: On input $\textsf{esk}^\prime$, it first samples $\mathbf{b}^\prime \gets \textsf{getEnroll}^{\mathcal{O}_{\mathcal{B}}}()$ and outputs $\textsf{Enroll}(\textsf{esk}^\prime, \mathbf{b}^\prime)$. This oracle is only useful when $\textsf{option}$ does not include $\mathcal{O}_{\mathcal{B}}$.

	\item $\mathcal{O}_\textsf{Probe}^\prime (\cdot)$: On input $\textsf{psk}^\prime$, it first samples $\mathbf{b}^\prime \gets \textsf{getProbe}^{\mathcal{O}_{\mathcal{B}}}()$ and outputs $\textsf{Probe}(\textsf{psk}^\prime, \mathbf{b}^\prime)$. This oracle is only useful when $\textsf{option}$ does not include $\mathcal{O}_{\mathcal{B}}$. This oracle and $\textsf{psk}$ should not be given at the same time; otherwise, there exists a trivial attack with a winning rate $\textsf{TP}$ by returning $\mathcal{O}_{\textsf{Probe}}^\prime (\textsf{psk})$.
	
\end{itemize}

%The requirement that the adversary should return a $\mathbf{\tilde{z}}$ that is not equal to any previous answer of $\mathcal{O}_\textsf{Probe}$ is to prevent a trivial attack that leverages \textsf{TP} or \textsf{FP} when it is not negligible.
%If \textsf{option} includes $\mathcal{O}_\mathcal{B}$ and either $\textsf{psk}$ or $\mathcal{O}_\textsf{Probe}$, the adversary can enjoy a winning rate \textsf{TP}. Therefore, we rule out the case that $\textsf{option}$ includes both $\textsf{psk}$ and $\mathcal{O}_\mathcal{B}$, and we forbid the adversary to return what $\mathcal{O}_\textsf{Probe}$ returns.
%If \textsf{option} has only $\textsf{psk}$ or $\mathcal{O}_\textsf{Probe}$, the $\textsf{UF}$ adversary $\mathcal{A}$ in Algorithm \ref{alg:adv:FP} can still enjoy a winning rate $\textsf{FP}$, if we place no restriction on the adversary's answer. Therefore, we only consider $\textsf{psk}$ in \textsf{option} when \textsf{FP} is negligible, and we restrict the adversary's answer when $\mathcal{O}_\textsf{Probe}$ is given.

We define the advantage of an adversary $\mathcal{A}$ in the $\textsf{UF}_{\Pi, \mathbb{B}, \textsf{option}}$ game of a scheme $\Pi$ associated with a family $\mathbb{B}$ of distributions as
\[
	\Adv^{\textsf{UF}}_{\Pi, \mathbb{B}, \mathcal{A}, \textsf{option}} := \Pr[\textsf{UF}_{\Pi, \mathbb{B}, \textsf{option}}(\mathcal{A}) \to 1]
\]

An authentication scheme $\Pi$ associated with a family $\mathbb{B}$ of distributions is called \emph{\textsf{option}-unforgeable} (\textsf{option}-UF) if for any PPT adversary $\mathcal{A}$,
\[
	\Adv^{\textsf{UF}}_{\Pi, \mathbb{B}, \mathcal{A}, \textsf{option}} = \negl.
\]

For the rest of this work, if the scheme $\Pi$, the family $\mathbb{B}$ of distributions, and the auxiliary information $\textsf{option}$ are clear from context, we omit the subscript and write the game as $\textsf{UF}(\mathcal{A})$. This abbreviation also holds for all other games.

\paragraph{Choice of \textsf{option}}
\noindent Consider a \textsf{UF} adversary $\mathcal{A}$ in Algorithm \ref{alg:adv:FP}. The $\textsf{option}$ includes either $\textsf{psk}$ or $\mathcal{O}_{\textsf{Probe}}$. The advantage of $\mathcal{A}$ is
\begin{align*}
	& \Pr[\textsf{UF}_{\Pi, \mathbb{B}, \textsf{option}}(\mathcal{A}) \to 1] \\ 
	=\; & \Pr [ \textsf{Verify}(\textsf{Compare}(\textsf{csk}, \mathbf{c_x}, \mathbf{c_y})) = 1 ] \\
	\leq\; & \Pr \left [
		\begin{aligned}
			& \mathcal{B} \getsdollar \mathbb{B}, \mathbb{B} \gets \mathbb{B} \setminus \mathcal{B} \\
			& \mathbf{b} \gets \textsf{getEnroll}^{\mathcal{O}_{\mathcal{B}} }() \\
			& \mathcal{B}^\prime \getsdollar \mathbb{B} \\
			& \mathbf{b}^\prime \gets \textsf{getProbe}^{\mathcal{O}_{\mathcal{B}^\prime} }()
		\end{aligned}
		: \textsf{Verify}(\textsf{BioCompare}(\mathbf{b}, \mathbf{b}^\prime)) = 1 \right ] + \negl. \\
	=\; & \textsf{FP} + \negl.
\end{align*}

\noindent If $\textsf{psk}$ and $\mathcal{O}_{\mathcal{B}}$ are given at the same time, the adversary can even win with a probability \textsf{TP} by returning a probe from the distribution $\mathcal{B}$. To prevent such trivial attacks, we add the following requirements:

\begin{itemize}
	\item \textsf{option} only includes $\textsf{psk}$ when \textsf{FP} is negligible.

	\item \label{item:requirement} The adversary is not allowed to return what $\mathcal{O}_\textsf{Probe}$ returns.
	
	\item \textsf{option} cannot include both $\textsf{psk}$ and $\mathcal{O}_{\mathcal{B}}$.
\end{itemize}

%If $\textsf{option}$ includes $\mathcal{O}_{\textsf{Enroll}}^\prime$ and either $\textsf{psk}$ or $\mathcal{O}_\textsf{Probe}$, and if we place no restriction on an \textsf{UF} game adversary's answer, the adversary in Algorithm \ref{alg:adv:FP2} can win with a probability
%\[
	%\Pr[ \textsf{Verify}( \textsf{BioCompare}(\mathbf{x}^{(0)}, \mathbf{y}) ) = 1 \mid \textsf{Verify}( \textsf{BioCompare}(\mathbf{x}^{(1)}, \mathbf{y}) ) = 1]
%\]
%where $\mathbf{x}^{(0)}, \mathbf{x}^{(1)}$ are generated from $\textsf{encodeEnroll}^{\mathcal{O}_\mathcal{B}}()$ and $\mathbf{y} \gets \textsf{encodeProbe}^{\mathcal{O}_{\mathcal{B}^\prime}}() $.
%This value is in general not negligible.
%The expected number of repetitions is $\mathbb{E}_{\mathcal{B} \getsdollar \mathbb{B}}\left[\frac{1}{\textsf{FP}(\mathcal{B})} \right]$. If $\textsf{FP}(\mathcal{B})$ is non-negligible, the adversary can return the answer in an expected polynomial time. A similar adversary also exists when $\textsf{option}$ includes $\mathcal{O}_{\textsf{Probe}}^\prime$ and $\mathcal{O}_\textsf{Probe}$.

 
%\begin{figure}[h]
%\centering
	%\begin{minipage}[t]{0.6\linewidth}
	%\centering
	%\begin{algorithm}[H]
	%\caption{$\mathcal{A}^{\mathcal{O}_\textsf{Enroll}^\prime}(\textsf{psk})$ (or  $\mathcal{A}^{\mathcal{O}_\textsf{Enroll}^\prime, \mathcal{O}_\textsf{Probe}}$ ) }
	%\label{alg:adv:FP2}
	%\begin{algorithmic}[1]
		%\State $\textsf{esk}^\prime, \textsf{psk}^\prime, \textsf{csk}^\prime \gets \textsf{Setup}(1^\lambda)$

		%\Repeat
		
			%\State $\mathcal{B}^\prime \getsdollar \mathbb{B}$
		
			%\State $\mathbf{y}^\prime \gets \textsf{encodeProbe}^{\mathcal{O}_{\mathcal{B}^\prime }}()$ 

			%\State $\mathbf{c_y}^\prime \gets \textsf{Probe}(\textsf{psk}^\prime, \mathbf{y}^\prime)$

			%\State $\mathbf{c_x}^\prime \gets \mathcal{O}_\textsf{Enroll}^\prime (\textsf{esk}^\prime)$

		%\Until{ $\textsf{Verify}(\textsf{Compare}(\textsf{csk}^\prime, \mathbf{c_x}^\prime, \mathbf{c_y}^\prime )) = 1$ }

		%\State $\mathbf{c_y} \gets \textsf{Probe}(\textsf{psk}, \mathbf{y}^\prime)$  \Comment{ or $\mathbf{c_y} \gets \mathcal{O}_\textsf{Probe}(\mathbf{y}^\prime)$ } 

		%\State \Return $\mathbf{c_y}$
	%\end{algorithmic}
	%\end{algorithm}
	%\end{minipage}
	
%\end{figure}

\begin{figure}[h]
\centering
	\begin{minipage}[t]{0.6\linewidth}
	\centering
	\begin{algorithm}[H]
	\caption{$\mathcal{A}(\textsf{psk})$ ( or $\mathcal{A}^{\mathcal{O}_\textsf{Probe}}$ ) }
	\label{alg:adv:FP}
	\begin{algorithmic}[1]
		\State $\mathcal{B}^\prime \getsdollar \mathbb{B}$
		
		\State $\mathbf{b}^\prime \gets \textsf{getProbe}^{\mathcal{O}_{\mathcal{B}^\prime }}()$

		\State $\mathbf{c_y} \gets \textsf{Probe}(\textsf{psk}, \mathbf{b}^\prime)$ \Comment{or $\mathbf{c_y} \gets \mathcal{O}_\textsf{Probe}(\mathbf{b}^\prime)$ }

		\State \Return $\mathbf{c_y}$
	\end{algorithmic}
	\end{algorithm}
	\end{minipage}
	
\end{figure}


\paragraph{UF Security with Digital Signature}

We note that we can achieve UF security by a similar approach in \cite{cryptoeprint:2023/481} with a digital signature scheme. Given any authentication scheme $\Pi$ and an sEUF-CMA digital signature scheme $\textsf{Sig} = (\textsf{Sig.KeyGen}, \textsf{Sig.Sign}, \textsf{Sig.Verify})$, consider the following scheme $\Pi^\prime$.

\begin{itemize}

	\item $\textsf{Setup}^\prime (1^\lambda)$: Run $(\textsf{esk}, \textsf{psk}, \textsf{csk}) \gets \textsf{Setup}(1^\lambda)$ and $(\textsf{sk}_{\textsf{Sig}}, \textsf{pk}_{\textsf{Sig}}) \gets \textsf{Sig.KeyGen}(1^\lambda)$. Output $\textsf{esk}^\prime \gets \textsf{esk}$, $\textsf{psk}^\prime \gets (\textsf{psk}, \textsf{sk}_{\textsf{Sig}})$, $\textsf{csk}^\prime \gets \textsf{csk}$.

	\item $\textsf{Enroll}^\prime$: The same as $\textsf{Enroll}$.

	\item $\textsf{Probe}^\prime (\textsf{psk}^\prime, \mathbf{b}^\prime)$: Run $\mathbf{c_y} \gets \textsf{Probe}(\textsf{psk}, \mathbf{b}^\prime)$ and $\sigma \gets \textsf{Sig.Sign}(\textsf{sk}_{\textsf{Sig}}, \mathbf{c_y})$. Output $\mathbf{c_y}^\prime \gets (\mathbf{c_y}, \sigma)$.

	\item $\textsf{Compare}^\prime (\textsf{csk}, \mathbf{c_x}, \mathbf{c_y}^\prime)$: If $\textsf{Sig.Verify}(\textsf{pk}_{\textsf{Sig}}, \mathbf{c_y}, \sigma) = 1$, output $\textsf{Compare}(\textsf{csk}, \mathbf{c_x}, \mathbf{c_y})$; otherwise, output $\bot$.

\end{itemize}

An $\textsf{UF}_\textsf{option}$ adversary has to forge a signature $\sigma$ to win the game, so the scheme is $\textsf{option}$-UF for any $\textsf{option}$ that does not include $\textsf{psk}$. 

\begin{theorem}
\label{thm:sEUF-CMA-esk-csk}
	Let $\textsf{option} = \{ \textsf{esk}, \textsf{csk}, \mathbf{c_x}, \mathcal{O}_\mathcal{B}, \mathcal{O}_{\textsf{Probe}} \}$. For any authentication scheme $\Pi$, $\Pi^\prime$ is $\textsf{option}$-UF. 
\end{theorem}

Note that Theorem \ref{thm:sEUF-CMA-esk-csk} also holds for the following trivial authentication scheme for any biometric layer.
\begin{itemize}
	\item $\textsf{Setup} (1^\lambda)$: $\textsf{esk} = \textsf{psk} = \textsf{csk}$ are all empty strings.
	\item $\textsf{Enroll} (\textsf{esk}, \mathbf{b}) \to \mathbf{b}$.
	\item $\textsf{Probe} (\textsf{psk}, \mathbf{b}^\prime) \to \mathbf{b}^\prime$.
	\item $\textsf{Compare} (\textsf{csk}, \mathbf{c}, \mathbf{c}^\prime) = \textsf{BioCompare}(\mathbf{c}, \mathbf{c}^\prime)$.
\end{itemize}


%-------------------


\subsection{Indistinguishability}
\label{sec:ind_game}

In the game of indistinguishability, we model the ability of an authentication server who tries to identify the user, which describes the privacy leakage of the scheme. The adversary $\mathcal{A}$ is given oracles to two biometric distributions $\mathcal{B}^{(0)}$ and $ \mathcal{B}^{(1)}$ and \textsf{option} that depends on our threat model. It tries to guess from either $\mathcal{B}^{(0)}$ or $ \mathcal{B}^{(1)}$ the enrollment or probe messages are generated. The whole game $\textsf{IND}_{\Pi, \mathbb{B}, \textsf{option}}$ is defined in Algorithm \ref{alg:ind_game}.

\begin{figure}[h]
\centering

	\begin{minipage}[t]{0.55\textwidth}
	\begin{algorithm}[H]
	\caption{$\textsf{IND}_{\Pi, \mathbb{B}, \textsf{option}}(\mathcal{A})$}
	\label{alg:ind_game}
	\begin{algorithmic}[1]
		\State $b \getsdollar \{0, 1\}$

		\State $\mathcal{B}^{(0)} \getsdollar \mathbb{B}, \quad \mathbb{B} \gets \mathbb{B} \setminus \mathcal{B}^{(0)}$

		\State $\mathcal{B}^{(1)} \getsdollar \mathbb{B}, \quad \mathbb{B} \gets \mathbb{B} \setminus \mathcal{B}^{(1)}$

		\State $\textsf{esk}, \textsf{psk}, \textsf{csk} \gets \textsf{Setup}(1^\lambda)$

		\State $\mathbf{b} \gets \textsf{getEnroll}^{\mathcal{O}_{\mathcal{B}^{(b)}}}()$

		\State $\mathbf{c_x} \gets \textsf{Enroll}(\textsf{esk}, \mathbf{b})$

		%\For{$i = 1$ to $t$}

			%\State ${\mathbf{b}^\prime}^{(i)} \gets \textsf{getProbe}^{\mathcal{O}_{\mathcal{B}^{(b)}}}() $
		
			%\State $\mathbf{c_y}^{(i)} \gets \textsf{Probe}( \textsf{psk}, {\mathbf{b}^\prime}^{(i)} )$

		%\EndFor

		%\State In Device-of-User Model:
		
			%\State \hspace{\algorithmicindent} $\tilde{b} \gets \mathcal{A}^{\mathcal{O}_{\mathcal{B}^{(0)}}, \mathcal{O}_{\mathcal{B}^{(1)}} } ( \textsf{csk}, \mathbf{c_x}, \{ \mathbf{c_y}^{(i)} \}_{i=1}^t )$

		%\State In Device-of-Domain Model:
		
			\State $\tilde{b} \gets \mathcal{A}^{\mathcal{O}_{\mathcal{B}^{(0)}}, \mathcal{O}_{\mathcal{B}^{(1)}}} ( \textsf{option} )$

		\State \Return $1_{\tilde{b} = b}$
	\end{algorithmic}
	\end{algorithm}
	\end{minipage}

%\caption{The \textsf{IND} Game}
\label{fig:ind_game}
\end{figure}

The auxiliary information \textsf{option} can be nothing or include $\textsf{esk}, \textsf{psk}, \textsf{csk}, \mathbf{c_x} $ or the following oracles:

\begin{itemize}

	\item $\mathcal{O}_{\mathbf{c_y}}$: It first samples a biometric sample $\mathbf{b}^{\prime} \getsdollar \textsf{getProbe}^{\mathcal{B}^{(b)} }()$ and outputs $\mathbf{c_y} \gets \textsf{Probe}(\textsf{psk}, \mathbf{b}^{\prime})$.

	\item $\mathcal{O}_\textsf{Enroll}(\textsf{esk}, \cdot)$: On input $\mathbf{b}^\prime$, it outputs the enrollment message $\textsf{Enroll}(\textsf{esk}, \mathbf{b}^\prime)$.

	\item $\mathcal{O}_\textsf{Probe}(\textsf{psk}, \cdot)$: On input $\mathbf{b}^\prime$, it outputs the probe message $\textsf{Probe}(\textsf{psk}, \mathbf{b}^\prime)$.
	
	\item $\mathcal{O}_\textsf{CompVrfy}(\textsf{csk}, \cdot, \cdot)$: On input $\mathbf{c}$ and $\mathbf{c}^\prime$, it first computes $s \gets \textsf{Compare}(\textsf{csk}, \mathbf{c}, \mathbf{c}^\prime)$ and outputs $\textsf{Verify}(s)$.
\end{itemize}

\noindent Note that at least one of $\mathbf{c_x}$ and $\mathcal{O}_{\mathbf{c_y}}$ should be given; otherwise, IND security is trivial.

We define the advantage of an adversary $\mathcal{A}$ in the $\textsf{IND}_{\Pi, \mathbb{B}, \textsf{option}}$ game of a scheme $\Pi$ associated with a family of distributions $\mathbb{B}$ as
\[
	\Adv^{\textsf{IND}}_{\Pi, \mathbb{B}, \mathcal{A}, \textsf{option}} := \left |\Pr[\textsf{IND}_{\Pi, \mathbb{B}, \textsf{option}}(\mathcal{A}) \to 1] - \frac{1}{2} \right|.
\]

An authentication scheme $\Pi$ associated with a family $\mathbb{B}$ of distributions is called \emph{\textsf{option}-indistinguishable} (\textsf{option}-IND) if for any PPT adversary $\mathcal{A}$,
\[
	\Adv^{\textsf{IND}}_{\Pi, \mathbb{B}, \mathcal{A}, \textsf{option}} = \negl.
\]


\paragraph{Trivial Attacks}
We note that if $\textsf{TP} - \textsf{FP} > \frac{1}{\poly}$, there are trivial attacks.

\begin{theorem}
\label{thm:ind-tp-fp}
Given any distribution family $\mathbb{B}$ that $\textsf{TP} - \textsf{FP} > \frac{1}{\poly}$, $\Pi$ is not $\textsf{option}$-IND for
\begin{itemize}
	\item $\textsf{option} = \{\mathbf{c_x}, \text{ either } \textsf{csk} \text{ or } \mathcal{O}_{\textsf{CompVrfy}}, \text{ either } \textsf{psk} \text{ or } \mathcal{O}_{\textsf{Probe}} \}$

	\item $\textsf{option} = \{\mathcal{O}_{\mathbf{c_y}}, \text{ either } \textsf{csk} \text{ or } \mathcal{O}_{\textsf{CompVrfy}}, \text{ either } \textsf{esk} \text{ or } \mathcal{O}_{\textsf{Enroll}} \}$
\end{itemize}

\end{theorem}

\begin{proof}

Let $\textsf{option}$ be the first case that includes $\mathbf{c_x}$. The second case when $\mathcal{O}_{\mathbf{c_y}}$ is given can be analyzed analogously. Consider the adversary $\mathcal{A}$ in the $\textsf{IND}_{\textsf{option}}$ game in Algorithm \ref{alg:ind-tp-fp}. When the challenge bit $b = 0$, the probability that $\mathcal{A}$ wins is $\textsf{TP}$. When the challenge bit $b = 1$, the probability that $\mathcal{A}$ wins is $1 - \textsf{FP}$. Now, 

\[
	\Adv_{\Pi, \mathbb{B}, \mathcal{A}, \textsf{option}}^{\textsf{IND}} = \left| \Pr[ \textsf{IND}_\Pi(\mathcal{A}) \to 1 ] - \frac{1}{2} \right| = \left| \frac{1}{2} (\textsf{TP} + 1 - \textsf{FP}) - \frac{1}{2} \right| > \frac{1}{\poly}.
\]

\begin{figure}[h]
\centering

	\begin{minipage}[t]{0.85\textwidth}
	\begin{algorithm}[H]
	\caption{$\mathcal{A}^{\mathcal{O}_{\mathcal{B}^{(0)}}, \mathcal{O}_{\mathcal{B}^{(1)}}} (\mathsf{psk}, \mathsf{csk}, \mathbf{c_x})$ or $\mathcal{A}^{\mathcal{O}_{\mathcal{B}^{(0)}}, \mathcal{O}_{\mathcal{B}^{(1)}}, \mathcal{O}_{\mathsf{Probe}}, \mathcal{O}_{\mathsf{CompVrfy}} } (\mathbf{c_x})$}
	\label{alg:ind-tp-fp}
	\begin{algorithmic}[1]

		\State $\mathbf{b}^{(0)} \gets \mathsf{getProbe}^{\mathcal{O}_{\mathcal{B}^{(0)}}}()$
		
		\State $\mathbf{c_y}^{(0)} \gets \mathsf{Probe}(\mathsf{psk}, \mathbf{b}^{(0)})$ \Comment{or $\mathbf{c_y}^{(0)} \gets \mathcal{O}_{\mathsf{Probe}}(\mathbf{b}^{(0)})$ }
		
		\State $r \gets \textsf{Verify}( \textsf{Compare}(\textsf{csk}, \mathbf{c_x}, \mathbf{c_y}^{(0)}) ) $ \Comment{or $r \gets \mathcal{O}_{\textsf{CompVrfy}}( \mathbf{c_x}, \mathbf{c_y}^{(0)}) $}
		
		\State \Return $1 - r$
	\end{algorithmic}
	\end{algorithm}
	\end{minipage}

\end{figure}

\end{proof}


\paragraph{Necessity of IND Security}

Recall the trivial authentication scheme for any biometric layer we introduced in Section \ref{sec:uf_game}. By Theorem \ref{thm:sEUF-CMA-esk-csk}, we can obtain an authentication scheme $\Pi^\prime$ that is $\mathsf{option}$-UF for any $\mathsf{option}$ that does not include $\textsf{psk}$.
However, the enrollment and probe messages leak biometric vectors $\mathbf{b}$ and $\mathbf{b}^\prime$ and compromise privacy. Obviously, this scheme is not \textsf{option}-IND for an \textsf{option} that includes either $\mathbf{c_x}$ or $\mathcal{O}_{\mathbf{c_y}}$ when $\mathsf{TP} - \mathsf{FP} > \frac{1}{\poly}$.

\paragraph{IND Security for a Particular Biometric Layer}

Let $\mathsf{getEnroll}^{\mathcal{O}_{\mathcal{B}}}(), \textsf{getProbe}^{\mathcal{O}_{\mathcal{B}}}()$ be such that
\[
	\textsf{getEnroll}^{\mathcal{O}_{\mathcal{B}}}() \to \mathbf{b}^*_{\mathcal{B}} \oplus \mathcal{E}_{\textsf{Enroll}}  \quad \text{and} \quad \textsf{getProbe}^{\mathcal{O}_{\mathcal{B}}}() \to \mathbf{b}^*_{\mathcal{B}} \oplus \mathcal{E}_{\textsf{Probe}}
\]
where $\mathbf{b}^*_{\mathcal{B}} \in \{0, 1\}^k$ is a fixed vector only dependent on $\mathcal{B}$, and $\mathcal{E}_{\mathsf{Enroll}}, \mathcal{E}_{\mathsf{Probe}} \subseteq \{0, 1\}^k$ are some \emph{error distributions} independent of $\mathcal{B}$. Let $\mathsf{BioCompare}(\mathbf{b}, \mathbf{b}^\prime) \to 1_{\mathsf{HD}(\mathbf{b}, \mathbf{b}^\prime) \leq \tau}$. Then
\[
	\textsf{TP} = \Pr[ \textsf{HW}(\mathbf{b}^*_{\mathcal{B}} \oplus \mathcal{E}_{\textsf{Enroll}} \oplus \mathbf{b}^*_{\mathcal{B}} \oplus \mathcal{E}_{\textsf{Probe}}) \leq \tau ] = \Pr[ \textsf{HW}(\mathcal{E}_{\textsf{Enroll}} \oplus \mathcal{E}_{\textsf{Probe}}) \leq \tau ]
\]

\noindent We note that previous works such as \cite{10.1145/1030083.1030096,cryptoeprint:2014/394} model biometric template vectors in a similar way.

Now, for this biometric layer, we can construct a simple but IND secure authentication scheme $\Pi$ with the following cryptographic layer.

\begin{itemize}

	\item $\mathsf{Setup} (1^\lambda)$: Sample $\mathbf{r} \getsdollar \{0, 1\}^k$. Output $\mathsf{esk} = \mathsf{psk} \gets \mathbf{r}$, $\mathsf{csk} \gets \epsilon$.

	\item $\mathsf{Enroll}(\mathsf{esk}, \mathbf{b})$: Output $\mathbf{b} \oplus \mathbf{r}$.
	
	\item $\mathsf{Probe}(\mathsf{psk}, \mathbf{b}^\prime)$: Output $\mathbf{b}^\prime \oplus \mathbf{r}$.

	\item $\mathsf{Compare} (\mathsf{csk}, \mathbf{c_x}, \mathbf{c_y})$: If $\mathsf{HD}(\mathbf{c_x}, \mathbf{c_y}) \leq \tau$, return $1$; otherwise, return $0$. 

\end{itemize}
The correctness of $\Pi$ holds by
\[
	\textsf{HD}(\mathbf{c_x}, \mathbf{c_y}) = \textsf{HW}(\mathbf{b} \oplus \mathbf{r} \oplus \mathbf{b}^\prime \oplus \mathbf{r}) = \textsf{HW}(\mathbf{b} \oplus \mathbf{b}^\prime) = \textsf{BioCompare}(\mathbf{b}, \mathbf{b}^\prime).
\]

\begin{theorem}
\label{thm:rh:ind:particular-biometri-layer}
Let $\mathsf{option} = \{\mathsf{csk}, \mathbf{c_x}, \mathcal{O}_{\mathbf{c_y}}\}$. The authentication scheme $\Pi$ is \textsf{option}-IND.

\end{theorem}

\begin{proof}

Let $\mathbf{b}^*_0$ and $\mathbf{b}^*_1$ be the fixed vectors of $\mathcal{B}^{(0)}$ and $\mathcal{B}^{(1)}$ in the \textsf{IND} game, respectively. Given any adversary, assume that the number of its queries to $\mathcal{O}_{\mathbf{c_y}}$ is bounded by $t$. For any $\mathbf{v}, \mathbf{v}^{(1)}, \cdots, \mathbf{v}^{(t)}$,
\begin{align*}
	& \Pr[ \mathbf{c_x} = \mathbf{v},\; \mathbf{c_y}^{(1)} = \mathbf{v}^{(1)},\; \cdots,\; \mathbf{c_y}^{(t)} = \mathbf{v}^{(t)} \mid b = 0,\; \mathbf{b}^*_0,\; \mathbf{b}^*_1 ] \\
	&= \Pr[ \mathbf{b}^*_0 \oplus \mathcal{E}_{\textsf{Enroll}} \oplus \mathbf{r} = \mathbf{v},\; \mathbf{b}^*_0 \oplus \mathcal{E}_{\textsf{Probe}} \oplus \mathbf{r} =\mathbf{v}^{(1)},\; \cdots,\; \mathbf{b}^*_0 \oplus \mathcal{E}_{\textsf{Probe}} \oplus \mathbf{r} =\mathbf{v}^{(t)} \mid \mathbf{b}^*_0,\; \mathbf{b}^*_1] \\ 
	&= \Pr[ \mathbf{r} = \mathbf{v} \oplus \mathbf{b}^*_0 \oplus \mathcal{E}_{\textsf{Enroll}} =\mathbf{v}^{(1)} \oplus \mathbf{b}^*_0 \oplus \mathcal{E}_{\textsf{Probe}} = \cdots = \mathbf{v}^{(t)} \oplus \mathbf{b}^*_0 \oplus \mathcal{E}_{\textsf{Probe}} \mid \mathbf{b}^*_0,\; \mathbf{b}^*_1] \\ 
	&= \Pr[ \mathbf{r} = \mathbf{v} \oplus \mathbf{b}^*_1 \oplus \mathcal{E}_{\textsf{Enroll}} = \mathbf{v}^{(1)} \oplus \mathbf{b}^*_1 \oplus \mathcal{E}_{\textsf{Probe}} = \cdots = \mathbf{v}^{(t)} \oplus \mathbf{b}^*_1 \oplus \mathcal{E}_{\textsf{Probe}} \mid \mathbf{b}^*_0,\; \mathbf{b}^*_1] \\ 
	&= \Pr[ \mathbf{b}^*_1 \oplus \mathcal{E}_{\textsf{Enroll}} \oplus \mathbf{r} = \mathbf{v},\; \mathbf{b}^*_1 \oplus \mathcal{E}_{\textsf{Probe}} \oplus \mathbf{r} =\mathbf{v}^{(1)},\; \cdots,\; \mathbf{b}^*_1 \oplus \mathcal{E}_{\textsf{Probe}} \oplus \mathbf{r} =\mathbf{v}^{(t)} \mid \mathbf{b}^*_0,\; \mathbf{b}^*_1] \\ 
	&= \Pr[ \mathbf{c_x} = \mathbf{v},\; \mathbf{c_y}^{(1)} = \mathbf{v}^{(1)},\; \cdots,\; \mathbf{c_y}^{(t)} = \mathbf{v}^{(t)} \mid b = 1,\; \mathbf{b}^*_0,\; \mathbf{b}^*_1  ]
\end{align*}
Hence, the adversary cannot distinguish between $\mathbf{c_x}, \mathbf{c_y}^{(1)}, \cdots, \mathbf{c_y}^{(t)}$ generated from $\mathcal{B}^{(0)}$ and $\mathcal{B}^{(1)}$.

\end{proof}

\paragraph{IND Security with Public-Key Encryption}

We note that we can achieve IND security with a public-key encryption scheme. We first recall the \emph{multi-challenge IND-CPA} security for a public-key encryption scheme.

\begin{definition}[Multi-challenge IND-CPA]

A public-key encryption scheme $\mathsf{PKE} = (\mathsf{PKE.KeyGen}, \mathsf{PKE.Enc}, \mathsf{PKE.Dec})$ is multi-challenge IND-CPA if for any adversary $\mathcal{A}$ , the advantage of $\mathcal{A}$ in the game $\mathsf{MC-IND-CPA}_{\mathsf{PKE}}$ is

\[
	\Adv^{\mathsf{MC-IND-CPA}}_{\mathsf{PKE}, \mathcal{A}} := \left| \Pr[ \mathsf{MC-IND-CPA}_{\mathsf{PKE}}(\mathcal{A}) \to 1 ] - \frac{1}{2} \right| = \negl.
\]

\begin{figure}[h]
\centering

	\begin{minipage}[t]{0.6\textwidth}
	\begin{algorithm}[H]
	\caption{$\textsf{MC-IND-CPA}_{\mathsf{PKE}} (\mathcal{A})$ }
	\label{alg:MC-IND-CPA}
	\begin{algorithmic}[1]
		\State $b \getsdollar \{0, 1\}$

		\State $(\mathsf{sk}_{\mathsf{PKE}}, \mathsf{pk}_{\mathsf{PKE}}) \gets \mathsf{PKE.KeyGen}(1^\lambda)$
		
		\State $\tilde{b} \gets \mathcal{A}^{\mathcal{O}_{\textsf{MC-Enc}}}(\mathsf{pk}_{\mathsf{PKE}})$ 
		
		\State \Return $1_{\tilde{b} = b}$
	\end{algorithmic}
	\end{algorithm}
	\end{minipage}

\end{figure}
\end{definition}

\begin{itemize}
	\item $\mathcal{O}_{\textsf{MC-Enc}}(\cdot, \cdot)$: On input $(\mathsf{m}_0, \mathsf{m}_1)$, it outputs $\mathsf{PKE.Enc}(\mathsf{pk}_{\mathsf{PKE}}, \mathsf{m}_b)$
\end{itemize}

Given any authentication scheme $\Pi$ and a multi-challenge IND-CPA public-key encryption scheme $\mathsf{PKE} = (\mathsf{PKE.KeyGen}, \allowbreak \mathsf{PKE.Enc}, \mathsf{PKE.Dec})$, consider the following scheme $\Pi^\prime$. 

\begin{itemize}

	\item $\mathsf{Setup}^\prime (1^\lambda)$: Run $(\mathsf{esk}, \mathsf{psk}, \mathsf{csk}) \gets \mathsf{Setup}(1^\lambda)$ and $(\mathsf{sk}_{\mathsf{PKE}}, \mathsf{pk}_{\mathsf{PKE}}) \gets \mathsf{PKE.KeyGen}(1^\lambda)$. Output $\mathsf{esk}^\prime \gets ( \mathsf{esk}, \mathsf{pk}_{\mathsf{PKE}} ) $, $\mathsf{psk}^\prime \gets (\mathsf{psk}, \mathsf{pk}_{\mathsf{PKE}})$, $\mathsf{csk}^\prime \gets (\mathsf{csk}, \mathsf{sk}_{\mathsf{PKE}} )$.

	\item $\mathsf{Enroll}^\prime (\mathsf{esk}^\prime, \mathbf{b})$: Run $\mathbf{c_x} \gets \mathsf{Enroll}(\mathsf{esk}, \mathbf{b})$ and $\mathsf{ct}_{\mathbf{x}} \gets \mathsf{PKE.Enc}(\mathsf{pk}_{\mathsf{PKE}}, \mathbf{c_x})$. Output $\mathbf{c_x}^\prime \gets \mathsf{ct}_{\mathbf{x}}$.

	\item $\mathsf{Probe}^\prime (\textsf{psk}^\prime, \mathbf{b}^\prime)$: Run $\mathbf{c_y} \gets \textsf{Probe}(\textsf{psk}, \mathbf{b}^\prime)$ and $\textsf{ct}_{\mathbf{y}} \gets \textsf{PKE.Enc}(\textsf{pk}_{\textsf{PKE}}, \mathbf{c_y})$. Output $\mathbf{c_y}^\prime \gets \textsf{ct}_{\mathbf{y}}$.

	\item $\textsf{Compare}^\prime (\textsf{csk}^\prime, \mathbf{c_x}^\prime, \mathbf{c_y}^\prime)$: First decrypt $\mathbf{c_x} \gets \textsf{PKE.Dec}(\textsf{sk}_{\textsf{PKE}}, \mathbf{c_x}^\prime)$ and  $\mathbf{c_y} \gets \textsf{PKE.Dec}(\textsf{sk}_{\textsf{PKE}}, \mathbf{c_y}^\prime)$. Output $\textsf{Compare}(\textsf{csk}, \mathbf{c_x}, \mathbf{c_y})$.
\end{itemize}

\begin{theorem}
\label{thm:mc-ind-cpa:ind-esk-psk}
	Let $\textsf{option} = \{ \textsf{esk}, \textsf{psk}, \mathbf{c_x}, \mathcal{O}_{\mathbf{c_y}} \}$. For any authentication scheme $\Pi$, $\Pi^\prime$ is $\textsf{option}$-IND. 
\end{theorem}

\begin{proof}

Given an adversary $\mathcal{A}$ in the $\textsf{IND}_{\textsf{option}}$ game, consider the reduction adversary $\mathcal{R}$ in Algrothm \ref{red:mc-ind-cpa:ind-esk-psk-ox-oc} which plays the $\textsf{MC-IND-CPA}_{\textsf{PKE}}$ game by running $\mathcal{A}$. $\mathcal{R}$ simulates $\mathcal{O}_{\mathbf{c_y}}$ by the following steps.

\begin{enumerate}
	\item Sample ${\mathbf{b}^{\prime}}^{(0)} \gets \textsf{getProbe}^{\mathcal{O}_{\mathcal{B}^{(0)}}}()$ and ${\mathbf{b}^{\prime}}^{(1)} \gets \textsf{getProbe}^{\mathcal{O}_{\mathcal{B}^{(1)}}}()$.
	\item Run $\mathbf{c_y}^{(0)} \gets \textsf{Probe}(\textsf{psk}, {\mathbf{b}^{\prime}}^{(0)} )$ and $\mathbf{c_y}^{(1)} \gets \textsf{Probe}(\textsf{psk}, {\mathbf{b}^{\prime}}^{(1)} )$
	\item Output $\mathcal{O}_{\textsf{MC-Enc}}(\mathbf{c_y}^{(0)}, \mathbf{c_y}^{(1)})$.
\end{enumerate}

\begin{figure}[h]
\centering
	
	\begin{minipage}[t]{0.7\linewidth}
	\centering
	\begin{algorithm}[H]
	\caption{$\mathcal{R}^{\mathcal{O}_{\textsf{MC-Enc}}}(\textsf{pk}_{\textsf{PKE}})$}
	\label{red:mc-ind-cpa:ind-esk-psk-ox-oc}
	\begin{algorithmic}[1]
		\State $\mathcal{B}^{(0)} \getsdollar \mathbb{B}, \quad \mathbb{B} \gets \mathbb{B} \setminus \mathcal{B}^{(0)}$
		
		\State $\mathcal{B}^{(1)} \getsdollar \mathbb{B}, \quad \mathbb{B} \gets \mathbb{B} \setminus \mathcal{B}^{(1)}$

		\State $\textsf{esk}, \textsf{psk}, \textsf{csk} \gets \textsf{Setup}(1^\lambda)$

		\State $\mathbf{b}^{(0)} \gets \textsf{getProbe}^{\mathcal{O}_{\mathcal{B}^{(0)}}}(), \mathbf{c_x}^{(0)} \gets \textsf{Enroll}(\textsf{esk}, \mathbf{b}^{(0)})$
		
		\State $\mathbf{b}^{(1)} \gets \textsf{getProbe}^{\mathcal{O}_{\mathcal{B}^{(1)}}}(), \mathbf{c_x}^{(1)} \gets \textsf{Enroll}(\textsf{esk}, \mathbf{b}^{(1)})$
		
		\State $\mathbf{c_x}^\prime \gets \mathcal{O}_{\textsf{MC-Enc}}(\mathbf{c_x}^{(0)}, \mathbf{c_x}^{(1)})$

		\State $\textsf{esk}^\prime \gets (\textsf{esk}, \textsf{pk}_{\textsf{PKE}}), \textsf{psk}^\prime \gets (\textsf{psk}, \textsf{pk}_{\textsf{PKE}})$

		\State $ \tilde{b} \gets {\mathcal{A}}^{\mathcal{O}_{\mathcal{B}^{(0)}}, \mathcal{O}_{\mathcal{B}^{(1)}}, \mathcal{O}_{\mathbf{c_y}} } (\textsf{esk}^\prime, \textsf{psk}^\prime, \mathbf{c_x}^\prime)$ \label{alg:red:ind-cpa:ind-esk-psk-ox-oc}

		\State \Return $\tilde{b}$

	\end{algorithmic}
	\end{algorithm}
	\end{minipage}
	
\end{figure}

\noindent Since $\mathcal{R}$ simulates an $\textsf{IND}_{\textsf{option}}$ game for $\mathcal{A}$, the advantage of $\mathcal{A}$ is
\[
	\Adv^{\textsf{IND}}_{\Pi, \mathbb{B}, \mathcal{A}, \textsf{option}} = \Adv^{\textsf{MC-IND-CPA}}_{\textsf{PKE}, \mathcal{R}} = \negl.
\]

\end{proof}


%-------------------
% Security Analysis: fh-IPFE-based Instantiation
\section{Security Analysis: fh-IPFE-based Instantiation}
\label{sec:security_analysis:fh-IPFE}
%%%%%%%%%%%%%%%%%%%%

% Project Name: Semester Project Fall 2024 for EPFL
% File: security_analysis_fh-IPFE.tex
% Author: Keng-Yu Chen

%%%%%%%%%%%%%%%%%%%%

\begin{definition}[Function-Hiding Inner Product Functional Encryption (adapted from \cite{cryptoeprint:2016/440}) ]
\label{def:fh-IPFE}
	A \emph{function-hiding inner product functional encryption} (fh-IPFE) scheme $\sf FE$ for a field $\mathbb{F}$ and input length $k$ is composed of PPT algorithms $\textsf{FE.Setup}$, $\textsf{FE.KeyGen}$, $\textsf{FE.Enc}$, and $\textsf{FE.Dec}$:

	\begin{itemize}
	
		\item $\textsf{FE.Setup}(1^\lambda) \to \textsf{msk}, \textsf{pp}$: It outputs the public parameter $\textsf{pp}$ and the master secret key $\textsf{msk}$.
	
		\item $\textsf{FE.KeyGen}(\textsf{msk}, \textsf{pp}, \mathbf{x}) \to \textsf{sk}_{\mathbf{x}}$: It generates the functional decryption key $\textsf{sk}{\mathbf{x}}$ for an input vector $\mathbf{x} \in \mathbb{F}^k$. 
	
		\item $\textsf{FE.Enc}(\textsf{msk}, \textsf{pp}, \mathbf{y}) \to \textsf{ct}_{\mathbf{y}}$: It encrypts the input vector $\mathbf{y} \in \mathbb{F}^k$ to the ciphertext $\textsf{ct}_{\mathbf{y}}$. 
	
		\item $\textsf{FE.Dec}(\textsf{pp}, \textsf{sk}_{\mathbf{x}}, \textsf{ct}_{\mathbf{y}}) \to z$: It outputs a value $z \in \mathbb{F}$ or an error symbol $\bot$.
	
	\end{itemize}
	
	\paragraph{\textbf{Correctness}} An fh-IPFE scheme \textsf{FE} is \emph{correct} if for all non-zero $ \mathbf{x}, \mathbf{y} \in \mathbb{F}^k \setminus \{\mathbf{0}\}$, let $(\textsf{msk}, \textsf{pp}) \gets \textsf{FE.Setup}(1^\lambda)$, we have
	\[
		\textsf{FE.Dec}( \textsf{pp}, \textsf{FE.KeyGen}(\textsf{msk}, \textsf{pp}, \mathbf{x}), \textsf{FE.Enc}(\textsf{msk}, \textsf{pp}, \mathbf{y}) ) = \langle \mathbf{x}, \mathbf{y} \rangle \in \mathbb{F}.
	\]

\end{definition}

%-------------------

\subsection{Instantiation with an fh-IPFE Scheme}
\label{sec:fh-IPFE-instantiation}

Let $\textsf{FE} = (\textsf{FE.Setup}, \textsf{FE.KeyGen}, \textsf{FE.Enc}, \textsf{FE.Dec})$ be an fh-IPFE scheme. Following \cite{cryptoeprint:2023/481}, we can instantiate a biometric authentication scheme using $\textsf{FE}$ with the distance metric the Euclidean distance.
Let $\textsf{getEnroll}^{\mathcal{O}_{\mathcal{B}}}()$ and $\textsf{getProbe}^{\mathcal{O}_{\mathcal{B}}}()$ both output vectors in $ \{0, 1, \cdots, m \}^k$ for all biometric distributions $\mathcal{B} \in \mathbb{B}$. 
For a pre-defined real number $\tau \geq 0$, define
\[
	\textsf{BioCompare}(\mathbf{b}, \mathbf{b}^\prime) \to \| \mathbf{b} - \mathbf{b}^\prime\|^2 \quad \text{and} \quad 
	\textsf{Verify}(s) \to 
	\begin{cases} 
		1 & \text{if } \sqrt{s} \leq \tau \\
		0 & \text{if } \sqrt{s} > \tau
	\end{cases}.
\]

Now, let the associated field of $\textsf{FE}$ be $\mathbb{Z}_q$, where $q$ is a prime number larger than the maximum possible Euclidean distance $m^2 \cdot k$. The scheme is instantiated as follows.

\begin{itemize}

	\item $\textsf{Setup}(1^\lambda)$: It calls $\textsf{FE.Setup}(1^\lambda) \to \textsf{msk}, \textsf{pp}$ and outputs $\textsf{esk} \gets (\textsf{msk}, \textsf{pp})$, $\textsf{psk} \gets (\textsf{msk}, \textsf{pp})$ and $\textsf{csk} \gets \textsf{pp}$.

	\item $\textsf{Enroll}(\textsf{esk}, \mathbf{b})$: On input a template vector $\mathbf{b} = (b_1, b_2, \cdots, b_k)$, the algorithm first encodes it as $\mathbf{x} = (x_1, x_2, \cdots, x_{k+2}) = (b_1, b_2, \cdots, b_k, 1, \|\mathbf{b}\|^2)$. Next, it calls $\textsf{FE.KeyGen}(\textsf{msk}, \textsf{pp}, \mathbf{x}) \to \textsf{sk}_{\mathbf{x}}$ and outputs $\mathbf{c_x} \gets \textsf{sk}_{\mathbf{x}}$.

	\item $\textsf{Probe}(\textsf{psk}, \mathbf{b}^\prime)$: On input a template vector $\mathbf{b}^\prime = (b_1^\prime, b_2^\prime, \cdots, b_k^\prime)$, the algorithm first encodes it as $\mathbf{y} = (y_1, y_2, \cdots, y_{k+2}) = (-2b_1^\prime, -2b_2^\prime, \cdots, -2b_k^\prime, \|\mathbf{b}^\prime\|^2, 1)$. Next, it calls $\textsf{FE.Enc}(\textsf{msk}, \textsf{pp}, \mathbf{y}) \to \textsf{ct}_{\mathbf{y}}$ and outputs $\mathbf{c_y} \gets \textsf{ct}_{\mathbf{y}}$.

	\item $\textsf{Compare}(\textsf{csk}, \mathbf{c_x}, \mathbf{c_y)}$: It calls $\textsf{FE.Dec}(\textsf{pp}, \mathbf{c_x}, \mathbf{c_y}) \to s$ and outputs the value $s$.

\end{itemize}

By the correctness of the functional encryption scheme $\sf FE$, we have
\[
	s = \textsf{FE.Dec}(\textsf{pp}, \mathbf{c_x}, \mathbf{c_y}) = \langle \mathbf{x}, \mathbf{y} \rangle = \sum_{i=1}^k -2b_ib_i^\prime + \|\mathbf{b}\|^2 + \|\mathbf{b}^\prime\|^2 = \| \mathbf{b} - \mathbf{b}^\prime \|^2.
\]
which is equal to $\textsf{BioCompare}(\mathbf{b}, \mathbf{b}^\prime)$. Therefore, if two templates $\mathbf{b}$ and $\mathbf{b}^\prime$ are close enough such that $\|\mathbf{b} - \mathbf{b}^\prime\| \leq \tau$, the scheme results in $r = 1$, a successful authentication.

Instantiated with an fh-IPFE scheme in this way, the comparison secret key $\textsf{csk}$ is public, and the enrollment secret key $\textsf{esk}$ and probe secret key $\textsf{psk}$ are the same. Anyone with access to the enrollment message $\mathbf{c_x}$ and either $\textsf{esk}$ or $\textsf{psk}$ can probe any (invalidly encoded) $\mathbf{y}^{\prime} \in \Z_q^{k+2}$ and find $\langle \mathbf{x}, {\mathbf{y}^\prime} \rangle$ to get partial or full information about the biometric template $\mathbf{b}$. Even if the adversary has no $\textsf{esk}$ or $\textsf{psk}$, if it can sample ciphertexts $\mathbf{c_{y}}$ corresponding to some unknown random vectors $\mathbf{y}$, and if the field size $q$ is not large enough, it can also find a forged $\mathbf{c_{y^*}}$ such that $\langle \mathbf{x}, \mathbf{y^*} \rangle \leq \tau$ with a good probability to impersonate the user by sampling many times offline.

We note that the construction in \cite{cryptoeprint:2023/481} is applying Theorem \ref{thm:sEUF-CMA-esk-csk} on this instantiation. The user holds \textsf{esk} and \textsf{psk} while the server holds \textsf{csk}, the public parameter of the functional encryption scheme.
%-------------------

Let $\Pi$ be an authentication scheme instantiated by an fh-IPFE scheme \textsf{FE} for a field $\mathbb{F} = \Z_q$. In the following, we discuss the UF and IND security of $\Pi$ in this section. For this, we first define two security notions of \textsf{FE}\footnote{In security definition, the vectors lie in $\Z_q^k$, but we consider $\Z_q^{k+2}$ when discussing the instantiation $\Pi$.}.

%-------------------

\subsection{fh-IND Security of \textsf{FE}}

Given an fh-IPFE scheme \textsf{FE}, we define the \textsf{fh-IND} game \cite{cryptoeprint:2016/440} in Algorithm \ref{alg:ind-fh-IPFE}.

\begin{figure}[h]
\centering

	\begin{minipage}[t]{0.4\textwidth}
	\begin{algorithm}[H]
	\caption{$\textsf{fh-IND}_{\textsf{FE}}(\mathcal{A})$}
	\label{alg:ind-fh-IPFE}
	\begin{algorithmic}[1]
		\State $b \getsdollar \{0, 1\}$

		\State $\textsf{msk}, \textsf{pp} \gets \textsf{FE.Setup}(1^\lambda)$

		\State $\tilde{b} \gets \mathcal{A}^{\mathcal{O}_{\textsf{KeyGen}}, \mathcal{O}_{\textsf{Enc}}} ( \textsf{pp} )$

		\State \Return $1_{\tilde{b} = b}$
	\end{algorithmic}
	\end{algorithm}
	\end{minipage}

\label{fig:ind-fh-IPFE}
\end{figure}

\begin{itemize}

	\item $\mathcal{O}_{\textsf{KeyGen}}(\cdot, \cdot)$: On input pair $(\mathbf{x}^{(0)}, \mathbf{x}^{(1)})$, where $\mathbf{x}^{(0)}, \mathbf{x}^{(1)} \in \Z_q^k \setminus \{\mathbf{0}\}$, it outputs $\textsf{FE.KeyGen}(\textsf{msk}, \textsf{pp}, \mathbf{x}^{(b)} )$.

	\item $\mathcal{O}_{\textsf{Enc}}(\cdot, \cdot)$: On input pair $(\mathbf{y}^{(0)}, \mathbf{y}^{(1)})$, where $\mathbf{y}^{(0)}, \mathbf{y}^{(1)} \in \Z_q^k \setminus \{\mathbf{0}\}$, it outputs $\textsf{FE.Enc}(\textsf{msk}, \textsf{pp}, \mathbf{y}^{(b)} )$.

\end{itemize}

\noindent To avoid trivial attacks, we consider \emph{admissible adversaries}.

\begin{definition}[Admissible Adversary]

	Let $\mathcal{A}$ be an adversary in an \textsf{fh-IND} game, and let $ (\mathbf{x}_1^{(0)}, \mathbf{x}_1^{(1)}), \cdots, (\mathbf{x}_{Q_K}^{(0)}, \mathbf{x}_{Q_K}^{(1)})$ be its queries to $\mathcal{O}_{\textsf{KeyGen}}$ and $(\mathbf{y}_1^{(0)}, \mathbf{y}_1^{(1)}), \cdots, (\mathbf{y}_{Q_E}^{(0)}, \mathbf{y}_{Q_E}^{(1)})$ be its queries to $\mathcal{O}_{\textsf{Enc}}$.
	We say $\mathcal{A}$ is \emph{admissible} if $\forall i \in [Q_K], \forall j \in [Q_E]$, we have
\[
	\left\langle {\mathbf{x}^{(0)}_{i}}, {\mathbf{y}^{(0)}_{j}} \right\rangle = \left\langle {\mathbf{x}^{(1)}_{i}}, {\mathbf{y}^{(1)}_{j}} \right\rangle
\]

\end{definition}


\begin{definition}[fh-IND Security]

	An fh-IPFE scheme \textsf{FE} is called fh-IND secure if for any admissible adversary $\mathcal{A}$, the advantage of $\mathcal{A}$ in the $\textsf{fh-IND}$ game in Algorithm \ref{alg:ind-fh-IPFE} is

\[
	\Adv_{\textsf{FE}, \mathcal{A}}^{\textsf{fh-IND}} := \left| \Pr[\textsf{fh-IND}_{\textsf{FE}}(\mathcal{A}) \to 1 ] - \frac{1}{2} \right| = \negl.
\]

\end{definition}

We note that fh-IND security is a standard notion for an fh-IPFE, and constructions in \cite{cryptoeprint:2015/1255, 10.1007/978-3-319-45871-7_24, cryptoeprint:2016/440} are proven fh-IND secure. However, fh-IND security may not be sufficient for the UF security of the instantiation in Section \ref{sec:fh-IPFE-instantiation}. 

\begin{theorem}
\label{thm:fh-IPFE-not-uf}
An instantiation $\Pi$ using the construction in \cite{cryptoeprint:2016/440} is not \textsf{option}-UF secure for any \textsf{option}.

\end{theorem}

\noindent We recall the construction in \cite{cryptoeprint:2016/440} in Appendix \ref{sec:fh-IPFE-construction}.

\begin{proof}

Let $\mathcal{A}$ be a \textsf{UF} game adversary that returns $(K_1, K_2) = (1, (1, \cdots, 1))$. Then, in the decryption,
\[
	D_1 = e(g_1, g_2)^{0} = 1 \quad \text{and} \quad D_2 = e(g_1, g_2)^0 = 1
\]
As $D_1^0 = D_2$, the decryption returns $0$ and let the adversary win the game with probability $1$.

\end{proof}

%-------------------

\subsection{RUF Security of \textsf{FE}}
\label{sec:security_analysis:fh-IPFE:ruf}

We also define the $\textsf{RUF}^{\mathcal{O}}_\textsf{FE}$ game in Algorithm \ref{alg:ruf-fh-IPFE}.

\begin{figure}[H]
\centering

	\begin{minipage}[t]{0.55\textwidth}
	\begin{algorithm}[H]
	\caption{$\textsf{RUF}^{\mathcal{O}}_{\textsf{FE}}(\mathcal{A})$}
	\label{alg:ruf-fh-IPFE}
	\begin{algorithmic}[1]
		\State $\mathbf{r} \getsdollar \mathbb{F}^{k}$ \label{alg:oracle-ruf-fh-IPFE:r}

		\State $\textsf{msk}, \textsf{pp} \gets \textsf{FE.Setup}(1^\lambda)$

		\State $\textsf{sk}_{\mathbf{r}} \gets \textsf{FE.KeyGen}(\textsf{msk}, \textsf{pp}, \mathbf{r})$

		\State $\mathbf{\tilde{z}} \gets \mathcal{A}^{\mathcal{O}} ( \textsf{pp}, \textsf{sk}_{\mathbf{r}} )$

		\If{$\mathbf{\tilde{z}}$ is equal to any output of $\mathcal{O}^\prime_{\textsf{Enc}}$ }
			
			\State \Return $0$
		
		\EndIf

		\State $s \gets \textsf{FE.Dec}(\textsf{pp}, \textsf{sk}_{\mathbf{r}}, \mathbf{\tilde{z}} )$

		\State \Return $1_{s \neq \bot}$
	\end{algorithmic}
	\end{algorithm}
	\end{minipage}

\end{figure}

The oracle $\mathcal{O}$ can be nothing or include the following options based on the threat model.

\begin{itemize}

	\item $\mathcal{O}^\prime_{\textsf{KeyGen}}(\cdot)$: On input $\mathbf{x}^\prime$, it outputs $\textsf{FE.KeyGen}(\textsf{msk}, \textsf{pp}, \mathbf{x}^\prime)$.
	
	\item $\mathcal{O}^\prime_{\textsf{Enc}}(\cdot)$: On input $\mathbf{y}^\prime$, it outputs $\textsf{FE.Enc}(\textsf{msk}, \textsf{pp}, \mathbf{y}^\prime)$. The adversary is required to return $\mathbf{\tilde{z}}$ that is not equal to any output of this oracle.
\end{itemize}

\begin{definition}[RUF Security]

	An fh-IPFE scheme \textsf{FE} is called $\mathcal{O}$-RUF secure if for any adversary $\mathcal{A}$, the advantage of $\mathcal{A}$ in the $\textsf{RUF}^{\mathcal{O}}_\textsf{FE}$ game in Algorithm \ref{alg:ruf-fh-IPFE} is

\[
	\Adv_{\textsf{FE}, \mathcal{A}}^{\textsf{RUF}, \mathcal{O}} := \Pr[\textsf{RUF}^{\mathcal{O}}_{\textsf{FE}}(\mathcal{A}) \to 1 ] = \negl.
\]

\noindent We say $\textsf{FE}$ is RUF secure if it is $\{ \mathcal{O}^\prime_{\textsf{KeyGen}}, \mathcal{O}^\prime_{\textsf{Enc}} \}$-RUF secure.

\end{definition}

\subsubsection{Achievability of RUF Security}

We note that RUF security is a new security notion of fh-IPFE but can be achieved with a digital signature scheme. Let $\textsf{Sig} = ( \textsf{Sig.KeyGen}, \textsf{Sig.Sign}, \textsf{Sig.Verify} )$ be an sEUF-CMA signature scheme. By adding $\textsf{Sig}$, an fh-IPFE scheme $\textsf{FE}$ can be upgraded to an RUF scheme $\textsf{FE}^\prime$. 

\begin{itemize}

	\item $\textsf{FE}^\prime \textsf{.Setup}(1^\lambda)$: Run $\textsf{FE.Setup}(1^\lambda) \to (\textsf{msk}, \textsf{pp})$ and $\textsf{Sig.KeyGen}(1^\lambda) \to (\textsf{sk}_{\textsf{Sig}}, \textsf{pk}_{\textsf{Sig}} )$. Output $\textsf{msk}^\prime = (\textsf{msk}, \textsf{sk}_\textsf{Sig})$ and $\textsf{pp}^\prime = (\textsf{pp}, \textsf{pk}_{\textsf{Sig}})$.

	\item $\textsf{FE}^\prime \textsf{.KeyGen}(\textsf{msk}^\prime, \mathbf{x})$: Run $\textsf{FE.KeyGen}(\textsf{msk}, \mathbf{x}) \to \textsf{sk}_{\mathbf{x}}$ and output $\textsf{sk}_{\mathbf{x}}^\prime \gets \textsf{sk}_{\mathbf{x}}$.

	\item $\textsf{FE}^\prime \textsf{.Enc}(\textsf{msk}^\prime, \mathbf{y})$: Run $\textsf{FE.Enc}(\textsf{msk}, \mathbf{y}) \to \textsf{ct}_{\mathbf{y}}$ and sign $\textsf{ct}_{\mathbf{y}}$ by $\textsf{Sig.Sign}(\textsf{sk}_{\textsf{Sig}}, \textsf{ct}_{\mathbf{y}}) \to \sigma$. Output $\textsf{ct}_{\mathbf{y}}^\prime = (\textsf{ct}_{\mathbf{y}}, \sigma)$.

	\item $\textsf{FE}^\prime \textsf{.Dec}(\textsf{pp}^\prime, \textsf{sk}_{\mathbf{x}}^\prime, \textsf{ct}_{\mathbf{y}}^\prime )$: Output the decryption $\textsf{FE.Dec}(\textsf{pp}, \textsf{sk}_{\mathbf{x}}, \textsf{ct}_{\mathbf{y}})$ if the verification $\textsf{Sig.Verify}(\textsf{pk}_{\textsf{Sig}}, \textsf{ct}_{\mathbf{y}}, \sigma ) = 1$. Otherwise, output $\bot$.

\end{itemize}

\begin{theorem}

For any fh-IPFE $\textsf{FE}$, $\textsf{FE}^\prime$ is an RUF secure fh-IPFE.

\end{theorem}

\begin{proof}

Given an adversary $\mathcal{A}$ in the $\textsf{RUF}^{\mathcal{O}_{\textsf{KeyGen}}^\prime, \mathcal{O}_{\textsf{Enc}}^\prime}_{\textsf{FE}^\prime}$ game, consider the reduction adversary $\mathcal{R}$ in Algorithm \ref{alg:red:sEUF-CMA:RUF} which plays the \textsf{sEUF-CMA} game of $\textsf{Sig}$. $\mathcal{R}$ is given a verification public key $\textsf{pk}_{\textsf{Sig}}$ and a signing oracle $\mathcal{O}_{\textsf{Sig}}$ and returns a forged message-signature pair that is not equal to any previous answer of $\mathcal{O}_{\textsf{Sig}}$. To run $\mathcal{A}$, $\mathcal{R}$ simulates each oracle in the following way.

\begin{itemize}
	\item $\mathcal{O}_{\textsf{KeyGen}}^\prime(\mathbf{x}^\prime)$: Return $\textsf{FE.KeyGen}(\textsf{msk}, \mathbf{x})$.

	\item $\mathcal{O}_{\textsf{Enc}}^\prime(\mathbf{y}^\prime)$: Run $\textsf{FE.Enc}(\textsf{msk}, \mathbf{y}) \to \textsf{ct}_{\mathbf{y}}$ and call the signing oracle $\mathcal{O}_{\textsf{Sign}}(\textsf{ct}_{\mathbf{y}}) \to \sigma$. Output $\textsf{ct}_{\mathbf{y}}^\prime = (\textsf{ct}_{\mathbf{y}}, \sigma)$.
\end{itemize}

\begin{figure}[h]
\centering
	
	\begin{minipage}[t]{0.5\linewidth}
	\centering
	\begin{algorithm}[H]
	\caption{$\mathcal{R}^{\mathcal{O}_{\textsf{Sign}}}( \textsf{pk}_{\textsf{Sig}} )$}
	\label{alg:red:sEUF-CMA:RUF}
	\begin{algorithmic}[1]

		\State $\mathbf{r} \gets \mathbb{F}^k$

		\State $\textsf{msk}, \textsf{pp} \gets \textsf{FE.Setup}(1^\lambda)$

		\State $\textsf{sk}_{\mathbf{r}} \gets \textsf{FE.KeyGen}(\textsf{msk}, \textsf{pp}, \mathbf{r})$

		\State $\textsf{pp}^\prime \gets (\textsf{pp}, \textsf{pk}_{\textsf{Sig}})$

		\State $\mathbf{\tilde{z}} \gets {\mathcal{A}}^{\mathcal{O}_{\textsf{KeyGen}}^\prime, \mathcal{O}_{\textsf{Enc}}^\prime } (\textsf{pp}^\prime, \textsf{sk}_{\mathbf{r}})$ \label{alg:red:sEUF-CMA:RUF:A}

		\State Parse $(\textsf{ct}_{\mathbf{z}}, \sigma^\prime) \gets \mathbf{\tilde{z}}$

		\State \Return $(\textsf{ct}_{\mathbf{z}}, \sigma^\prime)$
	\end{algorithmic}
	\end{algorithm}
	\end{minipage}
	
\end{figure}

$\mathcal{R}$ perfectly simulates a $\textsf{RUF}$ game for $\mathcal{A}$, and if $\mathcal{A}$ wins the $\textsf{RUF}$ game, $(\textsf{ct}_{\mathbf{z}}, \sigma^\prime)$ is not equal to any previous answer of $\mathcal{O}_{\textsf{Enc}}^\prime$, and therefoere not equal to any previous message-signature pair $(\textsf{ct}_{\mathbf{y}}, \sigma)$ given by the signing oracle $\mathcal{O}_{\textsf{Sign}}$. Now, since $\textsf{Sig}$ is sEUF-CMA secure,
\[
	\Pr[\textsf{RUF}^{ \mathcal{O}^\prime_{\textsf{KeyGen}}}(\mathcal{A}) \to 1] \leq \Pr[ \textsf{Sig.Verify}(\textsf{pk}_{\textsf{Sig}}, \textsf{ct}_{\mathbf{z}}, \sigma^\prime) = 1 ] = \negl.
\]


\end{proof}

%-------------------


\subsection{UF Security of $\Pi$}
\label{sec:security_analysis:fh-IPFE:uf}

We first consider \textsf{option}-UF security when $\textsf{option}$ includes $\mathcal{O}_\textsf{Enroll}$. Note that in this instantiation, $\textsf{csk}$ is the public parameter $\textsf{pp}$ of \textsf{FE} and assumed to be given to all adversaries. 

\begin{theorem}
\label{thm:fh-IPFE:ind-ruf-OB-Enroll}
	Let $\textsf{option} = \{ \textsf{csk}, \mathbf{c_x}, \mathcal{O}_\mathcal{B}, \mathcal{O}_{\textsf{Enroll}} \}$. For any distribution family $\mathbb{B}$, if \textsf{FE} is fh-IND secure and $\mathcal{O}^\prime_{\textsf{KeyGen}}$-RUF secure, then $\Pi$ is $\textsf{option}$-UF secure. 
\end{theorem}


\begin{proof}

Given an adversary $\mathcal{A}$ in the $\textsf{UF}_{\textsf{option}}$ game, consider the reduction adversary $\mathcal{R}$ in Algorithm \ref{alg:red:ind-uf-OB-Enroll} which plays the \textsf{fh-IND} game. $\mathcal{R}$ runs $\mathcal{A}$ and simulates $\mathcal{O}_{\textsf{Enroll}}( \textsf{esk}, \mathbf{b}^\prime )$ by first encoding $\mathbf{b}^\prime = (b_1^\prime, \cdots, b_k^\prime)$ into $\mathbf{x}^\prime = (b_1^\prime, \cdots, b_k^\prime, 1, \|\mathbf{b}^\prime\|^2)$ and calling $\mathcal{O}_{\textsf{KeyGen}}(\mathbf{x}^\prime, \mathbf{x}^\prime)$ given in the \textsf{fh-IND} game.  Note that since $\mathcal{R}$ never calls $\mathcal{O}_{\textsf{Enc}}$, it is an admissible adversary.

\begin{figure}[h]
\centering
	
	\begin{minipage}[t]{0.5\linewidth}
	\centering
	\begin{algorithm}[H]
	\caption{$\mathcal{R}^{\mathcal{O}_{\textsf{KeyGen}}, \mathcal{O}_{\textsf{Enc}}}(\textsf{pp})$}
	\label{alg:red:ind-uf-OB-Enroll}
	\begin{algorithmic}[1]
		\State $\mathcal{B} \getsdollar \mathbb{B}, \quad \mathbb{B} \gets \mathbb{B} \setminus \mathcal{B}$ \label{alg:red:ind-uf-OB-Enroll:B}

		\State $\mathbf{b} = (b_1, \cdots, b_k) \gets \textsf{getEnroll}^{\mathcal{O}_{\mathcal{B}}}()$

		\State $\mathbf{x} \gets (b_1, \cdots, b_k, 1, \|\mathbf{b}\|^2)$

		\State $\mathbf{r} \getsdollar \mathbb{F}^{k+2}$
		
		\State $\textsf{sk} \gets \mathcal{O}_{\textsf{KeyGen}}(\mathbf{x}, \mathbf{r})$ \label{alg:red:ind-uf-OB-Enroll:sk}

		\State ${\mathbf{\tilde{z}}} \gets {\mathcal{A}}^{\mathcal{O}_\mathcal{B}, \mathcal{O}_\textsf{Enroll} } ( \textsf{pp}, \textsf{sk} )$ \label{alg:red:ind-uf-OB-Enroll:A}

		\State $s \gets \textsf{FE.Dec}( \textsf{pp}, \textsf{sk}, \mathbf{\tilde{z}} )$

		\If{$\textsf{Verify}(s) = 1$} \label{alg:red:ind-uf-OB-Enroll:verify}
			\State \Return $\tilde{b} = 0$
		\Else
			\State \Return $\tilde{b} \getsdollar \{0, 1\}$
		\EndIf

	\end{algorithmic}
	\end{algorithm}
	\end{minipage}
	
\end{figure}

	If the challenge bit $b = 0$, then $\mathcal{R}$ perfectly simulates a $\textsf{UF}_{\textsf{option}}$ game for $\mathcal{A}$. Therefore, the probability that $\textsf{Verify}(s) = 1$ in Line \ref{alg:red:ind-uf-OB-Enroll:verify} is $\Pr[\textsf{UF}_{\textsf{option}}(\mathcal{A}) \to 1]$.

	For the case when the challenge bit $b = 1$, consider an adversary $\mathcal{A}^\prime$ in Algorithm \ref{alg:adv:ind-uf-OB-Enroll} in the $\textsf{RUF}^{ \mathcal{O}^\prime_{\textsf{KeyGen}} }$ game. $\mathcal{A}^\prime$ runs Line \ref{alg:red:ind-uf-OB-Enroll:B} and \ref{alg:red:ind-uf-OB-Enroll:A} of $\mathcal{R}$ and simulates $\mathcal{O}_{\textsf{Enroll}}( \textsf{esk}, \mathbf{b}^\prime )$ by first encoding $\mathbf{b}^\prime$ into $\mathbf{x}^\prime$ as before and calling $\mathcal{O}_\textsf{KeyGen}^\prime(\mathbf{x}^\prime)$ given in the $\textsf{RUF}^{ \mathcal{O}^\prime_{\textsf{KeyGen}} }$ game. 


\begin{figure}[h]
\centering

	\begin{minipage}[t]{0.45\textwidth}
	\begin{algorithm}[H]
	\caption{${\mathcal{A}^\prime}^{\mathcal{O}^\prime_{\textsf{KeyGen}} }(\textsf{pp}, \textsf{sk}_\mathbf{r})$}
	\label{alg:adv:ind-uf-OB-Enroll}
	\begin{algorithmic}[1]
		\State $\mathcal{B} \getsdollar \mathbb{B}, \quad \mathbb{B} \gets \mathbb{B} \setminus \mathcal{B}$ 
		
		\State $\mathbf{\tilde{z}} \gets {\mathcal{A}}^{\mathcal{O}_{\mathcal{B}}, \mathcal{O}_\textsf{Enroll} } (\textsf{pp}, \textsf{sk}_{\mathbf{r}})$

		\State \Return $\mathbf{\tilde{z}}$
	\end{algorithmic}
	\end{algorithm}
	\end{minipage}

\end{figure}

Now, if the challenge bit $b = 1$, then $\mathcal{R}$ perfectly simulates $\mathcal{A}^\prime$ in the $\textsf{RUF}^{ \mathcal{O}^\prime_{\textsf{KeyGen}} }$ game. The probability that $\textsf{Verify}(s) = 1$ in Line \ref{alg:red:ind-uf-OB-Enroll:verify} is smaller than $\Pr[s \neq \bot] = \Pr[\textsf{RUF}^{\mathcal{O}^\prime_{\textsf{KeyGen}}}_{\textsf{FE}}(\mathcal{A}^\prime) \to 1 ]$

In conclusion,
\begin{align*}
	\Pr[\textsf{fh-IND}(\mathcal{R}) \to 1] 
	&= \Pr[b = 0] \cdot \left( \Pr[\textsf{Verify}(s) = 1 \mid b = 0] + \frac{1}{2} \cdot \Pr[\textsf{Verify}(s) = 0 \mid b = 0] \right) \\
	&\quad + \Pr[b = 1] \cdot \frac{1}{2} \cdot \Pr[\textsf{Verify}(s) = 0 \mid b = 1] \\
	&= \frac{1}{2} + \frac{1}{4} \left( \Pr[\textsf{Verify}(s) = 1 \mid b = 0] - \Pr[\textsf{Verify}(s) = 1 \mid b = 1] \right) \\
	&\geq \frac{1}{2} + \frac{1}{4} \left( \Pr[\textsf{UF}_\textsf{option}(\mathcal{A}) \to 1] - \Pr[\textsf{RUF}^{\mathcal{O}^\prime_{\textsf{KeyGen}}}_{\textsf{FE}}(\mathcal{A}^\prime) \to 1 ] \right)
\end{align*}

\noindent Since both $\Adv_{\textsf{FE}, \mathcal{R}}^\textsf{fh-IND} = \left| \Pr[\textsf{fh-IND}(\mathcal{R}) \to 1] - \frac{1}{2} \right|$ and $\Adv_{\textsf{FE}, \mathcal{A}^\prime}^{\textsf{RUF},\mathcal{O}^\prime_{\textsf{KeyGen}}} = \Pr[\textsf{RUF}^{\mathcal{O}^\prime_{\textsf{KeyGen}}}_{\textsf{FE}}(\mathcal{A}^\prime) \to 1 ]$ are negligilbe,
\[
	\Pr[\textsf{UF}_\textsf{option}(\mathcal{A}) \to 1] \leq 4 \cdot \Adv_{\textsf{FE}, \mathcal{R}}^\textsf{fh-IND} + \Adv_{\textsf{FE}, \mathcal{A}^\prime}^{\textsf{RUF},\mathcal{O}^\prime_{\textsf{KeyGen}}} = \negl.
\]

\end{proof}


For $\textsf{option}$ that includes $\mathcal{O}_{\textsf{Probe}}$, we first note that for any $d \in \Z_q$ and any non-zero vector $\mathbf{r} \in \Z_q^{k+2} \setminus \{\mathbf{0}\}$, there exists a vector $\mathbf{y} \in \Z_q^{k+2}$ such that $\langle \mathbf{r}, \mathbf{y} \rangle = d$.

\begin{theorem}
\label{thm:fh-IPFE:ind-ruf-OB-Probe}
	Let $\textsf{option} = \{\textsf{csk}, \mathbf{c_x}, \mathcal{O}_\mathcal{B}, \mathcal{O}_\textsf{Probe}\}$. For any distribution family $\mathbb{B}$, if \textsf{FE} is fh-IND secure and $\mathcal{O}^\prime_{\textsf{Enc}}$-RUF secure, then $\Pi$ is $\textsf{option}$-UF secure. 
\end{theorem}


\begin{proof}

[There exists a flaw in this proof. Line \ref{alg:adv:ind-uf-OB-Probe:Dec} in Algorithm \ref{alg:adv:ind-uf-OB-Probe} is not feasible for current fh-IPFE constructions. I am now trying to solve this problem. Details are in Appendix \ref{sec:fixing-proof-thm-OB-Probe}]

Given an adversary $\mathcal{A}$ in the $\textsf{UF}_\textsf{option}$ game, consider the reduction adversary $\mathcal{R}$ in Algorithm \ref{alg:red:ind-uf-OB-Probe} which plays the \textsf{fh-IND} game. $\mathcal{R}$ runs $\mathcal{A}$ and simulates $\mathcal{O}_{\textsf{Probe}}$ in the following way.

\begin{itemize}

	\item $\mathcal{O}_{\textsf{Probe}}( \textsf{psk}, \mathbf{b}^\prime )$: On input $\mathbf{b}^\prime = (b_1^\prime, \cdots, b_k^\prime)$, it first encodes it as $\mathbf{y}^\prime = (-2b_1^\prime, \cdots, \allowbreak -2b_k^\prime, \|\mathbf{b}^\prime\|^2, 1)$. Next, it computes $d \gets \langle \mathbf{x}, {\mathbf{y}^\prime} \rangle$ and finds a vector $\mathbf{y}^{\prime\prime}$ such that $\langle \mathbf{r}, {\mathbf{y}^{\prime\prime}} \rangle = d$. Finally, it calls $\mathcal{O}_{\textsf{Enc}}(\mathbf{y}^\prime, {\mathbf{y}^{\prime\prime}})$, which is given by the \textsf{fh-IND} game, and returns the result.

\end{itemize}

\noindent Note that $(\mathbf{x}, \mathbf{r})$ is the only query of $\mathcal{R}$ to $\mathcal{O}_{\textsf{KeyGen}}$, and for any query $( \mathbf{y}^\prime, {\mathbf{y}^{\prime\prime}} )$ to $\mathcal{O}_{\textsf{Enc}}$, it satisfies $\langle \mathbf{x}, {\mathbf{y}^\prime} \rangle = \langle \mathbf{r}, {\mathbf{y}^{\prime\prime}} \rangle$. Hence, $\mathcal{R}$ is an admissible adversary.

\begin{figure}[h]
\centering
	
	\begin{minipage}[t]{0.6\linewidth}
	\centering
	\begin{algorithm}[H]
	\caption{$\mathcal{R}^{\mathcal{O}_{\textsf{KeyGen}}, \mathcal{O}_{\textsf{Enc}}}(\textsf{pp})$}
	\label{alg:red:ind-uf-OB-Probe}
	\begin{algorithmic}[1]
		\State $\mathcal{B} \getsdollar \mathbb{B}, \quad \mathbb{B} \gets \mathbb{B} \setminus \mathcal{B}$ \label{alg:red:ind-uf-OB-Probe:B}

		\State $\mathbf{b} = (b_1, \cdots, b_k) \gets \textsf{getEnroll}^{\mathcal{O}_{\mathcal{B}}}()$

		\State $\mathbf{x} \gets (b_1, \cdots, b_k, 1, \|\mathbf{b}\|^2)$

		\State $\mathbf{r} \getsdollar \mathbb{F}^{k+2}$

		\State $\textsf{sk} \gets \mathcal{O}_{\textsf{KeyGen}}(\mathbf{x}, \mathbf{r})$

		\State ${\mathbf{\tilde{z}}} \gets \mathcal{A}^{\mathcal{O}_{\mathcal{B}}, \mathcal{O}_{\textsf{Probe}} } ( \textsf{pp}, \textsf{sk})$

		\If{$\mathbf{\tilde{z}}$ is equal to any output of $\mathcal{O}_\textsf{Probe}$}

			\State \Return $\bot$

		\EndIf

		\State $s \gets \textsf{FE.Dec}( \textsf{pp}, \textsf{sk}, \mathbf{\tilde{z}} )$

		\If{$\textsf{Verify}(s) = 1$} \label{alg:red:ind-uf-OB-Probe:verify}
			\State \Return $\tilde{b} = 0$
		\Else
			\State \Return $\tilde{b} \getsdollar \{0, 1\}$
		\EndIf

	\end{algorithmic}
	\end{algorithm}
	\end{minipage}
	
\end{figure}

If the challenge bit $b = 0$, then $\mathcal{R}$ perfectly simulates a $\textsf{UF}_\textsf{option}$ game for $\mathcal{A}$. Therefore, the probability that $\textsf{Verify}(s) = 1$ in Line \ref{alg:red:ind-uf-OB-Probe:verify} is $\Pr[\textsf{UF}_\textsf{option}(\mathcal{A}) \to 1]$.

For the case when the challenge bit $b = 1$, consider an adversary $\mathcal{A}^\prime$ in Algorithm \ref{alg:adv:ind-uf-OB-Probe} in the $\textsf{RUF}^{ \mathcal{O}^\prime_{\textsf{Enc}} }$ game. $\mathcal{A}^\prime$ runs $\mathcal{A}$ and simulates $\mathcal{O}_{\textsf{Probe}}$ in the following way.

\begin{itemize}

	\item $\mathcal{O}_{\textsf{Probe}}( \textsf{psk}, \mathbf{b}^\prime )$: It first encodes $\mathbf{b}^\prime$ into $\mathbf{y}^\prime$ as before. Next, it computes $d \gets \langle \mathbf{x}^{(*)}, {\mathbf{y}^\prime} \rangle$ and finds a vector $\mathbf{y}^{\prime\prime}$ such that $\langle \mathbf{r}, {\mathbf{y}^{\prime\prime}} \rangle = d$. Finally, it calls $\mathcal{O}^\prime_{\textsf{Enc}} (\mathbf{y}^{\prime\prime} )$ , which is given by the $\textsf{RUF}^{ \mathcal{O}^\prime_{\textsf{Enc}} }$ game, and returns the result.

\end{itemize}

\begin{figure}[h]
\centering
	
	\begin{minipage}[t]{0.85\linewidth}
	\centering
	\begin{algorithm}[H]
	\caption{$ {\mathcal{A}^\prime}^{ \mathcal{O}^\prime_{\textsf{Enc}} } (\textsf{pp}, \textsf{sk}_{\mathbf{r}}) $}
	\label{alg:adv:ind-uf-OB-Probe}
	\begin{algorithmic}[1]
		\State $\mathcal{B} \getsdollar \mathbb{B}, \quad \mathbb{B} \gets \mathbb{B} \setminus \mathcal{B}$ \label{alg:adv:ind-uf-OB-Probe:B}
		
		\State $\mathbf{b}^{(*)} \gets \textsf{getEnroll}^{ \mathcal{O}_{\mathcal{B}} } ()$
		
		\State $\mathbf{x}^{(*)} \gets (b_1^{(*)}, \cdots, b_k^{(*)}, 1, \|\mathbf{b}^{(*)}\|^2)$

		\State Sample $k+2$ linearly independent vectors $\{ \mathbf{e}^{(i)} \}_{i=1}^{k+2}$.

		\For{$i=1$ to $k+2$}
			\State $\textsf{ct}^{(i)} \gets \mathcal{O}^\prime_{\textsf{Enc}}(\mathbf{e}^{(i)})$.

			\State $d_i \gets \textsf{FE.Dec}(\textsf{pp}, \textsf{sk}_{\mathbf{r}}, \textsf{ct}^{(i)})$. \label{alg:adv:ind-uf-OB-Probe:Dec}
		\EndFor

		\State Find the vector $\mathbf{r}$ by solving the linear system $\{ \langle \mathbf{r}, {\mathbf{e}^{(i)}} \rangle = d_i \}_{i=1}^{k+2}$.\label{alg:adv:ind-uf-OB-Probe:r}

		\If{$\mathbf{r} = \mathbf{0}$}

			\State \Return $\bot$

		\EndIf

		\State ${\mathbf{\tilde{z}}} \gets \mathcal{A}^{\mathcal{O}_{\mathcal{B}}, \mathcal{O}_{\textsf{Probe}} } (\textsf{pp}, \textsf{sk}_{\mathbf{r}})$
		
		\State \Return ${\mathbf{\tilde{z}}}$
	\end{algorithmic}
	\end{algorithm}
	\end{minipage}
	
\end{figure}

To make $\mathcal{R}$ simulate $\mathcal{A}^\prime$ in the $\textsf{RUF}^{ \mathcal{O}^\prime_{\textsf{Enc}} }$ game, we still need to ensure two conditions.

\begin{itemize}

	\item $\mathbf{r} \neq \mathbf{0}$. Otherwise, $\mathcal{A}^\prime$ cannot simulate $\mathcal{O}_\textsf{Probe}$. 

	\item $\mathbf{\tilde{z}} \neq \textsf{ct}^{(i)}$ for all $i$. The answers of $\mathcal{O}_\textsf{Probe}$ have already been checked in $\mathcal{R}$. 
\end{itemize}

Let $\mathcal{A}^\prime$ play a tweaked $\textsf{RUF}_\textsf{FE}^{\mathcal{O}^\prime_{\textsf{Enc}}}$ game which does not check that $\mathbf{\tilde{z}}$ is not equal to $\mathbf{c}^{(i)}$ for all $i$. That is, the game only checks whether $\mathbf{\tilde{z}}$ is not equal to any output of $\mathcal{O}^\prime_\textsf{Enc}$ called by $\mathcal{O}_\textsf{Probe}$ of $\mathcal{A}$. Let the returned value of this game be $V$. We have Equation \ref{equ:ind-uf-OB-Probe:1} and \ref{equ:ind-uf-OB-Probe:2}. The former one is a relation between $\mathcal{R}$ playing $\textsf{fh-IND}$ game when the challenge bit $b=1$ and $V$, and the latter is a relation between $\mathcal{A}^\prime$ playing a regular $\textsf{RUF}_\textsf{FE}^{\mathcal{O}^\prime_{\textsf{Enc}}}$ game and the tweaked one.

\begin{gather}
	\Pr[\textsf{Verify}(s) = 1 \mid b = 1 \wedge \mathbf{r} \neq \mathbf{0}] = \Pr[V = 1] \label{equ:ind-uf-OB-Probe:1} \\
	\Pr[\textsf{RUF}_\textsf{FE}^{\mathcal{O}^\prime_{\textsf{Enc}}}(\mathcal{A}^\prime) \to 1] = \Pr\left[ V = 1 \mid \bigwedge_{i=1}^{k+2} \mathbf{\tilde{z}} \neq \textsf{ct}^{(i)} \right] \label{equ:ind-uf-OB-Probe:2}
\end{gather}

\noindent For Equation \ref{equ:ind-uf-OB-Probe:1}, consider that

\begin{align*}
	\Pr[\textsf{Verify}(s) = 1 \mid b = 1]
	&= \Pr[\textsf{Verify}(s) = 1 \mid b = 1 \wedge \mathbf{r} \neq \mathbf{0}] \cdot \Pr[\mathbf{r} \neq \mathbf{0}] \\
	&+ \Pr[\textsf{Verify}(s) = 1 \mid b = 1 \wedge \mathbf{r} = \mathbf{0}] \cdot \Pr[\mathbf{r} = \mathbf{0}] \\
	&\leq \Pr[V = 1] + \Pr[\mathbf{r} = 0] \\
	&= \Pr[V = 1] + \frac{1}{q^{k+2}} 
\end{align*}

\noindent For Equation \ref{equ:ind-uf-OB-Probe:2}, consider that

\begin{align*}
	\Pr[\textsf{RUF}_\textsf{FE}^{\mathcal{O}^\prime_{\textsf{Enc}}}(\mathcal{A}^\prime) \to 1] 
	&= \Pr\left[ V = 1 \mid \bigwedge_{i=1}^{k+2} \mathbf{\tilde{z}} \neq \textsf{ct}^{(i)} \right] \\ 
	& \geq \Pr[V = 1] - \Pr \left[ \neg  \left( \bigwedge_{i=1}^{k+2} \mathbf{\tilde{z}} \neq \textsf{ct}^{(i)} \right) \right] \\
	& = \Pr[V = 1] - \Pr \left[ \bigvee_{i=1}^{k+2} \mathbf{\tilde{z}} = \textsf{ct}^{(i)} \right] \\
	& \geq \Pr[V = 1] - \sum_{i=1}^{k+2} \Pr[\mathbf{\tilde{z}} = \textsf{ct}^{(i)}].
\end{align*}

\noindent Note that each $\textsf{ct}^{(i)}$ is an encryption of of some uniform non-zero vector $\mathbf{e}^{(i)}$. Also note that distinct vectors in $\Z_q^{k+2}$ will have different encryptions due to the correctness of $\textsf{FE}$. Therefore, $\Pr[\mathbf{\tilde{z}} = \textsf{ct}^{(i)}] \leq \frac{1}{q^{k+2} - 1}$ and
\[
	\Pr[\textsf{RUF}_\textsf{FE}^{\mathcal{O}^\prime_{\textsf{Enc}}}(\mathcal{A}^\prime) \to 1] \geq \Pr[V = 1] - \frac{k+2}{q^{k+2}-1}.
\]

\noindent Combining both results from Equation \ref{equ:ind-uf-OB-Probe:1} and \ref{equ:ind-uf-OB-Probe:2}, we derive
\[
	\Pr[\textsf{Verify}(s) = 1 \mid b = 1] \leq \Pr[V = 1] + \frac{1}{q^{k+2}} \leq \Pr[\textsf{RUF}_\textsf{FE}^{\mathcal{O}^\prime_{\textsf{Enc}}}(\mathcal{A}^\prime) \to 1] + \frac{k+2}{q^{k+2}-1} + \frac{1}{q^{k+2}}.
\]

Finally, similar to the proof of Theorem \ref{thm:fh-IPFE:ind-ruf-OB-Enroll}, we derive
\begin{align*}
	\Pr[\textsf{fh-IND}(\mathcal{R}) \to 1]
	&= \frac{1}{2} + \frac{1}{4} \left( \Pr[\textsf{Verify}(s) = 1 \mid b = 0] - \Pr[\textsf{Verify}(s) = 1 \mid b = 1] \right) \\
	&\geq \frac{1}{2} + \frac{1}{4} \left( \Pr[\textsf{UF}_\textsf{option}(\mathcal{A}) \to 1] - \Pr[\textsf{RUF}_\textsf{FE}^{\mathcal{O}^\prime_{\textsf{Enc}}}(\mathcal{A}^\prime) \to 1] - \frac{k+2}{q^{k+2}-1} - \frac{1}{q^{k+2}} \right).
\end{align*}

\noindent Since both $\Adv_{\textsf{FE}, \mathcal{R}}^\textsf{fh-IND} = \left| \Pr[\textsf{fh-IND}(\mathcal{R}) \to 1] - \frac{1}{2} \right|$ and $\Adv_{\textsf{FE}, \mathcal{A}^\prime}^{\textsf{RUF}, \mathcal{O}^\prime_{\textsf{Enc}}} = \Pr[\textsf{RUF}^{\mathcal{O}^\prime_{\textsf{Enc}}}_{\textsf{FE}}(\mathcal{A}^\prime) \to 1 ]$ are negligible,
\[
	\Pr[\textsf{UF}_\textsf{option}(\mathcal{A}) \to 1] \leq 4 \cdot \Adv_{\textsf{FE}, \mathcal{R}}^\textsf{fh-IND} + \Adv_{\textsf{FE}, \mathcal{A}^\prime}^{\textsf{RUF}, \mathcal{O}^\prime_{\textsf{Enc}}} + \frac{k+2}{q^{k+2}-1} + \frac{1}{q^{k+2}} = \negl.
\]

\end{proof}

Unfortunately, for the instantiation in Section \ref{sec:fh-IPFE-instantiation}, we cannot achieve UF security when the adversary has \textsf{psk}, even if the false positive rate is negligible. The adversary can simply compute $\mathbf{c} \gets \textsf{Probe}(\textsf{psk}, \mathbf{0})$ and return $\mathbf{c}$. The same results also hold for an $\textsf{option}$ that includes $\textsf{esk}$ since both $\textsf{psk}$ and $\textsf{esk}$ are equal to $\textsf{msk}$ and allow the adversary to run $\textsf{FE.Enc}(\textsf{msk}, \textsf{pp}, \mathbf{v})$ for any vector $\mathbf{v}$. We state this result formally in the following theorem.

\begin{theorem}

Let $\textsf{option}$ include $\textsf{esk}$ or $\textsf{psk}$. For any distribution family $\mathbb{B}$ and functional encryption $\textsf{FE}$, $\Pi$ is not \textsf{option}-UF secure.

\end{theorem}



\subsection{UF security with \cite{cryptoeprint:2016/440}}

In this section, we show that we can achieve a concrete UF secure scheme by the fh-IPFE scheme given in \cite{cryptoeprint:2016/440}. In a high-level overview, two main reasons make the scheme not UF secure:
\begin{itemize}
	\item The construction \cite{cryptoeprint:2016/440} allows anyone to generate a ciphertext that corresponds to a zero vector, which is described in Theorem \ref{thm:fh-IPFE-not-uf}. This makes the fh-IPFE scheme not even $\emptyset$-RUF secure.

	\item With the master secret key $\mathsf{msk}$, one can reconstruct the vector $\mathbf{x}$ that a secret key $\mathsf{sk}_{\mathbf{x}}$ corresponds to or even submit an encryption $\mathsf{ct}_{\mathbf{y}}$ of a small vector $\mathbf{y}$. This is a drawback of all the schemes instantiated by fh-IPFE in the way described in Section \ref{sec:fh-IPFE-instantiation}. Therefore, we need to instantiate the authentication scheme using a different way.

	[I haven't found a solution for this problem yet, but I am now trying a few possible ways.]
\end{itemize}

We first upgrade the scheme in \cite{cryptoeprint:2016/440}, which is not $\emptyset$-RUF secure due to Theorem \ref{thm:fh-IPFE-not-uf}, to $\mathcal{O}_{\mathsf{KeyGen}}^\prime$-RUF secure. Firstly, for an fh-IPFE scheme $\mathsf{FE}$ over the field $\Z_q$, consider the following transformation $\mathsf{FE}^\prime$:

\begin{itemize}
	\item $\mathsf{FE}^\prime.\mathsf{Setup}$: The same as $\mathsf{FE.Setup}$.

	\item $\mathsf{FE}^\prime.\mathsf{KeyGen}(\mathsf{pp}, \mathsf{msk}, \mathbf{x})$: Sample $\sigma \getsdollar \Z_q$. Return $\mathsf{sk}_{\mathbf{x} \| \sigma} \gets \mathsf{FE.KeyGen}(\mathsf{msk}, \mathsf{pp}, (\mathbf{x} \| \sigma))$ and $\sigma$, where $\mathbf{x} \| \sigma$ is appending $\sigma$ to $\mathbf{x}$.
	
	\item $\mathsf{FE}^\prime.\mathsf{Enc}(\mathsf{pp}, \mathsf{msk}, \mathbf{y})$: Return $\mathsf{ct}_{\mathbf{y} \| 1} \gets \mathsf{FE.Enc}(\mathsf{msk}, \mathsf{pp}, (\mathbf{y} \| 1))$, where $\mathbf{y} \| 1$ is appending the constant $1$ to $\mathbf{y}$.

	\item $\mathsf{FE}^\prime.\mathsf{Dec}(\mathsf{pp}, \mathsf{sk}_{\mathbf{x} \| \sigma}, \sigma, \mathbf{ct}_{\mathbf{y}\| 1})$: Return $\mathsf{FE.Dec}(\mathsf{pp}, \mathsf{sk}_{\mathbf{x}\|\sigma}, \mathbf{c}_{\mathbf{y} \| 1}) - \sigma \bmod q$.

\end{itemize}

\noindent One can show that if $\mathsf{FE}$ is fh-IND secure, $\mathsf{FE}^\prime$ is also fh-IND secure. Moreover, for the concrete construction \cite{cryptoeprint:2016/440}, we prove that $\mathsf{FE}^\prime$ is also $\mathcal{O}_{\mathsf{KeyGen}}^\prime$-RUF secure.

\begin{theorem}
\label{thm:fh-IPFE:2016440-OKeyGen-RUF}
Let $\mathsf{FE}$ be the scheme described in \cite{cryptoeprint:2016/440}. If $\textsf{FE}$ is fh-IND secure, then $\mathsf{FE}^\prime$ is $\mathcal{O}_{\mathsf{KeyGen}}^\prime$-RUF secure.
\end{theorem}

Recall that $\mathsf{FE}$ is proven fh-IND secure in the generic group model. In addition, the decryption of $\mathsf{FE}$ requires speciflying a polynomially-bounded set $S$. It searches $s \in S$ such that $D_1^s = D_2$, where $D_1$ and $D_2$ are elements in a group $\mathbb{G}_T$ of order $q$ that is in exponential of $\lambda$. If no such $s$ is found, the decryption returns $\bot$.

\begin{proof}

Given an adversary $\mathcal{A}$ in the $\textsf{RUF}^{\mathcal{O}^\prime_{\textsf{KeyGen}}}_{\mathsf{FE}^\prime}$ game, consider the reduction adversary $\mathcal{R}$ in Algorithm \ref{alg:red:2016440-OKeyGen-RUF} which plays the \textsf{fh-IND} game of the scheme $\mathsf{FE}$. $\mathcal{R}$ simulates $\mathcal{O}_\textsf{KeyGen}^\prime(\mathbf{x}^\prime)$ by sampling $\sigma \getsdollar \Z_q$ and querying $\mathcal{O}_{\textsf{KeyGen}}(\mathbf{x}^\prime \| \sigma, \mathbf{x}^\prime \| \sigma)$.

\begin{figure}[htp]
\centering
	
	\begin{minipage}[t]{0.5\linewidth}
	\centering
	\begin{algorithm}[H]
	\caption{$\mathcal{R}^{\mathcal{O}_{\textsf{KeyGen}}, \mathcal{O}_{\textsf{Enc}}}(\textsf{pp})$}
	\label{alg:red:2016440-OKeyGen-RUF}
	\begin{algorithmic}[1]
		\State $\mathbf{r}^{(0)}, \mathbf{r}^{(1)} \getsdollar \Z_q^k$

		\State $\sigma \getsdollar \Z_q$
		
		\State $\textsf{sk} \gets \mathcal{O}_{\textsf{KeyGen}}(\mathbf{r}^{(0)} \| \sigma, \mathbf{r}^{(1)} \sigma)$ 

		\State ${\mathbf{\tilde{z}}} \gets {\mathcal{A}}^{\mathcal{O}^\prime_{\textsf{KeyGen}}} (\textsf{pp}, \textsf{sk}, \sigma)$

		\State $\mathbf{r}^\prime \getsdollar \Z_q^{k+1}$

		\State $\textsf{sk}^\prime \gets \mathcal{O}_{\textsf{KeyGen}}(\mathbf{r}^{(0)} \| \sigma, \mathbf{r}^\prime)$

		\State $s^\prime \gets \textsf{FE.Dec}( \textsf{pp}, \textsf{sk}^\prime, \mathbf{\tilde{z}} ) - \sigma \bmod q$ \label{alg:red:2016440-OKeyGen-RUF:s}

		\State \Return $1_{s^\prime = \bot}$ \label{alg:red:2016440-OKeyGen-RUF:verify}
	
	\end{algorithmic}
	\end{algorithm}
	\end{minipage}
	
\end{figure}

In Algorithm \ref{alg:red:2016440-OKeyGen-RUF}, let
\begin{itemize}
	\item $\mathsf{sk} = (g_1^{\alpha_0}, g_1^{\alpha_1}, \cdots, g_1^{\alpha_{k+1}})$.

	\item $\mathsf{sk}^\prime = (g_1^{\alpha_0^\prime}, g_1^{\alpha_1^\prime}, \cdots, g_1^{\alpha_{k+1}^\prime})$.

	\item $s = \mathsf{FE}^\prime.\mathsf{Dec}(\mathsf{pp}, \mathsf{sk}, \mathbf{\tilde{z}}) = \mathsf{FE.Dec}(\mathsf{pp}, \mathsf{sk}, \mathbf{\tilde{z}}) - \sigma \bmod q$.

\end{itemize}

Since $\mathcal{R}$ simulates a $\textsf{RUF}^{\mathcal{O}^\prime_{\textsf{KeyGen}}}_{\mathsf{FE}^\prime}$ game for $\mathcal{A}$, regardless of the challenge bit, we have 
\[
	\Pr[s \neq S] = \Pr[s \in S] = \Pr[ \mathsf{RUF}^{\mathcal{O}^\prime_{\mathsf{KeyGen}}}_{\mathsf{FE}^\prime}(\mathcal{A}) \to 1]
\]
Moreover, if $s \neq \bot$, let $\mathbf{\tilde{z}} = (g_2^{\beta_0}, g_2^{\beta_1}, \cdots g_2^{\beta_{k+1}})$, we have
\[
	\prod_{i=1}^{k+1} e(g_1^{\alpha_i}, g_2^{\beta_i}) = g_T^{\sum_{i=1}^{k+1} \alpha_i \beta_i}
	\implies \mathsf{FE.Dec}(\mathsf{pp}, \mathsf{sk}, \mathbf{\tilde{z}}) = \begin{cases} 
		\frac{\sum_{i=1}^{k+1} \alpha_i \beta_i}{\alpha_0 \beta_0} & \text{if } \beta_0 \neq 0 \\
		0 & \text{if } \beta_0 = 0 \\
	\end{cases}
\]
Similarly, $\mathsf{FE.Dec}(\mathsf{pp}, \mathsf{sk}^\prime, \mathbf{\tilde{z}}) = \frac{\sum_{i=1}^{k+1} \alpha_i^\prime \beta_i}{\alpha_0^\prime \beta_0}$ or $0$. If the challenge bit $b = 0$, since $\mathsf{sk}$ and $\mathsf{sk}^\prime$ correspond to the same vector $\mathbf{r}^{(0)} \| \sigma$, we have 
\[
	\frac{\alpha_0^\prime}{\alpha_0} = \frac{\alpha_1^\prime}{\alpha_1} = \cdots = \frac{\alpha_{k+1}^\prime}{\alpha_{k+1}},
\]
which implies $\mathsf{FE.Dec}(\mathsf{pp}, \mathsf{sk}, \mathbf{\tilde{z}}) = \mathsf{FE.Dec}(\mathsf{pp}, \mathsf{sk}^\prime, \mathbf{\tilde{z}})$ and $s = s^\prime$.

Now, we analyze the advantage of $\mathcal{R}$. If the challenge bit $b = 0$, the probability that $s \neq \bot$ is
\begin{align*}
	\Pr[s^\prime \neq \bot \mid b = 0] 
	&\geq \Pr[s^\prime \in S \mid s \in S, b = 0] \cdot \Pr[s \in S \mid b = 0] \\
	&= 1 \cdot \Pr[ \mathsf{RUF}^{\mathcal{O}^\prime_{\mathsf{KeyGen}}}_{\mathsf{FE}^\prime}(\mathcal{A}) \to 1] 
\end{align*}
If the challenge bit $b = 1$,
\begin{align*}
	\Pr[s^\prime \neq \bot \mid b = 1] 
	&\leq \Pr[s^\prime \in S \mid s \in S, b = 1] + \Pr[s \notin S \mid b = 1] \\
	&\leq \Pr[\mathsf{FE.Dec}(\mathsf{pp}, \mathsf{sk}^\prime, \mathbf{\tilde{z}}) - \sigma \bmod q \in S \mid s \in S, b = 1] \\
	&\quad +  1 - \Pr[\mathsf{RUF}^{\mathcal{O}^\prime_{\mathsf{KeyGen}}}_{\mathsf{FE}^\prime}(\mathcal{A}) \to 1] \\
\end{align*}
Note that $\mathbf{r}^\prime, \mathsf{sk}^\prime$ and thus all $\alpha^\prime_i$ are independent of $\mathbf{\tilde{z}}$ and all $\beta_i$. Hence,
\begin{align*}
	& \Pr[\mathsf{FE.Dec}(\mathsf{pp}, \mathsf{sk}^\prime, \mathbf{\tilde{z}}) - \sigma \bmod q \in S \mid s \in S, b = 1] \\
	& \leq \Pr \left[ \frac{\sum_{i=1}^{k+1} \alpha_i^\prime \beta_i}{\alpha_0^\prime \beta_0} - \sigma \bmod q \in S \mid s \in S, b = 1, \beta_0 \neq 0 \right] + \Pr \left[0 - \sigma \bmod q \in S \right] \\
	& \leq \Pr \left[ \frac{\sum_{i=1}^{k+1} \alpha_i^\prime \beta_i}{\alpha_0^\prime \beta_0} - \sigma \bmod q \in S \mid s \in S, b = 1, \beta_0 \neq 0 \right] + \Pr \left[0 - \sigma \bmod q \in S \right] \\
	& \leq  \frac{|S|}{q} + \frac{|S|}{q} \\
\end{align*}

In conclusion,
\begin{align*}
	\Pr[\textsf{fh-IND}(\mathcal{R}) \to 1] 
	&= \frac{1}{2} \cdot \Pr[s^\prime \neq \bot \mid b = 0] + \frac{1}{2} \cdot \Pr[s^\prime = \bot \mid b = 1] \\
	&\geq \frac{1}{2} \cdot \Pr[ \mathsf{RUF}^{\mathcal{O}^\prime_{\mathsf{KeyGen}}}_{\mathsf{FE}^\prime}(\mathcal{A}) \to 1] + \frac{1}{2} \cdot \left( 1 - \frac{2|S|}{q} - (1 - \Pr[\mathsf{RUF}^{\mathcal{O}^\prime_{\mathsf{KeyGen}}}_{\mathsf{FE}^\prime}(\mathcal{A}) \to 1]) \right) \\
	&= \frac{1}{2} \cdot \Pr[ \mathsf{RUF}^{\mathcal{O}^\prime_{\mathsf{KeyGen}}}_{\mathsf{FE}^\prime}(\mathcal{A}) \to 1] - \frac{|S|}{q} + \frac{1}{2} \cdot \Pr[\mathsf{RUF}^{\mathcal{O}^\prime_{\mathsf{KeyGen}}}_{\mathsf{FE}^\prime}(\mathcal{A}) \to 1]) \\
	&= \Pr[ \mathsf{RUF}^{\mathcal{O}^\prime_{\mathsf{KeyGen}}}_{\mathsf{FE}^\prime}(\mathcal{A}) \to 1] - \frac{|S|}{q}
\end{align*}

\noindent Since $\Adv_{\textsf{FE}, \mathcal{R}}^\textsf{fh-IND} = \left| \Pr[\textsf{fh-IND}(\mathcal{R}) \to 1] - \frac{1}{2} \right|$ and $\frac{|S|}{q}$ are negligible,
\[
	\Pr[ \textsf{RUF}^{\mathcal{O}^\prime_{\textsf{KeyGen}}, \gamma}(\mathcal{A}) \to 1]  \leq \frac{|S|}{q} + \Adv_{\textsf{FE}, \mathcal{R}}^\textsf{fh-IND} = \negl.
\]

\end{proof}

To instantiate the authentication scheme $\Pi$ using the way described in Section \ref{sec:fh-IPFE-instantiation} with $\mathsf{FE}^\prime$, we assume that $s \in S$ for all integer $s \leq \tau^2$, the pre-defined threshold in the biometric layer. Along with Theorem \ref{thm:fh-IPFE:ind-ruf-OB-Enroll}, we have the following result.

\begin{corollary}
\label{cor:fh-IPFE:2016440-OB-Enroll}

Let $\textsf{option} = \{ \textsf{csk}, \mathbf{c_x}, \mathcal{O}_\mathcal{B}, \mathcal{O}_{\textsf{Enroll}} \}$, and let $\mathsf{FE}$ be the scheme described in \cite{cryptoeprint:2016/440}. If $\mathsf{FE}$ is fh-IND secyre, then for any distribution family $\mathbb{B}$, the authentication scheme instantiated by $\mathsf{FE}^\prime$ is $\textsf{option}$-UF secure.

\end{corollary}


%-------------------


\subsection{IND Security of $\Pi$}
\label{sec:security_analysis:fh-IPFE:IND}

For the IND security, we first consider the following definition and assumption on the biometric distribution family $\mathbb{B}$.

\begin{definition}
For an authentication scheme $\Pi$, a distribution $\mathcal{B} \in \mathbb{B}$, and an integer $t$, define the distribution $\mathcal{D}_\mathcal{B}(t)$ as
\[
	\mathcal{D}_\mathcal{B}(t) = \left( \textsf{BioCompare}(\mathbf{b}, \mathbf{b}^{(1)}), \textsf{BioCompare}(\mathbf{b}, \mathbf{b}^{(2)}), \cdots, \textsf{BioCompare}(\mathbf{b}, \mathbf{b}^{(t)}) \right)
\]
where $\mathbf{b} \gets \textsf{getEnroll}^{\mathcal{O}_\mathcal{B}}()$ and $ \mathbf{b}^{(i)} \gets \textsf{getProbe}^{\mathcal{O}_\mathcal{B}}()$ for all $i \in [t]$.

\end{definition}

\begin{assumption}
\label{assump:equal_distance}
Let $t$ be an integer. Assume that for any two distributions $\mathcal{B}^{(0)}$ and $\mathcal{B}^{(1)}$ in the biometric distribution family $\mathbb{B}$, $\mathcal{D}_{ \mathcal{B}^{(0)} }(t)$ and $\mathcal{D}_{ \mathcal{B}^{(1)} }(t)$ are the same. 

\end{assumption}

Note that indistinguishability between $\mathcal{D}_{ \mathcal{B}^{(0)} }(t)$ and $\mathcal{D}_{ \mathcal{B}^{(1)} }(t)$ is a necessary condition to achieve \textsf{option}-IND security when $\textsf{option}$ includes $\textsf{csk}, \mathbf{c_x}$ and $\mathcal{O}_{\mathbf{c_y}}$ because
\[
	\left( \textsf{Compare}(\textsf{csk}, \mathbf{c_x}, \mathbf{c_y}^{(1)}), \cdots, \textsf{Compare}(\textsf{csk}, \mathbf{c_x}, \mathbf{c_y}^{(t)}) \right) = \mathcal{D}_{ \mathcal{B}^{(b)} }(t)
\]
where $b$ is the challenge bit.


\begin{theorem}
\label{thm:fh-IPFE:ind-ind}
Let $\textsf{option} = \{ \textsf{csk}, \mathbf{c_x},  \mathcal{O}_{\mathbf{c_y}} \}.$ For a distribution family $\mathbb{B}$ satisfying Assumption \ref{assump:equal_distance} and having a true positive rate $\textsf{TP} > \frac{1}{\poly}$, if \textsf{FE} is fh-IND secure, then $\Pi$ is \textsf{option}-IND secure.

\end{theorem}



\begin{proof}

Given an adversary $\mathcal{A}$ in the $\textsf{IND}_{\textsf{option}}$ game, consider the reduction adversary $\mathcal{R}$ in Algorithm \ref{alg:red:ind-ind} which plays the \textsf{fh-IND} game by running $\mathcal{A}$. $\mathcal{R}$ simulates $\mathcal{O}_{\mathbf{c_y}}$ by the following steps.

\begin{enumerate}

	\item Sample ${\mathbf{b}^\prime}^{(0)} \gets \textsf{getProbe}^{\mathcal{O}_{\mathcal{B}^{(0)}}}()$ and let $\mathbf{y}^{(0)} \gets (-2{b_1^\prime}^{(0)}, \cdots, -2{b_k^\prime}^{(0)}, \| {\mathbf{b}^\prime}^{(0)} \|^2, 1) $

	\item Repeat sampling ${\mathbf{b}^\prime}^{(1)} \gets \textsf{getProbe}^{\mathcal{O}_{\mathcal{B}^{(1)}}}()$ and let $\mathbf{y}^{(1)} \gets (-2{b_1^\prime}^{(1)}, \cdots, -2{b_k^\prime}^{(1)}, \| {\mathbf{b}^\prime}^{(1)} \|^2, 1)$ until $\langle \mathbf{x}^{(0)}, {\mathbf{y}^{(0)}} \rangle = \langle \mathbf{x}^{(1)}, {\mathbf{y}^{(1)}} \rangle$.
	
	\item Return $\textsf{ct}_{\mathbf{y}}^{(i)} \gets \mathcal{O}_{\textsf{Enc}}(\mathbf{y}^{(0)}, \mathbf{y}^{(1)})$.

\end{enumerate}

\begin{figure}[h]
\centering
	
	\begin{minipage}[t]{0.75\linewidth}
	\centering
	\begin{algorithm}[H]
	\caption{$\mathcal{R}^{\mathcal{O}_{\textsf{KeyGen}}, \mathcal{O}_{\textsf{Enc}}}(\textsf{pp})$}
	\label{alg:red:ind-ind}
	\begin{algorithmic}[1]
		\State $\mathcal{B}^{(0)} \getsdollar \mathbb{B}, \quad \mathbb{B} \gets \mathbb{B} \setminus \mathcal{B}^{(0)}$ \label{alg:red:ind-ind:B0}
		
		\State $\mathcal{B}^{(1)} \getsdollar \mathbb{B}, \quad \mathbb{B} \gets \mathbb{B} \setminus \mathcal{B}^{(1)}$ \label{alg:red:ind-ind:B1}

		\State $\mathbf{b}^{(0)} \gets \textsf{getEnroll}^{\mathcal{O}_{\mathcal{B}^{(0)}}}(), \mathbf{x}^{(0)} \gets (b_1^{(0)}, \cdots, b_k^{(0)}, 1, \|\mathbf{b}^{(0)}\|^2)$
		
		\State $\mathbf{b}^{(1)} \gets \textsf{getEnroll}^{\mathcal{O}_{\mathcal{B}^{(1)}}}(), \mathbf{x}^{(1)} \gets (b_1^{(1)}, \cdots, b_k^{(1)}, 1, \|\mathbf{b}^{(1)}\|^2)$
		
		\State $\textsf{sk} \gets \mathcal{O}_{\textsf{KeyGen}}(\mathbf{x}^{(0)}, \mathbf{x}^{(1)})$ \label{alg:red:ind-ind:cx}

		\State $\tilde{b} \gets {\mathcal{A}}^{\mathcal{O}_{\mathcal{B}^{(0)}}, \mathcal{O}_{\mathcal{B}^{(1)}}, \mathcal{O}_{\mathbf{c_y}} } (\textsf{pp}, \textsf{sk})$ \label{alg:red:ind-ind:A}

		\State \Return $\tilde{b}$

	\end{algorithmic}
	\end{algorithm}
	\end{minipage}
	
\end{figure}

\noindent Note that $(\mathbf{x}^{(0)}, \mathbf{x}^{(1)})$ is the only query of $\mathcal{R}$ to $\mathcal{O}_{\textsf{KeyGen}}$, and for any query $( \mathbf{y}^{(0)}, {\mathbf{y}^{(1)}} )$ to $\mathcal{O}_{\textsf{Enc}}$, it satisfies $\langle \mathbf{x}^{(0)}, {\mathbf{y}^{(0)}} \rangle = \langle \mathbf{x}^{(1)}, {\mathbf{y}^{(1)}} \rangle$. Hence, $\mathcal{R}$ is an admissible adversary.

We first show that the simulation of oracle $\mathcal{O}_{\mathbf{c_y}}$ is efficient. The probability that $\langle \mathbf{x}^{(0)}, {\mathbf{y}^{(0)}} \rangle = \langle \mathbf{x}^{(1)}, {\mathbf{y}^{(1)}} \rangle$ is satisfied is
\begin{align*}
	\Pr[\mathcal{D}_{\mathcal{B}^{(0)}}(1) = \mathcal{D}_{\mathcal{B}^{(1)}}(1)] 
	&\geq \sum_{i=0}^\tau \Pr[\mathcal{D}_{\mathcal{B}^{(0)}}(1) = i]^2 \qquad \text{(Assumption \ref{assump:equal_distance})} \\
	&\geq \frac{1}{\tau + 1} \cdot \left( \sum_{i=0}^\tau \Pr[\mathcal{D}_{\mathcal{B}^{(0)}}(1) = i] \right)^2 \\
	&= \frac{1}{\tau + 1} \cdot \left( \Pr \left[
		\begin{aligned}
			& \mathbf{b} \gets \textsf{getEnroll}^{\mathcal{O}_{\mathcal{B}^{(0)}} }() \\
			& \mathbf{b}^\prime \gets \textsf{getProbe}^{\mathcal{O}_{\mathcal{B}^{(0)}} }()
		\end{aligned}
		: \| \mathbf{b} - \mathbf{b}^\prime \| \leq \tau \right] \right)^2 \\
	&= \frac{ \textsf{TP}(\mathcal{B}^{(0)})^2 }{\tau + 1} = \frac{ \textsf{TP}^2 }{\tau + 1} \qquad \text{(Assumption \ref{assump:equal_distance})} 
\end{align*}
The expected number of repetitions is bounded above by $\frac{\tau + 1}{ \textsf{TP}^2 }$. Moreover, the probability that it is satisfied within $T$ repetitions is at least
\[
	1 - (1 - \frac{ \textsf{TP}^2 }{\tau + 1})^T \geq 1 - e^{-T \cdot \frac{ \textsf{TP}^2 }{\tau + 1}}
\]
We can reach a $1 - \negl.$ probability that the loop will end within $T$ times by setting a polynomial-size $T$.

Now, we show that $\mathcal{R}$ perfectly simulate an $\textsf{IND}_{\textsf{option}}$ game for $\mathcal{A}$. Assume that $\mathcal{A}$ makes $t$ queries to $\mathcal{O}_{\mathbf{c_y}}$ and receives probe messages $\{ \mathbf{c_y}^{(i)} \}_{i=1}^t = \{ \textsf{ct}_{\mathbf{y}}^{(i)} \}_{i=1}^t$. If the challenge bit $b$ of the \textsf{fh-IND} game is $0$, $\mathbf{c_x} = \textsf{sk}$ and $\mathbf{c_y}^{(i)}$ for all $i \in [t]$ are generated from $\mathcal{B}^{(0)}$ and have the same distributions as the inputs for an adversary in \textsf{IND} game.
If the challenge bit $b$ is $1$, we show that distributions of $\mathbf{c_x}, \{ \mathbf{c_y}^{(i)} \}_{i=1}^t$ also follow the same distribution given Assumption \ref{assump:equal_distance}.

Let $b^\prime \in \{0, 1\}$, define distributions 
\begin{align*}
	\mathbf{X}^{(b^\prime)} &= \{ \mathbf{b}^{(b^\prime)} \gets \textsf{getEnroll}^{\mathcal{O}_{\mathcal{B}^{(b^\prime)}}}() : \mathbf{x}^{(b^{\prime})} \gets (b_1^{(b^{\prime})}, \cdots, b_k^{(b^{\prime})}, 1, \|\mathbf{b}^{ (b^{\prime}) }\|^2) \} \\
	\mathbf{Y}^{ (b^\prime) }_i &= \{ \mathbf{b}^{(b^\prime)} \gets \textsf{getProbe}^{\mathcal{O}_{\mathcal{B}^{(b^\prime)}}}() : \mathbf{y}^{(b^{\prime})} \gets ( -2b_1^{(b^{\prime})}, \cdots, -2b_k^{(b^{\prime})}, \|\mathbf{b}^{ (b^{\prime}) }\|^2, 1) \} \\
	\{ \mathbf{Y}^{ (b^\prime) }_i \}_{i=1}^t &= (\mathbf{Y}^{ (b^\prime) }_1, \cdots, \mathbf{Y}^{ (b^\prime) }_t) \quad  \text{ ($t$ identical and independent distributions) }
\end{align*}

\noindent Note that for any $\{ d_i \}_{i=1}^t, d_i > 0$,
\begin{align*}
\Pr \left[ \bigwedge_{i=1}^t \left\langle \mathbf{X}^{(0)}, {\mathbf{Y}^{(0)}_i} \right\rangle = d_i^2 \right] 
&= \Pr \left[ \mathcal{D}_{\mathcal{B}^{(0)}}(t) = (d_1, \cdots, d_t) \right] \\
&= \Pr \left[ \mathcal{D}_{\mathcal{B}^{(1)}}(t) = (d_1, \cdots, d_t) \right] = \Pr \left[ \bigwedge_{i=1}^t \left\langle \mathbf{X}^{(1)}, {\mathbf{Y}^{(1)}_i} \right\rangle = d_i^2 \right]
\end{align*}

Now, let $\mathbf{Y}^\prime_i$ be the distribution of $\mathbf{y}^{(1)}$ derived in the $i$-th query to $\mathcal{O}_{\mathbf{c_y}}$. For any $\mathbf{x}$ and $\{ \mathbf{y}_i \}_{i=1}^t$,
\begin{align*}
	& \Pr [\mathbf{X}^{(1)} = \mathbf{x}, \mathbf{Y}^\prime_1 = \mathbf{y}_1, \cdots, \mathbf{Y}^\prime_t = \mathbf{y}_t] \\
	&= \sum_{d_1, \cdots, d_t} \left( \Pr \left[\mathbf{X}^{(1)} = \mathbf{x}, \mathbf{Y}^{(1)}_1 = \mathbf{y}_1, \cdots, \mathbf{Y}^{(1)}_t = \mathbf{y}_t \mid \bigwedge_{i=1}^t \left\langle \mathbf{X}^{(1)}, {\mathbf{Y}^{(1)}_i} \right\rangle = d_i^2 \right] \right. \\
	&\qquad \qquad \left. \times \Pr \left[\bigwedge_{i=1}^t \left\langle \mathbf{X}^{(0)}, {\mathbf{Y}_i^{(0)}} \right\rangle = d_i^2 \right] \right) \\
	&= \sum_{d_1, \cdots, d_t} \left( \Pr \left[\mathbf{X}^{(1)} = \mathbf{x}, \mathbf{Y}^{(1)}_1 = \mathbf{y}_1, \cdots, \mathbf{Y}^{(1)}_t = \mathbf{y}_t \mid \bigwedge_{i=1}^t \left\langle \mathbf{X}^{(1)}, {\mathbf{Y}^{(1)}_i} \right\rangle = d_i^2 \right] \right. \\
	&\qquad \qquad \left. \times \Pr \left[\bigwedge_{i=1}^t \left\langle \mathbf{X}^{(1)}, {\mathbf{Y}_i^{(1)}} \right\rangle = d_i^2 \right] \right) \\
	&= \Pr[\mathbf{X}^{(1)} = \mathbf{x}, \mathbf{Y}^{(1)}_1 = \mathbf{y}_1, \cdots, \mathbf{Y}^{(1)}_t = \mathbf{y}_t ]
\end{align*}

\noindent which implies $\mathcal{R}$ also perfectly simulate an $\textsf{IND}_{\textsf{option}}$ game for $\mathcal{A}$ when the challenge bit $b = 1$.

In conclusion, 
\[
	\Adv_{\textsf{FE}, \mathcal{R}}^\textsf{fh-IND} = \Adv_{\Pi, \mathbb{B}, \mathcal{A}, \textsf{option}}^\textsf{IND} = \negl.
\]
which holds for all adversaries $\mathcal{A}$ in the $\textsf{IND}_{\textsf{option}}$ game. This implies the \textsf{option}-IND security of $\Pi$.


\end{proof}



%-------------------
% Security Analysis: Relational Hash-based Instantiation
\section{Security Analysis: Relational Hash-based Instantiation}
\label{sec:security_analysis:rh}
%%%%%%%%%%%%%%%%%%%%

% Project Name: Semester Project Fall 2024 for EPFL
% File: security_analysis_rh.tex
% Author: Keng-Yu Chen

%%%%%%%%%%%%%%%%%%%%

\begin{definition}[Relational Hash (adapted from \cite{cryptoeprint:2014/394})]
\label{def:rh}
	Let $R$ be a relation over sets $X, Y$, and $Z$. A \emph{relational hash} scheme \textsf{RH} for $R$ consists of PPT algorithms \textsf{RH.KeyGen}, $\textsf{RH.HASH}_1$, $\textsf{RH.HASH}_2$, and \textsf{RH.Verify}:
	
	\begin{itemize}
	
		\item $\textsf{RH.KeyGen}(1^\lambda) \to \textsf{pk}$: It outputs a public hash key \textsf{pk}.  
			
		\item $\textsf{RH.Hash}_1(\textsf{pk}, \mathbf{x}) \to \mathbf{h_x}$: Given a hash key \textsf{pk} and $\mathbf{x} \in X$, it outputs a hash $\mathbf{h_x}$.

		\item $\textsf{RH.Hash}_2(\textsf{pk}, \mathbf{y}) \to \mathbf{h_y}$: Given a hash key \textsf{pk} and $\mathbf{y} \in Y$, it outputs a hash $\mathbf{h_y}$.

		\item $\textsf{RH.Verify}(\textsf{pk}, \mathbf{h_x}, \mathbf{h_y}, \mathbf{z}) \to r \in \{0, 1\}$: Given a hash key \textsf{pk}, two hashes $\mathbf{h_x}$ and $\mathbf{h_y}$, and $\mathbf{z} \in Z$, it verifies whether the relation among $\mathbf{x}, \mathbf{y}$ and $\mathbf{z}$ holds.

	\end{itemize}

	\paragraph{\textbf{Correctness}} A relational hash scheme \textsf{RH} is \emph{correct} if $\forall \mathbf{x}, \mathbf{y}, \mathbf{z} \in X \times Y \times Z$,
	\[
		\Pr \left [
			\begin{aligned} 
				 &\; \textsf{pk} \gets \textsf{RH.KeyGen}(1^\lambda) \\
				 &\; \mathbf{h_x} \gets \textsf{RH.Hash}_1(\textsf{pk}, \mathbf{x}) \\
				 &\; \mathbf{h_y} \gets \textsf{RH.Hash}_2(\textsf{pk}, \mathbf{y})
			\end{aligned} :
			\textsf{RH.Verify}(\textsf{pk}, \mathbf{h_x}, \mathbf{h_y}, \mathbf{z}) = R(\mathbf{x}, \mathbf{y}, \mathbf{z})
			\right ] = 1 - \negl.
	\]
\end{definition}

\noindent Note that $Z_\lambda$ is an auxiliary input. When the relation $R$ is over two sets $X \times Y$, we ignore $Z$ and write $\textsf{RH.Verify}(\textsf{pk}, \mathbf{h_x}, \mathbf{h_y})$.

%-------------------

\subsection{Instantiation with a Relational Hash Scheme}
\label{sec:rh-instantiation}

Let $\textsf{RH} = (\textsf{RH.KeyGen}, \textsf{RH.Hash}_1, \textsf{RH.Hash}_2, \textsf{RH.Verify})$ be a relational hash scheme for the relation $R^\tau$ of Hamming distance proximity parametrized by a constant $\tau$.
\[
	R^\tau = \{ (\mathbf{x}, \mathbf{y}) \mid \textsf{HD}(\mathbf{x}, \mathbf{y}) \leq \tau \wedge \mathbf{x}, \mathbf{y} \in \{0,1\}^k \}
\]
Note that here we ignore the third parameter $Z$.
Let $\textsf{getEnroll}^{\mathcal{O}_{\mathcal{B}}}()$ and $\textsf{getProbe}^{\mathcal{O}_{\mathcal{B}}}()$ both output vectors in $ \{0, 1\}^k$ for all biometric distributions $\mathcal{B} \in \mathbb{B}$, and let 
\[
	\textsf{BioCompare}(\mathbf{b}, \mathbf{b}^\prime) \to
	\begin{cases}
		1 & \text{if } (\mathbf{b}, \mathbf{b}^\prime) \in R^\tau \\
		0 & \text{if } (\mathbf{b}, \mathbf{b}^\prime) \notin R^\tau
	\end{cases} \quad \text{and} \quad
	\textsf{Verify}(s) \to s.
\]
Following \cite{cryptoeprint:2014/394}, we can instantiate a biometric authentication scheme using $\textsf{RH}$.  Let the biometric distribution $\mathcal{B} \subseteq \{0,1\}^k$.

\begin{itemize}

	\item $\textsf{Setup}(1^\lambda)$: It calls $\textsf{RH.KeyGen}(1^\lambda) \to \textsf{pk}$ and outputs $\textsf{esk} \gets \textsf{pk}$, $\textsf{psk} \gets \textsf{pk}$, and $\textsf{csk} \gets \textsf{pk}$.

	\item $\textsf{Enroll}(\textsf{esk}, \mathbf{b})$: Let $\mathbf{x} \gets \mathbf{b}$. It calls $\textsf{RH.Hash}_1(\textsf{pk}, \mathbf{x}) \to \mathbf{h_x}$ and outputs $\mathbf{c_x} \gets \mathbf{h_x}$.

	\item $\textsf{Probe}(\textsf{psk}, \mathbf{b}^\prime)$: Let $\mathbf{y} \gets \mathbf{b}$. It calls $\textsf{RH.Hash}_2(\textsf{pk}, \mathbf{y}) \to \mathbf{h_y}$ and outputs $\mathbf{c_y} \gets \mathbf{h_y}$.

	\item $\textsf{Compare}(\textsf{csk}, \mathbf{c_x}, \mathbf{c_y)}$: It calls $\textsf{RH.Verify}(\textsf{pk}, \mathbf{h_x}, \mathbf{h_y}) \to s$ and outputs the value $s$.

\end{itemize}

By the correctness of the relational hash scheme $\textsf{RH}$, we have (except for a negligible probability),
\[
	r = 1 \Leftrightarrow (\mathbf{x}, \mathbf{y}) = (\mathbf{b}, \mathbf{b}^\prime) \in R^\tau \Leftrightarrow \textsf{HD}(\mathbf{b}, \mathbf{b}^\prime) \leq \tau
\]

The idea behing this construction is that users hold a personal device which runs \textsf{Setup}, \textsf{Enroll}, and \textsf{Probe} using templates coming from a biometric sensor. The message $\mathbf{c_x}$ along with the comparison key $\textsf{csk} = \textsf{pk}$, which is assumed to be public, are sent to a server in order to enroll the user, and $\mathbf{c_y}$ is sent on authentication.


Now, let $\Pi$ be an authentication scheme instantiated by a relational hash scheme \textsf{RH}. We discuss the UF and IND security of $\Pi$ in the following subsections.

%-------------------

\subsection{UF Security of $\Pi$}
\label{sec:security_analysis:rh:uf}

We first recall the unforgeability \cite{cryptoeprint:2014/394} of a relational hash scheme.

\begin{definition}[Unforgeability]

A relational hash scheme \textsf{RH} is called \emph{unforgeable} for the distribution $\mathcal{X}$ if for any adversary $\mathcal{A}$, the following probability is negligible.
\[
	\Pr \left [
		\begin{aligned} 
			 &\; \mathbf{x} \getsdollar \mathcal{X} \\
			 &\; \textsf{pk} \gets \textsf{RH.KeyGen}(1^\lambda) \\
			 &\; \mathbf{h_x} \gets \textsf{RH.Hash}_1(\textsf{pk}, \mathbf{x}) \\
			 &\; \mathbf{\tilde{z}} \gets \mathcal{A}(\textsf{pk}, \mathbf{h_x})
		\end{aligned} :
		\textsf{RH.Verify}(\textsf{pk}, \mathbf{h_x}, \mathbf{\tilde{z}}) = 1
		\right ] = \negl.
\]

\noindent In this work, since we assume the existence of a family $\mathbb{B}$ of biometric distributions and interfaces to interact with it, we extend the notion to \emph{unforgeable for any adversary who has access to $\mathbb{B}$}.
\end{definition}

\begin{theorem}
\label{thm:rh:uf-uf-cx}

Let $\textsf{option} = \{\textsf{esk}, \textsf{psk}, \textsf{csk}, \mathbf{c_x} \}$. If \textsf{RH} is unforgeable for the distribution
\[
	\mathcal{X} = \{ \mathcal{B} \getsdollar \mathbb{B}: \mathbf{b} \gets {\sf getEnroll}^{\mathcal{O}_\mathcal{B}}() \},
\]
for any adversary who has access to $\mathbb{B}$, then $\Pi$ is $\textsf{option}$-UF secure. 

\end{theorem}

In \cite{cryptoeprint:2014/394}, the authors construct an $\textsf{RH}$ that is unforgeable for the uniform distribution over $\{0, 1\}^k$, under the hardness of some computational problems.

\begin{proof}

Recall that the distribution of $\mathbf{c_x}$ in the \textsf{UF} game of the instantiation of Section \ref{sec:rh-instantiation} is
\[
	\left \{
		\begin{aligned} 
			 & \mathcal{B} \getsdollar \mathbb{B} \\
			 & \textsf{pk} \gets \textsf{RH.KeyGen}(1^\lambda) \\
			 & \mathbf{x} = \mathbf{b} \gets \textsf{getEnroll}^{\mathcal{O}_\mathcal{B}}() 
		\end{aligned} :
		\mathbf{c_x} \gets \textsf{RH.Hash}_1(\textsf{pk}, \mathbf{x})
	\right \}
\]
Also recall that $\textsf{Verify}(\textsf{Compare}(\textsf{csk}, \mathbf{c_x}, \mathbf{\tilde{z}} )) = \textsf{RH.Verify}(\textsf{pk}, \mathbf{c_x}, \mathbf{\tilde{z}} )$.
The \textsf{option}-UF security is thus guaranteed by the unforgeability of \textsf{RH}.

\end{proof}

\paragraph{Remark}
As we mentioned in Section \ref{sec:uf_game}, an adversary with \textsf{psk} can enjoy a winning rate of the false positive rate \textsf{FP} of $\mathbb{B}$. Theorem \ref{thm:rh:uf-uf-cx} thus implies that if $\textsf{FP}$ is not negligible, there does not exist an \textsf{RH} that is unforgeable for the distribution $\{ \mathcal{B} \getsdollar \mathbb{B}: \mathbf{b} \gets {\sf getEnroll}^{\mathcal{O}_\mathcal{B}}() \}$ for any adversary who has access to $\mathbb{B}$.


Note that since $\textsf{esk}, \textsf{psk}$, and $ \textsf{csk}$ are all public in this instantiation, it is meaningless to discuss $\mathcal{O}_\textsf{Enroll}, \mathcal{O}_\textsf{Probe}$, or $\mathcal{O}_\textsf{log}$. In addition, for $\textsf{option}$ that includes $\mathcal{O}_\mathcal{B}$ or $\mathcal{O}_\textsf{Probe}^\prime$, as discussed in Section \ref{sec:uf_game}, we cannot achieve \textsf{option}-UF security since $\textsf{psk}$ is public in this instantiation.

For \textsf{option} that includes $\mathcal{O}_\textsf{Enroll}^\prime$, we notice that for the \textsf{RH} construction in \cite{cryptoeprint:2014/394}, there exists an invalid $\textsf{pk}^\prime$ such that $\textsf{RH.Hash}_1(\textsf{pk}^\prime, \mathbf{x})$ directly leaks $\mathbf{x}$. By returning $\textsf{RH.Hash}_2( \textsf{pk}, \mathbf{x} )$, one can break the $\textsf{UF}_{\textsf{option}}$ game with probability $1$.


%-------------------

\subsection{IND Security of $\Pi$}
\label{sec:security_analysis:rh:IND}

For the IND security, since $\textsf{esk}, \textsf{psk}$ and $\textsf{csk}$ are assumed to be public in this instantiation and should given in $\textsf{option}$, by applying Theorem \ref{thm:ind-tp-fp}, $\Pi$ is not $\textsf{option}$-IND secure for any \textsf{option} that includes $\mathbf{c_x}$ or $\mathcal{O}_{\mathbf{c_y}}$.

\begin{theorem}

Let $\textsf{option} = \{\textsf{esk}, \textsf{psk}, \textsf{csk}, \mathbf{c_x}\}$ or $\{\textsf{esk}, \textsf{psk}, \textsf{csk}, \mathcal{O}_{\mathbf{c_y}} \}$. For any distribution family $\mathbb{B}$ that $\textsf{TP} - \textsf{FP} > \frac{1}{\poly}$, and for any relational hash scheme \textsf{RH}, $\Pi$ is not $\textsf{option}$-IND secure.

\end{theorem}

\subsubsection{IND Security for a Particular Biometric Layer}

Recall that in Section \ref{sec:ind_game}, we introduce as an example a particular biometric layer:
\[
	\textsf{getEnroll}^{\mathcal{O}_{\mathcal{B}}}() \to \mathbf{b}^* + \mathcal{E}_{\textsf{Enroll}}  \quad \text{and} \quad \textsf{getProbe}^{\mathcal{O}_{\mathcal{B}}}() \to \mathbf{b}^* + \mathcal{E}_{\textsf{Probe}}
\]
where $\mathbf{b}^* \in \{0, 1\}^k$ is a fixed vector only dependent on $\mathcal{B}$, and $\mathcal{E}_{\textsf{Enroll}}, \mathcal{E}_{\textsf{Probe}} \subseteq \{0, 1\}^k$ are some \emph{error distributions} independent of $\mathcal{B}$.
With the same relational hash \textsf{RH} in Section \ref{sec:rh-instantiation}, we can instantiate another authentication scheme using \textsf{RH}.

\begin{itemize}

	\item $\textsf{Setup}(1^\lambda)$: It runs $\textsf{RH.KeyGen}(1^\lambda) \to \textsf{pk}$ and samples $\mathbf{r} \getsdollar \{0, 1\}^k$. Then it outputs $\textsf{esk} \gets (\textsf{pk}, \mathbf{r})$, $\textsf{psk} \gets (\textsf{pk}, \mathbf{r})$, and $\textsf{csk} \gets \textsf{pk}$.

	\item $\textsf{Enroll}(\textsf{esk}, \mathbf{b})$: Let $\mathbf{x} \gets \mathbf{b}$. It calls $\textsf{RH.Hash}_1(\textsf{pk}, \mathbf{x} + \mathbf{r}) \to \mathbf{h_x}$ and outputs $\mathbf{c_x} \gets \mathbf{h_x}$.

	\item $\textsf{Probe}(\textsf{psk}, \mathbf{b}^\prime)$: Let $\mathbf{y} \gets \mathbf{b}$. It calls $\textsf{RH.Hash}_2(\textsf{pk}, \mathbf{y} + \mathbf{r}) \to \mathbf{h_y}$ and outputs $\mathbf{c_y} \gets \mathbf{h_y}$.

	\item $\textsf{Compare}(\textsf{csk}, \mathbf{c_x}, \mathbf{c_y)}$: It calls $\textsf{RH.Verify}(\textsf{pk}, \mathbf{h_x}, \mathbf{h_y}) \to s$ and outputs the value $s$.

\end{itemize}
Correctness holds because
\[
	\textsf{Compare}(\textsf{csk}, \mathbf{c_x}, \mathbf{c_y)} = 1 \Leftrightarrow \textsf{HD}(\mathbf{x} + \mathbf{r}, \mathbf{y} + \mathbf{r}) \leq \tau \Leftrightarrow \textsf{HD}(\mathbf{x}, \mathbf{y}) \leq \tau = \textsf{BioCompare}(\mathbf{b}, \mathbf{b}^\prime).
\]

With the same argument in Theorem \ref{thm:rh:ind:particular-biometri-layer}, one can prove that this new scheme is now $\{\textsf{csk}, \mathbf{c_x}, \mathcal{O}_{\mathbf{c_y}}\}$-IND secure, albeit at the cost of requiring $\textsf{esk}$ and $\textsf{psk}$ to remain secret.


%-------------------

\newpage
\appendix

%-------------------

\section{Construction in \cite{cryptoeprint:2016/440}}
\label{sec:fh-IPFE-construction}

Let $\mathbb{G}_1$ and $\mathbb{G}_2$ be two groups of order a prime number $q$ with generators $g_1$ and $g_2$, respectively. Let $e: \mathbb{G}_1 \times \mathbb{G}_2 \to \mathbb{G}_T$ be a mapping to a target group $\mathbb{G}_T$ also of order $q$. 

\begin{definition}[Bilinear asymmetric group \cite{cryptoeprint:2016/440}]
\label{bilinear-group}

A tuple $(\mathbb{G}_1, \mathbb{G}_2, \mathbb{G}_T, q, e)$ is a \emph{bilinear asymmetric group} if the following hold.

\begin{itemize}

	\item Group operations in $\mathbb{G}_1, \mathbb{G}_2$, and $\mathbb{G}_T$ and mapping $e$ are efficiently computable.

	\item $e$ is bilinear. That is, for $x, y \in \Z_q$, $e(g_1^x, g_2^y) = e(g_1, g_2)^{xy}$.

	\item $e$ is non-degenerate. That is, $e(g_1, g_2) \neq 1$, the identity element of $\mathbb{G}_T$.

\end{itemize}

\end{definition}

For a vector $\mathbf{v} = (v_1, v_2, \cdots, v_n) \in \Z_q^n$ and a group element $g$ in group of order $q$, we write $g^\mathbf{v}$ to denote the vector of group elements $(g^{v_1}, g^{v_2}, \cdots, g^{v_n})$. Moreover, for $k \in \Z_q$ and $\mathbf{v}, \mathbf{w} \in \Z_q^n$, we write $(g^{\mathbf{v}})^k = g^{k \cdot \mathbf{v}}$ and $g^\mathbf{v} \cdot g^\mathbf{w} =  g^{\mathbf{v} + \mathbf{w}}$. Finally, the pairing operation is extended to vectors.
\[
	e(g_1^{\mathbf{v}}, g_2^{\mathbf{w}}) = \prod_{i \in [n]} e(g_1^{v_i}, g_2^{w_i}) = e(g_1, g_2)^{\langle \mathbf{v}, \mathbf{w} \rangle}.
\]

We now recall the fh-IPFE construction $\textsf{FE}$ in \cite{cryptoeprint:2016/440}.

\begin{itemize}

	\item $\textsf{FE.Setup}(1^\lambda)$: Sample an asymmetric bilinear group $(\mathbb{G}_1, \mathbb{G}_2, \mathbb{G}_T, q, e)$ and choose generators $g_1 \in \mathbb{G}_1$ and $g_2 \in \mathbb{G}_2$. Sample $\mathbf{B} \in \mathbb{GL}_n(\Z_q)$ and find $\mathbf{B}^* = \det(\mathbf{B}) \cdot (\mathbf{B}^{-1})^T$. Finally, output the public parameter $\textsf{pp} = (\mathbb{G}_1, \mathbb{G}_2, \mathbb{G}_T, q, e)$ and the master secret key $\textsf{msk} = (\textsf{pp}, g_1, g_2, \mathbf{B}, \mathbf{B}^*)$.
	
	\item $\textsf{FE.KeyGen}(\textsf{msk}, \textsf{pp}, \mathbf{x})$: Sample $\alpha \getsdollar \Z_q$ and output
	\[
		 \textsf{sk}_{\mathbf{x}} = (K_1, K_2) = \left( g_1^{\alpha \cdot \det(\mathbf{B})}, g_1^{\alpha \cdot \mathbf{x} \cdot \mathbf{B}} \right)
	\]
	
	\item $\textsf{FE.Enc}(\textsf{msk}, \textsf{pp}, \mathbf{y})$: Sample $\beta \getsdollar \Z_q$ and output
	\[
		 \textsf{ct}_{\mathbf{y}} = (C_1, C_2) = \left( g_2^{\beta}, g_2^{\beta \cdot \mathbf{y} \cdot \mathbf{B}^*} \right)
	\]
	
	\item $\textsf{FE.Dec}(\textsf{pp}, \textsf{sk}_{\mathbf{x}}, \textsf{ct}_{\mathbf{y}}) \to z$: Parse $\textsf{sk}_{\mathbf{x}} = (K_1, K_2)$ and $\textsf{ct}_{\mathbf{y}} = (C_1, C_2)$ and compute 
	\[
		D_1 = e(K_1, C_1) \quad \text{and} \quad D_2 = e(K_2, C_2)
	\]
	Solve the discrete logarithm to find $z$ such that $D_1^z = D_2$ and output $z$. If it fails to find such $z$, output $\bot$.

\end{itemize}

\paragraph{Correctness}
We have
\[
	D_1 = e(K_1, C_1) = e(g_1, g_2)^{\alpha \cdot \beta \cdot \det(\mathbf{B})}
\] and 
\[
	D_2 = e(K_2, C_2) = e(g_1, g_2)^{\alpha \cdot \beta \cdot \mathbf{x} \cdot \mathbf{B} \cdot (\mathbf{B}^*)^T \cdot \mathbf{y}^T \cdot } = e(g_1, g_2)^{\alpha \cdot \beta \cdot \det(\mathbf{B}) \cdot \mathbf{x}\mathbf{y}^T }.
\]
Therefore, $(D_1)^{\langle \mathbf{x}, \mathbf{y} \rangle} = D_2$.

\paragraph{Remark}
In this construction, $q$ is exponential to $\lambda$ to achieve security, and decryption relies on some priori knowledge of possible ranges of the inner product $\langle \mathbf{x}, \mathbf{y} \rangle$. For example, for the instantiation in Section \ref{sec:fh-IPFE-instantiation}, one can enumerate $z \in \{0, 1, \cdots, \tau \}$ and return $\bot$ when no valid $z \leq \tau$ such that $D_1^z = D_2$ is found.

%-------------------

\newpage

%-------------------

\section{$\gamma$-RUF Security of \textsf{FE}}

Let $\mathbb{F} = \Z_q$. We can extend the definition of the RUF security in Section \ref{sec:security_analysis:fh-IPFE:ruf} with an integer parameter $\gamma$.

\begin{figure}[H]
\centering

	\begin{minipage}[t]{0.55\textwidth}
	\begin{algorithm}[H]
	\caption{$\textsf{RUF}^{\mathcal{O}, \gamma}_{\textsf{FE}}(\mathcal{A})$}
	\label{alg:gamma-ruf-fh-IPFE}
	\begin{algorithmic}[1]
		\State $\mathbf{r} \getsdollar \mathbb{F}^{k}$

		\State $\textsf{msk}, \textsf{pp} \gets \textsf{FE.Setup}(1^\lambda)$

		\State $\textsf{sk}_{\mathbf{r}} \gets \textsf{FE.KeyGen}(\textsf{msk}, \textsf{pp}, \mathbf{r})$

		\State $\mathbf{\tilde{z}} \gets \mathcal{A}^{\mathcal{O}} ( \textsf{pp}, \textsf{sk}_{\mathbf{r}} )$

		\If{$\mathbf{\tilde{z}}$ is equal to any output of $\mathcal{O}^\prime_{\textsf{Enc}}$ }
			
			\State \Return $0$
		
		\EndIf

		\State $s \gets \textsf{FE.Dec}(\textsf{pp}, \textsf{sk}_{\mathbf{r}}, \mathbf{\tilde{z}} )$

		\State \Return $1_{s \leq \gamma}$
	\end{algorithmic}
	\end{algorithm}
	\end{minipage}

\end{figure}

Here, the game runs $1_{s \leq \gamma}$ by first viewing the field element $s \in \Z_q$ as a positive integer in $\{0, 1, \cdots, q-1 \}$ and comparing it with $\gamma$.

The oracle $\mathcal{O}$ can be nothing or include $\mathcal{O}^\prime_{\textsf{KeyGen}}(\cdot)$ and $\mathcal{O}^\prime_{\textsf{Enc}}(\cdot)$ based on the threat model as in Section \ref{sec:security_analysis:fh-IPFE:ruf}.

\begin{definition}[$\gamma$-RUF Security]

	An fh-IPFE scheme \textsf{FE} is called $\{ \mathcal{O}, \gamma \}$-RUF secure if for any adversary $\mathcal{A}$, the advantage of $\mathcal{A}$ in the $\textsf{RUF}^{\mathcal{O}, \gamma}_\textsf{FE}$ game in Algorithm \ref{alg:gamma-ruf-fh-IPFE} is
\[
	\Adv_{\textsf{FE}, \mathcal{A}}^{\textsf{RUF}, \mathcal{O}, \gamma} := \Pr[\textsf{RUF}^{\mathcal{O}, \gamma}_{\textsf{FE}}(\mathcal{A}) \to 1 ] = \negl.
\]

\noindent Note that if $\textsf{FE}$ is $\mathcal{O}$-RUF secure, it is $\{ \mathcal{O}, \gamma \}$-RUF secure for any integer $\gamma$.
\end{definition}

With the extension with $\gamma$, we can rewrite our results in Section \ref{sec:security_analysis:fh-IPFE}.


\begin{theorem}[Theorem \ref{thm:fh-IPFE:ind-ruf-OB-Enroll}]
\label{thm:fh-IPFE:ind-gamma-ruf-OB-Enroll}
	Let $\textsf{option} = \{ \textsf{csk}, \mathbf{c_x}, \mathcal{O}_\mathcal{B}, \mathcal{O}_{\textsf{Enroll}} \}$. For any distribution family $\mathbb{B}$, if \textsf{FE} is fh-IND and $\{ \mathcal{O}^\prime_{\textsf{KeyGen}}, \gamma \}$-RUF for a $\gamma \geq \tau^2$, then $\Pi$ is $\textsf{option}$-UF. 
\end{theorem}

\begin{theorem}[Theorem \ref{thm:fh-IPFE:ind-ruf-OB-Probe}]
\label{thm:fh-IPFE:ind-gamma-ruf-OB-Probe}
	Let $\textsf{option} = \{\textsf{csk}, \mathbf{c_x}, \mathcal{O}_\mathcal{B}, \mathcal{O}_\textsf{Probe}\}$. For any distribution family $\mathbb{B}$, if \textsf{FE} is fh-IND and $\{ \mathcal{O}^\prime_{\textsf{Enc}}, \gamma \}$-RUF for a $\gamma \geq \tau^2$, then $\Pi$ is $\textsf{option}$-UF. 
\end{theorem}

\subsection{Achievability of $\gamma$-RUF Security}

\begin{assumption}
\label{assump:only-return-valid-ct}
Assume that $\textsf{FE.Dec}(\textsf{pp}, \mathbf{c}, \mathbf{z})$ only returns when $\mathbf{z}$ corresponds to a \emph{non-zero} vector $\mathbf{v} \in \mathbb{F}^k$. More precisely, assume that for any $\mathbf{z}$, there can only be two possibilities.

\begin{itemize}
	\item Either there exists a vector $\mathbf{v} \in \mathbb{F}^k \setminus \{\mathbf{0}\}$ such that for any $\mathbf{x} \in \mathbb{F}^k, \textsf{sk}_{\mathbf{x}} \gets \textsf{FE.KeyGen}(\textsf{msk}, \textsf{pp}, \mathbf{x})$, 
	\[
		%\textsf{FE.Dec}(\textsf{pp}, \mathbf{c}, \mathbf{z}) = \textsf{FE.Dec}(\textsf{pp}, \mathbf{c}, \mathbf{c_v}).
		\textsf{FE.Dec}(\textsf{pp}, \textsf{sk}_{\mathbf{x}}, \mathbf{z}) = \langle \mathbf{x}, \mathbf{v} \rangle.
	\]
	\item Or for any $\mathbf{x} \in \mathbb{F}^k$ and $\textsf{sk}_{\mathbf{x}} \gets \textsf{FE.KeyGen}(\textsf{msk}, \textsf{pp}, \mathbf{x})$, $\textsf{FE.Dec}(\textsf{pp}, \textsf{sk}_{\mathbf{x}}, \mathbf{z}) \to \bot$.

\end{itemize}
Note that this implies $\textsf{FE}$ rejects the zero vector $\mathbf{0}$ as the input of $\textsf{FE.Enc}$.
\end{assumption}

\begin{theorem}
\label{thm:fh-IPFE:ind-OKeyGen-gamma-ruf}
Given Assumption \ref{assump:only-return-valid-ct}. If \textsf{FE} is fh-IND, then $\textsf{FE}$ is $\{ \mathcal{O}^\prime_{\textsf{KeyGen}}, \gamma \}$-RUF for any $\gamma \leq (1 - \frac{1}{\poly}) \cdot \|\mathbb{F}\|$.

\end{theorem}

\begin{proof}
Given an adversary $\mathcal{A}$ in the $\textsf{RUF}^{\mathcal{O}^\prime_{\textsf{KeyGen}}, \gamma}_{\textsf{FE}}$ game for a $\gamma \leq (1 - \frac{1}{P(\lambda)}) \cdot \|\mathbb{F}\|$, where $P(\lambda)$ is any polynomial. Let $t$ be an integer, consider the reduction adversary $\mathcal{R}$ in Algorithm \ref{alg:red:ind-OKeyGen-gamma-ruf} which plays the \textsf{fh-IND} game. $\mathcal{R}$ simulates $\mathcal{O}_\textsf{KeyGen}^\prime(\mathbf{x}^\prime)$ by $\mathcal{O}_\textsf{KeyGen}(\mathbf{x}^\prime, \mathbf{x}^\prime)$.
If there exists an $s_i \neq \bot$ in Line \ref{alg:red:ind-OKeyGen-gamma-ruf:s}, by Assumption \ref{assump:only-return-valid-ct}, let $\mathbf{\tilde{z}}$ correspond to a non-zero vector $\mathbf{\tilde{v}}$.

\begin{figure}[h]
\centering
	
	\begin{minipage}[t]{0.5\linewidth}
	\centering
	\begin{algorithm}[H]
	\caption{$\mathcal{R}^{\mathcal{O}_{\textsf{KeyGen}}, \mathcal{O}_{\textsf{Enc}}}(\textsf{pp})$}
	\label{alg:red:ind-OKeyGen-gamma-ruf}
	\begin{algorithmic}[1]
		\State $\mathbf{r}^{(0)}, \mathbf{r}^{(1)} \getsdollar \mathbb{F}^{k}$
		
		\State $\textsf{sk} \gets \mathcal{O}_{\textsf{KeyGen}}(\mathbf{r}^{(0)}, \mathbf{r}^{(1)})$ 

		\State ${\mathbf{\tilde{z}}} \gets {\mathcal{A}}^{\mathcal{O}^\prime_{\textsf{KeyGen}}} (\textsf{pp}, \textsf{sk})$

		\For{$i = 1$ to $t$}
		
			\State $\mathbf{r}_i \getsdollar \mathbb{F}^{k}$

			\State $\textsf{sk}_i \gets \mathcal{O}_{\textsf{KeyGen}}(\mathbf{r}^{(0)}, \mathbf{r}_i)$

			\State $s_i \gets \textsf{FE.Dec}( \textsf{pp}, \textsf{sk}_i, \mathbf{\tilde{z}} )$ \label{alg:red:ind-OKeyGen-gamma-ruf:s}
	
		\EndFor	
		
		\If{$\bigwedge_{i=1}^t s_i \leq \gamma$} \label{alg:red:ind-OKeyGen-gamma-ruf:verify}
			\State \Return $\tilde{b} = 0$
		\Else
			\State \Return $\tilde{b} \getsdollar \{0, 1\}$
		\EndIf

	\end{algorithmic}
	\end{algorithm}
	\end{minipage}
	
\end{figure}


If the challenge bit $b = 0$, then by Assumption \ref{assump:only-return-valid-ct}, any $s_i \neq \bot$ in Line \ref{alg:red:ind-OKeyGen-gamma-ruf:s} implies all $s_i \neq \bot$ and $s_i = s_j$ for any $i, j$. Therefore, the probability that all $s_i \leq \gamma$ in Line \ref{alg:red:ind-OKeyGen-gamma-ruf:verify} is
\begin{align*}
	\Pr\left[ \bigwedge_{i=1}^t s_i \leq \gamma \mid b = 0 \right]
	&= \Pr\left[ s_1 \neq \bot \mid b = 0 \right] \cdot \Pr \left[ s_1 \leq \gamma \mid b = 0 \wedge s_1 \neq \bot \right] \\
	&= \Pr \left[ s_1 \neq \bot \mid b = 0 \right] \cdot \Pr \left[ \langle \mathbf{r}^{(0)}, \mathbf{\tilde{v}} \rangle \leq \gamma \mid b = 0 \wedge s_1 \neq \bot \right] \\
	&= \Pr \left[s_1 \neq \bot \mid b = 0 \right] \cdot \Pr \left[ \textsf{FE.Dec}(\textsf{pp}, \textsf{sk}, \mathbf{\tilde{z}}) \leq \gamma \mid b = 0 \wedge s_1 \neq \bot \right] \\
	&= \Pr \left[s_1 \neq \bot \mid b = 0 \right] \cdot \Pr \left[ \textsf{RUF}^{\mathcal{O}^\prime_{\textsf{KeyGen}}, \gamma}(\mathcal{A}) \to 1 \mid b = 0 \wedge s_1 \neq \bot \right] \\ 
	&= \Pr \left[ \textsf{RUF}^{\mathcal{O}^\prime_{\textsf{KeyGen}}, \gamma}(\mathcal{A}) \to 1 \right] 
\end{align*}

If the challenge bit $b = 1$, for any $i \in [t]$,
\begin{align*}
	\Pr[ s_i \leq \gamma \mid b = 1 ]
	&= \Pr[s_i \neq \bot \mid b = 1] \cdot \Pr[ s_i \leq \gamma \mid b = 1 \wedge s_i \neq \bot] \\
	&= \Pr[s_i \neq \bot \mid b = 1] \cdot \Pr[ \langle \mathbf{r}_i, \mathbf{\tilde{v}} \rangle \leq \gamma \mid b = 1 \wedge s \neq \bot]
\end{align*}
Note that $\mathbf{r}_i$ is independent of $\mathbf{\tilde{z}}$ and thus independent of $\mathbf{\tilde{v}}$. Hence, $\Pr[ \langle \mathbf{r}_i, \mathbf{\tilde{v}} \rangle \leq \gamma \mid b = 1 \wedge s_i \neq \bot] = \frac{\gamma}{\| \mathbb{F} \|}$ and
\[
	\Pr \left[ \bigwedge_{i=1}^t s_i \leq \gamma \mid b = 1 \right] = \Pr \left[ \bigwedge_{i=1}^t s_i \neq \bot \mid b = 1 \right] \cdot \left( \frac{\gamma}{\| \mathbb{F} \|} \right)^t \leq \left( \frac{\gamma}{\| \mathbb{F} \|} \right)^t
\]

In conclusion,
\begin{align*}
	\Pr[\textsf{fh-IND}(\mathcal{R}) \to 1] 
	&= \frac{1}{2} + \frac{1}{4} \left( \Pr \left[ \bigwedge_{i=1}^t s_i \leq \gamma \mid b = 0 \right] - \Pr \left[ \bigwedge_{i=1}^t s_i \leq \gamma \mid b = 1 \right] \right) \\
	&\geq \frac{1}{2} + \frac{1}{4} \left( \Pr[ \textsf{RUF}^{\mathcal{O}^\prime_{\textsf{KeyGen}}, \gamma}(\mathcal{A}) \to 1] - \left( \frac{\gamma}{\| \mathbb{F} \|} \right)^t \right) \\
	&\geq \frac{1}{2} + \frac{1}{4} \left( \Pr[ \textsf{RUF}^{\mathcal{O}^\prime_{\textsf{KeyGen}}, \gamma}(\mathcal{A}) \to 1] -  e^{- t \cdot (1 - \frac{\gamma}{\| \mathbb{F} \|}) } \right) \\
	&\geq \frac{1}{2} + \frac{1}{4} \left( \Pr[ \textsf{RUF}^{\mathcal{O}^\prime_{\textsf{KeyGen}}, \gamma}(\mathcal{A}) \to 1] -  e^{- \frac{t}{P(\lambda)} } \right)
\end{align*}

\noindent Take $t$ be any integer larger than $P(\lambda) \cdot \lambda$. Since $\Adv_{\textsf{FE}, \mathcal{R}}^\textsf{fh-IND} = \left| \Pr[\textsf{fh-IND}(\mathcal{R}) \to 1] - \frac{1}{2} \right|$ and $e^{-t \cdot \frac{1}{P(\lambda)}} < e^{-\lambda}$ are negligible,
\[
	\Pr[ \textsf{RUF}^{\mathcal{O}^\prime_{\textsf{KeyGen}}, \gamma}(\mathcal{A}) \to 1]  \leq e^{- \frac{t}{P(\lambda)} } + 4 \cdot \Adv_{\textsf{FE}, \mathcal{R}}^\textsf{fh-IND}  = \negl.
\]

\end{proof}

%-------------------

\newpage

%-------------------


\section{Summary of UF and IND Security}

\begin{table}[ht]
\centering
\begin{tabular}{c c c} 
	
	\toprule

	& \textbf{UF} & \textbf{IND} \\
	
	\midrule

	\textsf{fh-IND} & \xmark \; (Thm \ref{thm:fh-IPFE-not-uf}) & $\{ \textsf{csk}, \mathbf{c_x},  \mathcal{O}_{\mathbf{c_y}} \}$ (Thm \ref{thm:fh-IPFE:ind-ind}) \\
	
	\midrule
	
	\textsf{fh-IND}, $\mathcal{O}^\prime_{\textsf{KeyGen}}$-RUF & $\{ \textsf{csk}, \mathbf{c_x}, \mathcal{O}_\mathcal{B}, \mathcal{O}_{\textsf{Enroll}} \}$ (Thm \ref{thm:fh-IPFE:ind-ruf-OB-Enroll}) & $\{ \textsf{csk}, \mathbf{c_x},  \mathcal{O}_{\mathbf{c_y}} \}$ (Thm \ref{thm:fh-IPFE:ind-ind}) \\
	
	\midrule
	
	\textsf{fh-IND}, $\mathcal{O}^\prime_{\textsf{Probe}}$-RUF & $\{ \textsf{csk}, \mathbf{c_x}, \mathcal{O}_\mathcal{B}, \mathcal{O}_{\textsf{Probe}} \}$ (Thm \ref{thm:fh-IPFE:ind-ruf-OB-Probe}) & $\{ \textsf{csk}, \mathbf{c_x},  \mathcal{O}_{\mathbf{c_y}} \}$ (Thm \ref{thm:fh-IPFE:ind-ind}) \\
	
	\midrule
	
	\textsf{fh-IND} + sEUF-CMA \textsf{Sig} & $\{ \textsf{esk}, \textsf{csk}, \mathbf{c_x}, \mathcal{O}_\mathcal{B}, \mathcal{O}_{\textsf{Probe}} \}$ (Thm \ref{thm:sEUF-CMA-esk-csk}) & $\{ \textsf{csk}, \mathbf{c_x},  \mathcal{O}_{\mathbf{c_y}} \}$ (Thm \ref{thm:fh-IPFE:ind-ind}) \\
	
	\midrule
	
	\textsf{fh-IND} + MC-IND-CPA \textsf{PKE} & \xmark \; (Thm \ref{thm:fh-IPFE-not-uf})  & \makecell{$\{ \textsf{csk}, \mathbf{c_x},  \mathcal{O}_{\mathbf{c_y}} \}$ (Thm \ref{thm:fh-IPFE:ind-ind}) \\ $\{ \textsf{esk}, \textsf{psk}, \mathbf{c_x}, \mathcal{O}_{\mathbf{c_y}} \}$ (Thm \ref{thm:mc-ind-cpa:ind-esk-psk})} \\
	
	\midrule
	
	\makecell{\textsf{fh-IND} + sEUF-CMA \textsf{Sig} \\ + MC-IND-CPA \textsf{PKE}} & $\{ \textsf{esk}, \textsf{csk}, \mathbf{c_x}, \mathcal{O}_\mathcal{B}, \mathcal{O}_{\textsf{Probe}} \}$ (Thm \ref{thm:sEUF-CMA-esk-csk}) & \makecell{$\{ \textsf{csk}, \mathbf{c_x},  \mathcal{O}_{\mathbf{c_y}} \}$ (Thm \ref{thm:fh-IPFE:ind-ind}) \\ $\{ \textsf{esk}, \textsf{psk}, \mathbf{c_x}, \mathcal{O}_{\mathbf{c_y}} \}$ (Thm \ref{thm:mc-ind-cpa:ind-esk-psk})} \\
	
	\bottomrule

\medskip
\end{tabular}
\caption{fh-IPFE. We can try to provide \textsf{option}-UF with \textsf{psk}.}
\label{table:fh-IPFE}
\end{table}



\begin{table}[ht]
\centering
\begin{tabular}{c c c} 
	
	\toprule

	& \textbf{UF} & \textbf{IND} \\
	
	\midrule

	Unforgeable for $\mathcal{X}$ & $\{ \textsf{esk}, \textsf{psk}, \textsf{csk}, \mathbf{c_x} \}$ (Thm \ref{thm:rh:uf-uf-cx}) & \xmark \; (Thm \ref{thm:ind-tp-fp}) \\
	
	\midrule
	
	\makecell{Unforgeable for $\mathcal{X}$ \\ + sEUF-CMA \textsf{Sig}} & \makecell{ $\{ \textsf{esk}, \textsf{psk}, \textsf{csk}, \mathbf{c_x} \}$ (Thm \ref{thm:rh:uf-uf-cx}) \\ $\{ \textsf{esk}, \textsf{csk}, \mathbf{c_x}, \mathcal{O}_\mathcal{B}, \mathcal{O}_{\textsf{Probe}} \}$ (Thm \ref{thm:sEUF-CMA-esk-csk}) } & \xmark \; (Thm \ref{thm:ind-tp-fp}) \\
	
	\midrule
	
	\makecell{ Unforgeable for $\mathcal{X}$ \\+ MC-IND-CPA \textsf{PKE}} & $\{ \textsf{esk}, \textsf{psk}, \textsf{csk}, \mathbf{c_x} \}$ (Thm \ref{thm:rh:uf-uf-cx}) & \makecell{$\{ \textsf{esk}, \textsf{psk}, \mathbf{c_x}, \mathcal{O}_{\mathbf{c_y}} \}$ (Thm \ref{thm:mc-ind-cpa:ind-esk-psk})} \\
	
	\midrule
	
	\makecell{Unforgeable for $\mathcal{X}$ \\ + sEUF-CMA \textsf{Sig} \\ + MC-IND-CPA \textsf{PKE}} & \makecell{ $\{ \textsf{esk}, \textsf{psk}, \textsf{csk}, \mathbf{c_x} \}$ (Thm \ref{thm:rh:uf-uf-cx}) \\ $\{ \textsf{esk}, \textsf{csk}, \mathbf{c_x}, \mathcal{O}_\mathcal{B}, \mathcal{O}_{\textsf{Probe}} \}$ (Thm \ref{thm:sEUF-CMA-esk-csk}) }  & \makecell{$\{ \textsf{esk}, \textsf{psk}, \mathbf{c_x}, \mathcal{O}_{\mathbf{c_y}} \}$ (Thm \ref{thm:mc-ind-cpa:ind-esk-psk})} \\
	
	\bottomrule

\medskip
\end{tabular}
\caption{RH. We can try to provide \textsf{option}-IND with \textsf{csk}.}
\label{table:rh}
\end{table}

%-------------------

\newpage

%-------------------


\section{One-Way Game}

\begin{figure}[h]
\centering

	\begin{minipage}[t]{0.55\textwidth}
	\begin{algorithm}[H]
	\caption{$\textsf{OW}_{\Pi, \mathbb{B}, \textsf{option}}(\mathcal{A})$}
	\label{alg:ow_game}
	\begin{algorithmic}[1]

		\State $\mathcal{B} \getsdollar \mathbb{B}, \quad \mathbb{B} \gets \mathbb{B} \setminus \mathcal{B}$

		\State $\textsf{esk}, \textsf{psk}, \textsf{csk} \gets \textsf{Setup}(1^\lambda)$

		\State $\mathbf{b} \gets \textsf{getEnroll}^{\mathcal{O}_{\mathcal{B}}}()$

		\State $\mathbf{c_x} \gets \textsf{Enroll}(\textsf{esk}, \mathbf{b})$

		\State $\mathbf{\tilde{b}} \gets \mathcal{A}( \textsf{option} )$

		\State $s \gets \textsf{BioCompare}(\mathbf{b}, \mathbf{\tilde{b}})$
		\State \Return $ \textsf{Verify}(s) $
	\end{algorithmic}
	\end{algorithm}
	\end{minipage}

\label{fig:ow_game}
\end{figure}

The auxiliary information \textsf{option} can be nothing or include $\textsf{esk}, \textsf{psk}, \textsf{csk}, \mathbf{c_x}$ or the oracles $\mathcal{O}_{\mathbf{c_y}}, \mathcal{O}_{\textsf{Enroll}}, \mathcal{O}_{\textsf{Probe}}, \mathcal{O}^\prime_{\textsf{Enroll}}, \mathcal{O}^\prime_{\textsf{Probe}}$. We can also consider providing $\mathbf{c_x}^{(i)} \gets \textsf{Enroll}(\textsf{esk}_i, \mathbf{b})$ for different honest $\textsf{esk}_i$. Note that the \textsf{OW} game is trivial if \textsf{FP} is not negligible.

We define the advantage of an adversary $\mathcal{A}$ in the $\textsf{OW}_{\Pi, \mathbb{B}, \textsf{option}}$ game of a scheme $\Pi$ associated with a family $\mathbb{B}$ of distributions as
\[
	\Adv^{\textsf{OW}}_{\Pi, \mathbb{B}, \mathcal{A}, \textsf{option}} := \Pr[\textsf{OW}_{\Pi, \mathbb{B}, \textsf{option}}(\mathcal{A}) \to 1]
\]

An authentication scheme $\Pi$ associated with a family $\mathbb{B}$ of distributions is called \emph{\textsf{option}-one-way} (\textsf{option}-OW) if for any PPT adversary $\mathcal{A}$,
\[
	\Adv^{\textsf{OW}}_{\Pi, \mathbb{B}, \mathcal{A}, \textsf{option}} = \negl.
\]

Its relation with UF security is as follows.

\begin{theorem}
	Let $\textsf{option}$ include $\textsf{psk}$. If $\Pi$ is $\textsf{option}$-UF, then $\Pi$ is $\textsf{option}$-OW.
\end{theorem}

\begin{proof}
	Suppose $\Pi$ is not $\textsf{option}$-OW, the adversary can derive $\tilde{\mathbf{b}}$ such that 
	\[
		\textsf{Verify}(\textsf{BioCompare}(\mathbf{b}, \tilde{\mathbf{b}})) = 1
	\]
	With $\textsf{psk}$, the adversary can further generate $\tilde{\mathbf{c_x}} \gets \textsf{Probe}(\textsf{psk}, \tilde{\mathbf{b}})$ to win the \textsf{option-UF} game.
\end{proof}

\begin{assumption}
\label{assump:consistent}
	Assume that for any $\mathcal{B}^{(0)}, \mathcal{B}^{(1)} \in \mathbb{B}$, $\mathcal{B}^{(0)} \neq \mathcal{B}^{(1)}$, and $\tilde{\mathbf{b}}$ in the domain of $\mathsf{BioCompare}$, let $\mathbf{b}^{(0)}, \mathbf{b}^{(0) \prime} \gets \mathsf{getEnroll}^{\mathcal{O}_{\mathcal{B}^{(0)}}}()$ and $\mathbf{b}^{(1) \prime} \gets \mathsf{getEnroll}^{\mathcal{O}_{\mathcal{B}^{(1)}}}()$, then
	\begin{align*}
		& \Pr[\mathsf{Verify}(\mathsf{BioCompare}(\mathbf{b}^{(0)\prime}, \tilde{\mathbf{b}})) = 1 \mid \mathsf{Verify}(\mathsf{BioCompare}(\mathbf{b}^{(0)}, \tilde{\mathbf{b}})) = 1] \\
		& - \Pr[\mathsf{Verify}(\mathsf{BioCompare}(\mathbf{b}^{(1)}, \tilde{\mathbf{b}})) = 1 \mid \mathsf{Verify}(\mathsf{BioCompare}(\mathbf{b}^{(0)}, \tilde{\mathbf{b}})) = 1] \geq \frac{1}{\poly}.
	\end{align*}
\end{assumption}

\begin{theorem}
	Given Assumption \ref{assump:consistent}. If $\Pi$ is $\textsf{option}$-IND, then $\Pi$ is $\textsf{option}$-OW.
\end{theorem}

\begin{proof}
	Suppose $\Pi$ is not \textsf{option}-OW, we can construct an adversary in the \textsf{option}-{IND} game that can derive $\tilde{\mathbf{b}}$ such that 
	\[
		\textsf{Verify}(\textsf{BioCompare}(\mathbf{b}, \tilde{\mathbf{b}})) = 1
	\]
	where $\mathbf{b} \gets \textsf{getEnroll}^{\mathcal{O}_{\mathcal{B}^{(b)}}}()$ and $b$ is the challenge bit. By assumption \ref{assump:consistent}, this implies that if the adversary runs $\mathbf{b}^{(0)\prime} \gets \mathsf{getEnroll}^{\mathcal{O}_{\mathcal{B}^{(0)}}}()$ and $\mathbf{b}^{(1) \prime} \gets \mathsf{getEnroll}^{\mathcal{O}_{\mathcal{B}^{(1)}}}()$, then
	\begin{gather*}
		\Pr[\mathsf{Verify}(\mathsf{BioCompare}(\mathbf{b}^{(b)\prime}, \tilde{\mathbf{b}})) = 1]  - \Pr[\mathsf{Verify}(\mathsf{BioCompare}(\mathbf{b}^{(1-b)\prime}, \tilde{\mathbf{b}})) = 1] \geq \frac{1}{\poly}.
	\end{gather*}
	By comparing the results of the two cases, the adversary can find $b$ with a probability greater than $\frac{1}{\poly}$.
\end{proof}
We have seen some examples that authentication schemes using current constructions of fh-IPFE and RH may not be OW-secure.

\paragraph{fh-IPFE}

Suppose $\Pi$ is instantiated by the construction in \cite{cryptoeprint:2016/440} in the way described in Section \ref{sec:fh-IPFE-instantiation}. The scheme is not $\mathcal{O}_{\textsf{Enroll}}^\prime$-OW. The adversary can set the fake $\textsf{esk}^\prime$ to include the matrix $\mathbf{B} = \mathbf{I}$, the size-$(k+2)$ identity matrix, and query $\mathcal{O}_{\textsf{Enroll}}^\prime$ to get
\[
	g_1^{\alpha \cdot \det(\mathbf{B})}, g_1^{\alpha \cdot \mathbf{x} \cdot \mathbf{B}} = g_1^{\alpha}, (g_1^{\alpha \cdot {x_1}}, g_1^{\alpha \cdot {x_2}}, g_1^{\alpha \cdot {x_{k+2}}}) 
\]
where $\alpha$ is an unknown element in $\Z_q$, and $\mathbf{x} = (x_1, \cdots, x_{k+2}) = (b_1, \cdots, b_k, 1, \|\mathbf{b}\|^2)$. Since $b_i$ is bounded in $\{0, 1, \cdots, m\}$ for some fixed integer $m$ for each $i$, the adversary can raise the power of $g_1^{\alpha}$ to retrive all the coefficients of $\mathbf{b}$. Similarly, the scheme is not $\mathcal{O}_{\textsf{Probe}}^\prime$-OW.

In addition, any fh-IPFE scheme instantiated in the way described in Section \ref{sec:fh-IPFE-instantiation} is neither $\{ \mathbf{c_x}, \textsf{esk} \}$-OW nor $\{ \mathbf{c_x}, \textsf{psk} \}$-OW. Note that $\textsf{esk} = \textsf{psk} = \textsf{msk}$, the master secret key of the fh-IPFE. With $\textsf{msk}$, the adversary can generate encryptions of any vector and find $\mathbf{x}$ by solving linear equations of inner products.

\paragraph{RH}

Suppose $\Pi$ is instantiated by the construction in \cite{cryptoeprint:2014/394} in the way described in Section \ref{sec:rh-instantiation}. The scheme is neither $\mathcal{O}_{\textsf{Enroll}}^\prime$-OW nor $\mathcal{O}_{\textsf{Probe}}^\prime$-OW. Recall that the enrollment key $\textsf{esk}$ is the public key $\textsf{pk}$ of the relational hash scheme. In the construction of \cite{cryptoeprint:2014/394},
\begin{itemize}
	\item $\textsf{pk}$ includes an encoding algorithm $\textsf{Encode}$ of an error correcting code.
	\item $\mathbf{h_x}$ includes $\mathbf{x} + \textsf{Encode}(\mathbf{r})$ for some random $\mathbf{r} \getsdollar \{0,1\}^k$.
\end{itemize}
Now, the adversary can set the fake $\textsf{esk}^\prime$ to include an invalid $\textsf{Encode}$ such that $\mathbf{x} + \textsf{Encode}(\mathbf{r})$ reveals $\mathbf{x} = \mathbf{b}$. 

The scheme is $\{ \textsf{esk}, \textsf{psk}, \textsf{csk}, \mathbf{c_x} \}$-OW, which is given in \cite[Theorem 4]{cryptoeprint:2014/394}.

%-------------------

\newpage

%-------------------

\section{Fixing the Proof of Theorem \ref{thm:fh-IPFE:ind-ruf-OB-Probe}}

In Theorem \ref{thm:fh-IPFE:ind-ruf-OB-Probe}, we simulate the oracle $\mathcal{O}_{\textsf{Probe}}$ in the adversary $\mathcal{A}^\prime$ in Algorithm \ref{alg:adv:ind-uf-OB-Probe} in the $\textsf{RUF}^{ \mathcal{O}^\prime_{\textsf{Enc}} }$ game by the following steps:

\begin{enumerate}

\item Sample $k+2$ independent vectors $\mathbf{e}^{(1)}, \cdots, \mathbf{e}^{(k+2)}$.

\item For $i \in [k+2]$, $\textsf{ct}^{(i)} \gets \mathcal{O}^\prime_{\textsf{Enc}}(\mathbf{e}^{(i)})$.

\item For $i \in [k+2]$,  $d_i \gets \textsf{FE.Dec}(\textsf{pp}, \textsf{sk}_{\mathbf{r}}, \mathbf{c}^{(i)})$, where $\textsf{sk}_{\mathbf{r}}$ is $\textsf{FE.KeyGen}(\textsf{msk}, \textsf{pp}, \mathbf{r})$.

\item Find the vector $\mathbf{r}$ by solving the linear system $\{ \langle \mathbf{r}, {\mathbf{e}^{(i)}} \rangle = d_i \}_{i=1}^{k+2}$.

\item On query $\mathcal{O}_{\textsf{Probe}}(\mathbf{b}^\prime)$, first encode $\mathbf{b}^\prime$ into $\mathbf{y}^\prime$ and find $d \gets \langle \mathbf{x}^{*}, {\mathbf{y}^\prime} \rangle$. Then find a vector $\mathbf{y}^{\prime\prime}$ such that $ \langle \mathbf{r}, {\mathbf{y}^{\prime\prime}} \rangle = d$. Return $\mathcal{O}^\prime_{\textsf{Enc}}(\mathbf{y}^{\prime\prime})$.

\end{enumerate}

However, $\textsf{FE.Dec}$ of constructions in \cite{cryptoeprint:2015/1255, 10.1007/978-3-319-45871-7_24, cryptoeprint:2016/440} rely on some \emph{prior knowledge} of the inner product. Their basic idea is to find $d$ given $g$ and $g^d$, where $g$ is in a group of exponential size. Therefore, $\textsf{FE.Dec}(\textsf{pp}, \textsf{sk}_{\mathbf{r}}, \mathbf{c}^{(i)})$ will probably return $\bot$, representing that $d$ is too large to find.

For this problem, I have three proposals:
\begin{enumerate}
	\item Let the adversary choose $\mathbf{r}$.

	\item Modify the $\textsf{RUF}$ game. 

	\item Do nothing. Claim \cite{cryptoeprint:2015/1255, 10.1007/978-3-319-45871-7_24, cryptoeprint:2016/440} are not ideal \textsf{FE}.
\end{enumerate}

\subsection{Solution I}
Consider the \textsf{SUF} game.
\begin{figure}[H]
\centering

	\begin{minipage}[t]{0.55\textwidth}
	\begin{algorithm}[H]
	\caption{$\textsf{SUF}^{\mathcal{O}}_{\textsf{FE}}(\mathcal{A} = (\mathcal{A}_1, \mathcal{A}_2))$}
	\begin{algorithmic}[1]
		\State $\mathbf{r}, \textsf{st} \gets \mathcal{A}_1(1^\lambda)$

		\If{$\mathbf{r} = \mathbf{0}$}
			\State \Return $\bot$
		\EndIf

		\State $\textsf{msk}, \textsf{pp} \gets \textsf{FE.Setup}(1^\lambda)$

		\State $\textsf{sk}_{\mathbf{r}} \gets \textsf{FE.KeyGen}(\textsf{msk}, \textsf{pp}, \mathbf{r})$

		\State $\mathbf{\tilde{z}} \gets \mathcal{A}_2^{\mathcal{O}} ( \textsf{st}, \textsf{pp}, \textsf{sk}_{\mathbf{r}} )$

		\If{$\mathbf{\tilde{z}}$ is equal to any output of $\mathcal{O}^\prime_{\textsf{Enc}}$ }
			
			\State \Return $0$
		
		\EndIf

		\State $s \gets \textsf{FE.Dec}(\textsf{pp}, \textsf{sk}_{\mathbf{r}}, \mathbf{\tilde{z}} )$

		\State \Return $1_{s \neq \bot}$
	\end{algorithmic}
	\end{algorithm}
	\end{minipage}

\end{figure}

$\mathbf{r} \neq 0$ is because for constructions \cite{10.1007/978-3-319-45871-7_24, cryptoeprint:2016/440}, there exists an algorithm $\textsf{RandEnc}(\textsf{pp})$ that can generate ciphertexts $\textsf{FE.Enc}(\textsf{msk}, \mathbf{r}^\prime)$ of a random unknown vector $\mathbf{r}^\prime$.

I call this security property \emph{selective unforgeability (SUF)}. An SUF \textsf{FE} is RUF since the adversary $\mathcal{A}_1$ can choose $\mathbf{r} \getsdollar \mathbb{F}^k$. We can also add a signature scheme to an \textsf{FE} to make it SUF. Moreover, if we use SUF security, we do not need fh-IND security in our main results in \ref{sec:security_analysis:fh-IPFE:uf}. One can reduce \textsf{option}-UF security of $\Pi$ to SUF security of $\textsf{FE}$ in a simple and intuitive way.

\begin{theorem}[Revision of Theorem \ref{thm:fh-IPFE:ind-ruf-OB-Enroll}]
	Let $\textsf{option} = \{ \mathbf{c_x}, \textsf{csk}, \mathcal{O}_\mathcal{B}, \mathcal{O}_{\textsf{Enroll}} \}$. For any distribution family $\mathbb{B}$, if either one of the following is satisfied:
	\begin{itemize}
		\item \textsf{FE} is both fh-IND and $\mathcal{O}^\prime_{\textsf{KeyGen}}$-RUF (Theorem \ref{thm:fh-IPFE:ind-ruf-OB-Enroll})
		\item \textsf{FE} is $\mathcal{O}^\prime_{\textsf{KeyGen}}$-SUF
	\end{itemize}
then $\Pi$ is $\textsf{option}$-UF. 
\end{theorem}

\begin{theorem}[Revision of Theorem \ref{thm:fh-IPFE:ind-ruf-OB-Probe}]
	Let $\textsf{option} = \{ \mathbf{c_x}, \textsf{csk}, \mathcal{O}_\mathcal{B}, \mathcal{O}_{\textsf{Probe}} \}$. For any distribution family $\mathbb{B}$, if \textsf{FE} is $\mathcal{O}^\prime_{\textsf{Enc}}$-SUF, then $\Pi$ is $\textsf{option}$-UF. 
\end{theorem}

We can also leave both RUF and SUF security.

\subsection{Solution II}

Instead of sampling $\mathbf{r} \getsdollar \mathbb{F}^{k}$, one samples $\mathbf{r} \getsdollar \{0, 1\}^{k+2}$.
\begin{enumerate}

	\item Pick $k+2$ random one-hot vectors $\mathbf{e}^{(i)} \gets (0, \cdots, \overset{i\text{th}}{u_i}, \cdots, 0)$, where $u \getsdollar \mathbb{F}$.

	\item For $i \in [k+2]$, $\textsf{ct}^{(i)} \gets \mathcal{O}^\prime_{\textsf{Enc}}(\mathbf{e}^{(i)})$.

	\item For $i \in [k+2]$, $d_i \gets \textsf{FE.Dec}(\textsf{pp}, \textsf{sk}_{\mathbf{r}}, \textsf{ct}^{(i)})$. Note that $d_i$ can only be $u_i$ or $0$.

	\item Find the vector $\mathbf{r} \in \{0, 1\}^{k+2}$.

	\item On query $\mathcal{O}_{\textsf{Probe}}(\mathbf{b}^\prime)$, first encode $\mathbf{b}^\prime$ into $\mathbf{y}^\prime$ and find $d \gets \langle \mathbf{x}^{*}, {\mathbf{y}^\prime} \rangle$. Then find a vector $\mathbf{y}^{\prime\prime}$ such that $\langle \mathbf{r}, {\mathbf{y}^{\prime\prime}} \rangle = d$. Return $\mathcal{O}^\prime_{\textsf{Enc}}(\mathbf{y}^{\prime\prime})$.

\end{enumerate}

\subsection{Solution III}

We can say these fh-IPFE constructions \cite{cryptoeprint:2015/1255, 10.1007/978-3-319-45871-7_24, cryptoeprint:2016/440} are not \emph{ideal}. Their $\textsf{FE.Dec}$ often aborts on unrestricted inputs. On the contrary, constructions like \cite{10.1007/978-3-030-90567-5_33} do not need discrete logarithm in $\textsf{FE.Dec}$. Unfortunately, it is not \textsf{fh-IND} secure.

%-------------------

\newpage

%-------------------

\section{Trace users with different identities}

The work \cite{simoens2012framework} discusses a security concept related to reusability. It is called \emph{Trace users with different identities}. It is about when a user registers multiple records of its biometrics on the database. The server or compromised database should not be able to find that two records correspond to the same person.

Based on this concept, I design the following games. $\mathcal{S}_{\textsf{Enroll}}$ and $\mathcal{S}_{\textsf{Probe}}$ are two simulators that do not have access to $\mathbb{B}$.

\begin{figure}[H]
\centering

	\begin{minipage}[t]{0.45\textwidth}
	\begin{algorithm}[H]
	\caption{$\textsf{REU}_{\Pi, \mathbb{B}}(\mathcal{A})$}
	\begin{algorithmic}[1]
		\State $b \getsdollar \{0, 1\}$

		\State $\mathcal{B} \getsdollar \mathbb{B}, \mathbb{B} \gets \mathbb{B} \setminus \mathcal{B}$

		\State $\tilde{b} \gets \mathcal{A}^{\mathcal{O}_{\textsf{Reg}}, \mathcal{O}_{\textsf{auth}}}(1^\lambda)$
 
		\State \Return $1_{\tilde{b} = b}$
	\end{algorithmic}
	\end{algorithm}
	\end{minipage}

\end{figure}

\begin{itemize}
	\item $\mathcal{O}_{\textsf{Reg}}$: It maintains a table $\mathcal{T}$ and a counter $i$ initialized to $0$ at the beginning. On query, it updates $i \gets i+1$, and behaves depending on $b$:
	\begin{itemize}
		\item If $b = 0$: It generates key triplets $(\textsf{esk}_i, \textsf{psk}_i, \textsf{csk}_i) \gets \textsf{Setup}(1^\lambda)$, samples an enrollment template $\mathbf{b} \getsdollar \textsf{getEnroll}^{\mathcal{O}_{\mathcal{B}}}()$, stores an entry $\textsf{psk}_i$ in $\mathcal{T}[i]$, and outputs $\mathbf{c_x}_i \gets \textsf{Enroll}(\textsf{esk}_i, \mathbf{b})$ and $\textsf{csk}_i$.

		\item If $b = 1$: It generates key triplets $(\textsf{esk}_i, \textsf{psk}_i, \textsf{csk}_i) \gets \textsf{Setup}(1^\lambda)$, samples a biometric distribution $\mathcal{B}_i \getsdollar \mathbb{B}$ and an enrollment template $\mathbf{b}_i \getsdollar \textsf{getEnroll}^{\mathcal{O}_{\mathcal{B}_i}}()$, stores an entry $(\textsf{psk}_i, \mathcal{B}_i)$ in $\mathcal{T}[i]$, and outputs $\mathbf{c_x}_i \gets \textsf{Enroll}(\textsf{esk}_i, \mathbf{b}_i)$ and $\textsf{csk}_i$.
	\end{itemize}

	\item $\mathcal{O}_{\textsf{auth}}(i)$: This oracle has access to the table $\mathcal{T}$. On input $i$, it retrieves the entry $\mathcal{T}[i]$ and behaves depending on $b$:
	\begin{itemize}
		\item If $b = 0$: Let $\textsf{psk}_i \gets \mathcal{T}[i]$. It samples a probe template $\mathbf{b} \getsdollar \textsf{getProbe}^{\mathcal{O}_{\mathcal{B}}}()$ and outputs $\mathbf{c_y}_i \gets \textsf{Probe}(\textsf{psk}_i, \mathbf{b})$.

		\item If $b = 1$: Let $(\textsf{psk}_i, \mathcal{B}_i) \gets \mathcal{T}[i]$. It samples a probe template $\mathbf{b}_i \getsdollar \textsf{getProbe}^{\mathcal{O}_{\mathcal{B}_i}} ()$ and outputs $\mathbf{c_y}_i \gets \textsf{Probe}(\textsf{psk}_i, \mathbf{b}_i)$.
	\end{itemize}
	
\end{itemize}
If any adversary $\mathcal{A}$ cannot recover the bit $b$; that is, if $\Pr[ \textsf{REU}(\mathcal{A}) \to 1 ] = \negl.$, then a real-world server cannot distinguish whether a list of enrollment records all corresponding to the same person or not.

To provide more power for the adversary, we can also consider the following oracles.

\begin{itemize}
	\item $\mathcal{O}_{\textsf{Reg}}^\prime(\textsf{esk}^\prime)$: This oracle maintains a table $\mathcal{T}$ and a counter $i$ initialized to $0$ at the beginning. If $\textsf{esk}^\prime$ has been queried before, it aborts. Otherwise, it updates $i \gets i+1$ and behaves depending on $b$:
	\begin{itemize}
		\item If $b = 0$: It samples an enrollment template $\mathbf{b} \getsdollar \textsf{getEnroll}^{\mathcal{O}_{\mathcal{B}}}()$ and outputs $\mathbf{c_x} \gets \textsf{Enroll}(\textsf{esk}^\prime, \mathbf{b})$.

		\item If $b = 1$: It samples a biometric distribution $\mathcal{B}_i \getsdollar \mathbb{B}$ and an enrollment template $\mathbf{b}_i \getsdollar \textsf{getEnroll}^{\mathcal{O}_{\mathcal{B}_i}}()$, stores $\mathcal{B}_i$ in $\mathcal{T}[i]$, and outputs $\mathbf{c_x}_i \gets \textsf{Enroll}(\textsf{esk}^\prime, \mathbf{b}_i)$.
	\end{itemize}

	\item $\mathcal{O}_{\textsf{auth}}^\prime(i, \textsf{psk}^\prime)$: This oracle has access to the table $\mathcal{T}$ and behaves depending on $b$:
	\begin{itemize}
		\item If $b = 0$: It samples a probe template $\mathbf{b} \getsdollar \textsf{getProbe}^{\mathcal{O}_{\mathcal{B}}}()$ and outputs $\mathbf{c_y} \gets \textsf{Probe}(\textsf{psk}^\prime, \mathbf{b})$.

		\item If $b = 1$: Let $\mathcal{B}_i \gets \mathcal{T}[i]$. It samples a probe template $\mathbf{b}_i \getsdollar \textsf{getProbe}^{\mathcal{O}_{\mathcal{B}_i}} ()$ and outputs $\mathbf{c_y}_i \gets \textsf{Probe}(\textsf{psk}^\prime, \mathbf{b}_i)$.
	\end{itemize}
	
\end{itemize}

We forbid the adversary to query $\mathcal{O}_{\textsf{Reg}}^\prime$ on the same $\textsf{esk}^\prime$ twice to avoid trivial attacks. Without this restriction, the adversary can generate honest key triplet $(\textsf{esk}, \textsf{psk}, \textsf{csk})$, ask for two records $\mathbf{c_x}_1, \mathbf{c_x}_2$ both corresponding to $\textsf{esk}$, and use $\mathbf{c_y} \gets \mathcal{O}_{\textsf{auth}}^\prime(1, \textsf{psk})$ and $s \gets \textsf{Compare}(\textsf{csk}, \mathbf{c_x}_2, \mathbf{c_y})$ to know the challenge bit $b$. If $b = 0$, $\textsf{Verify}(s) \to 1$ with probability $\textsf{TP}$; otherwise, $\textsf{Verify}(s) \to 1$ with probability $\textsf{FP}$.

%-------------------

\newpage

%-------------------


\iffalse

\section{Achievability of $\mathcal{O}^\prime_{\textsf{Enc}}$-RUF Security}

\begin{assumption}
\label{assump:random_key_ct}

Let $\textsf{FE}$ be an fh-IPFE scheme and $(\textsf{msk}, \textsf{pp}) \gets \textsf{FE.Setup}(1^\lambda)$. Assume that given $\textsf{pp}$, there exist PPT algorithms $\textsf{RandKeyGen}$ and $\textsf{RandEnc}$ that can generate $\textsf{FE.KeyGen}(\textsf{msk}, \mathbf{r})$ and $\textsf{FE.Enc}(\textsf{msk}, \mathbf{r})$ for some random vector $\mathbf{r} \in \mathbb{F}^k$, respectively.

\end{assumption}

Note that constructions in \cite{10.1007/978-3-319-45871-7_24, cryptoeprint:2016/440} satisfy Assumption \ref{assump:random_key_ct}.

\begin{theorem}
\label{thm:fh-IPFE:ind-OEnc-ruf}
Given Assumption \ref{assump:random_key_ct}. If \textsf{FE} is fh-IND and $\emptyset$-RUF, then $\textsf{FE}$ is also $\mathcal{O}^\prime_{\textsf{Enc}}$-RUF.

\end{theorem}

\begin{proof}

Given an adversary $\mathcal{A}$ in the $\textsf{RUF}^{\mathcal{O}^\prime_{\textsf{Enc}}}_{\textsf{FE}}$ game, consider the reduction adversary $\mathcal{R}$ in Algorithm \ref{alg:red:ind-OEnc-ruf} which plays the \textsf{fh-IND} game. $\mathcal{R}$ simulates $\mathcal{O}_\textsf{Enc}^\prime(\mathbf{y}^\prime)$ by first sampling a $\mathbf{r}^\prime \gets \mathbb{F}^k$ and returning $\mathcal{O}_\textsf{Enc}(\mathbf{y}^\prime, \mathbf{r}^\prime)$.

\begin{figure}[h]
\centering

	\begin{minipage}{0.4\linewidth}
	\centering
	\begin{algorithm}[H]
	\caption{$\mathcal{R}^{\mathcal{O}_{\textsf{KeyGen}}, \mathcal{O}_{\textsf{Enc}}}(\textsf{pp})$}
	\label{alg:red:ind-OEnc-ruf}
	\begin{algorithmic}[1]
		\State $\mathbf{c} \gets \textsf{RandKeyGen}(\textsf{pp})$. 

		\State ${\mathbf{\tilde{z}}} \gets {\mathcal{A}}^{\mathcal{O}^\prime_{\textsf{Enc}}} (\textsf{pp}, \mathbf{c})$

		\State $s \gets \textsf{FE.Dec}( \textsf{pp}, \mathbf{c}, \mathbf{\tilde{z}} )$
		
		\If{$ s \neq \bot$}
			\State \Return $\tilde{b} = 0$
		\Else
			\State \Return $\tilde{b} = 1$
		\EndIf

	\end{algorithmic}
	\end{algorithm}
	\end{minipage}
	\hspace{0.05\textwidth}
	\begin{minipage}{0.4\linewidth}
	\centering
	\begin{algorithm}[H]
	\caption{$\mathcal{A}^\prime (\textsf{pp})$}
	\label{alg:adv:ind-OEnc-ruf}
	\begin{algorithmic}[1]
		\State $\mathbf{c} \gets \textsf{RandKeyGen}(\textsf{pp})$. 

		\State ${\mathbf{\tilde{z}}} \gets {\mathcal{A}}^{\mathcal{O}^\prime_{\textsf{Enc}}} (\textsf{pp}, \mathbf{c})$

		\State \Return $\mathbf{\tilde{z}}$
	\end{algorithmic}
	\end{algorithm}
	\end{minipage}
	
\end{figure}

By Assumption \ref{assump:random_key_ct}, $\mathbf{c}$ looks like an honest key of some random vector $\mathbf{r}$. If the challenge bit $b = 0$, $\mathcal{R}$ perfectly simulates an $\textsf{RUF}^{\mathcal{O}^\prime_{\textsf{Enc}}}_{\textsf{FE}}$ game for $\mathcal{A}$ and $\Pr[\tilde{b} = 0 \mid b = 0] = \Pr[ \textsf{RUF}^{\mathcal{O}^\prime_{\textsf{Enc}}}_{\textsf{FE}}(\mathcal{A}) \to 1 ]$. On the other hand, if the challenge bit $b = 1$, then $\mathcal{R}$ simulates an $\textsf{RUF}^{\emptyset}_{\textsf{FE}}$ adversary $\mathcal{A}^\prime$ in Algorithm \ref{alg:adv:ind-OEnc-ruf}. $\mathcal{A}^\prime$ runs $\mathcal{A}$ and simulates $\mathcal{O}^\prime_{\textsf{Enc}}(\mathbf{y}^\prime)$ by simply returning $\textsf{RandEnc}(\textsf{pp})$. Therefore, $\Pr[\tilde{b} = 0 \mid b = 1] = \Pr[ \textsf{RUF}^{\emptyset}_{\textsf{FE}}(\mathcal{A}^\prime) \to 1 ]$.

In conclusion,
\begin{align*}
	\Pr[\textsf{fh-IND}(\mathcal{R}) \to 1] 
	&= \Pr[b = 0] \cdot \Pr[\tilde{b} = 0 \mid b = 0] + \Pr[b = 1] \cdot \Pr[\tilde{b} = 1 \mid b = 1] \\
	&= \frac{1}{2} \left( \Pr[\textsf{RUF}^{\mathcal{O}^\prime_{\textsf{Enc}}}_{\textsf{FE}}(\mathcal{A}) \to 1] + 1 - \Pr[ \textsf{RUF}^{\emptyset}_{\textsf{FE}}(\mathcal{A}^\prime) \to 1 ] \right) \\
	&= \frac{1}{2} + \frac{1}{2} \left( \Pr[\textsf{RUF}^{\mathcal{O}^\prime_{\textsf{Enc}}}_{\textsf{FE}}(\mathcal{A}) \to 1] - \Pr[ \textsf{RUF}^{\emptyset}_{\textsf{FE}}(\mathcal{A}^\prime) \to 1 ] \right)
\end{align*}

Since $\Adv_{\textsf{FE}, \mathcal{R}}^\textsf{fh-IND} = \left| \Pr[\textsf{fh-IND}(\mathcal{R}) \to 1] - \frac{1}{2} \right|$ and $\Adv_{\textsf{FE}, \mathcal{A}^\prime }^{\textsf{RUF}, \emptyset} = \Pr[\textsf{RUF}^{\emptyset}_{\textsf{FE}}(\mathcal{A}^\prime) \to 1 ]$ is negligible,
\[
	\Pr[ \textsf{RUF}^{\mathcal{O}^\prime_{\textsf{Enc}}}(\mathcal{A}) \to 1]  = 2 \cdot \Adv_{\textsf{FE}, \mathcal{R}}^\textsf{fh-IND} + \Adv_{\textsf{FE}, \mathcal{A}^\prime }^{\textsf{RUF}, \emptyset} = \negl.
\]

\end{proof}

With a similar proof, one can also show that

\begin{theorem}
\label{thm:fh-IPFE:ind-OEnc-gamma-ruf}
Given Assumption \ref{assump:random_key_ct}. If \textsf{FE} is fh-IND and $\{ \emptyset, \gamma \}$-RUF, then $\textsf{FE}$ is also $\{ \mathcal{O}^\prime_{\textsf{Enc}}, \gamma \}$-RUF.

\end{theorem}

Since $\{ \mathcal{O}^\prime_{\textsf{KeyGen}}, \gamma \}$-RUF implies $\{ \emptyset, \gamma \}$-RUF, with Theorem \ref{thm:fh-IPFE:ind-OKeyGen-gamma-ruf}, we have the following corollary.

\begin{corollary}
\label{cor:fh-IPFE:ind-OKeyGen-OEnc-ruf}
Given Assumption \ref{assump:only-return-valid-ct} and \ref{assump:random_key_ct}. If \textsf{FE} is fh-IND, then $\textsf{FE}$ is both $\{ \mathcal{O}^\prime_{\textsf{KeyGen}}, \gamma \}$-RUF and $\{ \mathcal{O}^\prime_{\textsf{Enc}}, \gamma \}$-RUF for any $\gamma \leq (1 - \frac{1}{\poly}) \cdot \|\mathbb{F}\|$.
\end{corollary}


%-------------------

\newpage

%-------------------

\section{Instantiation using Other Primitives}

\begin{itemize}
	\item Homomorphic encryption (HE) \cite{10.1007/978-3-642-01957-9_7, 10.1007/978-3-642-40588-4_5, pradel2021privacypreservingbiometricmatchingusing}:
	\begin{itemize}
		\item \cite{10.1007/978-3-642-40588-4_5}: The server owns the secret key for the HE. $\textsf{Compare}(\textsf{csk}, \mathbf{c_x}, \mathbf{c_y})$ is split into two phases:
			\begin{itemize}
				\item With $\mathbf{c_x}$ and $\mathbf{c_y}$, homomorphically compute the encryption of $\textsf{HW}(\mathbf{b} \oplus \mathbf{b}^\prime)$.
				\item Use $\textsf{csk}$, which is the secret key of the HE, to recover $\textsf{HW}(\mathbf{b} \oplus \mathbf{b}^\prime)$.
			\end{itemize}
			If we want to hide the biometrics $\mathbf{b}, \mathbf{b}^\prime$ from the server, two phases of $\textsf{Compare}$ have to be run by two parties.
			\begin{itemize}
				\item The first one only has $\mathbf{c_x}$ without $\textsf{csk}$.
				\item The other one only has $\textsf{csk}$ without $\mathbf{c_x}$.
			\end{itemize}
		\item \cite{10.1007/978-3-642-01957-9_7, pradel2021privacypreservingbiometricmatchingusing}: The user owns the secret key for the HE. The server homomorphically computes the encryption of $\textsf{HW}(\mathbf{b} \oplus \mathbf{b}^\prime) + R$, where $R$ is a random value, and sends it to the user. The user decrypts the value $\textsf{HW}(\mathbf{b} \oplus \mathbf{b}^\prime) + R$, sends it back to the server. The server recovers $\textsf{HW}(\mathbf{b} \oplus \mathbf{b}^\prime)$ by subtracting from $R$.

	\end{itemize}

	\item Fuzzy extractor \cite{10.1145/1030083.1030096, 7980010}: This requires either the enrollment phase passes a public string $Q$ to the authentication phase, or the server sends $Q$ to the user on authentication.

	\item Oblivious transfer \cite{cryptoeprint:2012/586}: The server knows $\mathbf{b}$. Run $k$ OT protocols, where $k$ is the length of $\mathbf{b}$ and $\mathbf{b}^\prime$. At the end, the user will have some random value $R$. The server will have $T = \textsf{HW}(\mathbf{b} \oplus \mathbf{b}^\prime) + R$. The user sends $R$ to the server, and the server recovers $\textsf{HW}(\mathbf{b} \oplus \mathbf{b}^\prime) = T - R$.

\end{itemize}

\fi

%-------------------

\newpage

%-------------------

%-------------------
%% Backup File

% \input{backup.tex}

%-------------------
%% Reference List
\pagebreak

\nocite{*}
\printbibliography


\end{document}
